\chapter{Medida de Lebesgue -  versi\'on 2022}

COMPLETAR CON DATA DE LA  VERSION PDF DE LAS NOTAS DE CLASES QUE TIENEN
ALGUNO AGREGADOS EN COLORES!!!


\section{Preliminares}
Sea $\rr^d$ el espacio eucl\'ideo de dimensi\'on $d$. 

Si $x \in \rr^d$, entonces $x=(x_1,x_2,\ldots,x_d)$ siendo $x_i\in \rr$.

La \emph{norma} de $x$ se define como $|x|=\left(x_1^2+x_2^2+\ldots+x_d^2\right)^{1/2}$ y la \emph{distancia} de $x$ a $y$
se calcula mediante $d(x,y)=|x-y|$.

Si $E\subset \rr^d$, el \emph{complemento} de $E$ es 
$E^C=\left\{ x:x\notin E\right\}$.

Si $E,F\subset \rr^d,$ se tiene que 
$$E-F=E\cap F^C=\left\{x: x\in E \wedge x \notin F\right\}$$ y 
$$d(E,F)=\inf \left\{d(x,y):x\in E, y \in F \right\}.$$

Si $E \subset \rr^d$, entonces $\diametro(E)=\sup\left\{ d(x,y): x,y \in E \right\}$.

Ahora, la \emph{bola abierta} de centro $x$ y radio $r$ est\'a dada por \[B_r(x)=\left\{y \in \rr^d: d(x,y)<r  \right\}.\]

Si $E \subset \in \rr^d$ se dice \emph{abierto} si $\forall x \in E$, $\exists r>0:$\, $B(x,r)\subset E$. 

Y $F \subset \rr^d$ se denomina \emph{cerrado} si y s\'olo si $F^C$ es abierto.

\begin{itemize}
    \item Si $\{E_{\lambda}\}_{\lambda \in \Lambda}$ son abiertos $\Rightarrow$ $ \bigcup\limits_{\lambda \in \Lambda} E_{\lambda}$ 
    es abierto. 
    \item Si $\Lambda$ es finito $\Rightarrow$ $ \bigcap\limits_{\lambda \in \Lambda} E_{\lambda}$ es abierto.
    \item Si los conjuntos $E_{\lambda}$ son cerrados, se obtienen conjuntos cerrados \emph{intercambiando} uniones por intersecciones.
\end{itemize}

Si $E \subset \rr^d$ se dice \emph{acotado} si $E\subset B$ para alguna bola $B$.

Y, $E \subset \rr^d $ es \emph{compacto} si es cerrado y acotado.


\begin{teorema}{}[Cubrimiento Heine-Borel]
Si $E \subset \rr^d$ es compacto y $E\subset \bigcup\limits_{\alpha} \mathcal{O}_{\alpha}$
con $\mathcal{O}_{\alpha}$ abiertos $\forall \alpha$, entonces existen finitos $\alpha:\alpha_1,\alpha_2, \ldots,\alpha_N$, tal que 
$E \subset \bigcup\limits_{i=1}^N \mathcal{O}_{\alpha_i}$.
\end{teorema}

\begin{itemize}
    \item $x \in \rr^d$ es un \emph{punto l\'imite} \'o \emph{punto de clausura} de $E\subset \rr^d$
    si $\forall r>0$,\; $B(x,r)\cap E \neq \emptyset$.
    \item $x \in \rr^d$ es un \emph{punto aislado} de $E\subset \rr^d$ si $\exists r>0$ tal que $B(x,r)\cap E=\{x\}$.
    \item $x \in \rr^d$ es \emph{interior} a $E$ si $\exists r>0$ tal que $B(x,r) \subset E$.
\end{itemize}

Luego, se definen los siguientes conjuntos 
\begin{itemize}
    \item $E^{\circ}=\left\{x| x \;\mbox{ es interior a }\; E \right\}$.
    \item $\overline{E}=\left\{x| x \;\mbox{ es punto l\'imite de }\; E \right\}$.
    \item     $\partial E=\overline{E}-E^{\circ}$.
\end{itemize}

\begin{ejercicio}{}
\begin{enumerate}
    \item $\overline{E}$ es cerrado;
    \item $E$ es cerrado si y s\'olo su $E=\overline{E}$;
    \item $E$ es abierto si  s\'olo su $E=E^{\circ}$;
    \item $\partial E = \partial E^C$;
    \item $E^{\circ} = \overline{E^C}$.
\end{enumerate}
\end{ejercicio}

Por \'ultimo, un conjunto $E \subset \rr^d$ se llama \emph{perfecto} si no tiene puntos aislados.

\section{Rect\'angulos y cubos}
\begin{definicion}{}
$R \subset \rr^d$ es un \emph{rect\'angulo} si 
\[
R=[a_1,b_1]\times\ldots\times [a_n,b_n],
\]
donde $a_j\leq b_j$, para $j=1,2,\ldots,d$.
\end{definicion}

\begin{observacion}{}
Por definici\'on un rect\'angulo $R$ es cerrado y tiene lados paralelos a los ejes. Si 
\begin{itemize}
    \item $d=1$, $R$ es un intervalo cerrado;
    \item $d=2$, $R$ es un rect\'angulo cerrado con lados paralelos a los ejes;
    \item $d=3$. $R$ es un paralelep\'ipedo.
\end{itemize}
\end{observacion}

Si $R$ es rect\'angulo tal que la longitud de lado es $b_i-a_i$, $i=1,\ldots,d$, entonces el \emph{Volumen} de $R$ est\'a dado por
\[
|R|=(b_1-a_1)\ldots(b_d-a_d).
\]

\begin{itemize}
    \item Un \emph{rect\'angulo abierto} se define del modo natural.
    \item Un \emph{cubo} es un rect\'angulo con todos los lados de la misma longitud $l$. \\
    Si $Q$ es un cubo con lados de longitud $l$, entonces $|Q|=l^d$.
\end{itemize}

Una familia de rect\'angulos $\{R_{\lambda}\}_{\lambda \in \Lambda}$ se dice \emph{casi disjunta} si 
\[
R_{\lambda_1}^{\circ} \cap R_{\lambda_2}^{\circ}=\emptyset, \;\; \lambda_1,\lambda_2 \in \Lambda.
\]

Para $N>0$, consideramos el conjunto 
\[
\frac{1}{N}\zz=\left\{ \frac{m}{N}: m\in \zz   \right\}.
\]

\textbf{AGREGAR DIBUJITO DE LOS VALORES DEL CONJUNTO.}


Sea $I$ un intervalo de $\rr$ con longitud $l$. Queremos estimar 
\[
\# \left(I \cap \frac{1}{N}\zz \right).
\]

\begin{lema}{lema:cantidad-puntos-en-intervalo}
\[
Nl-1 \leq \# \left(I \cap \frac{1}{N}\zz \right) \leq Nl+1.\]
\end{lema}

\begin{demo}

Supongamos que $k=\# \left(I \cap \frac{1}{N}\zz \right)$.

Si $k>0$ y $a_1,\ldots,a_k \in I \cap \frac{1}{N}\zz$ tales que 
$a_1<a_2<\ldots<a_k$ y $a_1=\frac{J}{N},a_2=\frac{J+1}{N},\ldots,a_k=\frac{J+k-1}{N}$, con 
$I=[a,b]$. 

Por un lado, se tiene que
\[
a\leq \frac{J}{N}\leq \frac{J+k-1}{N}\leq b \Rightarrow
\frac{k-1}{N}\leq b-a=l.
\]
Por otro, 
\[
\frac{J-1}{N}< a\leq b<\frac{J+k}{N} \Rightarrow
l=b-a< \frac{k+1}{N}.
\]
\end{demo}

\begin{ejercicio}{}
Probar el Lema \ref{lema:cantidad-puntos-en-intervalo} para el caso $k=0$.
\end{ejercicio}

\begin{corolario}{}
Si $I$ es un intervalo de longitud $l$, entonces
\[
\lim\limits_{N \to \infty} \frac{1}{N}\# \left(I \cap \frac{1}{N}\zz \right)=l=|I|.
\]
\end{corolario}

Si $R$ es un rect\'angulo de $\rr^d$ con 
\[
R=\underbrace{[a_1,b_1]}_{I_1}\times \ldots \times \underbrace{[a_d,b_d]}_{I_d},
\]
entonces 
\[
\#\left(R \cap \frac{1}{N}\zz^d \right)=
\# \left(I_1 \cap \frac{1}{N}\zz \right)\ldots
\# \left(I_d \cap \frac{1}{N}\zz \right)
\]
y por tanto
\[
\lim\limits_{N \to \infty} \frac{1}{N^d}
\# \left(R \cap \frac{1}{N}\zz^d \right)
=|R|.
\]
Como 
\[
\# \left(R^{\circ} \cap \frac{1}{N}\zz^d \right) 
\leq
\# \left(R \cap \frac{1}{N}\zz^d \right)
\leq 
\# \left(R^{\circ} \cap \frac{1}{N}\zz^d \right) +2^d.
\]

\textbf{EN LA HOJA 7B DE LAS NOTAS DE CLASE HAY UNA DEDUCCI\'ON DEL $2^d$. AGREGAR!!!}


Tambi\'en tenemos que 
\[
\lim\limits_{N \to \infty} \frac{1}{N^d}
\# \left(R^{\circ} \cap \frac{1}{N}\zz^d \right)
=|R|.
\]

\begin{corolario}{}
Si $R$ es un rect\'angulo y $R= \bigcup\limits_{j=1}^M R_j$ con 
$\{R_j\}_{j=1}^M$ una familia casi disjunta de rect\'angulos, entonces
\[
|R|=\sum\limits_{j=1}^M |R_j|.
\]
\end{corolario}

\begin{demo}
$\leq)$
\[
\begin{split}
|R|=&
\lim\limits_{N \to \infty} \frac{1}{N^d}
\# \left(R \cap \frac{1}{N}\zz^d \right)
\\
\leq & 
\lim\limits_{N \to \infty} \frac{1}{N^d}
\sum\limits_{j=1}^M 
\# \left(R_j \cap \frac{1}{N}\zz^d \right)
\\
=&
\sum\limits_{j=1}^M 
\lim\limits_{N \to \infty} \frac{1}{N^d}
\# \left(R_j \cap \frac{1}{N}\zz^d \right)
\\
=&
\sum\limits_{j=1}^M |R_j|.
\end{split}
\]

$\geq) $
\[\begin{split}
\sum\limits_{j=1}^M |R_j|=&
 \sum\limits_{j=1}^M
\lim\limits_{N \to \infty} 
\frac{1}{N^d} \# \left(R_j^{\circ} \cap \frac{1}{N}\zz^d \right)
\\
=&
\lim\limits_{N \to \infty} 
\frac{1}{N^d} 
\sum\limits_{j=1}^M
\# \left(R_j^{\circ} \cap \frac{1}{N}\zz^d \right)
\\
=&
\lim\limits_{N \to \infty} \frac{1}{N^d}
\# \left(\left(\bigcup\limits_{j=1}^M  R_j^{\circ}\right)
\cap \frac{1}{N} \zz^d\right)
\\
\leq& |R|.
\end{split}
\]
\end{demo}

\begin{lema}{lema:medida-rectangulo-incluido-en-union}
Si $R$ es rect\'angulo y $R\subset \bigcup\limits_{j=1}^M R_j$ donde
$R_j$ son rect\'angulos, entonces
\[
|R|\leq \sum\limits_{j=1}^M |R_j|. 
\]
\end{lema}

\begin{demo}
La prueba de este lema que como ejercicio para el lector. 
\end{demo}


\begin{ejercicio}{}
Demostrar el  Lema \ref{lema:medida-rectangulo-incluido-en-union}.
\end{ejercicio}


\begin{teorema}{teo:abierto-union-intervalos-abiertos-disj}
Todo conjunto abierto $\mathcal{O}$ de $\rr$ es uni\'on numerable \emph{\'unica} de intervalos abiertos disjuntos.
\end{teorema}

\begin{demo}
Sea $x \in \mathcal{O}$ y definimos 
\[
I_x=\bigcup \left\{  
I:\,I\mbox{ es intervalo abierto, }\, I\subset \mathcal{O}, x \in I
\right\}.
\]
\begin{enumerate}
    \item $I_x$ es abierto.
    \item $I_x$ es intervalo. En efecto, si $y,z\in I$ $\Rightarrow \exists\, I_1,I_2$ intervalos tales que $I_1,I_2\subset \mathcal{O}$, 
    $x \in I_1\cap I_2$, $y \in I_1$ y $z\in I_2$. 
    Entonces $I_1\cup I_2$ es intervalo e $I_1\cup I_2 \in \mathcal{O}$. Luego  $[y,z]\subset I_x$. 
    \item Si $I_x \cap I_y \neq \emptyset \Rightarrow I_x=I_y$.
    \'Esto se obtiene a partir de que $I_x \cup I_y$ es intervalo.
    \item\label{it:cantidad-numerable-intervalos} Hay a lo sumo una cantidad numerable de intervalos disjuntos.
    \item \label{it:desc-unica}La descomposici\'on en intervalos abiertos disjuntos es \'unica.
    \end{enumerate}
\end{demo}

\begin{ejercicio}{}
Demostrar  los apartados \ref{it:cantidad-numerable-intervalos} y \ref{it:desc-unica} de la prueba del Teorema \ref{teo:abierto-union-intervalos-abiertos-disj}.
\end{ejercicio}


La generalizaci\'on a $\rr^d$ con $d>1$ presenta algunas dificultades.

\begin{teorema}{}
Todo conjunto abierto de $\rr^d$ puede escribirse como uni\'on numerable de cubos casi disjuntos.
\end{teorema}

\begin{demo}
Sea $\mathcal{O}\in \rr^d$ abierto. Vamos a construir una familia $\mathcal{G}$ de cubos casi disjuntos con $\mathcal{O}=\bigcup\limits_{Q \in \mathcal{G}} Q$.

Procedemos inductivamente en etapas, 

\underline{Primera etapa.}
Consideremos la familia de cubos con v\'ertices en $\zz^d$ que cubre $\rr^d$.
\\
Todos aquellos cubos que quedan contenidos en $\mathcal{O}$ se agregan a $\mathcal{G}$.  Los  cubos  que son disjuntos con $\mathcal{O}$ se tiran y los que tienen parte en $\mathcal{O}$ y en $\mathcal{O}^C$ se dejan  como candidatos.

\underline{$k$-\'esima  etapa.}
Tomamos los candidatos de la etapa $k-1$, dividimos sus lados en $2$ y procedemos como en la primera etapa. 

\textbf{
AGREGAR DIBUJITOS!!!!}

La familia $\mathcal{G}$ es numerable y casi disjunta. 

Sea $x\in \mathcal{O}$. Luego,  $x$ pertenece a un cubo $Q$ de cada etapa $N$ cuyos lados miden $2^{-N}$. Si $N$ es suficientemente grande, se tiene  $Q\subset \mathcal{O}$. Entonces, o bien, $Q$ es agregado a $\mathcal{G}$ en la etapa $N$ o un padre de $Q$ fue agregado en una etapa anterior.
 \end{demo}
 
 \section{Expresiones s-\'adicas}
 
 El objetivo de esta secci\'on es representar n\'umeros reales mediante series num\'ericas.
 
 Sea $d=2,3,\ldots$. 
 
 Supongamos que $x \in [0,1)$. Dividamos el intervalo  $[0,1]$ en $d$ partes iguales, o sea, 
 \[
 0<\frac{1}{d}<\frac{2}{d}<\ldots<\frac{d-1}{d}<1.
 \]
 Ahora, existe un \'unico $j$ tal que 
 \[
 x \in \left[\frac{j}{d}, \frac{j-1}{d}\right).
 \]
 A ese valor de $j$ lo llamamos $a_1$. 
 Y, observamos que \[ \left|x-\frac{a_1}{d}\right|<\frac{1}{d}.\]
 
 A continuaci\'on,  dividimos $\left[\frac{a_1}{j}, \frac{a_1+1}{j} \right)$ en $d$ partes iguales y obtenemos
 \[
 \frac{a_1}{d}<\frac{a_1}{d}+\frac{1}{d^2}<\frac{a_1}{d}+\frac{2}{d^2}
 < \ldots<\frac{a_1}{d}+\frac{d-1}{d^2}<\frac{a_1+1}{d}.
 \]
 Nuevamente, hay un \'unico $j$ tal que 
 \[x \in \left[\frac{a_1}{d}+\frac{j}{d^2}, \frac{a_1}{d}+\frac{j+1}{d^2}\right).
 \]
 Ahora, llamamos $a_2$  a ese valor de $j$. Y notamos que 
 \[
 \left|x-\left(\frac{a_1}{d}+\frac{a_2}{d^2}\right)\right|<\frac{1}{d^2}.
 \]
 Continuando con este procedimiento, obtenemos una sucesi\'on $a_1,a_2,\ldots,a_n$ tal que 
 \[
 \left|  
 x-\left( \frac{a_1}{d}+\frac{a_2}{d^2}+\ldots+\frac{a_n}{d^n}\right)
 \right|<\frac{1}{d^n}.
 \]
 De este modo, habremos encontrado $a_j \in \{ 0,1,2,\ldots,d-1\}$ tales que 
 \begin{equation}\label{eq:expresion-d-adica}
 x=\sum\limits_{j=1}^{\infty} \frac{a_j}{d^j}.
 \end{equation}
 El desarrollo dado por  \eqref{eq:expresion-d-adica} se denomina \emph{expresi\'on  d-\'adica} de $x$. 
 
 Cuando $d=10$, \eqref{eq:expresion-d-adica} se llama \emph{expresi\'on decimal}.
 
Mientras que si $d=2$, \eqref{eq:expresion-d-adica} se denomina \emph{expresi\'on binaria}.
 
La expresi\'on d-\'adica \eqref{eq:expresion-d-adica} suele escribirse
\[
x=\left(0.a_1a_2\ldots\right)_d.
\]

Lamentablemente, la expresi\'on d-\'adica de $x$ no es \'unica. El problema se presenta con las sucesiones que toman el valor $d-1$ de un momento en adelante, pues
\[
\sum\limits_{j=k}^{\infty} \frac{d-1}{d^j}=
\frac{d-1}{d^k} \sum\limits_{j=0}^{\infty} \frac{1}{d^j}=
\frac{d-1}{d^k} \frac{1}{1-\frac{1}{d}}=\frac{1}{d^{k-1}}.
\]
Luego
\[
\left(0.a_1a_2\ldots a_{k-1} (d-1) (d-1)\dots\right)_d =
\left(0.a_1a_2\ldots (a_{k-1} +1) 0\dots\right)_d.
\]
 Si $a_{k-1}+1=d$, entonces 
 \[
 \left(0.a_1a_2\ldots (a_{k-1} +1)\dots\right)_d=
 \left(0.a_1a_2\ldots (a_{k-2} +1) 0\dots\right)_d.
\]
Si $a_{k-2}+1=d$, se procede de la misma manera. 


 \subsection{Conjunto de Cantor}
 Sea $\mathscr{C}_0=[0,1]$. 
 Entonces
 \[
 \begin{split}
 &\mathscr{C}_1=[0,1]-\left(\frac{1}{3},\frac{2}{3}\right)
 \\
  &\mathscr{C}_2=\mathscr{C}_1-\left(\frac{1}{9},\frac{2}{9}\right)
   -\left(\frac{7}{9},\frac{8}{9}\right)
   \\
   &\vdots
   \\
   &\mathscr{C}=\bigcap\limits_{k=1}^{\infty} \mathscr{C}_k.
 \end{split}
 \]
 
 \begin{proposicion}{prop:conjunto-Cantor-propiedades}
 $\mathscr{C}$ es compacto, totalmente disconexo y perfecto. 
 Adem\'as, $\# \mathscr{C}=\# \rr$.
 \end{proposicion}
 
 \begin{ejercicio}{}
 Demostrar la Proposici\'on \ref{prop:conjunto-Cantor-propiedades}.
 \end{ejercicio}
 
 \section{Medida exterior}
 
 \begin{definicion}{}
 Dado $E\subset \rr^d$, su \emph{medida exterior} est\'a dada por 
 \[
 m_*(E)=\inf \sum\limits_{j=1}^{\infty} |Q_j|,
 \]
 donde el \'infimo se toma obre todos los cubrimientos  $E\subset \bigcup\limits_{j=1}^{\infty} Q_j$ de $E$ por medio de cubos.
 \end{definicion}
 
 Claramente $0\leq m_{*}(E)\leq +\infty$
 
 \begin{observacion}{}
 \begin{enumerate}
    \item  Es importante permitir uniones infinitas.
     \item Se pueden usar cubrimientos por rect\'angulos o bolas.
 \end{enumerate}
  \end{observacion}
  
  \begin{ejemplo}{}
  La medida exterior de un punto es cero.
  \end{ejemplo}
  
  \begin{ejemplo}{ejem:medida-ext-cubo}
  Si $Q$ es un cubo, entonces \[m_{*}(Q)=|Q|.\] 
  
  $\leq)$
  Como $Q$ se cubre a s\'i mismo, se tiene que $m_{*}(Q)\leq |Q|$.
  
  $\geq)$ 
  Sea $\left\{Q_j\right\}_{j=1}^{\infty}$ un cubrimiento por cubos de $Q$. Construyamos cubos abiertos $S_j \supset Q_j$ tal que 
  \[
  |S_j|\leq (1+\epsilon) |Q_j|.
  \]
  Como $Q$ es compacto, hay finitos $S_j$ que cubren $Q$, entonces
  \[
  Q\subset \bigcup\limits_{j=1}^N S_j \subset 
  \bigcup\limits_{j=1}^N \overline{S_j}.
  \]
  Por el Lema \ref{lema:medida-rectangulo-incluido-en-union}, tenemos
  \[
  \begin{split}
  |Q|\leq 
  \sum\limits_{j=1}^N |\overline{S_j}|&=
  \sum\limits_{j=1}^N |S_j|
  \\
  &\leq (1+\epsilon) \sum\limits_{j=1}^N |Q_j|\\
  &\leq   (1+\epsilon) \sum\limits_{j=1}^{\infty} |Q_j|.
  \end{split}
  \]
  Como $\epsilon$ es arbitrario, obtenemos
  \[
  |Q|\leq \sum\limits_{j=1}^{\infty} |Q_j|.
  \]
  \end{ejemplo}
  
  \begin{ejemplo}{}
  Si $R\subset \rr^d$ es un rect\'angulo, entonces \[m_{*}(R)=|R|.\]
  
  $\geq)$
  Como en el Ejemplo \ref{ejem:medida-ext-cubo}, se tiene que 
  $|R|\leq m_{*}(R)$.
  
  $\leq)$ Sea $R=I_1\times I_2\times\ldots \times I_d$. Entonces
  \[
  \begin{split}
  |R|&=\lim\limits_{N\to \infty}
  \frac{1}{N^d} \#\left(R \cap \frac{1}{N}\zz^d \right)
  \\
  &=\lim\limits_{N\to \infty}
  \frac{1}{N} \#\left(I_1 \cap \frac{1}{N}\zz \right)\ldots 
  \left(I_d \cap \frac{1}{N}\zz \right).
  \end{split}
  \]
  Si $I_j\cap \frac{1}{N}\zz=
  \left\{
  \frac{\alpha_j}{N},\frac{\alpha_j+1}{N},
  \ldots, \frac{\alpha_j+\beta_j-1}{N}
  \right\}$, entonces
  \[
  \#\left( I_j \cap \frac{1}{N} \zz \right)=\beta_j.
    \]
    Sea $\mathscr{G}_N$ la familia de rect\'angulos que generan
    \[
    \frac{\alpha_j-1}{N},\frac{\alpha_j}{N},\ldots,\frac{\alpha_j+\beta_j}{N}.
    \]
  Hay $(\beta_1+2),\ldots, (\beta_d+2)$ de ellos y miden $\frac{1}{N^d}$.
  Luego, 
  \[
  |R|=\lim\limits_{N\to \infty} \frac{\beta_1}{N}\ldots \frac{\beta_d}{N}.
  \]
  Pero
  \[
  \begin{split}
  m_{*}(R)
  &\leq 
  \lim\limits_{N\to \infty} \sum\limits_{Q \in \mathscr{G}_N} |Q|
  \\
  &=\lim\limits_{N \to \infty} \frac{(\beta_1+2)\ldots (\beta_N+2)}{N^d}
  \\
  &=
  \lim\limits_{N \to \infty}
  \left\{ 
  \frac{\beta_1\beta_2\ldots\beta_d}{N^d}
  \right.
  \\
  &+\frac{2}{N^d}(\beta_2\ldots\beta_d+\beta_1\beta_3\ldots\beta_d+\ldots+\beta_1\beta_2\ldots\beta_{d-1})
  \\
  &+\left.
  \frac{4}{N^d}(\beta_3\ldots \beta_d+\ldots+\beta_1\ldots\beta_{d-2})\right\}
  \\
  &=|R|
  \end{split}
  \]
  \end{ejemplo}
  
  \begin{ejemplo}{}
  $m_{*}(\rr^d)=+\infty$.
  
  Un cubrimiento de $\rr^d$ es un cubrimiento de cualquier cubo $Q$ de modo que 
  \[
  |Q|=m_{*}(Q)\leq m_{*}(\rr^d).
  \]
  Podemos encontrar un cubo $Q_k$ de $\rr^d$ tal que $|Q_k|\xrightarrow[k \to \infty]{} +\infty$.
  
  Por ejemplo, $Q_k=[0,k]^d=
  \underbrace{[0,k]\times[0,k]\times \ldots \times [0,k]}_{d\;\mbox{ intervalos de }\rr}$ con $|Q_k|=k^d$.
    \end{ejemplo}
    
    \begin{ejemplo}{}
    Si $\mathscr{C}$ es el conjunto de Cantor, entonces $m_{*}(\mathscr{C})=0$.
    \begin{demo}
    Se tiene que $\mathscr{C}\subset \mathscr{C}_k$ donde $\mathscr{C}_k$ es una uni\'on de $2^k$ intervalos de longitud $\frac{1}{3^k}$. 
    Luego, 
    \[
    m_{*}(\mathscr{C})\leq \left(\frac{2}{3}\right)^k \xrightarrow[k \to \infty]{}0.
    \]
    \end{demo}
    $\mathscr{C}$ es un  conjunto \emph{infinito no numerable} de $\rr$     con medida exterior nula.
    \end{ejemplo}
    
    \begin{ejemplo}{}
    $m_{*}(\qq)=0$.
    \textbf{COPIAR LA JUSTIFICACION QUE ESTA EN EL PDF!!!}
    \end{ejemplo}
    
    \subsection{Propiedades de la medida exterior}
    
    \begin{observacion}{}
    Sea $E \subset \rr^d$ tal que  $m_*(E)<+\infty$. 
    Entonces, dado $\epsilon>0$, existe un cubrimiento por cubos  tal que $E \subset \bigcup\limits_{j=1}^{\infty} Q_j$ y 
    \[
    \sum\limits_{j=1}^{\infty} m_{*}(Q_j)< m_{*}(E)+\epsilon.
    \]

    Si $m_{*}(E)=+\infty$, se tiene
    \[
    \sum\limits_{j=1}^{\infty} m_{*}(Q_j)\leq m_{*}(E)+\epsilon.
    \]
    \end{observacion}
        
    \begin{observacion}{}[Monoton\'ia]
    Si $E_1\subset E_2$, entonces $m_{*}(E_1)\leq m_{*}(E_2)$.
    
    El resultado se obtiene a partir de que 
    \[
   \left \{ \sum\limits_{j=1}^{\infty} |Q_j|: E_2\subset \bigcup\limits_{j=1}^{\infty} Q_j
    \right\}
    \subset
     \left\{ \sum\limits_{j=1}^{\infty} |Q_j|: E_1\subset \bigcup\limits_{j=1}^{\infty} Q_j
    \right\}.
    \]
    \end{observacion}
    
    \begin{observacion}{}[Numerable $\sigma$-sub-aditividad]\label{obs:sigma-subaditividad}
    Si $E=\bigcup\limits_{j=1}^{\infty} E_j$, entonces 
    \[m_{*}(E)\leq \sum\limits_{j=1}^{\infty} m_{*}(E_j).\]
    \begin{demo}
    Podemos suponer que $m_{*}(E_j)<\infty$.
    
    Sean  $\{Q_{j,k}\}_{k=1}^{\infty}$ las familias de cubos tales que 
    \[
    \sum\limits_{k=1}^{\infty} |Q_{jk}|<m_{*}(E_j)+\frac{\epsilon}{2^j}.
    \]
    Luego $\{Q_{jk}\}_{j, k=1}^{\infty}$ es un cubrimiento por cubos de $\bigcup\limits_{j=1}^{\infty} E_j$
    y por ende de $E$. 
    Luego, 
    \[
    m_{*}(E)\leq \sum\limits_{j,k=1}^{\infty} |Q_{jk}|=
    \sum\limits_{j=1}^{\infty} \sum\limits_{k=1}^{\infty} |Q_{jk}|
    \leq \sum\limits_{j=1}^{\infty} m_{*}(E_j)+\epsilon.
    \]
       \end{demo}
        \underline{Recordar que:} si $\alpha_{jk}\geq 0$ y $\sum\limits_{j,k=1}^{\infty} \alpha_{jk}<\infty$, entonces
    \[
    \sum\limits_{j,k=1}^{\infty} \alpha_{jk}=
    \sum\limits_{j=1}^{\infty} \sum\limits_{k=1}^{\infty}\alpha_{jk}.
    \]
    \end{observacion}
    
    \begin{observacion}{}\label{obs:def-medida-ext-abierto}
    Si $R\subset \rr^d$, entonces
    \[
    m_{*}(E)=
    \inf\{
    m_{*}(\mathcal{O}): \mathcal{O}\supset E\;\mbox{ y }\;\mathcal{O}\;\mbox{ abierto} 
    \}.
    \]
    
    \begin{demo}
    $\leq)$
    Sale a partir de la monoton\'ia pues $E \subset \mathscr{O}$ implica $m_{*}(E)\leq m_{*}(\mathscr{O})$.
    
    $\geq)$
    Sea $\epsilon>0$. Luego, existen cubos  $Q_j$  tales que $E\subset \bigcup\limits_{j=1}^{\infty}Q_j$ y 
    \[
    \sum\limits_{j=1}^{\infty} m_{*}(Q_j)< m_{*}(E)+\frac{\epsilon}{2}.
    \]
    Sean $\widetilde{Q}_j$ cubos abiertos tales que $Q_j \subset \widetilde{Q}_j$ y 
    \[
    m_{*}(Q_j)\leq m_{*}(\widetilde{Q}_j)<m_{*}(Q_j)+\frac{\epsilon}{2}.
    \]
    Entonces 
    $\mathcal{O}=\bigcup\limits_{j=1}^{\infty} \widetilde{Q}_j$         es abierto y 
        \[
        \begin{split}
        m_{*}(\mathscr{O})
        &\leq \sum\limits_{j=1}^{\infty}m_{*}(\widetilde{Q}_j)
        \\
        &<  \sum\limits_{j=1}^{\infty} m_{*}(Q_j)+\frac{\epsilon}{2}
        \\
        &<m_{*}(E)+\epsilon.
        \end{split}
        \]
    \end{demo}
    \end{observacion}
    
    \begin{observacion}{}
       Si $E=E_1\cup E_2$ con $d(E_1,E_2)>0$, entonces
    \[
    m_{*}(E)=m_{*}(E_1)+m_{*}(E_2).
    \]
    \begin{demo}
    $\leq)$ Inmediata a partir de la Observaci\'on \ref{obs:sigma-subaditividad}.
    
    $\geq)$
          Si $m_{*}(E)=\infty$,  la desigualdad es trivial. 
          
          Supongamos que $m_{*}(E)<\infty$. \\
    Sea $0<\delta <d(E_1, E_2)$ y sea $\epsilon>0$.
    Consideremos un cubrimiento de $E$ tal que 
    \[E\subset \bigcup\limits_{j=1}^{\infty} Q_j\;\mbox{  y }\;
    \sum\limits_{j=1}^{\infty} m_{*}(Q_j)<m_{*}(E)+\epsilon.\]
    Podemos suponer que $\diametro(Q_j)<\delta$, porque de no presentarse el caso, dividimos en m\'as cubos hasta lograr el di\'ametro requerido.
    \\
    Entonces, cada cubo $Q_j$ puede intersecar a s\'olo uno de los conjuntos $E_1$ \'o $E_2$.

    Sean \[
    J_1=\left\{j|Q_j \cap E_1\neq \emptyset\right\}
        \]
        y 
    \[
    J_2=\nn-J_1.
    \]
Entonces
    \[
E_1\subset \bigcup\limits_{j\in J_1} Q_j \;\;\mbox { y }\;\;
E_2\subset \bigcup\limits_{j \in J_2} Q_j.
    \]
    Luego, obtenemos
    \[
    m_{*}(E)+\epsilon>\sum\limits_{j=1}^{\infty} m_{*}(Q_j)
    =\sum\limits_{j \in J_1} m_{*}(Q_j)+\sum\limits_{j\in J_2} m_{*}(Q_j)
    \geq 
    m_{*}(E_1)+m_{*}(E_2).
    \]
    \end{demo}
    \end{observacion}
    
\begin{observacion}\label{obs:sigma-aditividad-medida-ext}
Si $E=\bigcup\limits_{j=1}^{\infty} Q_j$ con los  $Q_j$ casi disjuntos, entonces
\[
m_{*}(E)=\sum\limits_{j=1}^{\infty} m_{*}(Q_j).\]
\begin{demo}
$\leq)$ Inmediata a partir de la Observaci\'on \ref{obs:sigma-subaditividad}.

$\geq)$
Sea $\epsilon>0$ y sean $\widetilde{Q}_j\subset Q_j^{\circ}$ cubos cerrados tales que 
\[
m_{*}(\widetilde{Q}_j)>m_{*}(Q_j^{\circ})-\frac{\epsilon}{2^j}.
\]
Entonces $d(\widetilde{Q}_{j_1},\widetilde{Q}_{j_2})>0$.
Luego, 
\[
%\begin{split}
m_{*}(E) =m_{*}\left(\bigcup\limits_{j=1}^{\infty} Q_j\right)
\geq m_{*}\left(\bigcup\limits_{j=1}^{N} Q_j\right)
%\\
\geq m_{*}\left(\bigcup\limits_{j=1}^{N} \widetilde{Q}_j\right)
=\sum\limits_{j=1}^N m_{*}(\widetilde{Q}_j),
%\end{split}
\]
a partir de lo cual tenemos
\[
m_{*}(E)\geq \lim\limits_{N\to \infty}\sum\limits_{j=1}^N  m_{*}(\widetilde{Q}_j)
=\sum\limits_{j=1}^{\infty} m_{*}(\widetilde{Q}_j)
\geq 
\sum\limits_{j=1}^{\infty} \left( m_{*}({Q}_j^{\circ})-\frac{\epsilon}{2^j}\right)
=
\sum\limits_{j=1}^{\infty} m_{*}({Q}_j)-\epsilon.
\]
La desigualdad buscada se consigue haciendo $\epsilon \to 0$.
\end{demo}

\end{observacion}

La demostraci\'on de la Observaci\'on \ref{obs:sigma-aditividad-medida-ext} se aplica a los conjuntos abiertos pues si
$\mathcal{O}$ es abierto, entonces
\[
\mathcal{O}=\bigcup\limits_{j=1}^{\infty} Q_j,
\]
donde los cubos $Q_j$ son casi disjuntos.


\section{Conjuntos medibles y medida de Lebesgue}

\begin{definicion}{}
Un conjunto $E\subset \rr^d$ se llama medible si $\forall \epsilon>0$ existe un abierto $\mathcal{O}\supset E$
con 
\[
m_{*}(\mathcal{O}-E)<\epsilon.
\]
\end{definicion}

Si $E$ es medible, se define $m(E):=m_{*}(E)$.

\begin{proposicion}{}
Si $\mathcal{O}$ es abierto, entonces $\mathcal{O}$ es medible.
\end{proposicion}

\begin{proposicion}{}
\begin{enumerate}
    \item\label{it:conj-medida-cero-medible} Si $m_{*}(E)=0$, entonces $E$ es medible. 
    \item Si $F\subset E$ y $m_{*}(E)=0$, entonces $F$ es medible.
\end{enumerate}
\end{proposicion}

\begin{demo}
\begin{enumerate}
    \item Por la Observaci\'on \ref{obs:def-medida-ext-abierto}, 
    dado $\epsilon>0$, existe $\mathcal{O}$ abierto tal que 
    \[
    m_{*}(\mathcal{O})<m_{*}(E)+\epsilon.
    \]
    Ahora, como $\mathcal{O}-E \subset \mathcal{O}$ y $m_{*}(E)=0$, entonces 
    \[
    m_{*}(\mathcal{O}-E)<\epsilon.
    \]

\item  A partir de que $F\subset E$ y $m_{*}(E)=0$, se tiene que $m_{*}(F)=0$. 
Luego, por \ref{it:conj-medida-cero-medible}, $F$ es medible.
\end{enumerate}
\end{demo}

\begin{observacion}{}
$\mathscr{C}$ es medible.
\end{observacion}