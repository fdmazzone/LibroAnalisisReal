\documentclass{book}
\usepackage{amssymb,amsmath}
\usepackage{polyglossia}
\setmainlanguage{spanish} % Idioma principal
\usepackage{theorem}
\usepackage{times}
\usepackage{array}
\usepackage{graphicx}
\usepackage{hyperref}
\usepackage{multirow}
\usepackage{fancyhdr}
%\usepackage[cp1252]{inputenc}
\usepackage{hhline}
\usepackage{multicol}
\usepackage[a4paper,driver=xetex,top=4.5cm,head=4.5cm, bottom=2cm,%
layouthoffset=0mm, left=2.5cm, right=2.5cm,marginparwidth=0cm]{geometry}
%\usepackage{bm}
%\usepackage{tabular}
\usepackage{fontspec}
 \usepackage[breakable,many]{tcolorbox}
\defaultfontfeatures{Ligatures=TeX}
\usepackage{empheq}
 \setromanfont{Roboto Condensed}
 \usepackage{float}
 \usepackage{mathrsfs} 
%
%
% \renewcommand{\familydefault}{\sfdefault}
%\renewcommand{\familydefault}{\sfdefault}

%%%%%%%%%Estilo de la pagina%%%%%%%%%%%%%%%%%%%%%%%%%%%%%%%%%%%
%%%%%%%%%%%%%%%%%%%%%%%%%%%%%%%%%%%%%%%%%%%%%%%%%%%%%%%%%%%%%%%%%%
% \newcounter{ejer}
% 
% {\theorembodyfont{\normalfont}
% \newtheorem{ejercicio}[ejer]{Ejercicio}}

\newcommand{\rr}{\mathbb{R}}
\newcommand{\qq}{\mathbb{Q}}
\newcommand{\nn}{\mathbb{N}}



\DeclareMathOperator{\atan2}{atan2}
%\DeclareMathOperator{\sen}{sen}
\DeclareMathOperator{\sign}{sign}
\DeclareMathOperator{\sn}{sn}
\DeclareMathOperator{\SO}{SO}
%\DeclareMathOperator{\arcsen}{arcsen}
\DeclareMathOperator{\Or}{O}

\usepackage[framemethod=TikZ]{mdframed}
%%%%%%%%%%%%%%%%%%%%%%%%%%%%%%
%Theorem

%% Ejercicio
\newcounter{ejer} \setcounter{ejer}{0}
\renewcommand{\theejer}{\arabic{ejer}}
\newenvironment{ejer}[2][]{%
\vspace{5pt}
\refstepcounter{ejer}%
\ifstrempty{#1}%
{%
% \mdfsetup{%
% frametitle={%
% \tikz[baseline=(current bounding box.east),outer sep=-0pt]
% \node[anchor=east,rectangle,fill=green!50]
{\noindent\bfseries Ejercicio~\theejer}.}
%
{%
% \mdfsetup{%
% frametitle={%
% \tikz[baseline=(current bounding box.east),outer sep=0pt]
% \node[anchor=east,rectangle,fill=green!50]
{\noindent\bfseries  Ejercicio~\theejer:~#1}.}%
%
%\mdfsetup{innertopmargin=10pt,linecolor=green!50,%
%linewidth=2pt,topline=true,%
%frametitleaboveskip=\dimexpr-\ht\strutbox\relax
%}
%\begin{mdframed}[]
\relax%
\label{#2}}{\vspace{5pt}}%\end{mdframed}}

%Theorem
\newcounter{theo}[chapter] \setcounter{theo}{0}
\renewcommand{\thetheo}{\arabic{section}.\arabic{theo}}
\newenvironment{theo}[2][]{%
\refstepcounter{theo}%
\ifstrempty{#1}%
{\mdfsetup{%
frametitle={%
\tikz[baseline=(current bounding box.east),outer sep=0pt]
\node[anchor=east,rectangle,fill=blue!20]
{\strut Teorema~\thetheo};}}
}%
{\mdfsetup{%
frametitle={%
\tikz[baseline=(current bounding box.east),outer sep=0pt]
\node[anchor=east,rectangle,fill=blue!20]
{\strut Teorema~\thetheo:~#1};}}%
}%
\mdfsetup{innertopmargin=10pt,linecolor=blue!20,%
linewidth=2pt,topline=true,%
frametitleaboveskip=\dimexpr-\ht\strutbox\relax
}
\begin{mdframed}[]\relax%
\label{#2}}{\end{mdframed}}
%%%%%%%%%%%%%%%%%%%%%%%%%%%%%%
%Lemma
\newcounter{lem}[chapter] \setcounter{lem}{0}
\renewcommand{\thelem}{\arabic{section}.\arabic{lem}}
\newenvironment{lem}[2][]{%
\refstepcounter{lem}%
\ifstrempty{#1}%
{\mdfsetup{%
frametitle={%
\tikz[baseline=(current bounding box.east),outer sep=0pt]
\node[anchor=east,rectangle,fill=green!20]
{\strut Lemma~\thelem};}}
}%
{\mdfsetup{%
frametitle={%
\tikz[baseline=(current bounding box.east),outer sep=0pt]
\node[anchor=east,rectangle,fill=green!20]
{\strut Lemma~\thelem:~#1};}}%
}%
\mdfsetup{innertopmargin=10pt,linecolor=green!20,%
linewidth=2pt,topline=true,%
frametitleaboveskip=\dimexpr-\ht\strutbox\relax
}
\begin{mdframed}[]\relax%
\label{#2}}{\end{mdframed}}
%%%%%%%%%%%%%%%%%%%%%%%%%%%%%%
%% Definicion
\newcounter{defini}[chapter] \setcounter{defini}{1}
\renewcommand{\thedefini}{\arabic{section}.\arabic{defini}}
\newenvironment{definicion}[2][]{%
\refstepcounter{defini}%
\ifstrempty{#1}%
{\mdfsetup{%
frametitle={%
\tikz[baseline=(current bounding box.east),outer sep=0pt]
\node[anchor=east,rectangle,fill=green!20]
{\strut Definición~\thedefini};}}
}%
{\mdfsetup{%
frametitle={%
\tikz[baseline=(current bounding box.east),outer sep=0pt]
\node[anchor=east,rectangle,fill=green!20]
{\strut Definición~\thedefini:~#1};}}%
}%
\mdfsetup{innertopmargin=10pt,linecolor=green!20,%
linewidth=2pt,topline=true,%
frametitleaboveskip=\dimexpr-\ht\strutbox\relax
}
\begin{mdframed}[]\relax%
\label{#2}}{\end{mdframed}}

%Proof
\newenvironment{prf}{\noindent\emph{Dem.}}{$\square$ \newline\vspace{5pt}}


%Corolario
\newcounter{cor}[chapter] \setcounter{cor}{0}
\renewcommand{\thecor}{\arabic{section}.\arabic{cor}}
\newenvironment{cor}[2][]{%
\refstepcounter{cor}%
\ifstrempty{#1}%
{\mdfsetup{%
frametitle={%
\tikz[baseline=(current bounding box.east),outer sep=0pt]
\node[anchor=east,rectangle,fill=green!20]
{\strut Corolario~\thelem};}}
}%
{\mdfsetup{%
frametitle={%
\tikz[baseline=(current bounding box.east),outer sep=0pt]
\node[anchor=east,rectangle,fill=green!20]
{\strut Corolario~\thelem:~#1};}}%
}%
\mdfsetup{innertopmargin=10pt,linecolor=green!20,%
linewidth=2pt,topline=true,%
frametitleaboveskip=\dimexpr-\ht\strutbox\relax
}
\begin{mdframed}[]\relax%
\label{#2}}{\end{mdframed}}

\tcbset{highlight math style={enhanced,
  colframe=red!60!black,colback=yellow!50!white,arc=4pt,boxrule=1pt,
  drop fuzzy shadow}}
  
  
  
  
  
  
  


\pagestyle{fancyplain}

 \renewcommand{\sectionmark}[1]
                 {\markright{\thesection\ #1}}


% \lhead[\fancyplain{}{\bfseries\thepage}]
%       {\fancyplain{}{\bfseries\rightmark}}
%
 \rhead[\fancyplain{}{\bfseries\leftmark}]{\fancyplain{}{\bfseries}}




 \lhead[\fancyplain{}{ \includegraphics[scale=.3]{EscudoUNLPam.png}}]{\fancyplain{}{ \includegraphics[scale=.3]{EscudoUNLPam.png}}}

\cfoot{}





  
  
  
  
  
  
  
\begin{document}


\hyphenation{excen-tri-ci-dad}


\begin{large}
\begin{bfseries} % \begin{scshape}
        \noindent Depto de Matem\'atica.\\
        Primer Cuatrimestre de 2022\\                                                                                                                                                                                                                                                                                                                                                
        Teoría de la Medida \\
        Práctica 8: Espacios $L^p$

%\end{scshape}
\end{bfseries}
\end{large}
\par\noindent\rule{\textwidth}{.5pt}








  \begin{ejer}{} Demostrar que el conjunto de las funciones acotadas sobre el conjunto $X$,
  denotado $B(X)$, con la norma 
  $$\parallel f \parallel=\sup_{x\in X}|f(x)| $$
  es un espacio vectorial normado.
	\end{ejer}




\begin{ejer}{} Sea $E\subseteq \rr^n$ medible tal que $m(E)<\infty$.
\\
Supóngase que $\{f_n\}_{ \{ n \in \nn\} }$ es una sucesión de funciones de $L^1(E)$
y que \linebreak $f\in L^1(E)$.
Probar que:
\begin{enumerate}
	\item Si $f_n \stackrel {u}{\longrightarrow}f$, entonces $f_n \stackrel {L^1(E)}{\longrightarrow}f$.
	\item Si $f_n \stackrel {L^1(E)}{\longrightarrow}f$, entonces $f_n \stackrel {m}{\longrightarrow}f$.
	\item Si $f_n \stackrel {L^1(E)}{\longrightarrow}f$, entonces existe $\{f_{n_k}\}\subseteq \{f_n\}$
	tal que  $f_{n_k}{\to}f$ c.t.p.
 \end{enumerate}
\end{ejer}


\begin{ejer}{} Demostrar que:
	\begin{enumerate}
\item $\|\, \cdot\,\|_{\infty}$ es una norma.
\item $L^{\infty}(E)$ es un álgebra.
\item $L^{\infty}(E)$ es un espacio de Banach.  
%\item Las funciones simples son densas en $L^{\infty}(E)$.
	\end{enumerate}
	\end{ejer}



\begin{ejer}{} Sean $g\in L^{\infty}(E)$ y \;\;$\ell_g:L^1(E)\rightarrow \rr$\;\; 
tal que \;\;$\ell_g(f)=\int_E (fg)(x)\,dx$\;\; para
\\ 
$f \in L^1(E)$, entonces
$\ell_g$ es lineal.
\end{ejer}


\begin{ejer}{} Sea $g$ una función medible tal que $fg\in L^1(E)$ $\;\forall f \in L^1(E)$, entonces $g \in L^{\infty}(E)$.
\end{ejer}


\begin{ejer}{}
%\textcolor{red}{Yo creo Sonia que esto estaría bueno dejarlo para los $L^p$}
	Si $\varphi(x) f(x)$ es integrable sobre $E$ para cualquier función $f$ integrable sobre $E$, entonces
  existe una constante finita $C$, tal que $|\varphi(x)|\leq C$ en c.t.p $x$ de $E$.
	\end{ejer}


\begin{ejer}{} Probar que: 
		\begin{enumerate}
	\item 
$\|T\|_{L(X,Y)}=\sup\limits_{\|x\|_{X}\leq 1}\{\|Tx\|_{Y}\}$
es una norma.
	\item 
 $\|T\|_{L(X,Y)}=\sup\limits_{\|x\|_{X}=1}\|Tx\|_{Y}=
\sup\limits_{x \not =0}\frac{\|Tx\|_{Y}}{\|x\|_{X}}=
\\
\inf\{M:\|Tx\|_{Y}\leq M \|x\|_{X}, x\in X\}$.
\item $L(X,Y)$ es un espacio de Banach siempre que $Y$ sea un espacio de Banach.
		\end{enumerate}
\end{ejer}
		
		
\begin{ejer}{} 
Sea $H$ un espacio de Hilbert.
 \\
 Si $x,y\in H$ entonces vale la {\it{Identidad del Paralelogramo:}}
\\
 $\|x+y\|^2+\|x-y\|^2=2(\|x\|^2+\|y\|^2). $
 \end{ejer}
 
%%%%%%%%%%%%%%
\begin{ejer}{}
 Probar que:
	\begin{enumerate}
  \item $\frac{1}{x}\in L^2((1,\infty))$ y $\frac{1}{x}\not \in L^1((1,\infty))$;
  \item $\frac{1}{\sqrt{|x|}}\in L^1((-1,1))$ y $\frac{1}{\sqrt{|x|}}\not \in L^2((-1,1))$.
	\end{enumerate}
\end{ejer}


\begin{ejer}{} 
Si $1\leq p \leq \infty$, demostrar que
	%\begin{enumerate}
		$L^p(E)$ es un espacio de Banach.
		%\item  Las funciones continuas de soporte compacto son densas en $L^p(E)$.
	%\end{enumerate} 
	\end{ejer}
	
	
	\begin{ejer}{}
	Probar la desigualdad de Minkowski usando la desigualdad de H\"older.
	\end{ejer}
	
	\begin{ejer}{}
 Demostrar que si $m(E)<\infty$ y $1\leq p\leq q \leq \infty$, entonces 
$L^{q}(E)\subset L^{p}(E)$.
\\
{\it Sugerencia:} Si $q<\infty$, verificar 
$$\left\|\frac{f}{|E|^{\frac{1}{p}}}\right\|_p\leq
\left\|\frac{f}{|E|^{\frac{1}{q}}}\right\|_q$$
usando Jensen.
Tambi\'en se puede probar el ejercicio usando la desigualdad de H\"older.
\end{ejer}

%%\item Si $f \in L^{1}(\rr^n)$, se define la {\bf transformada de Fourier} de $f$ como
%\[
%\mathcal{F}(f)(y)=(2\pi)^{-\frac{n}{2}}\int_{\rr^n} e^{ix \cdot y}f(x)\,dx.
%\]
%Probar que $\mathcal{F}$ es una transformación lineal continua de $L^{1}(\rr^n)$ en el espacio de las funciones continuas que se anulan en el infinito.


\end{document}
