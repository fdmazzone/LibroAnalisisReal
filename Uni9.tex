\chapter{Espacios $L^p$}

\section{Espacio de funciones integrables}
Sea
$
L^1=L^1(E)=
\{
f: f \; \mbox{ es integrable en } E
\}.
$

$L^1(E)$ es un espacio vectorial donde se define la relaci\'on de equivalencia
\[
f \sim g \Longleftrightarrow f=g \;\mbox{ en c.t.p. de } E.
\]   

 
La clase de equivalencia de $f$  que suele notarse con ``$(f)$'', pero en un abuso de notaci\'on, se indicar\'a  $f$
y $L^1(E)/_{\sim}$ se denotar\'a con $L^1(E)$.

Para $f\in L^1(E)$, definimos
\[
\|f\|_1=\int_E|f(x)|\,dx.
\]

$\| \cdot \|_1$ es una funci\'on no negativa sobre $L^1(E)$ que cumple
\begin{enumerate}
    \item [H1)] $\|f+g\|_1\leq \|f\|_1+\|g\|_1$.
    \item [H2)] $\|\lambda f\|_1=|\lambda|\|f\|_1$.
    \item [H3)] $\|f\|_1=0$ si y s\'olo si f=0.
\end{enumerate}

As\'i $(L^1(E),\|\cdot\|_1)$ es un espacio normado. 
Y se define 
\[
d(f,g)=\|f-g\|_1.
\]

Si un espacio normado es completo se denomina espacio de \emph{Banach}. 

\begin{teorema}{}\label{teorema:L1-completo}
$L^1(E)$ es un espacio de Banach.
\end{teorema}

\begin{demo}
En la prueba 
%del Teorema\ref{teorema:L1-completo}, 
usaremos el siguiente ejercicio.

\begin{ejercicio}{}
Sea $(X,\| \cdot\|)$ un espacio normado. \\
$(X,\| \cdot\|)$ es un espacio de Banach si y s\'olo si cada vez que 
\[
\sum\limits_{n=1}^{\infty} \|x_n\|<\infty,
\]
existe $x \in X$ tal que $\sum\limits_{n=1}^{\infty} x_n$.
\end{ejercicio}

Sea pues $f_i \in L^1(E)$ tal que 
\[
\sum\limits_{i=1}^{\infty} \|f_i\|_1<\infty.
\]

Sea \[ \Phi(x)=\sum\limits_{i=1}^{\infty} |f_i(x)|.\]

$\Phi \in L^1(E)$ y entonces $\Phi$ es finita en casi todo punto. Por lo tanto, 
\[
S(x)=\sum\limits_{i=1}^{\infty} f_i(x),
\]
est\'a bien definida excepto sobre un conjunto de medida nula.
%
Adem\'as $|S|\leq \Phi$ y  $S\in L^1(E)$.
%
Ahora, si 
\[
S_j(x)=\sum\limits_{i=1}^j f_i(x),
\]
tenemos que $S_j \to S$ en c.t.p. y $|S_j-S|\leq 2\Phi \in L^1(E)$ y por lo tanto, por Convergencia Mayorada de Lebesgue (Corolario \ref{cor:conv-mayorada}), 
\[
\|S_j-S\|_1=\int_E |S_j-S|\,dx \to 0 \mbox{ cuando } j \to \infty.
\]
\end{demo}

\begin{teorema}{}
Sea $f_i \in L^1(E)$ tal que $f_i \xrightarrow{L^1(E)} f$, entonces existe una subsucesi\'on $f_{i_k}$ tal que $f_{i_k}\to f$ en c.t.p.
\end{teorema}

\begin{demo}
Por la desigualdad de Chebyshev, se tiene
\[
m(\{x:|f(x)-f_i(x)|>\epsilon \})\leq 
\frac{1}{\epsilon}\int_E |f(x)-f_i(x)|\,dx \xrightarrow{i \to \infty}0.
\]
\'Esto muestra que si $f_i \xrightarrow{L^1(E)} f$ entonces
$f_i \xrightarrow{m} f$.
Luego, existe una subsucesi\'on $f_{i_k}\to f$ en c.t.p.
\end{demo}

\subsection{Conjuntos densos en $L^1(E)$}

\begin{teorema}[Simples $\Rightarrow$ densas en $L^1(E)$]{}
El conjunto de las funciones simples es denso en $L^1(E)$.
\end{teorema}

\begin{demo}
Sea $f\in L^1(E)$. Existe una sucesi\'on $\{f_i\}$ de funciones simples tal que $|f_i|\leq |f|$ y $f_i \to f$. Luego, $f_i \in L^1(E)$ y $|f-f_i|\leq 2|f|$. As\'i, por  Convergencia Mayorada de Lebesgue (Corolario \ref{cor:conv-mayorada}), $f_i \to f$ en $L^1(E)$.
\end{demo}

\begin{definicion}{}
Diremos que $\varphi$ es una funci\'on escalonada si 
$\varphi=\sum\limits_{i=1}^N \alpha_i \chi_{I_i}$
con $I_i$ intervalos.
\end{definicion}

\begin{teorema}[Escalonadas $\Rightarrow$ densas en $L^1(E)$]{}
El conjunto de las funciones escalonadas es denso en $L^1(E)$.
\end{teorema}

\begin{demo}
Bastar\'ia ver que $\chi_{E_0}$ es l\'imite de funciones simples para cada $E_0\subset E$ con $m(E_0)<\infty$.
Ahora bien, ya se demostr\'o que existe una sucesi\'on  de conjuntos elementales $A_k$, i.e. $A_k=\bigcup\limits_{j=1}^{N_k} I_j^k$ tales que 
\[
m\left(E_0\Delta A_k\right) \to 0,
\]
siendo
\[
m\left(E_0\Delta A_k\right)=
\int \left|\chi_E-\chi_{A_k}\right|\,dx=
\bigintss \left|
\chi_E-\sum\limits_{j=1}^{N_k} \chi_{I_j^k}
\right|\,dx.
\]
\end{demo}

\begin{ejercicio}{}
El conjunto de las funciones continuas con soporte compacto es denso en $L^1(\rr^n)$.
\end{ejercicio}

\begin{ejercicio}{}
$L^1(E)$ es separable.
\end{ejercicio}

\section{Espacio de funciones esencialmente acotadas}

La función medible $f$ es esencialmente acotada sobre $E$ si existe $M>0$ tal que
\begin{equation}\label{eq:defi-norma-infinito}
|f(x)|\leq M \mbox { en c.t.p. de  } E.
\end{equation}

$M$ se llama cota esencial de $f$.  Si $f$ es esencialmente acotada sobre $E$, denotamos por $\|f\|_{\infty}$ al \'infimo de los $M$ que satisfacen \eqref{eq:defi-norma-infinito}.

\begin{ejercicio}{}
$\|f\|_{\infty}$ es cota esencial de $f$ y de hecho es la m\'a chica de las cotas.
\end{ejercicio}


Si $f$ no es esencialmente acotada ponemos $\|f\|_{\infty}=\infty$.

Notamos con $L^{\infty}(E)$ al conjunto de todas las funciones esencialmente acotadas.

\begin{ejercicio}{}
\begin{enumerate}
    \item $L^{\infty}(E)$ es espacio vectorial.
    \item $\left(L^{\infty}(E),\|\cdot\|_{\infty}\right)$ es un espacio de Banach.
    \item Las funciones simples son densas en $L^{\infty}(E)$.
\end{enumerate}
\end{ejercicio}

\begin{ejemplo}{}
Las funciones escalonadas no son densas  en $L^{\infty}(E)$.

Por ejemplo, si $\varphi$ es escalera en $\rr$ se tiene  $\|\chi_{\qq}-\varphi\|_{\infty}\geq \frac{1}{2}$. 

\textbf{EN LAS NOTAS MANUSCRITAS APARECE ALGO ASI...QUE NO SE VE BIEN..FALTA ALGO...!!!}

En efecto, si tomamos un abierto $G$ tal que $\qq\subset G$ y $|G|<\frac{1}{2}$.

\[ 
\left|\frac{1}{2}-\alpha_n\right|\leq \|f-\varphi_n \|_{\infty}<\frac{1}{2}
\]
\end{ejemplo}

Si $f\in L^1(E)$ y $g \in L^{\infty}(E)$ entonces $fg\in L^{\infty}(E)$
y 
\[
\int_E |fg|\,dx \leq \|f\|_1\|g\|_{\infty}.
\]
As\'i est\'a bien definida la funci\'on 
\[
\begin{split}
\ell_g:L^1(E)\to \rr\\
\ell_g(f)=\int_E fg\,dx.
\end{split}
\]
$\ell_g$ es lineal y verifica 
\[
|\ell_g(f)|\leq \|f\|_1 \|g\|_{\infty}.
\]

Diremos que una funci\'on lineal $T$ de un espacio normado $X$ en un espacio normado $Y$ es acotada si existe $M>0$ tal que 
\[
\|Tx\|_Y \leq M \|x\|_X.
\]

\begin{proposicion}{}
$T$ es acotada si y s\'olo si $T$ es continua.
\end{proposicion}

\begin{demo}
$\Rightarrow)$ Sean $\epsilon>0,$  $\delta=\frac{\epsilon}{M}$ y 
$\|x-y\|_X <\epsilon.$
Entonces
\[
\|Tx-Ty\|_Y = \|T(x-y)\|_Y \leq M \|x-y\|_X<\epsilon.
\]
$\Leftarrow)$ $T(0)=0$ y $T$ es continua en cero. Luego, existe $\delta>0$ tal que si 
\[
\|x\|_X \Rightarrow \|Tx\|_Y<1.
\]
Sea $x$ cualquiera, entonces $\left\|\frac{\delta}{2} \frac{x}{\|x\|_X}\right\|_X<\delta$. 
Luego, 
\[
\left\|T\left( \frac{\delta}{2} \frac{x}{\|x\|_X}   \right)\right\|_Y
\]
entonces
\[ 
\|Tx\|_Y \leq \frac{2}{\delta}\|x\|_X.
\]
\end{demo}

Al conjunto de todas las funciones lineales y acotadas entre $X$ e $Y$ se lo denota $L(X,Y)$.

$L(X,Y)$ es espacio vectorial y podemos definir la norma
\[
\begin{split}
\|T\|=&\sup\left\{ \|Tx\|: \|x\|\leq 1\right\}\\
=&\sup\left\{ \|Tx\|: \|x\| = 1\right\}\\
=&\sup\left\{ \frac{\|Tx\|}{\|x\|}: x \in X\right\}\\
=&\inf\left\{ M: \|Tx\|\leq M \|x\|\;\forall x \in X\right\}.
\end{split}
\]

\begin{ejercicio}{}
Probar la equivalencia entre las definiciones de $\|T\|.$
\end{ejercicio}

\begin{ejercicio}{}
Si $Y$ es completo, entonces $L(X,Y)$ tambi\'en lo es.
\end{ejercicio}

As\'i $L(X,\rr)$ es un espacio de Banach que se llama \emph{dual} de $X$ y se denota por $X^*$,

Diremos que $X$ e $Y$ son isom\'etricos si existe un isomorfismo $T:X\to Y$ tal que $\|Tx\|_Y=\|x\|_X$. 

Escribiremos $X \cong Y$.

\begin{teorema}{teo:dual-de-L1}
$(L^1)^* \cong L^{\infty}.$
\\
Com\'unmente se escribe, por abuso de notaci\'on, $(L^1)^{*}=L^{\infty}$.
\end{teorema}

\begin{demo}
Consideremos
\[
\begin{split}
\ell:L^{\infty}  & \longrightarrow (L^1)^*
\\
g & \longmapsto \ell_g.
\end{split}
\]
Ahora 
\[
\|\ell(g)\|_{(L^1)^*}=\|\ell_g\|_{(L^1)^*}=
\sup\limits_{\|f\|_1 =1} |\ell_g(f)| \leq 
\sup \limits_{\|f\|_1=1} \|g\|_{\infty} \|f\|_1 =\|g\|_{\infty}.
\]
Sea ahora $g$ no nula, entonces dado $\epsilon>0$ tenemos
\[
m\left(\{x: |g|>\|g\|_{\infty}-\epsilon|  \}\right)>0.
\]
Si $A \subset \left(\{x \big|\; |g|>\|g\|_{\infty}-\epsilon|  \}\right)$
con $0<m(A)<\infty$ y definimos 
\[
f=\sgn(g) \frac{1}{m(A)} \chi_A,
\]
entonces
\[
\begin{split}
    \|\ell_g\|_{(L^1)^*} \geq& \ell_g(f)=\int_E gf\,dx 
    \\
    &= \frac{1}{m(A)} \int_A |g|\,dx\geq \|g\|_{\infty}-\epsilon.
\end{split}
\]
Luego, $\|\ell_g\|_{(L^1)^*}=\|g\|_{\infty}$.

La prueba de la segunda parte del teorema se ver\'a m\'as adelante.
\end{demo}

\section{Espacio de funciones de cuadrado integrable}

Denotamos por $L^2(E)$ al conjunto de todas las funciones $f$ medibles sobre $E$ tales que 
\[ 
\int_E |f|^2\,dx <\infty.
\]
Si $f,g \in L^2(E)$ entonces $fg \in  L^1(E)$ pues
\[
fg \leq \frac{1}{2}\left(f^2+g^2\right).
\]
Luego, $(f+g)^2 \in L^1(E)$ y entonces se dice que $L^2(E)$ es espacio vectorial. 

Para $f,g\in L^2(E),$ definimos 
\[
(f,g)=\int_E fg\,dx.
\]

$(\, ,\, )$ es un producto escalar o interno sobre $\rr$, \'esto es, 
$(\, ,\, )$ satisface
\begin{enumerate}
    \item $(f,g+h)=(f,g)+(f,h)$,
    \item $(f,g)=(g,f)$,
    \item $(f,f)\geq 0$,
    \item $(f,f)=0$ si y s\'olo si $f=0$.
\end{enumerate}

Si ponemos $\|f\|_2=(f,f)^{1/2}$, obtenemos una norma y se satisface la desigualdad de Cauchy-Schwarz
\[
(f,g)\leq \| f \|_2\|g\|_2.
\]
Un espacio vectorial $H$ sobre el que se tiene definido un producto escalar $(\, ,\, )$ el cual da origen a una norma que hace a $H$ completo se llama espacio de \emph{Hilbert}.

\begin{teorema}{}
$L^2(E)$ es espacio de Hilbert.
\end{teorema}

\begin{demo}{}
Resta ver la completitud. 

Sea $f_n$ una sucesi\'on de Cauchy, entonces 
dado $\epsilon>0$ tenemos que 
\[
m\left(\left\{x: |f_n(x)-f_m(x)|>\epsilon \right\} \right)
\leq \frac{1}{\epsilon^2} \int_E |f_n(x)-f_m(x)|^2\,dx.
\]
As\'i, $\{f_n\}$ es fundamental en medida. Por lo tanto, 
existe $f$ tal que $f_n \xrightarrow[]{m} f$ y 
existe una subsucesi\'on $\{n_k\}$ tal que $f_{n_k}\to f$ en c.t.p.de $E$.

A continuaci\'on veremos  que $f \in L^2(E)$ y $f_n \to f$ en $L^2(E)$.

Como $\{f_n\}$ es sucesi\'on de Cauchy en $L^2(E)$, entonces est\'a acotada en $L^2(E)$, o sea, existe $M>0$ tal que $\|f\|_2\leq M$. 
Luego, por el Lema de Fatou (Lema \ref{lema:Fatou}), 
\[
\int_E |f|^2\,dx\leq \liminf\limits_{k \to \infty} \int_E |f_{n_k}|^2\,dx \leq M^2.
\]
Nuevamente, por el Lema de Fatou (Lema \ref{lema:Fatou}), llegamos a 
\[
\int_E |f_n-f|^2\,dx\leq \liminf\limits_{k \to \infty} \int_E |f_n-f_{n_k}|^2\,dx \leq \epsilon,
\]
si $n$ es grande.
\end{demo}

\subsection{Algunos hechos sobre espacios de Hilbert}

\begin{teorema}{}\label{teo:existe-proyeccion-sobre-H}
Sean $H$ un espacio de Hilbert,  $M$ un subsespacio cerrado de $H$ y  $x \in H$. Entonces existe $m \in M$ tal que 
\[
\|x-m\| \leq \|x-h\|\; \forall h \in M.
\]
\end{teorema}

\begin{demo}
Sea $m_k \in M$ tal que   \textbf{CHEQUEAR LA SIGUIENTE FORMULA!!!, en el pdf NO SE VE BIEN!!!}
\[ \|x-m_k\| \leq  d\|x_1\|+\frac{1}{k}. \]
Utilizamos la \emph{Identidad del Paralelogramo}
\[ 
\left\| \frac{x+y}{2} \right\|^2 +\left\| \frac{x-y}{2} \right\|^2
= \frac{1}{2}\left(\| x \|^2+ \|y\|^2  \right), 
\]
con los vectores $x-m_k$
\[ 
\left\| \frac{x-m_k+x-m_j}{2} \right\|^2 +\left\| \frac{x-m_k-(x-m_j)}{2} \right\|^2
= \frac{1}{2}\left(\| x+m_k \|^2+ \|x-m_j\|^2  \right), 
\]
y obtenemos
\[
\left\| \frac{m_j-m_k}{2} \right\|^2 \leq \frac{1}{2} \left(d_k^2 +d_j^2\right)-d^2
\xrightarrow[]{k,j \to \infty}0.
\]
De este modo, resulta que $\{m_j\}$ es una sucesi\'on de Cauchy, entonces $m_j \to m$ 
y se satisface lo pedido.
\end{demo}

\begin{definicion}{}
$m$ del Teorema \ref{teo:existe-proyeccion-sobre-H} se llama la proyecci\'on de $x$ sobre $M$ y se denota $P_M(x)$.
\end{definicion}

\begin{teorema}{}
Sean $H$ y $M$ como en el Teorema \ref{teo:existe-proyeccion-sobre-H}. 
Entonces son equivalentes
\begin{enumerate}
    \item $m=P_M(x)$,
    \item $x-m \perp M$, o sea  $(x-m,h =0)$ $\forall h \in M$.
\end{enumerate}
\end{teorema}

\begin{demo}
$1) \Rightarrow 2)$
Sea $h\in M$. Entonces
\[
\begin{split}
\|x-(m+\lambda h)  \|^2 \geq \| x-m\|^2 \Leftrightarrow
\\
\|x-m\|^2-2\lambda (x-m,h)+\lambda^2 \|h\|^2\geq \|x-m\|^2.
\end{split}
\]
A partir de aqu\'i, se deduce que $(x-m,h)\geq 0$. Luego, como \'esto vale para $-h$ se concluye $2)$.

$2) \Rightarrow 1)$ Sea $h\in M$. Luego
\[
\|x-m\|^2=(x-m,x-m)=(x-m,x-h)\leq \|x-m\| \|x-h\|.
\]
Si $v \in H$ entonces $\ell_v: H \to \rr$ definida por $\ell_v(u)=(u,v)$
satisface $l_v \in H^*$.
\end{demo}


\begin{teorema}{}[Representaci\'on de Riesz]
La aplicaci\'on $v \longmapsto \ell_v$ es una isometr\'ia suryectiva de
$H$ en $H^*$.
\end{teorema}

\begin{demo}
Se tiene 
\[
\|\ell_v\|=\sup\limits_{\|u\|\leq 1} \ell_v(u) =
\sup\limits_{\|v\|\leq 1} (u,v)\leq \|v\|.
\]
Adem\'as
\[
\|\ell_v\|\geq \left\|\ell_v\left(\frac{v}{\|v\|}\right) \right\|
=\|v\|.
\]
Claramente, $\ell$ es lineal. Resta ver que $\ell$ es suryectiva. 

Sea $\ell \in H^*$ y supongamos que $\ell \neq 0$. Definimos
\[
N=\{ x \in H: \ell(x)=0  \}.
\]
Como $\ell $ es lineal y continua, $N$ es un subespacio cerrado y $N \neq H$. Sea $y_0 \notin N$. Existe $x \in N$ tal que \[y:=y_0-x \perp N.\]
Podemos suponer $\|y\|=1$. 
Sea $x \in H$ entonces
\[
x=\left( x- \frac{\ell(x)}{\ell(y)} y \right)+\frac{\ell(x)}{\ell(y)} y=:
u+\alpha y.
\]
Notar que $u \in N$. Sea $v=\ell(y) y$ entonces
\[
\ell(\alpha y)=(\alpha y,v).
\]
Si $x \in N$ entonces $(v,x)=0$. Luego, $\forall x \in H$ tenemos
\[
\ell(x)=\ell(u)+\ell(\alpha y)=(u,v)+(\alpha y,v)=(x,v).
\]
\end{demo}

\begin{lema}{lem:fg-en-L-infinito-para-f-en-L1}
Si 
$
\int_E |fg|\,dx <\infty
$
para toda $f \in L^1(E)$, entonces $g \in L^{\infty}(E)$.
\end{lema}

\begin{demo}
Supongamos que $g \notin L^{\infty}(E)$. Sea $A_i$ tal que $0<m(A_i)<\infty$ y 
\[
A_i\subset \left\{x: 2^i< |g(x)|\right\}.
\]
Consideramos  \[f = \sum\limits_{i=1}^{\infty} \frac{1}{2^im(A_i)} \chi_{A_i} \] tal que $f \in L^1(E)$. 
Luego, $|fg| \notin L^1(E)$.
\end{demo}

Ahora completaremos la demostraci\'on de
%l Teorema \ref{teo:dual-de-L1}: 
$(L^1)^*=L^{\infty}$.

\begin{demo}{}[Segunda parte de la prueba del Teorema \ref{teo:dual-de-L1}]
Sea $\ell \in (L^1)^*$. Supongamos que $m(E)<\infty$.
Por la desigualdad de Cauchy-Schwarz tenemos
\[
|\ell(f)|\leq \|\ell\|\|f\|_1=\|\ell\| \int_E |f|\,dx 
\leq \|\ell\| \left[m(E)\right]^{1/2}\|f\|_2, 
\]
si $f \in L^2(E)$.
Entonces $\ell$ define un elemento de $(L^2)^*$. Luego, existe $g \in L^2(E)$ con 
\[
\ell(f)=\int_E fg\,dx=\ell_g(f), \;\forall f \in L^2(E).
\]
Como $\ell$ es lineal y continua, 
\[
\int_E |fg|\,dx \leq \|\ell\| \|f\|_1, \;\forall f \in L^2(E).
\]
Si $f \in L^1(E)$, definimos $f_n=\min\{n, |f|\}$. Luego,  se tiene que $f_n \in L^2(E)$ y $f_n \nearrow |f|$. Entonces, por el Teorema de Beppo-Levi (Teorema \ref{teo:Beppo-Levi}), se obtiene
\[
\int_E |f| |g|\,dx =\lim\limits_{n \to \infty} \int_E f_n g\,dx 
\leq \lim\limits_{n \to \infty} \|\ell\| \|f_n\|_1
=\|\ell\| \|f\|_1.\]
Por el Lema \ref{lem:fg-en-L-infinito-para-f-en-L1}, resulta que $g \in L^{\infty}(E)$.
Como $L^2 \cap L^1$ es denso en $L^1$ tenemos $\ell-\ell_g \in L^1$.

Si $m(E)=\infty$, escribimos $E=\bigcup\limits_{k=1}^{\infty} E_k$
con $m(E_k)<\infty$ y $E_k \cap E_j=\emptyset$ si $k \neq j$.
\end{demo}





\section{Finalmente...los espacios $L^p$}

Sea $1\leq p<\infty$.  Dado $E$ medible y $f:E\to \rr$, definimos 
\[
\|f\|_p=\left\{ \int_E |f(x)|^p\,dx \right\}^{\frac{1}{p}}.
\]

Las funciones que verifican $\|f\|_p<\infty$ forman la clase $L^p(E)$.

$L^p(E)$ es un espacio normado con $\|\cdot \|_p$ como norma. A  continuaci\'on, demostraremos la desigualdad triangular. 

Supondremos que $\|f\|_p,\|g\|_p>0$. Como $t^p$ es convexa para $1\leq <\infty$ tenemos 
\[
\left( \frac{|f(x)|+|g(x)|}{\|f\|_p+\|g\|_p}   \right)^p \leq 
\frac{\|f\|_p}{\|f\|_p+\|g\|_p} \frac{|f|^p}{\|f(x)\|_p}+  
\frac{\|g\|_p}{\|f\|_p+\|g\|_p} \frac{|g(x)|^p}{\|g\|_p}.
\]
Integrando sobre $E$ obtenemos
\[
\int_E \left(\frac{|f(x)|+|g(x)|}{\|f\|_p+\|g\|_p}\right)^p \,dx\leq 1.
\]
A partir de aqu\'i, despejando se deduce la desigualdad triangular tambi\'en llamada \emph{desigualdad de Minkowski}.

\begin{ejercicio}{}
$L^p(E)$ es espacio de Banach.
\end{ejercicio}

Sean $1<p<\infty$ y $q$ tal que $\frac{1}{p}+\frac{1}{q}=1$. \\$q$ se llama exponente conjugado de $p$ y satisface  $1<q<\infty$.

\begin{lema}{}
 Si $a,b\geq 0$ se cumple que
\[ab\leq \frac{a^p}{p}+\frac{b^q}{q}.\]
\end{lema}

\begin{demo}
Empleando propiedades de las funciones exponenciales y logar\'itmicas se puede escribir
\[
ab=e^{\ln a} e^{\ln b}=e^{\frac{\ln a^p}{p}} e^{\frac{\ln b^p}{p}}=
e^{\frac{\ln a^p}{p}+\frac{\ln b^p}{p}}.
\]
Dado que $e^x$ es funci\'on convexa, se tiene que 
\[
e^{\frac{\ln a^p}{p}+\frac{\ln b^p}{p}}
\leq 
\frac{1}{p} e^{\ln a^p}+\frac{1}{q}e^{\ln b^q}=\frac{a^p}{p}+\frac{b^q}{q}.
\]
\end{demo}

Si ahora $f \in L^p$ y $g \in L^q$, reemplazamos $a$ por $\frac{|f(x)|}{\|f\|_p}$ y $b$ por $\frac{|g(x)|}{\|g\|_p}$,   
integramos sobre $E$ y obtenemos
\[
\int_E \frac{|f(x)|}{\|f\|_p} \frac{|g(x)|}{\|g\|_p} \,dx \leq \frac{1}{p}+\frac{1}{q}=1.
\]
A partir de aqu\'i, deducimos que $fg\in L^1(E)$ y se satisface 
\[
\int_E fg \,dx \leq \|f\|_p \|g\|_q, \mbox{ \textbf{Desigualdad de H\"older.}}
\]

Otra desigualdad importante es la  que se presenta a continuaci\'on.

\begin{lema}{}[Desigualdad de Jensen]
 Sea $f \in L^1(E)$ con $0<m(E)<\infty$ y sea $\varphi:\rr \to \rr$ convexa. Entonces
 \[
 \varphi\left( \frac{1}{m(E)}\int_E f(x)\,dx \right) \leq \frac{1}{m(E)} \int_E \varphi(f)\,dx.
 \]
 \end{lema}

\begin{demo}
Como $\varphi$ es funci\'on convexa, para todo $x_0\in \rr$ existe $a\in \rr$ tal que 
\[
\varphi(x)\geq \varphi(x_0)+a(x-x_0).
\]
Luego,
\[
\begin{split}
\frac{1}{m(E)} \int_E \varphi(f)\,dx &\geq 
\frac{1}{m(E)}\int_E \varphi(x_0)+a(f(x)-x_0)\,dx
\\
&=\varphi(x_0)+a\frac{1}{m(E)}\int_E f(x)\,dx -ax_0.
\end{split}
\]
Ahora, tomamos $x_0=\frac{1}{m(E)} \int_E f(x)\,dx$ y  conseguimos la desigualdad buscada.
\end{demo}

\begin{teorema}{}
Sea $f$ medible sobre $E$ y $1\leq p\leq \infty$.
Entonces
\begin{enumerate}
    \item Si $f \in L^p$, entonces 
   \begin{equation}\label{eq:defi-alternativa-norma-p}
        \|f \|_p=\sup\limits_{g} \int_E fg\, dx,
   \end{equation}
   donde el supremo se toma sobre todas las funciones $g$ tales que
   $\|g\|_q\leq 1$.
   \item Si el supremo en \eqref{eq:defi-alternativa-norma-p} es finito entonces  $f \in L^p$ y dicho supremo es $\|f \|_p$.
   \end{enumerate}
\end{teorema}

\begin{demo}
El caso $p=\infty$ ya fue analizado, as\'i que veremos el caso $1\leq p<\infty$.
\begin{enumerate}
    \item Por la desigualdad de H\"older tenemos
    \[\sup\limits_{\|g\|_q\leq 1} \int_E fg\,dx \leq \|f\|_p.\]
    Supongamos que $f \neq 0$, pues $f=0$ es trivial.
    Sea $g= \frac{|f|^{p-1}}{\|f\|_p^{p-1}} \sgn f$, entonces
    $g \in L^q$ y $\|g\|_q\leq 1$ y 
    \[ \int_E fg\, dx=\|f\|_p. \]
    \item Supongamos que $f\geq 0$ tal que $\|f\|_p=\infty.$
    Definimos 
    \[
    f_k(x)=\left\{
    \begin{array}{ll}
       0  &    \mbox{ si } |x|\geq k
       \\
    \min\{f(x),k\}     &  \mbox{ si } |x|\leq k,
    \end{array}
    \right.
    \]    
    entonces $f_k \in L^p$ y por el Teorema de Beppo-Levi (Teorema \ref{teo:Beppo-Levi}), tenemos
    $\|f_k\|_p \nearrow \|f\|_p=\infty. $
    Sea  $g_k\in L^q$ con $\|g_k\|_q=1$ y 
    \[
    \int_E f_g g_k\, dx=\|f_k\|_p.
    \] 
    Luego, 
    \[
    \sup\limits_{\|g_k\|_q\leq 1} \int_E fg\,dx \geq 
    \int_E fg_k\,dx \geq \int_E f_k g_k \,dx=\|f\|_p \nearrow\infty.
    \]
        \end{enumerate}
\end{demo}


Notar que si $g \in L^q$, entonces
\[\begin{split}
\ell_g: L^p &\longrightarrow \rr
\\
f &\longmapsto \int_E fg\,dx,
\end{split}
\]
es una funcional lineal acotada. 

\begin{teorema}{}[Representaci\'on de Riesz]
\[
[L^p(E)]^*=L^q(E).\]
\end{teorema}

\section{Funci\'on maximal de Hardy-Littlewood y teorema de difrenciaci\'on de Lebesgue}


\subsection{Un lema de cubrimiento}

\begin{lema}{}
Sea $E \subset \rr^n$ medible y supongamos que $E$ es cubierto por una uni\'on de bolas $B_i$, con $i\in I$, de di\'ametro acotado. Entonces  existe una subfamilia de bolas $B_1,B_2,\ldots,$ de la familia dada, disjuntas dos a dos y  tal que 
\[
\sum\limits_{k} m(B_k) \leq Cm(E),
\]
donde $C$ depende de la dimensi\'on $n$ ($5^-n$ funciona).
\end{lema}

\begin{demo}
Sea $B_1$ tal que 
\[
\diametro B_1 \geq \frac{1}{2} \sup\limits_{i \in I} \diametro B_i.
\]
Una vez que hemos elegido $B_k$, seleccionamos $B_{k+1}$ de modo que 
\[
\diametro B_{b_1}\geq \frac{1}{2} \diametro B_i,
\]
para toda $B_i$ que ....
\end{demo}

\textbf{
LAS NOTAS MANUSCRITAS SUBIDAS AL MOODLE LLEGAN HASTA AC\'A!!!!}