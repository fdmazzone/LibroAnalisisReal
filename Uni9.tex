\chapter{Espacios $L^p$}

\section{El espacio de funciones integrables}
Sea
$
L^1=L^1(E)=
\{
f|\mbox{ es integrable en } E
\}.
$

$L^1(E)$ es un espacio vectorial donde se define la relaci\'on de equivalencia
\[
f \sim g \Longleftrightarrow f=g \;\mbox{ en c.t.p. de } E.
\]   

 
La clase de equivalencia de $f$  que suele notarse con ``$(f)$'', pero en un abuso de notaci\'on, se indicar\'a  $f$
y $L^1(E)/_{\sim}$ se denotar\'a con $L^1(E)$.

Para $f\in L^1(E)$, definimos
\[
\|f\|_1=\int_E|f(x)|\,dx.
\]

$\| \cdot \|_1$ es una funci\'on no negativa sobre $L^1(E)$ que cumple
\begin{enumerate}
    \item [H1)] $\|f+g\|_1\leq \|f\|_1+\|g\|_1$.
    \item [H2)] $\|\lambda f\|_1=|\lambda|\|f\|_1$.
    \item [H3)] $\|f\|_1=0$ si y s\'olo si f=0.
\end{enumerate}

As\'i $(L^1(E),\|\cdot\|_1)$ es un espacio normado. 

Y se define 
\[
d(f,g)=\|f-g\|_1.
\]

Si un espacio normado es completo se denomina espacio de Banach. 

\begin{teorema}{}\label{teorema:L1-completo}
$L^1(E)$ es un espacio de Banach.
\end{teorema}

\begin{demo}
En la prueba 
%del Teorema\ref{teorema:L1-completo}, 
usaremos el siguiente ejercicio.

\begin{ejercicio}{}
Sea $(X,\| \cdot\|)$ un espacio normado. \\
$(X,\| \cdot\|)$ es un espacio de Banach si y s\'olo si cada vez que 
\[
\sum\limits_{n=1}^{\infty} \|x_n\|<\infty,
\]
existe $x \in X$ tal que $\sum\limits_{n=1}^{\infty} x_n$.
\end{ejercicio}

Sea pues $f_i \in L^1(E)$ tal que 
\[
\sum\limits_{i=1}^{\infty} \|f_i\|_1<\infty.
\]

Sea \[ \Phi(x)=\sum\limits_{i=1}^{\infty} |f_i(x)|.\]

$\Phi \in L^1(E)$ y entonces $\Phi$ es finita en casi todo punto. Por lo tanto, 
\[
S(x)=\sum\limits_{i=1}^{\infty} f_i(x),
\]
est\'a bien definida excepto sobre un conjunto de medida nula.
%
Adem\'as $|S|\leq \Phi$ y  $S\in L^1(E)$.
%
Ahora, si 
\[
S_j(x)=\sum\limits_{i=1}^j f_i(x),
\]
tenemos que $S_j \to S$ en c.t.p. y $|S_j-S|\leq 2\Phi \in L^1(E)$ y por lo tanto, por el Teorema de la Convergencia Mayorada de Lebesgue \ref{ AUN NO ESTA EN APUNTE}, 
\[
\|S_j-S\|_1=\int_E |S_j-S|\,dx \to 0 \mbox{ cuando } j \to \infty.
\]
\end{demo}

\begin{teorema}{}
Sea $f_i \in L^1(E)$ tal que $f_i \xrightarrow{L^1(E)} f$, entonces existe una subsucesi\'on $f_{i_k}$ tal que $f_{i_k}\to f$ en c.t.p.
\end{teorema}

\begin{demo}
Por la desigualdad de Chebyshev, se tiene
\[
m(\{x:|f(x)-f_i(x)|>\epsilon \})\leq 
\frac{1}{\epsilon}\int_E |f(x)-f_i(x)|\,dx \xrightarrow{i \to \infty}0.
\]
\'Esto muestra que si $f_i \xrightarrow{L^1(E)} f$ entonces
$f_i \xrightarrow{m} f$.
Luego, existe una subsucesi\'on $f_{i_k}\to f$ en c.t.p.
\end{demo}

\subsection{Conjuntos densos en $L^1(E)$}

\begin{teorema}[Simples $\Rightarrow$ densas en $L^1(E)$]{}
El conjunto de las funciones simples es denso en $L^1(E)$.
\end{teorema}

\begin{demo}
Sea $f\in L^1(E)$. Existe una sucesi\'on $\{f_i\}$ de funciones simples tal que $|f_i|\leq |f|$ y $f_i \to f$. Luego, $f_i \in L^1(E)$ y $|f-f_i|\leq 2|f|$. As\'i, por el Teorema de la Convergencia Mayorada de Lebesgue \ref{FALTA}, $f_i \to f$ en $L^1(E)$.
\end{demo}

\begin{definicion}{}
Diremos que $\varphi$ es una funci\'on escalonada si 
$\varphi=\sum\limits_{i=1}^N \alpha_i \chi_{I_i}$
con $I_i$ intervalos.
\end{definicion}

\begin{teorema}[Escalonadas $\Rightarrow$ densas en $L^1(E)$]{}
El conjunto de las funciones escalonadas es denso en $L^1(E)$.
\end{teorema}

\begin{demo}
Bastar\'ia ver que $\chi_{E_0}$ es l\'imite de funciones simples para cada $E_0\subset E$ con $m(E_0)<\infty$.
Ahora bien, ya se demostr\'o que existe una sucesi\'on  de conjuntos elementales $A_k$, i.e. $A_k=\bigcup\limits_{j=1}^{N_k} I_j^k$ tales que 
\[
m\left(E_0\Delta A_k\right) \to 0,
\]
siendo
\[
m\left(E_0\Delta A_k\right)=
\int \left|\chi_E-\chi_{A_k}\right|\,dx=
\bigintss \left|
\chi_E-\sum\limits_{j=1}^{N_k} \chi_{I_j^k}
\right|\,dx.
\]
\end{demo}

\begin{ejercicio}{}
El conjunto de las funciones continuas con soporte compacto es denso en $L^1(\rr^n)$.
\end{ejercicio}

\begin{ejercicio}{}
$L^1(E)$ es separable.
\end{ejercicio}

\section{Funciones esencialmente acotadas}


\section{Funciones de cuadrado integrable}