\chapter*{Apéndice}

\section{Topología}

\begin{teorema}[Principio de encaje de intervalos\index{Intervalos encajados}]{}  Sea $I_n=[a_n,b_n]\subset\mathbb{R}$ una sucessión de intervalos con las siguientes propiedades
\begin{enumerate}
 \item $\forall n\in\mathbb{N}: I_n\subset I_{n+1},$
 \item $\lim\limits_{n\to\infty}(b_n-a_n)=0.$
\end{enumerate}
Entonces $\bigcap_{n=1}^{\infty}I_n$ consiste de uno, y solo un, punto $x\in\mathbb{R}$.
 
\end{teorema}

\begin{proof}
 
\end{proof}





\begin{teorema}[Heine-Borel]{}\index[personas]{Heine}\index[personas]{Borel} Toda sucesión acotada de $\mathbb{R}$ 
tiene una subsucesión convergente.
 
\end{teorema}

\begin{proof} Uso encajes de intervalos.
 
\end{proof}


\begin{definicion}[Continuidad uniforme \index{Continuidad uniforme}]{} Sea $f:A\subset\mathbb{R}\to\mathbb{R}$ una función. Diremos que $f$ es uniformemente continua si 
\[
 \forall \epsilon>0\exists \delta>0 \forall x,y\in A:|x-y|<\delta\Rightarrow |f(x)-f(y)|<\epsilon.
 \]

\end{definicion}

\begin{ejemplo} Varios ilustrando diferencia con continuidad
 
\end{ejemplo}


\begin{teorema}{} Sea $f:[a,b]\to\mathbb{R}$ continua. Entonces $f$ es uniformemente continua. 
 \end{teorema}
\begin{proof} Uso Heine-Borel
 \end{proof}

