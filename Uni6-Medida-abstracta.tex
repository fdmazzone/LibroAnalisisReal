\chapter{Medidas abstractas}

\begin{quote}
 <<Me gustaría enfatizar nuevamente, antes de comenzar esta presentación, que la nueva definición será aplicable no solo a un espacio con $n$ dimensiones sino a un conjunto abstracto. Es decir, ni siquiera es necesario, por ejemplo, suponer que sabemos cuál es el límite de elementos en este conjunto>> 
\end{quote}
\begin{flushright}
 M. Frechet \index[personas]{Frechet}\\
Sur l’intégrale d’une fonctionnelle étendueà un ensemble abstrait\\
Bulletin de la S. M. F., tome  43 (1915), p. 248-265.
\end{flushright}

En todos los capítulos anteriores hemos tratado de fundar los conceptos que fuimos introduciendo relacionándolos con  conceptos que juzgamos los precedían. cuando decimos  ``preceder''contemplamos tanto el orden lógico de la construcción  como  el grado de abstracción de los objetos de estudio.

En esta unidad planteamos un salto cualitativo. Vamos abstraernos de la problemática que dió origen a la construcción de la medida en itengral de Lebesgue, esto es la noción de área, y consideraremos una teoría axiomática, donde postularemos como axiomas aquellas propiedades que se revelaron trascendentes en los capítulos anteriores. Este enfoque axiomático se abstrae a su vez de las entidades a las que pretendemos medir, en el sentido que ya no formularemos el concepto de medida para subconjuntos de $\mathbb{R}$, o el espacio euclideano $\mathbb{R}^n$. Introducieremos el concepto de \emph{espacio de medida}\index{Espacio de medida}  como una abstracción y veremos como este concepto induce un consecuente concepto de integral.  


Las nociones introducidas aquí fueron presentadas por primera vez por M. Frechet en el artículo del cual fue extraída la cita con la que comensamos el presente capítulo. 

La noción de medida abstracta es muy fructífera pues unifica multitud de instancias particulares de esta noción que aparecen en distintas áres de la matemática  además de contemplar la medida de Lebesgue. 

\section{Algebras, $\sigma$-álgebras y clases monótonas}

\begin{definicion}[Algebra de conjuntos]{def:algebra} \index{Algebra}
 Sea $X$ un conjunto y $\mathscr{A}\subset \mathcal{P}(X)$. Diremos que $\mathscr{A}$ es un \emph{álgebra} si:
 \begin{enumerate}
  \item $\emptyset\in\mathscr{A}$.
  \item $A\in\mathscr{A}\Rightarrow A^c\in\mathscr{A}$.
  \item $A_i\in\mathscr{A}$, $i=1,\ldots,n$, $\Rightarrow\bigcup\limits_{i=1}^n A_i\in\mathscr{A}$.
 \end{enumerate}

\end{definicion}


\begin{ejercicio}{} Demostrar que los siguientes ejemplos definen álgebras de conjuntos.
 \begin{enumerate}
  \item La colección de todas las uniones de una cantidad finita de intervalos de $\mathbb{R}$, donde por intervalo inlcuímos tanto acotados como  no y tanto abiertos como cerrados como ninguno de ambos.
  \item Como en el ejemplo anterior, pero con los extremos de los intervalos en $\mathbb{Z}\cup \{\pm \infty\}$ o $\mathbb{Q}\cup \{\pm \infty\}$.
    \item Más generalmente aún, como en los ejemplos anteriores, pero con los extremos de los intervalos en $A\cup \{\pm \infty\}$, donde $A\subset\mathbb{R}$.
 \end{enumerate}
\end{ejercicio}
 \begin{ejercicio}{} Demostrar que $\mathscr{A}$  es un álgebra si y solo si
 \begin{enumerate}
  \item $\emptyset\in\mathscr{A}$.
  \item $A\in\mathscr{A}\Rightarrow A^c\in\mathscr{A}$.
  \item $A_i\in\mathscr{A}$, $i=1,\ldots,n$, $\Rightarrow\bigcap\limits_{i=1}^nA_i\in\mathscr{A}$.
 \end{enumerate}
\end{ejercicio}

\begin{definicion}[Clases monótonas]{def:clase_monotona} \index{Clases monótonas}
 Sea $X$ un conjunto y $\mathscr{A}\subset\mathcal{P}(X)$. El conjunto $\mathscr{A}$ se llamará   \emph{clase monótona} si
 \begin{enumerate}
  \item $A_i\in\mathscr{A}$, $A_i\subset A_{i+1}$, $i=1,2,\ldots\Rightarrow \bigcup\limits_{i=1}^{\infty}A_i\in\mathscr{A}$. 
    \item $A_i\in\mathscr{A}$, $A_i\supset A_{i+1}$, $i=1,2,\ldots\Rightarrow \bigcap\limits_{i=1}^{\infty}A_i\in\mathscr{A}$. 
 \end{enumerate}

\end{definicion}

\begin{ejercicio}{} Demostrar que los siguientes ejemplos definen clases monótonas.
 \begin{enumerate}
  \item La colección de todos los intervalos de $\mathbb{R}$ de la forma $(a,+\infty)$ o $[a,+\infty)$ con $a\in [-\infty,+\infty)$.
  \item En $\mathbb{R}^n$ la colección de de todas las bolas, tanto cerradas o abiertas, de centro $0$ y radio $r\in[0,+\infty]$. 
  \item La colección de todos los subgrupos de un grupo dado $G$. 
 \end{enumerate}
\end{ejercicio}

Recordemos del capítulo anterior.
\begin{definicion}[$\sigma$-álgebra de conjuntos]{def:sigma_algebra} \index{$\sigma$-álgebra}
 Sea $X$ un conjunto y $\mathscr{A}\subset \mathcal{P}(X)$. Diremos que $\mathscr{A}$ es una \emph{$\sigma$-álgebra} si:
 \begin{enumerate}
  \item $\emptyset\in\mathscr{A}$.
  \item $A\in\mathscr{A}\Rightarrow A^c\in\mathscr{A}$.
  \item $A_i\in\mathscr{A}$, $i=1,\ldots$, $\Rightarrow\bigcup\limits_{i=1}^{\infty}A_i\in\mathscr{A}$.
 \end{enumerate}

\end{definicion}


\begin{ejercicio}{} Demostrar que  $\mathscr{A}$ es $\sigma$-álgebra si y sólo si es clase monótona y álgebra.
 
\end{ejercicio}
\begin{ejercicio}{ejer:sigma_alg_equiv} Demostrar que  $\mathscr{A}$ es $\sigma$-álgebra si y sólo si es álgebra y satisface que 
\begin{itemize}
 \item $E_n\in\mathscr{A}$, $n=1,2,\ldots$ y $E_i\cap E_j=\emptyset$, cuando $i\neq j$ implican que $\bigcup\limits_{n=1}^{\infty} E_n\in\mathscr{A}$.
\end{itemize}
\end{ejercicio}


\section{Medidas} 

\begin{definicion}{} Sea $X$ un conjunto y $\mathscr{A}$ una $\sigma$-álgebra. Una funcion $\mu:\mathscr{A}\to [0,+\infty]$ se llama una \emph{medida}\index{Medida} si  para toda colección numerable de subconjuntos $A_i\in\mathscr{A}$, $i=1,2,\ldots$, mutuamente disjuntos entre si, se satisface que 
$$\mu\left(\bigcup_{i=1}^{\infty}A_i\right)=\sum_{i=1}^{\infty}\mu\left(A_i\right).$$
Al triplete $(X,\mathscr{A},\mu)$ se lo denomina \emph{espacio de medida}.\index{Espacio de medida}. 
\end{definicion}


\begin{ejercicio}{}  Sea $(X,\mathscr{A},\mu)$ espacio de medida. Demostrar que se satisface las siguientes relaciones
\begin{enumerate}
 \item $\mu(\emptyset)=0$.
 \item $\mu(A\cup B)+\mu(A\cap B)=\mu(A)+\mu(B)$.
 \item Si $\mu(B)<\infty$ y $B\subset A$ entonces $\mu(A-B)=\mu(A)-\mu(B)$. 
\end{enumerate}

 
\end{ejercicio}

\begin{ejemplo}{} $(\mathbb{R},\mathscr{M},m)$, donde $\mathscr{M}$ denota la $\sigma$-algebra de los conjuntos medibles Lebesgue y $m$ la medida de Lebesgue, es un espacio de medida. Si en lugar de considerar la $\sigma$-álgebra $\mathscr{M}$ consideramos la $\sigma$-álgebra $\mathscr{B}$ de conjuntos medibles Borel, resulta en otro espacio de medida $(\mathbb{R},\mathscr{B},m)$, que no es más que la restricción de la medida a una sub-$\sigma$-álgebra.
 
\end{ejemplo}


\begin{ejercicio}[Medida de conteo]{def:med_cont} Sea $X$ es un conjunto y $\mathscr{A}=\mathcal{P}(X)$ la colección de todos los subconjuntos de $X$. Para $A\in\mathscr{A}$ escribamos $\mu(A)=\#A$, cuando $A$ es finito, y $\mu(A)=+\infty$ cuando $A$ no es finito. Demostrar que el triple $(X,\mathscr{A},\mu)$ es espacio de medida.
 
\end{ejercicio}
 
 
\begin{ejercicio}{def:med_cont} Sea $X=\mathbb{N}$  y $\mathscr{A}=\mathcal{P}(\mathbb{N})$ la colección de todos los subconjuntos de $\mathbb{N}$. Supongamos dada una función $f:\mathbb{N}\to [0,+\infty]$. Para $A\in\mathscr{A}$ escribamos 
\[\mu(A)=\sum_{n\in A}f(n), \] 
Demostrar que el triple $(\mathbb{N},\mathscr{A},\mu)$ es espacio de medida. ¿Qué resulta $\mu$ si $f(n)=1$ para todo $n$? ¿Qué condición debe satisfacer $f$ para que $\mu(A)<\infty$ para todo $A\subset\mathbb{N}$?
 
\end{ejercicio}

\begin{ejercicio}{def:med_cont_mult3} Sea $X=\mathbb{N}$  y $\mathscr{A}=\mathcal{P}(\mathbb{N})$ la colección de todos los subconjuntos de $\mathbb{N}$. Sea $k$ un natural fijo. Para $A\subset\mathbb{N}$ escribir

\[
 \mu(A)=\#\{ n\in A :  k|n \},
\]
es decir $\mu(A)$ cuenta cuantos multiplos de $k$ hay en $A$. Demostrar que $\mu$ es medida. Demostrar que esta medida es una instancia de las medidas introducidas en \eqref {def:med_cont}. 
\end{ejercicio}


Un ejemplo muy importante  es provisto por la siguiente proposición

\begin{proposicion}{prop:med_abs_cont}
 Sea $f:\mathbb{R}\to [0,+\infty]$ una función integrable y no negativa. El triplete
  $(\mathbb{R},\mathscr{M},\mu_f)$ es espacio de medida, donde $\mathscr{M}$ denota la $\sigma$-algebra de los conjuntos medibles Lebesgue y 
  \[\mu_f(A)=\int_A f(x)dx.\]
\end{proposicion}

\begin{demo}
 Sólo hay que demostrar que $\mu_f$ es una medida. Sean $A_i\in\mathscr{A}$, $i=1,\ldots$, mutuamente disjuntos. 
 
 \begin{ejercicio}{ejer:suma_simples}
  Verificar la siguiente relación
  \[
  \rchi_{\bigcup_{i=1}^{\infty}A_i}=\sum_{i=1}^{\infty}\rchi_{A_i}.
  \]


 \end{ejercicio}

 Luego por la intercambiabilidad entre integral y series de términos positivos

\begin{multline*}
 \mu_f\left(\bigcup_{i=1}^{\infty}A_i \right)=\int_{\bigcup_{i=1}^{\infty}A_i}f(x)dx=\int \rchi_{\bigcup_{i=1}^{\infty}A_i} f(x)dx\\=\int \sum_{i=1}^{\infty}\rchi_{A_i}dx= \sum_{i=1}^{\infty}\int\rchi_{A_i}dx= \sum_{i=1}^{\infty}\mu_f(A_i).
\end{multline*}

 
\end{demo}


\begin{ejercicio}[Delta de dirac]{ejer:delta_dirac}
 Sea $X$ un conjunto no vacío cualquiera, $a\in X$ un punto fijo y $\mathscr{A}=\mathcal{P}(X)$. Definimos $\delta_a:\mathscr{A}\to [0,+\infty]$ por
 \[
  \delta_a(A)=\left\{\begin{array}{ll} 1 &\text{ si } a\in A\\0 &\text{ si } a\notin A\end{array}\right.
 \]
 Demostrar que $(X,\mathscr{A},\delta)$ es un espacio de medida. La medida $\delta_a$
se denomina \emph{$\delta$ de Dirac}\index{$\delta$ de Dirac}\index[personas]{Dirac}. 
\end{ejercicio}
\marginnote{
\begin{center}
 \adjustimage{max size={0.9\linewidth}{0.9\paperheight}}{imagenes/dirac.jpg}\\
\end{center}
\small
Paul Adrien Maurice Dirac, (Brístol, Reino Unido, 8 de agosto de 1902-Tallahassee, Estados Unidos, 20 de octubre de 1984) fue un ingeniero eléctrico, matemático y físico teórico británico que contribuyó de forma fundamental al desarrollo de la mecánica cuántica y la electrodinámica cuántica.
}
\begin{definicion}[Completitud de medidas]{defi:med_completa}
 Un espacio de medida  se llama \emph{completo}\index{Medida!completa} si $A\subset B\in\mathscr{A}$ y $\mu(B)=0$ implican $A\in\mathscr{A}$.
\end{definicion}

\begin{ejemplo}{}  $(\mathbb{R},\mathscr{M},m)$ es un espacio de medida completo, mientras que $(\mathbb{R},\mathscr{B},m)$ no lo es.
\end{ejemplo}





\section{Medida exterior}

\begin{definicion}[Medida exterior]{defi:med_exterior}
Sea $X$ un conjunto no vacío. Una función $\mu^\star:\mathcal{P}(X)\to [0,+\infty]$ se denomina una \emph{medida exterior}\index{Medida!exterior} si satisface que  	
  \begin{description}
   \item[] $\mu^\star(\emptyset)=0$. 
   \item[Monotonía.] $A_1\subset A_2\Rightarrow \mu^\star(A_1)\leq \mu^\star(A_2)$ 
   \item[$\sigma$-subaditividad.] $A_j\subset X$, $j=1,2,\ldots$, $\Rightarrow  \mu^\star\left(\bigcup\limits_{j=1}^{\infty}A_j\right)\leq\sum_{j=1}^{\infty}\mu^\star(A_j)$.
   \end{description}


\end{definicion}

\begin{ejemplo}{} La medida exterior que definimos sobre subconjuntos de $\mathbb{R}$ es obviamente una medida exterior en el sentido de la definición anterior. 
\end{ejemplo}

 



\begin{definicion}[Conjuntos medibles de Carathéodory]{defi:med_cara}
 Sea $\mu^\star$ una medida exterior sobre $X$ y $E\subset X$. Diremos que $E$ es \emph{medible en el sentido de Charathéodory} \index{Conjunto!Medible Carathéodory}\index[personas]{Carathéodory} si para todo $A\subset X$ se cumple que
 \begin{equation}\label{eq:cond_cara}
  \mu^\star(A)=\mu^\star(A\cap E)+\mu^\star(A-E).
 \end{equation}

\end{definicion}

 \begin{observacion} A los efectos de chequear si un conjunto es medible es suficiente probar que se satisface la desigualdad 
   \begin{equation}\label{eq:cond_caraII}
  \mu^\star(A)\geq\mu^\star(A\cap E)+\mu^\star(A-E).
 \end{equation}
 para todo $A$ con medida exterior finita.
 \end{observacion}
 
 
 \begin{teorema}{teo:med_cara} Si $\mu^\star$ es una medida exterior sobre $X$ y $\mathscr{A}$ el conjunto de todos los subconjuntos de $X$ que son medibles según Carathéodory. Entonces  $\mathscr{A}$ es una $\sigma$-álgebra y $\mu^\star$ restringido a $\mathscr{A}$ es una medida. El espacio de medida $(X,\mathscr{A},\mu^\star)$ es completo
  
 \end{teorema}
 
 \begin{demo} Que $\emptyset\in\mathscr{A}$ es una afirmación inmediata. 
 
 Si uno escribe la condición de Carathéodory para $E^c$ queda exactamente igual que la respectiva condición para $E$. Esta observación justifica que $E\in \mathscr{A}$ implica que $E^c\in\mathscr{A}$.
 
 Sean $E_1,E_2\in\mathscr{A}$ y $A\subset X$. Entonces
 \begin{multline*}
  \mu^\star(A)\geq  \mu^\star(A\cap E_2)+ \mu^\star(A\cap E_2^c)\\
  \geq \mu^\star(A\cap E_1\cap E_2)+ \mu^\star(A\cap E_2\cap E_1^c)\\
  + \mu^\star(A\cap E_2^c\cap E_1)+ \mu^\star(A\cap E_2^c\cap E_1^c)
 \end{multline*} 
Ahora los conjuntos $E_2\cap E_1, E_2\cap E_1^c$ y $E_1\cap E_2^c$ son mutuamente disjuntos  y su unión es $E_1\cup E_2$. Esta observación aplicada a los tres primeros términos del último miembro de la desigualdad anterior implica
\[
  \mu^\star(A)\geq \mu^\star(A\cap (E_1\cup E_2))+ \mu^\star(A\cap (E_1\cup E_2)^c).
\]
De esta desiguldad concluímos que $E_2\cup E_2\in\mathscr{A}$. Esto a su vez implica que $\mathscr{A}$ es (al menos) un álgebra. Queremos ver que en realidad es $\sigma$-álgebra.

Supongamos que $E_1\cap E_2=\emptyset$. Tomando $A= E_1\cup E_2$ en \eqref{eq:cond_caraII} obtenemos
\[
 \mu^\star(E_1\cup E_2)\geq \mu^\star(E_1)+\mu^\star(E_2).
 \]

\begin{ejercicio}{ejer:med_ext_premed} Generalizar la desigualdad anterior de la siguiente forma. Si $E_j$, $j=1,\ldots,n$ son mutuamente disjuntos entonces
\[
 \mu^\star(E_1\cup\cdots\cup E_2)\geq \mu^\star(E_1)+\cdots+\mu^\star(E_n).
 \]
 \end{ejercicio}

Siguiendo con la demostración, sean $E_n\in\mathscr{A}$, $n=1,2,\ldots$ una colección numerables de conjuntos mutuamente disjuntos en $\mathscr{A}$. Tomemos $G_n=\bigcup\limits_{j=1}^nE_j$ y $G=\bigcup\limits_{j=1}^{\infty}E_j$.

Como $\mathscr{A}$ es un álgebra, $G_n\in\mathscr{A}$. Además usando sucesivamente la condición de Carathéodory
\begin{equation}\label{eq:dem_cara1}
\begin{split}
  \mu^\star(G_n\cap A) &\geq  \mu^\star(G_n\cap A\cap E_n)+ \mu^\star(G_n\cap A\cap E_n^c)\\
 &=\mu^\star( A\cap E_n)+ \mu^\star(G_{n-1}\cap A)\\
 &\geq \mu^\star( A\cap E_n)+ \mu^\star(A\cap E_{n-1})+ \mu^\star(G_{n-2}\cap A)\\
&\hspace{1cm}\vdots\\
&\geq \sum_{j=1}^n\mu^\star(A\cap E_j).
\end{split}
\end{equation}

Ahora deducimos que para todo $A\subset X$


\begin{align*}
\mu^\star( A)&\geq  \mu^\star(A\cap G_n)+ \mu^\star(A\cap G_n^c) & (G_n\in\mathscr{A})\\
& \geq \sum_{j=1}^n\mu^\star(A\cap E_j)+\mu^\star(A\cap G^c) & (\text{Ecuación \eqref{eq:dem_cara1}}, G_n\subset G)\\ 
\end{align*}

Tomando límite cuando $n\to\infty$
\begin{align*}
\mu^\star( A)&\geq  \sum_{j=1}^{\infty}\mu^\star(A\cap E_j)+\mu^\star(A\cap G^c) &  \\ 
             &\geq  \mu^\star\left(A\cap \bigcup_{j=1}^{\infty} E_j\right)+\mu^\star(A\cap G^c) & (\sigma-\text{subaditividad de } \mu^\star)  \\ 
             &\geq  \mu^\star\left(A\cap G \right)+\mu^\star(A\cap G^c) &   \\ 
\end{align*}
Luego $G\in\mathscr{A}$. Ahora el Ejercicio \ref{ejer:sigma_alg_equiv} implican que $\mathscr{A}$ es $\sigma$-álgebra. Además por el Ejercio\ref {ejer:med_ext_premed}, para $E_j$, $j=1,\ldots$ mutuamente disjuntos en $\mathscr{A}$.
\begin{align*}
 \mu^\star\left(\bigcup_{j=1}^{\infty} E_j\right) &\geq \mu^\star\left(\bigcup_{j=1}^{n} E_j\right) & (\text{monotonía de } \mu^\star)\\ 
 &= \sum_{j=1}^{n}  \mu^\star\left(E_j\right) & (\text{Ejercicio \ref{ejer:sigma_alg_equiv}}) \\ 
\end{align*}

Tomando límite cuando $n\to\infty$ obtenemos 
\[\mu^\star\left(\bigcup_{j=1}^{\infty} E_j\right) \geq \sum_{j=1}^{\infty}  \mu^\star\left(E_j\right)\]
Como la desigualdad inversa a la anterior es siempre cierta por la $\sigma$-subaditividad queda demostrado que $\mu^\star$ es medida sobre $\mathscr{A}$ y finalizada la demostración del teorema.   
 \end{demo}

\section{Premedidas}

\begin{definicion}[Premedida]{defi:premedida}
 Sea $X$ un cojunto no vacío y $\mathscr{A}_0$ un álgebra de subconjuntos de $X$. Diremos que una función $\mu_0: \mathscr{A}_0\to [0,+\infty]$ es una \textbf{premedida}\index{Premedida} si satisface que
 \begin{itemize}
  \item  Si $E_j\in\mathscr{A}_0$, $j=1,2,\ldots$ son mutuamente disjuntos y $\bigcup\limits_{j=1}^{\infty}E_j\in\mathscr{A}_0$ entonces
  \[\mu_0\left(\bigcup\limits_{j=1}^{\infty} E_j\right) = \sum\limits_{j=1}^{\infty}  \mu_0\left(E_j\right)\]
\end{itemize}

  
 
\end{definicion}


Podemos construir medidas a partir de premedidas.

\begin{lema}{lem:extension}
 Sea $\mu_0$ una premedida sobre el álgebra  $\mathscr{A}_0$ de subconjuntos de $X$. Definimos  $\mu^\star:\mathcal{P}(X)\to [0,+\infty]$ por
 \begin{equation}\label{eq:defi_med_ext}
  \mu^\star(E)=\inf\left\{ \sum\limits_{j=1}^{\infty}\mu_0(E_j)\mid E\subset \bigcup\limits_{j=1}^{\infty}E_j, E_j\in\mathcal{A}_0  \right\}
 \end{equation}
 Entonces $\mu^\star$ es una medida exterior que satisface que $\mu^\star(E)=\mu_0(E)$ para todo $E\in\mathcal{A}_0$ y que todo conjunto 
$E\in\mathcal{A}_0$ es medible en el sentido de Carathéodory. 
\end{lema}

 \begin{demo} 
  


\begin{ejercicio}{} Probar que $\mu^\star$ definida en \eqref{eq:defi_med_ext} define en efecto una medida exterior.
 
\end{ejercicio}

Veamos que la restricción de $\mu_{*}$ to $\mathcal{A}$ coincide con $\mu_{0}$. Supongamos que $E \in \mathcal{A}$. Siempre $\mu_{*}(E) \leq \mu_{0}(E)$ pues $E$ se cubre a si mismo. Probemos la desigualdad recíproca. Supongamos  $E \subset \bigcup_{j=1}^{\infty} E_{j}$, con $E_{j} \in \mathcal{A}$ para todo $j$. Definimos
$$
E_{k}^{\prime}=E \cap\left(E_{k}-\bigcup_{j=1}^{k-1} E_{j}\right)
.$$
Los conjuntos $E_{k}^{\prime}$ son mutuamente disjuntos, $E_{k}^{\prime}\in \mathcal{A}, E_{k}^{\prime} \subset E_{k}$ y $E=\bigcup\limits_{k=1}^{\infty} E_{k}^{\prime}$. Por la definición de premedida:
$$
\mu_{0}(E)=\sum_{k=1}^{\infty} \mu_{0}\left(E_{k}^{\prime}\right) \leq \sum_{k=1}^{\infty} \mu_{0}\left(E_{k}\right)
$$
Luego  $\mu_{0}(E) \leq \mu_{*}(E)$.

Por último probemos que los conjuntos en $\mathcal{A}$ son medibles para $\mu_{*}$. Sea $A\subset X$, $E \in \mathcal{A}$ y $\epsilon>0$. Por definición existen  $E_{1}, E_{2}, \ldots$ en $\mathcal{A}$ con $A \subset \bigcup\limits_{j=1}^{\infty} E_{j}$ y
$$
\sum\limits_{j=1}^{\infty} \mu_{0}\left(E_{j}\right) \leq \mu_{*}(A)+\epsilon
$$
Como $\mu_{0}$  finitamente  aditiva en $\mathcal{A}$ 
$$
\begin{aligned}
\sum_{j=1}^{\infty} \mu_{0}\left(E_{j}\right) &=\sum_{j=1}^{\infty} \mu_{0}\left(E \cap E_{j}\right)+\sum_{j=1}^{\infty} \mu_{0}\left(E^{c} \cap E_{j}\right) \\
& \geq \mu_{*}(E \cap A)+\mu_{*}\left(E^{c} \cap A\right)
\end{aligned}
$$

$\epsilon$ es arbitrario,  $\mu_{*}(A) \geq \mu_{*}(E \cap A)+\mu_{*}\left(E^{c} \cap A\right)$ que termina por probar el teorema. \end{demo}

 
\begin{teorema}[Extensión premedidas]{teo:extension}
  Sea $\mu_0$ una premedida sobre el álgebra  $\mathscr{A}_0$ de subconjuntos de $X$. Entonces existe una extensión $\mu$ de $\mu_0$ a la $\sigma$-algebra   $\mathscr{A}$ generada por  $\mathscr{A}_0$.
\end{teorema}
\begin{proof}  La medida exterior $\mu_{*}$ inducida por $\mu_{0}$ define una medida $\mu$ sobre la $\sigma$-álgebra de los conjuntos medibles según  Carathéodory. Por el Lema anterior  $\mu$ es medida sobre $\mathscr{A}$ que extiende $\mu_{0}$. 

\end{proof}



\section{Medidas $\sigma$-finitas}

\begin{definicion}{} Un espacio de medida $(X,\mathcal{A},\mu)$ se llama \index{Medida!$\sigma$-finita}\emph{$\sigma$-finita}  si existen conjuntos medibles $E_n$, $n=1,\ldots$, de medida finita tales que $X=\bigcup_{n=1}^{\infty}E_n$. 
 
\end{definicion}

\begin{ejercicio}{}
 Demostrar que la medida de Lebesgue sobre $\mathbb{R}$ es $\sigma$-finita. Demotrar que la medida de conteo del ejercicio \ref{def:med_cont}  es $\sigma$-finita si y sólo si $X$ es a lo sumo numerable.
\end{ejercicio}


\section{Medida de Lebesgue-Stieltjes}

Haremos una construcción más general  que produce una gran familia de medidas en $\mathbb{R}$ cuyo dominio es el $\sigma$-álgebra de Borel $\mathcal{B}_{\mathbb {R}}$. Tales medidas se denominan \emph{medidas de Borel}\index{medida!Borel} en $\mathbb{R}$.

Para motivar las ideas, supongamos que $\mu$ es una medida de Borel finita en $\mathbb{R}$, y sea $F(x)=\mu((-\infty, x])$ . La función $F$  se llama la \emph{función de distribución}\index{función!distribución} de $\mu$.  Entonces $F$ es creciente a y continua a la derecha, ya que $(-\infty, x]=\bigcap\limits_{n=1}^{\infty}\left(-\infty, x_{n}\right]$ siempre que $x_{n} \searrow x$.  Además, si $b>a,(-\infty, b]=(-\infty, a] \cup(a, b]$, entonces $\mu((a, b])=F(b) -F(a)$. Nuestro procedimiento será darle la vuelta a este proceso y construir una medida $\mu$ a partir de una función creciente continua por la derecha $F$. El caso especial $F(x)=x$ producirá la medida habitual de Lebesgue.

Los bloques de construcción de nuestra teoría serán los intervalos abiertos por la izquierda y cerrados por la derecha en $\mathbb{R}$, es decir, conjuntos de la forma $(a, b]$ o $(a, \infty)$ o $\varnothing$, donde $-\infty \leq a<b<\infty$. En esta sección nos referiremos a tales conjuntos como intervalos semi-abiertos. Claramente, la intersección de dos intervalos semi-abiertos es un intervalo semi-abierto, y el complemento de un intervalo semi-abierto es un intervalo semi-abierto o la unión disjunta de dos intervalos semi-abiertos. La colección $\mathcal{A}$ de uniones disjuntas finitas de intervalos semi-abiertos es un álgebra, y la $\sigma$-álgebra generada por $\mathcal{A}$ es la $\sigma$-algebra de Borel  $\mathcal{B}_{\mathbb{R}}$.


\begin{proposicion}{}
 Sea $F: \mathbb{R} \rightarrow \mathbb{R}$ creciente y continua por la derecha. Si $\left(a_{j}, b_{j}\right]$ $(j=1, \ldots, n)$ son intervalos semi-abiertos disjuntos, sea
$$
\mu_{0}\left(\bigcup_{1}^{n}\left(a_{j}, b_{j}\right]\right)=\sum_{1}^{n}\left[F\left(b_{j}\right)-F\left(a_{j}\right)\right]
$$
y sea $\mu_{0}(\varnothing)=0$. Entonces $\mu_{0}$ es una premedida en el álgebra $\mathcal{A} .$
\end{proposicion}

\begin{proof}
 Primero debemos verificar que $\mu_{0}$ esté bien definida, ya que los elementos de $\mathcal{A}$ se pueden representar en más de una forma como uniones disjuntas de intervalos semi-abiertos. Si $\left\{\left(a_{j}, b_{j}\right]\right\}_{j=1}^{n}$ son disjuntos y $\bigcup\limits_{j=1}^{n}\left( a_{j}, b_{j}\right]=(a, b]$, entonces, quizás después de volver a etiquetar el índice $j$, debemos tener $a=a_{1}<b_{1}=a_{2 }<b_{2}=\ldots<b_{n}=b$, entonces $\sum\limits_{j=1}^{n}\left[F\left(b_{j}\right)-F\left(a_{ j}\right)\right]=$ $F(b)-F(a)$. Más generalmente, si $\left\{I_{i}\right\}_{j=1}^{n}$ y $ \left\{J_{j}\right\}_{j=1}^{m}$ son dos familias  finitas de intervalos semi-abiertos disjuntos tales que $\bigcup\limits_{j=1}^{n} I_{i}=\bigcup\limits_{j=1}^{n} J_{j}$:
 
$$
\sum_{i} \mu_{0}\left(I_{i}\right)=\sum_{i, j} \mu_{0}\left(I_{i} \cap J_{j}\right)= \sum_{j} \mu_{0}\left(J_{j}\right) .
$$
Por lo tanto, $\mu_{0}$ está bien definida y es finitamente aditiva por construcción.
Queda por demostrar que si $\left\{I_{j}\right\}_{j=1}^{\infty}$ es una secuencia de intervalos semi-abiertos disjuntos con $\bigcup\limits_{j=1}^{\infty} I_{j} \in$ $\mathcal{A}$ entonces $\mu_{0}\left(\bigcup\limits_{j=1}^{\infty} I_{j}\right)=\sum\limits_{j=1}^{\infty } \mu_{0}\left(I_{j}\right)$. Dado que $\bigcup\limits_{j=1}^{\infty} I_{j}$ es una unión finita de  intervalos semi-abiertos, la sucesión $\left\{I_{j}\right\}_{j=1}^{\infty} $ se puede dividir en un número finito de subsucesiones de modo que la unión de los intervalos en cada subsucesión sea un único intervalo semi-abierto. Al considerar cada subsucesión por separado y usar la aditividad finita de $\mu_{0}$, podemos suponer que $\bigcup\limits_{j=1}^{\infty} I_{j}$ es un intervalo semi-abierto $I=(a, b]$. En este caso, tenemos
$$
\mu_{0}(I)=\mu_{0}\left(\bigcup_{1}^{n} I_{j}\right)+\mu_{0}\left(I \backslash \bigcup_{1} ^{n} I_{j}\right) \geq \mu_{0}\left(\bigcup_{1}^{n} I_{j}\right)=\sum_{1}^{n} \mu_{ 0}\left(I_{j}\right) .
$$
Haciendo $n \rightarrow \infty$, obtenemos $\mu_{0}(I) \geq \sum\limits_{j=1}^{\infty} \mu_0\left(I_{j}\right)$. 

Para probar la desigualdad inversa, supongamos primero que $a$ y $b$ son finitos, y fijemos $\epsilon>0$. Como $F$ es continua por la derecha, existe $\delta>0$ tal que $F(a+\delta)-F(a)<\epsilon$, y si $I_{j}=\left(a_{j} , b_{j}\right]$, para cada $j$ existe $\delta_{j}>0$ tal que $F\left(b_{j}+\delta_{j}\right)-F\left(b_{j}\right)<\epsilon 2^{-j}$ Los intervalos abiertos $\left(a_{j}, b_{j}+\delta_{j}\right)$ cubren el conjunto compacto $[a+\delta, b]$, por lo que hay un sub-cubrimiento finito. Al descartar cualquier $\left(a_{j}, b_{j}+\delta_{j}\right)$ que esté contenido en uno más grande y reetiquetando el índice $j$, podemos suponer que 
\begin{itemize}
\item  los intervalos $\left(a_{1}, b_{1}+\delta_{1}\right), \ldots,\left(a_{N}, b_{N}+\delta_{N}\right)$  cubren $[a+\delta, b]$,
\item  $b_{j}+\delta_{j} \in\left(a_{j+1}, b_{j+1}+\delta_{j+1}\right)$ para $j=1, \ldots , N-1$.
\end{itemize}
Pero entonces
$$
\begin{aligned}
\mu_{0}(I) &<F(b)-F(a+\delta)+\epsilon \\
& \leq F\left(b_{N}+\delta_{N}\right)-F\left(a_{1}\right)+\epsilon \\
&=F\left(b_{N}+\delta_{N}\right)-F\left(a_{N}\right)+\sum\limits_{j=1}^{N-1}\left[F\left( a_{j+1}\right)-F\left(a_{j}\right)\right]+\epsilon \\
& \leq F\left(b_{N}+\delta_{N}\right)-F\left(a_{N}\right)+\sum\limits_{j=1}^{N-1}\left[F\left (b_{j}+\delta_{j}\right)-F\left(a_{j}\right)\right]+\epsilon \\
&<\sum\limits_{j=1}^{N}\left[F\left(b_{j}\right)+\epsilon 2^{-j}-F\left(a_{j}\right)\right]+ \epsilon \\
&<\sum\limits_{j=1}^{\infty} \mu\left(I_{j}\right)+2 \epsilon .
\end{aligned}
$$
Dado que $\epsilon$ es arbitrario, demostramos el resultado cuando $a$ y $b$ son finitos. 

Si $a=-\infty$, para cualquier $M<\infty$ los intervalos $\left(a_{j}, b_{j}+\delta_{j}\right)$ cubren $[-M, b]$ , por lo que el mismo razonamiento da $F(b)-F(-M) \leq \sum\limits_{j=1}^{\infty} \mu_{0}\left(I_{j}\right)+2 \epsilon$, mientras que si $b=\infty$, para cualquier $M<\infty$ obtenemos igualmente $F(M)-F(a) \leq \sum\limits_{j=1}^{\infty} \mu_{0}\left( I_{j}\right)+2 \epsilon$. El resultado deseado sigue entonces dejando $\epsilon \rightarrow 0$ y $M \rightarrow \infty$.
\end{proof}


\section{Integración en espacio de medida}

\begin{definicion}[Funciones medibles]{def:func_medibles} Sea $(X,\mathcal{A},\mu)$ un espacio de medida.  Una función $f:X\to\overline{\rr}:=\rr\cup\{\pm\infty\}$ se llama \index{Función!medible}\emph{medible} si para todo $a\in\rr$ 
\[
 f^{-1}([-\infty,a))=\left\{x\in X\mid f(x)<a\right\}. 
\]

\end{definicion}

La mayoría de los resultados y definiciones  establecidos en el contexto de la medida de Lebesgue en $\rr$ se extienden si cambios al contexto de medidas abstractas. Enumeremos los más importantes.
\begin{itemize}
 \item El concepto de propiedad válida en casi todo punto.
 \item Funciones simples. son funciones de la forma
 \[
  \phi(x)=\sum_{k=1}^na_k\chi_{E_k},
 \]
 con $a_k\in\rr$ y $E_k\in\mathcal{A}$, $k=1,\ldots,n$. 


 \end{itemize}
 
 \begin{ejercicio}{} Demostrar que si $f:X\to\overline{\rr}$ es medibles entonces si
\end{ejercicio}
 
 
 \begin{itemize}
 \item Integral de funciones medibles no-negativas.
 \item Teorema de Beppo-Levi.
 \item Lema Fatou.
 \item Teorema de la convergencia mayorada de Lebesgue
\end{itemize}


%\end{document}

\section{Medidas producto}

Hemos sido capaces de definir una medida sobre subconjuntos de $\rr$ que posee  las propiedades que a priori queríamos que tuviese, particularmente  la invariancia por traslaciones y tal que $\mu([0,1])=1$. El conjunto $\rr$ es importante pues es el modelo del espacio euclideano unidimensional. Pero tan naturales como este conjunto son los espacios euclideanos $n$-dimensionales $\rr^n$, sobre los cuales no tenemos definidas medidas, más que aquella definida en el ejemplo \ref{def:med_cont}. En estos espacios es util tener una medida  invariante por traslaciones y tal que $\mu([0,1]^n)=1$. Vamos a construit tales medidas en esta sección. Como $\rr^n$ es elproducto cartesiano de $n$ copias de $\rr$, seremos capaces de construir una medida allí si somo capaces de construir medidas sobre productos cartesianos de espacios de medidas.  


\begin{definicion}[Rectángulos medibles]{def:rect-med}
 Dados dos espacios de medida $(X_i,\mathscr{A}_i,\mu_i)$, $i=1,2$, un \index{rectángulo medible}\emph{rectángulo medible} es un conjunto de la forma $R=A_1\times A_2$, con $A_i\in\mathscr{A}_i$, $i=1,2$. Sea $\mathscr{A}_0$  la colección de todos los conjuntos que se pueden expresar como unión de una cantidad finita de restángulos medibles. 
\end{definicion}

\begin{ejercicio}{} Demostrar que $\mathscr{A}_0$ es una álgebra.
\end{ejercicio}




\begin{definicion}{} Si  $A\times B$ es un rectángulo medible definimos 
\[
 \mu_0(A\times B)=\mu_1(A)\mu_2(B).
\]
\end{definicion}




\begin{ejercicio}{} Supongamos que $A\times B$ es un rectángulo medible que es unión disjunta $\bigcup\limits_{j=1}^{\infty}A_j\times B_j$ de otros rectángulos medibles $A_j\times B_j$, $j=1,\ldots$.  Demostrar que 
\begin{equation}\label{eq:chis_prod}
 \rchi_{A}\rchi_{B}=\sum\limits_{j=1}^{\infty} \rchi_{A_j}\rchi_{B_j} 
\end{equation}

\end{ejercicio}

\begin{proposicion}{} Supongamos que $A\times B$ es un rectángulo medible que es unión disjunta $\bigcup\limits_{j=1}^{\infty}A_j\times B_j$ de otros rectángulos medibles $A_j\times B_j$, $j=1,\ldots$.  Entonces
\begin{equation}\label{eq:chis_prod}
 \mu_0(A\times B)=\sum\limits_{j=1}^{\infty} \mu_0(A_j\times B_j). 
\end{equation}

\end{proposicion}


\begin{definicion}{}
Si $C\in\mathscr{A}_0$ con $C=\bigcup\limits_{j=1}^{n}A_j\times B_j$, $A_j\times B_j=\emptyset$, si $i\neq j$, definimos 
\[
 \mu_0(C)=\sum\limits_{j=1}^{n}\mu_0(A_j\times B_j)
\]

\end{definicion}





\begin{proposicion}[Buena definición]{} La función $\mu_0$ esta bien definida, i.e. si $C$ admite dos representaciones distintas   $C=\bigcup\limits_{j=1}^{n}A_j\times B_j=\bigcup\limits_{j=1}^{n}A'_j\times B'_j$ como unión de rectángulos medibles mutuamente disjuntos entonces
 \[
\sum_{j=1}^{n}\mu_0(A_j\times B_j)=\sum_{j=1}^{n}\mu_0(A'_j\times B'_j)
\]
\end{proposicion}

\begin{proposicion}{} La función $\mu_0$ es una premedida para el algebra $\mathscr{A}_0$.
\end{proposicion}
