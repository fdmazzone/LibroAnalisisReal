\chapter{Sucesiones, series de funciones y sus amigos}


\section{Sucesiones de funciones}
Sea $K$ un espacio métrico, usualmente $K\subset \mathbb{R}^n$ para algún $n$. 

Una colección $f_n:K\to \mathbb{R}$ para $n=1,2,3,\dots$ se llama sucesión de funciones.

Dada una sucesión de funciones $f_n$, $n=1,2,3,\dots$  y $x\in K$, $f_n(x)$
es una sucesión de números reales y como tal puede o no converger a cierto límite.

La mayor diferencia entre una sucesión de números reales y una sucesión de funciones es
el hecho que en una sucesión de funciones los términos de la sucesión cambian cuando la variable
$x$ cambia. 
Por lo tanto el límite también puede cambiar, en caso de existir, y por consiguiente el límite
también es una función de $x$.
De manera que es necesario tener presente que cuando una sucesión de funciones es evaluada en
un valor de  $x$ particular resulta una sucesión de números reales.

Supongamos que para todo $x \in K$ la sucesión de números reales $f_n(x)$ converge,  
es decir que existe el $\lim\limits_{n \to \infty} f_n(x)$ y lo denotaremos $f(x)$. 
En este caso diremos que $f_n$ converge puntualmente a $f$.
 
\begin{ejemplo}
La sucesión $f_n(x)=\frac{1}{1+nx^2}$ converge puntualmente 
\[f(x)=\left\{\begin{array}{ll}
0&x\neq 0
\\
1&x=0
\end{array}
\right.\]
\textbf{Justificación:}
\\
Claramente si $x=0$ tenemos que $f_n(0)=1$ para todo $n \in \mathbb{N}$ y entonces $\lim\limits_{n \to \infty}f_n(0)=1$.

Si $x\neq 0$ entonces $nx^2\to \infty$ cuando $n\to \infty$  y por lo tanto $\lim\limits_{n \to \infty} \frac{1}{1+nx^2}=0$.

Como vemos, la determinación de la convergencia puntual suele reducirse al cálculo de un límite. 
Para este propósito es lícito usar todas las técnicas estudiadas en cursos anteriores como puede ser la Regla de L'H\^opital.
\end{ejemplo}

\begin{ejemplo}
La sucesión $f_n(x)=\frac{n^2x-n^2}{1+n^2x}$ converge a $f(x)=\frac{x-1}{x}$ si $x \neq 0$.

Si $x=0$ no converge.

Es necesario ser cuidadoso con la justificación. Por ejemplo, la Regla de L'H\^opital  sólo puede usarse en casos de indeterminaciones.

Si $x=0$ no hay indeterminación pues $f_n(0)=-n^2 $ y 
$\lim\limits_{n \to \infty}f_n(0)=\lim\limits_{n \to \infty}(-n^2)=-\infty$. 

Es lícito decir $\lim\limits_{n \to \infty}f_n(0)=-\infty$ en lugar de que $f_n$ no converge en $x=0$.

Cuando $x=1$ tampoco hay indeterminación pues $f_n(1)=\frac{0}{1+n^2}$ y por tanto $\lim\limits_{n \to \infty} f_n(1)=0$. 
 
Si $x\neq 0$ y $x \neq 1$ podemos usar la Regla de L'H\^opital dado que se tiene la indeterminación $\frac{\infty}{\infty}$. 
En efecto, 
$\lim\limits_{n \to \infty}f_n(x)=
\lim\limits_{n \to \infty}\frac{n^2x-n^2}{1+n^2x}=
\lim\limits_{n \to \infty}\frac{2nx-2n}{2nx}=
\lim\limits_{n \to \infty}\frac{2x-2}{2x}=\frac{1-x}{x}
$
Observemos que si $f(x)=\frac{1-x}{x}$ entonces $f(1)=0$ y por tanto $\lim\limits_{n\to \infty}f_n(x)=f(x)$ $\forall x\neq 0$.

Suele ser útil graficar algunas funciones de la sucesión y la función límite, ya sea empleando los procedimientos aprendidos 
en materias anteriores o usando sympy.
\end{ejemplo}

Para el Ejemplo 1.....COPIAR TEXTO DE SYMPY Y GRÁFICOS!!!!

En los ejemplos anteriores se forma una ``montaña'' alrededor de un punto \emph{fijo} ($x=0$). 
Pero, puede ocurrir otro comportamiento que observaremos los siguientes ejemplos.

\begin{ejemplo}
Si  $f_n(x)=\left\{
\begin{array}{ll}
1&n\leq x\leq n+1
\\
0&\mbox{en otro caso}
\end{array}
\right.$
entonces $\lim\limits_{n \to \infty} f_n(x)=0$. 
\end{ejemplo} 

GRAFICAR CON SYMPY!!!

A partir del gráfico vemos que los términos de la sucesión $f_n(x)$ son ``montañas móviles'' de altura 1.


\begin{ejemplo}
Si  $f_n(x)=\frac{nx}{1+n^2x^2}$ en $[0,\infty)$
entonces 
$\lim\limits_{n \to \infty} \frac{nx}{1+n^2x^2}=
\lim\limits_{n\to \infty} \frac{x}{x}\frac{x}{\frac{1}{n}+nx^2}=
\lim\limits_{n \to \infty}\frac{x}{\frac{1}{n}+nx^2}=0.$ 
\end{ejemplo}

GRAFICAR CON SYMPY!!!
En este caso también se observa una montaña móvil. 

HACER EL ANÁLISIS CON LA DERIVADA!!!

\begin{ejemplo}
Si $f_n(x)=\sqrt{x^2+\frac{1}{n^2}}$, $x \in \mathbb{R}$ luego 
$f_n^{'}(x)=\frac{x}{\sqrt{x^2+\frac{1}{n^2}}}$. 
Entonces $f_n^{'}(x)>0$ en $(0,+\infty)$ y $f_n^{'}(x)<0$ en $(0,+\infty)$, de donde  $(0,+\infty)$ es intervalo de
crecimiento para cada $f_n(x)$ y $(-\infty,0)$ es intervalo de decrecimiento para cada $f_n(x)$.
Luego cada $f_n(x)$ tiene un mínimo en $x=0$ y el valor mínimo es $f_n(0)=\frac{1}{n}.$

Por otra parte, $\lim\limits_{n \to \infty} f_n(x)=\lim\limits_{n \to \infty} \sqrt{x^2+\frac{1}{n^2}}=\sqrt{x^2}=|x|$.
\end{ejemplo}

GRAFICO!!!!

En el Análisis Matemático, además de límites tenemos conceptos como continuidad, derivadas, integrales, etc.
Es común operar expresiones conjugando varios de ellos y queremos contar con relaciones entre ellos que permitan 
transformar las expresiones. 
Por ejemplo, ?`es importante el orden en que se realizan  las operaciones?
?`Es lo mismo tomar límite y luego derivar que hacerlo en el orden inverso? 
Si se tienen dos límites, ?`se pueden permutar?

\begin{ejemplo}
Si $f_n(x)=\sen(nx)$ para $x\in [0,\pi]$ entonces $f_n(0)=f_n(\pi)=0$ y la sucesión converge en estos valores.

Veamos que la sucesión de funciones dada no converge en ningún otro valor. 

Supongamos que $x\neq0$, $x\neq \pi$ y $\lim\limits_{n \to \infty}\sen(nx)=\alpha$. 

Si se tuviese $\alpha\neq 0$ entonces
\[
1=\lim\limits_{n \to \infty}\frac{\sen(2nx)}{\sen(x)}=2\lim\limits_{n \to \infty}\cos(nx)
\]
de donde $\lim\limits_{n \to \infty} \cos(nx)=\frac{1}{2}$.
Sin embargo, 
\[
\lim\limits_{n \to \infty}\cos(2nx)=
\lim\limits_{n \to \infty}2\cos^2(nx)-1=-\frac{1}{2}
\]
lo que nos lleva a una contradicción. 

Por lo tanto, $\lim\limits_{n \to \infty}\sen(nx)=0$.
Entonces 
$\lim\limits_{n \to \infty}|\cos(2nx)|=1
$
y 
\[
0=\lim\limits_{n \to \infty} |\sen[(n+1)x]|=
\lim\limits_{n \to \infty} |\sen(nx)\cos x+\cos(nx)\sen x|=|\sen x|
\]
y necesariamente  $x=0$ \'o $x=\pi$.
\end{ejemplo}


\begin{ejemplo}
En el Ejemplo\ref{}1 vimos que  la sucesión 
$f_n(x)=\frac{1}{1+nx^2}$ converge puntualmente 
\[f(x)=\left\{\begin{array}{ll}
0&x\neq 0
\\
1&x=0
\end{array}
\right.\]

Si calculamos 
\[
\lim\limits_{n\to \infty}\lim\limits_{x \to 0}f_n(x)=\lim\limits_{n \to \infty}1=1
\]
y a continuación permutamos los límites obtenemos
\[
\lim\limits_{x\to 0}\lim\limits_{n \to \infty}f (x)=\lim\limits_{x \to 0}=0
\]
Por lo tanto, la permutación de los límites produce resultados \textbf{distintos}.

También vemos que la función límite es discontinua a pesar de que cada $f_n(x)$ es continua para cada $n$.
\end{ejemplo}

\begin{ejemplo}
Con las funciones del Ejemplo\ref{} 3 tenemos
\[
\int_{-\infty}^{+\infty} \lim\limits_{n \to  \infty} f_n(x)\,dx=\int_{-\infty}^{\infty} 0\,dx=0
\]
mientras que 
\[
\lim\limits_{n \to \infty} \int_{-\infty}^{\infty} f_n(x)\,dx=\lim\limits_{n \to \infty} \int_{n}^{n+1}dx=1
\]
En este caso, la permutación entre la operación de integración y la de límite también produce resultados \textbf{distintos}.
\end{ejemplo}

\begin{ejemplo}
Cada $f_n(x)=\sqrt{x^2+\frac{1}{n^2}}$ del Ejemplo\ref{} 4/5 es derivable y las derivadas son
$f_n^{'}(x)=\frac{x}{\sqrt{x^2+\frac{1}{n^2}}}$. 
Si computamos
\[
\lim\limits_{n \to \infty} f_n^{'}(x)=\frac{x}{|x|}=
\left\{
\begin{array}{ll}
1&x>0
\\
-1&x<0
\end{array}
\right.\]
cuando $x\neq 0$.
Entonces la funci\'on límite $f(x)=\frac{x}{|x|}$ no es derivable en 0. Luego  
\[
0=\lim\limits_{n \to \infty}f_n^{'}(0)\neq \frac{d}{dx}(\lim\limits_{n \to \infty}f_n(x))|_{x=0}
\]
pues ni siquiera tiene sentido el miembro de la derecha.
\end{ejemplo}

Es así que tenemos  
\textbf{
Problema: 
Encontrar condiciones que permitan permutar las operaciones anteriores.}

HABRÍA QUE PONER EL PROBLEMA ANTERIOR EN UN CUADRITO. BUSCAR EL COMANDO!!!

Antes de atacar este problema vamos a presentar varios ejemplo \textit{famosos} de sucesiones.


OJO!!!! LO QUE SIGUE EN EL APUNTE NO TIENE EJEMPLOS FAMOSOS, VIENEN LAS SERIES!!!!!

\section{Series de funciones}
Dada una sucesión de funciones $f_n(x)$, $n=1,2,\dots$ podemos formar otra sucesión tomando las sumas acumuladas o
sumas parciales
\[
\begin{split}
s_1(x)&=f_1(x)
\\
s_2(x)&=f_1(x)+f_2(x)
\\
&\vdots
\\
s_n(x)&=f_1(x)+f_2(x)+\dots+f_n(x)
\end{split}
\]

Si la nueva sucesión $\{s_n(x)\}$ converge a $f$ se dice que la serie $\sum\limits_{n=1}^{\infty} f_n(x)$
converge a $f$ o que 
\[
\sum\limits_{n=1}^{\infty} f_n(x)=f(x).
\]

En pocas ocasiones se puede determinar que una serie converge hallando una expresión simple para $s_n(x)$ y
calculando su límite.

\begin{ejemplo}
Si $f_n(x)=c$ con $c \in \mathbb{R}$ un número independiente de $n$, entonces
\[
\begin{split}
s_1(x)&=f_1(x)=c
\\
s_2(x)&=f_1(x)+f_2(x)=2c
\\
&\vdots
\\
s_n(x)&=f_1(x)+f_2(x)+\dots+f_n(x)=nc.
\end{split}
\]
Entonces
\[
\lim\limits_{n \to \infty} s_n(x)=
\left\{
\begin{array}{lll}
0&si&c=0
\\
\infty&si&c\neq 0
\end{array}
\right.
\]
y por lo tanto la series converge sólo cuando $c=0$.
\end{ejemplo}

\begin{ejemplo}
Si $f_n(z)=z^n$ para $n=0,1,2,\dots$, luego 
$s_n(z)=f_0(z)+\dots+f_n(z)=1+z+\dots+z^n$ y $zs_n(z)=z+z^2+\dots+z^n+z^{n+1}$ entonces
$zs_n(z)-s_n(z)=z^{n+1}-1$ y por tanto $s_n(z)=\frac{z^{n+1}-1}{z-1}$.
De este modo, logramos expresar $s_n(z)$ en una fórmula relativamente sencilla. Ahora, 
\[
\lim\limits_{n \to \infty} s_n(z)=
\lim\limits_{n \to \infty} \frac{z^{n+1}-1}{z-1}=
\left\{\begin{array}{ll}
\frac{1}{1-z}&|z|<1
\\
\mbox{no converge}&|z|\geq 1
\end{array}
\right.
\]
\end{ejemplo}

Es interesante ver qué ocurre en $|z|=1$. 

Si $|z|=1$ entonces $z=e^{i\theta}=\cos \theta +i \sen \theta$ y 
\[
\begin{split}
s_n(z)=
\frac{z^{n+1}-1}{z-1}=
\frac{z^{n+1}}{z-1}-\frac{1}{z-1}=
\\
\frac{z^{n+1}(\overline z-1)}{|z-1|^2}-\frac{1}{z-1}=
\frac{ [\cos(n+1)\theta+i\sen(n+1)\theta](\overline z-1)}{|z-1|^2}-\frac{1}{z-1}=
\\
\cos(n+1)\theta \frac{\overline z-1}{|z-1|^2}+i \sen(n+1)\theta \frac{(\overline z-1)}{|z-1|^2}-\frac{1}{z-1}
\end{split}
\]
son funciones oscilantes como en el Ejemplo \ref{} 5 y medio. VER GRÁFICOS!!!

En los ejemplos anteriores pudimos justificar la convergencia calculando explícitamente el límite. 
Ésto es posible las menos de las veces. En materias anteriores se estudiaron criterios para la 
convergencia de series  numéricas. Estos criterios establecen condiciones, algunas necesarias, otras 
suficientes y algunas necesarias y suficientes para que la serie converja. Recordaremos algunos de ellos.

QUIZÁS EN ALGÚN RECUADRO!!!
\textit{Criterio del Resto}
Si  $\sum\limits_{n=1}^{\infty} a_n$ converge entonces $\lim\limits_{n \to \infty} a_n=0$.

Coomo es una condición  necesaria sólo sirve para decir cuándo una serie no converge.

\begin{ejemplo}
Si $f_n(x)= \sen(nx)$ para $x \in [0,\pi]$. 
Como ya vimos, $\sen(nx)$ no converge excepto para $x=0$ \'o $x=\pi$. 
Luego, $\sum\limits_{n=1}^{\infty} \sen(nx)$ no converge.
\end{ejemplo}

Como veremos más adelante, la serie $\sum\limits_{n=1}^{\infty}\frac{1}{n}$ no converge y sin embargo
$\lim\limits_{n \to \infty}\frac{1}{n}=0$. Entonces el \textit{Criterio del Resto} no sirve para determinar
la convergencia de una serie.

\textit{Convergencia Absoluta}
Si$\sum\limits_{n=1}^{\infty} |a_n|$ converge entonces $\sum\limits_{n=1}^{\infty} a_n$ converge.

\begin{ejemplo}
$\sum \limits_{n=1}^{\infty} (-1)^n z^n$ converge cuando $|z|<1$.
\end{ejemplo}

\textit{Cirterio de Comparación}
Si $0\leq a_n\leq b_n$ y $\sum\limits_{n=1}^{\infty} b_n$ converge entonces $\sum\limits_{n=1}^{\infty}a_n$ converge.
Dicho de otro modo, si $\sum\limits_{n=1}^{\infty} a_n$ diverge (su suma es $+\infty$) entonces 
$\sum\limits_{n=1}^{\infty}$ diverge.

\begin{ejemplo}
\begin{enumerate}
\item $\sum\limits_{n=1}^{\infty} \frac{1}{2^n} \sen(nx)$ converge pues $|\frac{1}{2^n} \sen(nx)|\leq \frac{1}{2^n}$.
\item $\sum\limits_{n=1}^{\infty} \frac{1}{n^2} \sin(nx)$ converge pues para $n>1$ tenemos
\[
\left|\frac{1}{n^2}\sin(nx)\right|\leq \frac{1}{n^2}\leq \frac{1}{(n-1)n}=\frac{1}{n-1}-\frac{1}{n}
\]
y
\[
\sum\limits_{n=2}^{m}\frac{1}{n-1}-\frac{1}{n}=1-\frac{1}{m}\to 1\mbox{  cuando  } m\to \infty
\]
Luego $\sum\limits_{n=1}^{\infty} \frac{1}{n^2}$ converge y 
$\sum\limits_{n=1}^{\infty} \frac{1}{n^2} \sen(nx)$ también.
\end{enumerate}
\end{ejemplo}
\section{Series de Fourier}