\chapter{Integral de Lebesgue}

\section{Definici\'on y propiedades inmediatas}

\begin{definicion}{defi:integral-de-Lebesgue}
Sean $E\subset \rr^n$,  $f:\rr^n\to \overline{\rr}$  y $f\geq 0$ sobre $E$ medible. 
\\
La integral de Lebesgue de $f$ se define  mediante
\[
\int_E f(x)\.dx=\sup\left\{ \sum\limits_{i=1}^{N} m(E_i) \right\},
\]
donde el supremo se toma sobre toda descomposici\'on del conjunto $E$ en uni\'on de conjuntos medibles $E_i$ y mutuamente disjuntos,  siendo 
\[\alpha_i =\inf\limits_{E_i}f.\]
\end{definicion}

Se usa la convenci\'on $0{.}(+\infty)=+\infty{.}0=0.$

De la Definici\'on \ref{defi:integral-de-Lebesgue} se deduce que si $f\equiv c\in \rr$ sobre $E$, 
entonces
\[ \int f\,dx=c m(E).\]

\begin{teorema}{}
Si $E=A\cup B$ y $f\geq 0$ en $E$, entonces
\[
\int_E f=\int_A f +\int_B f.
\]
\end{teorema}

\begin{demo}
Sean $E=\bigcup\limits_{i=1}^N E_i$ y $\alpha_i=\inf\limits_{E_i} f$.
  $A_i=A\cap E_i$ y $B_i=B\cap E_i$ y llamamos $\beta_i=\inf\limits_{A_i} f$ y $\gamma_i=\inf\limits_{B_i} f$. 
Entonces $\alpha_i \leq \beta_i$, $\alpha_i\leq \gamma_i$ y 
\[
\begin{split}
\sum\limits_{i=1}^N \alpha_i m(E_i)=
&\sum\limits_{i=1}^N \alpha_i (m(A_i)+m(B_i))
\\
\leq & 
\sum\limits_{i=1}^N \beta_i m(A_i)+\sum\limits_{i=1}^N \gamma_i m(B_i)
\\
\leq & \int_A f\,dx +\int_B f\,dx. 
\end{split}
\]
Luego 
\[\int_E f \leq \int_A f +\int_B f.
\]
Sean $A=\bigcup\limits_{i=1}^N A_i$, $B=\bigcup\limits_{i=1}^M B_i$, $\beta_i$ y $\gamma_i$ como antes. 
Entonces 
\[
\sum \limits_{i=1}^N \beta_i m(A_i) +\sum\limits_{i=1}^M \gamma_i m(B_i) \leq \int_E f.
\]
Luego
\[
\int_A f + \int_B f \leq \int_E f.
\]
\end{demo}

\begin{itemize}
    \item Si $0\leq f\leq g$, entonces 
    \[\int_E f \leq \int_E g,\] 
    pues $\inf\limits_{E_i} f \leq \inf\limits_{E_i} g.$
    \item Si $f\geq 0$ sobre $E$ y $A\subset E$, entonces \[\int_A f \leq \int_E f,\] pues 
    $\int_E f=\int_A f +\int_{E-A} f$.
    \item Si $m(E)=0$ y $f\geq 0$ sobre $E$, entonces
    \[\int_E f =0.\]
    \item Si $f=g$ en c.t.p. de $E$, entonces
    \[\int_E f =\int_E g.\]
    \begin{demo}
    Sea $A=\{f \neq g\}$ entonces $m(A)=0$ y 
    \[\int_E f =\int_{E-A} f =\int_{E-A} g=\int_E g.\]
    \end{demo}
    \item Si $f\geq 0$ en $E$, entonces
    \[\int_E  f = \int_{\rr^n}  f \chi_E\,dx.\]
    \item 
    Cuando $E= \rr^n$, escribimos 
    \[\int_{\rr^n} f = \int f.\]
    \item Si $E=(a,b)\subset \rr$, ponemos
     \[\int_{E} f = \int_a^b f.\]
    \end{itemize}
    
    \section{Integral de funciones simples}
    
    \begin{teorema}{}
    Sean $E=E_1 \dot{\cup} E_2\dot{\cup} \ldots \dot{\cup} E_N$ y 
    $\beta_i \in \rr,\; i=1,2,\ldots,N$, $\beta_i\geq 0$. Luego, 
    \[
    \int_E \sum\limits_{i=1}^N \beta_i \chi_{E_i}=
    \sum\limits_{i=1}^N \beta_i m(E_i)
    \]
    \end{teorema}
    
    \begin{demo}
    \[
    \int_E \sum\limits_{i=1}^N \beta_i \chi_{E_i}=
    \sum\limits_{j=1}^N \int_{E_j} \sum\limits_{i=1}^N \beta_i \chi_{E_i} =\sum\limits_{j=1}^N \beta_j m(E_j).
    \]
    \end{demo}
    
    \begin{teorema}{}
    Si $\varphi$  y $\psi$ son funciones simples no negativas y $c\in \rr^+$, entonces
    \[\begin{split}
            \int_E (\varphi+\psi) &= \int_E \varphi +\int_E \psi \;\;\mbox{  y   }\;\;\\
        \int_E c\varphi &=c\int_E \varphi.
        \end{split}
    \]
    \end{teorema}
    
    \begin{demo}
    Sean $\varphi=\sum\limits_{i=1}^N \alpha_i \chi_{E_i}$ y $\psi=\sum\limits_{j=1}^M \beta_j \chi_{F_j}$. 
    Tenemos 
    \[
    \varphi+\psi=
    \sum\limits_{i=1}^N\sum\limits_{j=1}^M (\alpha_i+\beta_j) \chi_{E_i \cap F_j},
    \]
    y entonces
    \[
    \begin{split}
    \int_E \varphi+\psi=& 
    \sum\limits_{i=1}^N\sum\limits_{j=1}^M (\alpha_i+\beta_j) m(E_i \cap F_j)
    \\
    =&\sum\limits_{i=1}^N \alpha_i \sum\limits_{j=1}^M m(E_i \cap F_j)+
    \sum\limits_{j=1}^M \beta_j \sum\limits_{i=1}^N m(E_i \cap F_j)
    \\
    =&\int_E \varphi +\int_E \psi.
    \end{split}
        \]
    Como $c\varphi=\sum\limits_{i=1}^N c \alpha_i \chi_{E_i}$, luego
    \[
    \int_E c\varphi = \sum\limits_{i=1}^N c\alpha_i m(E_i)=\alpha \int_E \varphi.
    \]
    \end{demo}
    
    \begin{teorema}{}
    Si $f\geq 0$, entonces
    \[\int_E f = \sup\limits_{0\leq \varphi\leq f} \int_E \varphi.\]
    \end{teorema}
    
    \begin{demo}
    Sean $E=E_1\cup E_2\cup\ldots \cup E_N$ y $\alpha_i=\inf\limits_{E_i} f$. 
    Supongamos que $\alpha_i<\infty$. Luego, si $\varphi=\sum\limits_{i=1}^N \alpha_i \chi_{E_i}$ tenemos 
    \[
    \int_E \varphi =\sum\limits_{i=1}^N \alpha_i m(E_i)\;\;
    \mbox{ y }\;\; 0\leq \varphi \leq f.
    \]
    En general, para $k\in \nn$ definamos 
    \[\alpha_{ik}=\min\{k,\alpha_i\}\]
    y 
    \[ \varphi_k=\sum\limits_{i=1}^n \alpha_{ik} \chi_{E_i}.\]
    Ahora, $0\leq \varphi_k \leq f$ y $\alpha_{i1}\leq \alpha_{i2}\leq \ldots \nearrow \alpha_i$. 
    LUego 
    \[
    \begin{split}
    \sum\limits_{i=1}^N \alpha_i m(E_i)=&
    \lim\limits_{k \to \infty} \sum\limits_{i=1}^N \alpha_{ik} m(E_i)
    \\
    =&\lim\limits_{k \to \infty} \int_E \varphi_k 
     \leq  \sup\limits_{0\leq \varphi \leq f} \int_E \varphi.
    \end{split}
    \]
    Luego
    \[
    \int_E f \leq \sup\limits_{0\leq \varphi \leq f} \int_E \varphi.
    \]    
    La otra desigualdad es inmediata.
    \end{demo}
    
    
    \begin{lema}{}
    Sean $f_1\leq f_2\leq \ldots$ funciones no negativas y $\varphi$ funci\'on simple no negativa tal que 
    \[
    \varphi \leq \lim\limits_{k \to \infty} f_k.
    \]
    \end{lema}
    
\begin{demo}
Sean $0\leq \alpha_1< \alpha_2<\ldots < \alpha_N$ los distintos valores ordenados que toma $\varphi$ y sea 
\[  
A_i=\{\varphi=\alpha_i\}.
\]
Supongamos probado el lema cuando $\alpha_1>0$. Entonces, sale para $\alpha_1=0$, pues aplicando el caso probado en $E-A_1$ tenemos 
\[
\int_E \varphi =\int_{E-A_1}\varphi 
\leq \lim\limits_{k\to \infty} \int_{E-A_1} f_k
\leq \lim\limits_{k \to \infty} \int_E f_k.
\]
Supongamos ahora que $\alpha_1>0$.\\ Sea $0<\epsilon$ con $0<\epsilon<\alpha_1$.
Luego $\varphi-\epsilon$ es una funci\'on simple que toma los valores $\alpha_i-\epsilon$ sobre $A_i$. 
Poniendo 
\[
E_k=\{\alpha \in E: f_k(x)> \varphi (x)-\epsilon \}.
\]
Por hip\'otesis
\[
E_k \subset E_{k+1}  \;\mbox { y }\; \bigcup\limits_{k=1}^{\infty}E_k
=E.\]
Luego $m(E_k) \to m(E)$ cuando $k \to \infty$.

Se presentan dos casos
\begin{enumerate}
    \item Si $m(E)<\infty$, tenemos $m(E-E_k) \xrightarrow[]{k \to \infty}0$
     y 
    \[
    \begin{split}
    \int_E f_k \geq \int_{E_k} f_k &\geq \int_{E_k} \varphi(x) -\epsilon
    \\
    &=\int_E \varphi(x)-\epsilon -\int_{E-E_k}\varphi(x)-\epsilon
    \\
    &\geq \int_E \varphi -\epsilon m(E)-(\alpha_N-\epsilon)m(E-E_k).
    \end{split}
         \]
    Luego 
    \[
    \lim\limits_{k \to \infty} \int_E f_k \geq \int_E \varphi-\epsilon m(E).
    \]
    Como $\epsilon$ es arbitrario, se obtiene la desigualdad del lema.
    \item Si $m(E)=\infty$, consideramos $m(E_k)\nearrow +\infty$
    y obtenemos
    \[
    \int_E f_k \geq \int_{E_k} f_k 
    \geq \int_{E_k} \varphi-\epsilon \geq (\alpha_1-\epsilon)m(E_k)\nearrow +\infty, 
    \]
    lo cual lleva a la desigualdad del lema.
\end{enumerate}
\end{demo}

\section{Paso al l\'imite bajo el signo de integral}

\begin{teorema}{teo:Beppo-Levi}[Beppo-Levi]
Sea $0\leq f_1\leq f_2\leq\ldots \nearrow f$. 
Entonces
\[ \int_E f =
\lim\limits_{k \to \infty} \int_E f_k.\]
\end{teorema}
    
    \begin{demo}
    Sea $\varphi$ funci\'on simple tal que $0\leq \varphi \leq f$. Luego
    \[ \varphi \leq \lim\limits_{k \to \infty} f_k,\]
    y por tanto
    \[
\int_E \varphi \leq \lim\limits_{k \to \infty} \int_E f_k.
    \]
    As\'i, llegamos a 
    \[
    \int_E f=
    \sup\limits_{0\leq \varphi\leq f} \int_E \varphi
    \leq \lim\limits_{k \to \infty} \int_E f_k.
    \]
La otra desigualdad es inmediata. 
    \end{demo}
    
    
    \begin{teorema}{}
    Si $f,g\geq 0$ y $c \in \rr$, entonces
    \[ 
    \int_E (f+g)= \int_E f+\int_E g \;\;\;\mbox{ y }\;\;\;
    \int_E cf=c\int_E f.
    \]
    \end{teorema}
    
     \begin{demo}
     Sean $\varphi_k$ y $\psi_k$ funciones simples tales que $\varphi_k \nearrow f$ y $\psi_k \nearrow g$.
     Luego $\varphi_k +\psi_k \nearrow f+g$ y entonces
     \[
     \begin{split}
         \int_E f+g =&\lim\limits_{k \to \infty} \int_E \varphi_k +\psi_k
         \\
         =&\lim\limits_{k \to \infty}\int_E \varphi_k
         +
         \lim\limits_{k \to \infty}\int_E \psi_k
         \\
         =&\int_E f +\int_E g.
     \end{split}
     \]
          \end{demo}
          
          \begin{corolario}{cor:Beppo-Levi}
          Si $f_k\geq 0$, $k=1,2,\ldots,$ entonces
          \[
          \int_E \sum\limits_{k=1}^{\infty} f_k
          =
          \sum\limits_{k=1}^{\infty} \int_E f_k.
          \]
          \end{corolario}
          
          \begin{demo}
          Sean $S_N=\sum\limits_{k=1}^N f_k$ y 
          $s=\sum\limits_{k=1}^{\infty} f_k=\lim\limits_{N \to \infty} S_N$. \\Luego
          $0\leq S_1 \leq S_2 \leq \ldots \nearrow s$. 
          De este modo, 
          \[
          \begin{split}
          \int_E s=&\lim\limits_{N \to \infty} \int_E S_N
          \\
          =&\lim\limits_{N \to  \infty} \sum\limits_{k=1}^N \int_E f_k
          = \sum\limits_{k=1}^{\infty} \int_E f_k.
          \end{split}
          \]
                    \end{demo}
                    
                    
                    \begin{lema}{lema:Fatou}[Lema de Fatou]
                    Si $f_k\geq 0$, entonces
                    \[
                    \int_E \liminf\limits_{k \to \infty} f_k
                    \leq 
                    \liminf\limits_{k\to \infty} \int_E f_k.
                    \]
                    \end{lema}
                    
                    \begin{demo}
                    Sean 
                    \[
                    g(x)=\liminf\limits_{k \to \infty} f_k\;\;
                    \mbox{  y }\;\;
                      g_k(x)=\inf\limits_{j\geq k} f_j.
                    \]
                    As\'i, 
                    $g_1\leq g_2\leq \ldots \nearrow g $ y 
                    \[ g(x)=\sup\limits_{k} g_k(x)=\lim\limits_{k \to \infty} g_k(x).  
                    \]
                    Por el Teorema \ref{teo:Beppo-Levi} y como $g_k \leq f_k$, se tiene 
                    \[
                    %\begin{split}
                    \int_E g=\lim\limits_{k \to \infty} \int_E g_k
                    =\liminf\limits_{k \to \infty} \int_E g_k
                   % \\
                    \leq \liminf\limits_{k \to \infty} \int_E f_k.
                    %\end{split}
                    \]
                    \end{demo}
                    
                    
                    \begin{corolario}{cor:Fatou}
                    Si $f_k \to f$ en $E$, entonces
                    \[
                    \int_E f \leq \liminf\limits_{k \to \infty}\int_E f_k.
                    \]
                    \end{corolario}
                    
                    \section{Integrales de funciones de distinto signo}
                    
                    \begin{definicion}{}
                    Si $f=f^+-f^-$, diremos que $f$ es integrable sobre $E$ si y s\'olo si
                    \[
                    \int_E f^+ \;\;\mbox{ y }\;\;
                    \int_E f^-
                    \]
                    son finitas. 
                    
                    En este caso, escribimos
                    \[
                    \int_E f =\int_E f^+ - \int_E f^-.
                    \]
                    \end{definicion}
                    
                    \begin{teorema}{}
                    $f$ es integrable sobre $E$ si y s\'olo si $|f|$ lo es.  Y, en este caso, vale
                    \[
                    \left|\int_E f\right|\leq \int_E |f|.
                    \]
                    \end{teorema}
                    
                    \begin{demo}
                    $\Rightarrow)$
                    Como $|f|=f^++f^-$, si $f$ es integrable entonces $|f|$ tambi\'en los es. 
                    
                    $\Leftarrow)$
                    Si $|f|$ es integrable, como $f^+\leq |f|$ y $f^-\leq |f|$, entonces $f$ tambi\'en resulta integrable.
                    \end{demo}
                    
                    \begin{teorema}{teo:diferencia-de-integrables}
                    Si $f_1,f_2\geq 0$ son integrables sobre $E$ y $f=f_1-f_2$, entonces $f$ es integrable y 
                    \[\int_E f_1-f_2 = \int_E f_1 - \int_E f_2.\]
                    \end{teorema}
                    
                    \begin{demo}
                    A partir de  que $f^+\leq f_1$ y $f^-f_2$ se deduce que $f$ es integrable.  
                    \\
                    Como $f=f^+-f^-=f_1-f_2$ entonces 
                    $f^+ + f_2=f_1 + f^-$ y 
                    \[
                    \int_E f^+ + \int_E f_2=
                    \int_E f_1 + \int_E f^-.
                    \]
                    \end{demo}
                    
                    \begin{teorema}{}
                    Si $f\geq 0$ es integrable y $|g|\leq f$, entonces $g$ es integrable. 
                    \end{teorema}
                    
                    \begin{demo}
                     La prueba sale a partir de que 
                    $g^+\leq f$ y $g^- \leq f$.
                    \end{demo}
                   
                   \begin{teorema}{}
                   Si $f$ y $g$ son integrables sobre $E$ y $c\in \rr$, entonces $f+g$ y $cf$ son integrables sobre $E$. Adem\'as, 
                   \[
                   \int_E f+g =\int_E f +\int_E g
                   \;\;\mbox{ y }\;\;
                   \int_E cf =c\int_E f.
                   \]
                 \end{teorema}
                    
                    \begin{demo}
                     Como $f+g=f^+ + g^+ -(f^-+g^-)$, aplicando el Teorema \ref{teo:diferencia-de-integrables} se obtiene que $f+g$ es integrable y         \[
                     \begin{split}
                     \int_E f+g =&\int_E f^{+} +g^{+}  - \int_E f^{-} + g^{-}
                     \\
                     =&\int_E f^{+} +\int_E g^{+}  - \int_E f^{-} -\int_E g^{-}
                     \\
                     =&\int_E f + \int_E g.
                     \end{split}
                     \]
                     
                     Si $c\geq 0$, entonces $cf=cf^{+}-cf^{-}$ y el resultado se obtiene por el Teorema \ref{teo:diferencia-de-integrables}.
                     
                     Si $c<0$, el resultado se obtiene a partir de que $cf=(-c)f^{-}-(-c)f^{+}$.
                    \end{demo}
                    
                    \begin{corolario}{}
                    Si $f$ y $g$ son integrables tales que $f\leq g$, entonces 
                    \[\int_E f \leq \int_E g.\]
                    \end{corolario}
                    
                    
                    As\'i, 
                    \[
                    L(E)=\{
                    f: f\;\mbox{ es integrable sobre }\;E
                    \}
                    \]
                    es un espacio vectorial y la aplicaci\'on
                    \[
                    f \longmapsto \int_E f
                    \]
                    es una aplicaci\'on lineal.
                    
                    \section{Convergencia Mayorada}
                    
                    
                    Si $f_k$ son integrables sobre $E$ y $f_k \to f$ en c.t.p. de $E$ siendo $f$  integrable en $E$, \textbf{en general no es cierto} que 
                    \[
                    \int_E f_k \to \int_E f.
                    \]
                    
                    \begin{ejemplo}{}
                    Si $f_k=k \chi_{(0,\frac{1}{k})}$ entonces $f_k \to 0$ puntualmente en $(0,1)$. Sin embargo,
                     \[
                    \int_{(0,1)} f_k = 1,\;\; \forall k\in \nn.
                    \]
                     y por lo tanto 
                     \[
                    \int_{(0,1)} f_k  \nrightarrow \int_{(0,1)} f.
                    \]
                    \end{ejemplo}
                    
                    \begin{teorema}{}[Convergencia Mayorada de Lebesgue]
                    Sea $\Phi$ func\'ion integrable sore $E$. 
                    Si $f_k$ son funciones integrables tales que 
                    \[|f_k|\leq \Phi\;\mbox{ en }\;E,\]    
                    entonces
                    $g=\liminf f_k$, $h=\limsup
                    f_k$ son integrables sobre $E$ y 
                    \[
                    \int_E \liminf f_k \leq 
                    \liminf \int_E f_k \leq\limsup \int_E f_k
                    \leq \int_E h.
                    \]
                    \end{teorema}
                    
                    \begin{demo}
                    Por hip\'otesis se tiene que $-\Phi \leq f_k \leq \Phi$, a partir de lo cual se deduce que 
                    $-\Phi \leq g\leq \Phi$ y  $-\Phi \leq h\leq \Phi$ y por lo tanto $g$ y $h$ resultan integrables sobre $E$.
                    
                    Por otra parte,  $f_k+\Phi \geq 0$ y $\Phi-f_k\geq 0$ y 
                    \[\begin{split}
                    \liminf\limits_{k \to \infty} (f_k+\Phi)=g+\Phi
                    \\
                    \liminf\limits_{k \to \infty} (\Phi-f_k)=\Phi-h.
                    \end{split}
                    \]
                    Ahora, por el Lema de Fatou (Lema \ref{lema:Fatou}) tenemos
                    \[
                    \int_E g + \int_E \Phi =
                    \int_E g+\Phi \leq \liminf\limits_{k \to \infty} \int_E f_k +\Phi=
                    \liminf\limits_{k \to \infty} \int_E f_k +\int_E \Phi.
                    \]
                    Luego 
                    \[
                    \int_E g \leq \liminf \int_E f_k.
                    \]
                    La otra desigualdad se obtiene de manera an\'aloga.
                    \end{demo}
                    
                    \begin{corolario}{cor:conv-mayorada}
                    Si $f_k \to f$ en cada punto de $E$ y $|f_k|\leq \Phi \in L(E)$, entonces $f \in L(E)$ y 
                    \[
                    \lim\limits_{k \to \infty}\int_E f_k = \int_E f.
                    \]
                    \end{corolario}
                    
                    
                    \section{La integral y los conjuntos de medida nula}
                    
                    Si $f,g\geq 0$ y $f=g$ en c.t.p de $E$, entonces
                    \[
                    \int_E f =\int_E g.
                    \]
                    
                    En particular, sobre un conjunto de medida nula cualquier $f$ medible e integrable  con integral que vale 0.
                    
                    Usaremos la \textbf{desigualdad de Chebyshev} que establece que si $f$ es medible sobre $E$ se cumple 
                    \[
                    m\left(\{ x\in E: |f(x)|>\lambda\}\right) \leq \frac{1}{\lambda} \int_E |f|.
                    \]
                    La prueba de la desigualdad de Chebyshev es sencilla, a saber, 
                    \[
                    \int_E |f| \geq \int_{\{|f|\geq \lambda\}} |f|
                    \geq \lambda m\left(\{|f|\geq \lambda\}\right).
                    \]
                    
                    \begin{teorema}{}
                    Si $f\geq 0$ sobre $E$ y $\int_E f=0$, entonces $f=0$ en c.t.p. de $E$.
                    \end{teorema}
                    
                    \begin{demo}
                    Sean $Z_k=\{f >\frac{1}{k}\}$ y $Z=\{f>0\}$, entonces 
                    \[
                    Z=\bigcup\limits_{k=1}^{\infty} Z_k.
                    \]
                    Ahora, 
                    \[
                    m(Z_k)=m\left(\left\{f>\frac{1}{k}\right\}\right)\leq k \int_E f=0,
                    \]
                    y en consecuencia  $m(Z)=0$.
                    \end{demo}
                    
                    \section{ Invariancia bajo traslaciones }
                    
                    \begin{teorema}{}
                Sea $f\geq 0$, entonces $\forall h \in \rr^n$ se tiene
                \begin{enumerate}
                    \item \label{it:invariancia-f-x+h} $\int f(x+h) =\int f(x)$,
                    \item \label{it:invariancia-f-y-E-x+h} $\int_E f(x+h) =\int_{E+h} f(x)$.
                \end{enumerate}
                    \end{teorema}
                    
                    \begin{demo}
                    \begin{enumerate}
                        \item 
                    \begin{itemize}
                        \item Si $f=\chi_E$, entonces 
                        $f(x+h)=\chi_E (x+h) =\chi_{E-h}(x).$
                        As\'i, 
                        \[
                        \int f(x+h)= \int \chi_{E-h} (x) =m(E-h)
                        \]
                        y 
                        \[
                        \int f(x) = \int \chi_E(x)=m(E),
                        \]
                        y son iguales
                        \item Si $f$ es simple, entonces 
                        \[  f=\sum\limits_{i=1}^N \alpha_i f_i, \;\; \alpha_i\geq 0,\;f_i=\chi_{E_i}.
                        \]
                        Luego, 
                        \[
                        \int f(x+h) =\sum\limits_{i=1}^N \alpha_i \int f_i(x+h)=
                        \sum\limits_{i=1}^N \alpha_i \int_{E_i} f_i= \int f.
                        \]
                        \item Sea $f$ arbitraria no negativa y sean $\varphi_k$ funciones simples no negativas tales que $\varphi_k \nearrow f$. Luego, 
                        \[
                        \varphi_k(x+h)\to f(x+h), \;\;\forall x \in \rr^n.
                        \]
                        Por el Teorema de Beppo-Levi (Teorema \ref{teo:Beppo-Levi}), se tiene
                        \[
                        \int f(x+h)\,dx=\lim\limits_{k \to \infty} \int \varphi_k(x+h)\,dx=
                        \lim\limits_{k\to \infty} \int \varphi_k(x) =\int f(x)\,dx.
                        \]
                    \end{itemize}
                    As\'i queda demostrado el item \ref{it:invariancia-f-x+h}.
                    \item es consecuencia del item \ref{it:invariancia-f-x+h}. En efecto, 
                    \[\begin{split} 
                    \int_E f(x+h) \,dx=& \int \chi_{E+h}(x+h) f(x+h) \,dx
                    \\
                    =&
                    \int \chi_{E+h} (x) f(x)\,dx 
                    \\
                    =&
                    \int_{E+h} f(x)\,dx.
                    \end{split}\]
                    \end{enumerate}
                    Por \'ultimo, los item \ref{it:invariancia-f-x+h} y   \ref{it:invariancia-f-y-E-x+h} son ciertos para $f\in L(E)$,  a partir de la usual descomposici\'on de $f$ dada por $f=f^{+}-f^{-}$.
                    \end{demo}
                    
                    \section{La integral como funci\'on de conjunto}
                    
                    \textbf{FALTA DEFINIR $\mathcal{M}$ o RECORDARLO!!!} 
                    
                    Sea $f\in L(\rr)$ y definimos $\Phi: \mathcal{M} \to \rr$
                    por
                    \[
                    \Phi(E)=\int_E f.
                    \]
                    
                    $\Phi$ se llama integral indefinida de $f$.
                    
                    \begin{teorema}{}
                    Si $E_j \in \mathcal{M}$ son mutuamente disjuntos y sea
                    $E=\bigcup\limits_{j=1}^{\infty}E_j$, entonces
                    \[\Phi(E)=\sum\limits_{j=1}^{\infty} \Phi(E_j).\]
                    \end{teorema}
                    
                    \begin{demo}
                    Si $f\geq 0$, tenemos 
                    $\sum\limits_{j=1}^{\infty} \chi_{E_j}=\chi_E$ y luego
                    \[
                    \Phi(E)=\int_E f = \int f \chi_E =\int \sum\limits_j f \chi_{E_j}=\sum\limits_{j} \int f \chi_{E_j}=\sum\limits_j \Phi(E_j).
                    \]
                    Si $f \in L(E)$, trabajamos con  $f=f^{+}-f^{-}$ donde 
                    $f^+$ y $f^-$ son funciones medibles no negativas.
                    \end{demo}
                    
                    \begin{definicion}{}
                    Si $X$ es un conjunto y $\Sigma$ es una sigma-\'algebra de subconjuntos de $X$. 
                    Una funci\'on $\mu:\Sigma \to \rr^+$ se llama medida si
                    \begin{enumerate}
                        \item $\mu(\emptyset)=0,$
                        \item $\mu(\bigcup\limits_j E_j) = \sum\limits_j \mu(E_j)$, donde $E_j \in \Sigma$ y son mutuamente disjuntos.
                    \end{enumerate}
                    \end{definicion}
                    
                    
                    Para una medida valen los teoremas de convergencia mon\'otona de conjuntos, se puede definir una integral y \emph{casi todo} lo visto para la medida de Lebesgue es v\'alido. Una propiedad que no siempre es cierta es la invariancia por traslaciones.
                    
                    Ahora bien,  $\Phi$ es una medida. 
                    La pregunta que surge es ?`ser\'a toda medida sobre
                    $\mathcal{M}$ de la forma de $\Phi$ para alguna $f$?\\
                    La respuesta a esta pregunta ser\'a dada por el \textbf{Teorema de Radom-Nikodim}.
                    
                    
                    La siguiente propiedad se llama \emph{continuidad absoluta}.
                    
                    \begin{teorema}{}
                    Sea $f \in L(\rr^n)$, entonces $\forall \epsilon>0$ \;$\exists \delta >0$ tal que 
                    \[
                    m(E)<\delta \Longrightarrow |\Phi(E)|<\epsilon.
                    \]
                    \end{teorema}
                    
                    \begin{demo}
                    Se puede suponer que $f\geq 0$. \\
                    Sea $f_k=\min\{f,k\}$, entonces $f_k \nearrow f$.
                    Por el Teorema de Beppo-Levi (Teorema \ref{teo:Beppo-Levi}) se tiene que 
                    $\int_E f_k \nearrow \int_E f$ y por tanto
                    $
                    \int_E f-f_k \to 0.
                    $
                    As\'i, existe $k \in \nn$ tal que 
                    \[
                    \int_E f-f_k < \frac{\epsilon}{2}.
                    \]
                    Sea $\delta <\frac{\epsilon}{2k}$. Luego, si $m(E)<\delta$ entonces 
                    \[
                    \int_E f =\int_E f-f_k +\int_E f_k <\frac{\epsilon}{2}+km(E)<\epsilon. 
                    \]
                    \end{demo}
                    
                    \section{Comparaci\'on con la integral de Riemann}
                    
                    Sea $f$ acotada en $[a,b]\subset \rr$. 
                    Si $a=x_0<x_1<x_2<\ldots<x_N=b$, llamamos suma inferior de Riemann $s$ y suma superior de Riemann $S$ a 
                    \[
                    s=\sum\limits_{i=1}^N m_i (x_i-x_{i-1})  \;\;\mbox{ y }\;\;
                     S=\sum\limits_{i=1}^N M_i (x_i-x_{i-1}),
                    \]
                    respectivamente, donde 
                    \[
                    m_i=\inf\limits_{[x_{i-1},x_i]} f\;\;\mbox{ y }\;\;
                    M_i=\sup\limits_{[x_{i-1},x_i]} f.
                    \]
                    
                    Una funci\'on $f$ se llama integrable seg\'un Riemann si y s\'olo si
                    $\forall \epsilon>0$ existe una partici\'on para la cual 
                    \[
                    S-s<\epsilon.
                    \]
                    
                    
                    La integral de Riemann se define 
                    \[ \R\, \int_a^b f =\sup s = \inf S. \]
                    
                    \begin{teorema}{}
                    Si $f$ es integrable Riemann sobre $[a,b]$, entonces
                    $f$ es medible e integrable  Lebesgue. Adem\'as, 
                    \[
                    \R  \,\int_a^b f = \int_a^b f.
                    \]
                    \end{teorema}
                    
                    \begin{demo}
        Si $a=x_0<x_1<x_2<\ldots<x_N=b$, definimos las funciones escalonadas 
        \[
        \varphi(x)=\sum\limits_{i=1}^N m_i \chi_{J_i} \;\;\mbox{ y }\;\;
        \psi(x)=\sum\limits_{i=1}^N M_i \chi_{J_i} 
        \]
        donde 
        $J_i=[x_{i-1},x_i]$. Entonces $\varphi \leq f \leq psi$ en c.t.p. de $[a,b]$.
        
        Si $f$ es integrable Riemann, existen dos sucesiones de funciones escaleras $\varphi_k$ y $\psi_k$ tales que 
        $\varphi_k \leq f \leq \psi_k$ y 
        \[
        \int_a^b \psi_k -\varphi_k <\frac{1}{k}.
        \]
        Adem\'as, si $g=\sup\varphi_k$ y $h=\inf \psi_k$, entonces
        $g$ y $h$ son borelianas y $g\leq f\leq h$.
        Adem\'as
        \[
        \int_a^b \varphi_k \leq \int_a^b f \leq \int_a^b \psi_k
        \]
        y 
        \[
        \R\,\int_a^b \varphi_k \leq\; \R\, \int_a^b f \leq\; \R\, \int_a^b \psi_k,
        \]
        de donde
        \[
        \left|
        \int_a^b f -\; \R\,\int_a^b f
        \right|\leq 
        \int_a^b \psi_k - \varphi_k <\frac{1}{k}.
        \]
                            \end{demo}
                            
                            \section{Integraci\'on parcial: Teorema de Fubini}
                            
        Si $u \in \rr^{n+m}$, pondremos $u=(x,y)$
        con $x \in \rr^n$ e $y \in \rr^m$. 
        
        Si $E\subset \rr^{n+m}$ e $y\in \rr^n$, entonces
        \[
        E_x=\{y \in \rr^m: (x,y)\in E\},
        \]
        se llama la \emph{secci\'on} de $E$ en $x$.
                An\'alogamente, se define $E_y$.
        
        Se puede demostrar que 
        \[
        \left(\bigcup\limits_{k=1}^{\infty} E_k \right)_x
        = \bigcup\limits_{k=1}^{\infty} \left(E_k\right)_x\]
        y
        \[
        \left(\bigcap\limits_{k=1}^{\infty} E_k \right)_x
        = \bigcap\limits_{k=1}^{\infty} \left(E_k\right)_x,\]
        para cualquier sucesi\'on de conjuntos $E_k$ contenidos en $\rr^{n+m}$. 
        
        Si $E_1\subset E_2$ entonces 
        \[
        \left(E_1\right)_x  \subset   \left(E_2\right)_x   
       \; \mbox{ y }\;
        \left(E_1-E_2\right)_x=\left(E_1\right)_x -\left(E_2\right)_x.
        \]
        
        \begin{teorema}{}[Principio de Cavalieri]
        Sea $E$ medible en $\rr^{n+m}$, entonces
        \begin{enumerate}
            \item $E_x$ es medible de $\rr^{n+m}$ en c.t.p. $x \in \rr^n$;
            \item $m\left(E_x\right)$ es medible como funci\'on de $x$;
            \item $m(E)=\int m\left(E_x\right)\, dx$.
        \end{enumerate}
        \end{teorema}
        
        \begin{demo}
        \emph{Primer Paso)} Si $E$ es un intervalo de $\rr^{n+m}$, supongamos que $E=I\times J$ con $I$ intervalo de $\rr^n$, $J$ intervalo de $\rr^m$, entonces $E_x=J$ $\forall x \in I$ y $E_x=\emptyset$ si $x \notin I$.
        As\'i, $E_x$ es conjunto medible $\forall x \in I$, $m(E_x)=\chi_I m(J)$ es funci\'on medible  en $x$ y 
        \[
        \int m(E_x)\,dx=m(J)m(I)=m(E).
        \]
        \emph{Segundo Paso)} Si $E$ es abierto, entonces $E_x=\bigcup\limits_{k=1}^{\infty} (I_k)_x$ con $I_k$ intervalos mutuamente disjuntos y donde $E_x=\bigcup\limits_{k=1}^{\infty} \left(I_k\right)_x$. 
        As\'i $E_x$ es conjunto medible. Adem\'as, 
        \[
        m(E_x)=\sum\limits_{k=1}^{\infty} m\left((I_k)_x\right)
        \]
        es una funci\'on medible de $x$ y 
        \[
        \int m(E_x)=\sum\limits_{k=1}^{\infty} \int m\left((I_k)_x\right)\,dx=
        \sum\limits_{k=1}^{\infty} m(I_k)=m(E).
        \]
        \emph{Tercer Paso)} Si $E$ es un conjunto acotado y de tipo $G_{\delta}$, entonces existe una bola $B$ y una sucesi\'on de conjuntos abiertos $G_k$ tales que 
        \[
        E\subset B\;\mbox{  y   }\;E=\bigcap\limits_{k=1}^{\infty}G_k.
        \]
        Tomando $G_k^{'}=B\cap G_1\cap \ldots \cap G_k$, podemos suponer
        \[B\supset G_1 \supset \ldots \supset E.\]
        Ahora, se tiene que 
        \[
        E_x=\bigcap\limits_{k=1}^{\infty} \left(G_k\right)_x
        \]
        es conjunto medible y 
        \[
        m(E_x)=\lim\limits_{k \to \infty} m\left((G_k)_x\right)
        \]
        es funci\'on medible de $x$. Adem\'as, 
        \[
        m\left((G_k)_x\right)\leq m(B_x)\in L(\rr^n).
        \]
        A continuaci\'on, por aplicaci\'on de Convergencia Mayorada (Corolario \ref{cor:conv-mayorada}), se obtiene 
        \[
        \int m(E_x)\,dx =\lim\limits_{k \to \infty} \int m\left((G_k)_x\right)\,dx =\lim\limits_{k \to \infty} m(G_k)=m(E).
        \]
        \emph{Cuarto Paso)}
        Supongamos que  $E$ es un conjunto de tipo $G_{\delta}$. 
        Sean $B_k=B(0,k)$ y $E_k=E\cap B_k \in G_{\delta}$, entonces $E=\bigcup\limits_{k=1}^{\infty} E_k$ y $E_1\subset E_2\subset \ldots$ Luego
        \[
        E_x=\bigcup\limits_{k=1}^{\infty} \left(E_k\right)_x
        \;\mbox{ y }\;
        (E_1)_x \subset (E_2)_x \subset \ldots
        \]
        De este modo, resulta que $E_x$ es conjunto medible y 
        \[m(E_x) = \lim\limits_{k \to \infty} m((E_k)_x).\]
                Como $m((E_k))_x$ es una sucesi\'on mon\'otona creciente de funciones medibles, por el Teorema de Beppo-Levi (Teorema \ref{teo:Beppo-Levi}) llegamos a 
     \[\int m(E_x)\,dx=\lim\limits_{k \to \infty} \int m\left(\left(E_k\right)_x\right)\,dx=
     \lim\limits_{k \to \infty} m(E_k)=m(E).
                    \]
    \emph{Quinto Paso)} Sea $E$ un conjunto de medida nula. Luego, existe $H\in G_{\delta}$ tal que $E\subset H$ y $m(H)=0$.
    A partir de 
    \[
    \int m(H_x)\,dx=m(H)=0,
    \]
    se tiene que $0\leq m(E_x)\leq m(H_x)=0$ en c.t.p. $x$. Luego, $m(E_x)=0$
    en c.t.p. $x$ y por lo tanto es funci\'on medible en $x$.
    Adem\'as, 
    \[
    \int m(E_x)\,dx \leq \int m(H_x)\,dx=0=m(E).\]

 \emph{Sexto Paso)} Sea $E$ medible. Entonces existen $H\in G_{\delta}$ y $Z$ de medida nula tal que $E=H-Z$. 
 Luego
 \[E_x=H_x -Z_x\]
 y $m(Z_x)=0$ en c.t.p. $x$, de donde $E_x$ es un conjunto medible en c.t.p. $X$ y 
 \[
 m(E_x)=m(H_x)-m(Z_x)=m(H_x)\;\mbox{  en c.t.p. } x.
 \]
 Es as\'i que, $m(E_x)$ es funci\'on medible siempre que $E_x$ sea medible y 
 definimos $m(E_x)=0$ para el caso en que $E_x$ no sea medible. Por \'ultimo,  \[
 \int m(E_x)\,dx=\int m(H_x)\,dx=m(H)=m(E).
 \]
\end{demo}

\begin{ejemplo}{}
\begin{enumerate}
    \item 
Sea $H$ un hiperplano de ecuaci\'on $a_1x_1+\ldots+a_nx_n=a$. 
\\Veamos por inducci\'on que $m(H)=0$.

El caso $n=1$ es trivial. 
Supongamos que $(a_2,a_3,\ldots,a_n)\neq 0$, luego 
\[H_{x_1}=\{(x_2,\ldots,x_n)| a_2x_2+\ldots+a_nx_n=a-a_1x_1\}\]
y 
\[
m(H)=\int m(H_{x_1})\,dx_1=0.
\]
\item El simple $S$ de altura $a$ es $x_1+x_2+\ldots+x_n\leq a$ con $x_i\geq 0$. 

Veamos por inducci\'on que $m(S)=\frac{a^n}{n!}$. 

El caso $n=1$ es trivial. Para  $n>1$, se tiene que $S_{x_1}=\emptyset$ si $x_1\notin [0,a]$ y si $x_1 \in [0,a]$ entonces 
$S_{x_1}=\{(x_2,\ldots,x_n)|x_2+\dots+x_n\leq a-x_1 \}$ es el simple de altura $a-x_1$. Luego, 
\[
m(S)=\int_0^a m(S_{x_1})\,dx_1= \int_0^a \frac{(a-x_1)^{n-1}}{(n-1)!}\,dx_1=\frac{a^n}{n!}.
\]     
\end{enumerate}   
\end{ejemplo}
   
   
   \begin{teorema}{teo:Fubini-Tonelli}[Fubini-Tonelli]
   Si $f(u)=f(x,y)$ es medible no negativa sobre $\rr^{n+m}$, entonces
   \begin{enumerate}
       \item $f(x,y)$ es medible en $y$ para c.t.p. $x$;
       \item $g(x)=\int_{\rr^n} f(x,y)\,dy$ es medible sobre $\rr^n$;
       \item 
       \[
       \int g(x)\,dx=\int dx \int f(x,y)\,dy=\int f(u)\,du.
       \]
   \end{enumerate}
      \end{teorema}     
      
      \begin{demo}
      \emph{Paso 1)}
      Si $f=\chi_E$, vale  \[ \chi_E(x,y)=\chi_{E_x} (y).\]
      $E_x$ es medible en c.t.p. $x$. As\'i, $f$ es medible en $y$ para c.t.p. $x$. Adem\'as
      \[
       \int\,dx \int \chi_{E_x}(y)\,dy =
       \int m(E_x)\,dx =m(E)=\int f(u)\,du.
      \]
      
       \emph{Paso 2)}
       Si $f$ es simple, entonces 
       \[f(x,y)=\sum\limits_{k=1}^N \alpha_k f_k(x,y),\]
       donde $\alpha_k\geq 0$ y $f_k$ son funciones caracter\'isticas 
       $\chi_{E_k}$.
       
       Por el \emph{Paso 1)}, $f_k(x,y)$ es medible en $y$ y $x$ est\'a en un conjunto de la forma $\rr^n-Z_k$, donde $m(Z_k)=0$. 
       Ahora, llamando $Z=\bigcup\limits_{k=1}^{N} Z_k$, $f$ es medible en $y$ siempre que $x \in \rr^n-Z$. O sea, $f$ resulta medible en $y$ para c.t.p. $x \in \rr^n$.

       Si $x \in \rr^n-Z$, la funci\'on 
       \[
       g(x)=\int f(x,y)\,dy=\sum\limits_{k=1}^N \alpha_k \int f_k(x,y)\,dy
       \]
        es medible.  Adem\'as, por el \emph{Paso 1)}, 
        \[
        \int g(x)\,dx=
        \sum\limits_{k=1}^N \alpha_k \int\, dx \int f_k(x,y)\,dy
        =
        \sum\limits_{k=1}^N \alpha_k \int f_k(u)\,du=\int f(u)\,du.
        \]
        
        \emph{Paso 3)}
        Si $f$ es medible no negativa, existe una sucesi\'on de funciones simples $f_k:\rr^{n+m}\to \rr$ tales que $0\leq f_1\leq f_2\leq \ldots$ y $f_k(u)\nearrow f(u)$ $\forall u\in \rr^{n+m}$.
        Dado que $f_k$ es simple, existe $Z_k$ tal que $f_k(x,y)$ es medible en $y$  si $x \notin Z_k$. 
        \\
        Sea $Z=\bigcup\limits_{k=1}^N Z_k$, entonces $m(Z)=0$ y cada $f_k(x,y)$ es medible en $y$ si $x \notin Z$. As\'i, $f$ es medible  en $y$ $\forall x \notin Z$. 
        Aplicando el Teorema de Beppo-Levi (Teorema \ref{teo:Beppo-Levi}), si $x \notin Z$  se tiene que
        \[
        g(x)=\int f(x,y)\,dy=\lim\limits_{k\to \infty} \int f_k(x,y)\,dy,
        \]      
        y $g$ es medible tomando la precauci\'on de definir $g=0$ en $Z$.
        \\
        Por \'ultimo, aplicando nuevamente  el Teorema de Beppo-Levi (Teorema \ref{teo:Beppo-Levi}), obtenemos
        \[
        \int g(x)\,dx=\lim\limits_{k \to \infty} \int\,dx \int f_k(x,y)\,dy
        =\lim\limits_{k \to \infty} \int f_k(u)\,du
        =\int f(u)\,du.
        \]
        \end{demo}
        
        \begin{teorema}{}[Fubini]
        Si $f(u)=f(x,y) \in L(\rr^{n+m})$, entonces
        \begin{enumerate}
            \item para casi todo $x$, $f(x,y)$ es integrable en $y$;
            \item la funci\'on $g(x)=\int f(x,y)\,dy$ es integrable en $x$;
            \item \[
            \int g(x)\,dx=\int \,dx \int f(x,y)\,dy=\int f(u)\,du
            \]
        \end{enumerate}
                \end{teorema}
               
                
                \begin{demo}
                 Por aplicaci\'on del Teorema de Fubini-Tonelli (Teorema \ref{teo:Fubini-Tonelli}) se tiene que 
                 \[
                 \int\,dx \int f^{+}(x,y)\,dy=\int f^{+}(u)\,du<\infty
                 \]
                 y 
                 \[
                  \int\,dx \int f^{-}(x,y)\,dy=\int f^{-}(u)\,du<\infty.
                 \]
                 Las funciones no negativas
\[
g_1(x)=\int f^{+}(x,y)\,dy\; \mbox{ y }\; 
g_2(x)=\int f^{-}(x,y)\,dy
\]
son integrables y por ende finitas en casi todo punto. 
Luego, 
\[g(x)=\int f(x,y)\,dy=g_1(x)-g_2(x)
\]
es integrable en $y$ para casi todo $x$. Adem\'as, $g \in L(\rr^n)$ y 
\[
\begin{split}
\int g(x)\,dx&=\int g_1(x)\,dx - \int g_2(x)\,dx
\\&=\int f^{+}(u)\,du - \int f^{-}(u)\,du=\int f(u)\,du.
\end{split}
\]
                \end{demo}
                
                \begin{corolario}{}
Si $f$ satisface que 
\[
\int\,dx \int |f(x,y)|\,dy<\infty\; \mbox{ \'o }\;
\int\,dy \int |f(x,y)|\,dx<\infty,
\]
vale el Teorema de Fubini.
\end{corolario}
        
        
Si $f$ es integrable sobre un conjunto $E\subset \rr^{n+m}$, la integral de $f$ sobre $E$ se calcula por medio de la f\'ormula
\[
\int_E f(u)\,du=\int\,dx \int_{E_x} f(x,y)\,dy.
\]

En efecto, $f\chi_E$ es integrable sobre todo el espacio $\rr^{n+m}$ y por consiguiente
\[
\begin{split}
\int_E f(u)\,du &=\int\int_E f(x,y)\,dx\,dy
\\
&=\int\int \chi_E(x,y)f(x,y)\,dx\,dy
\\
&=\int\,dx\int \chi_{E_x(y)}f(x,y)\,dy
\\
&=\int\,dx\int_{E_x} f(x,y)\,dy.
\end{split}
\]
