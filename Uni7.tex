\chapter{Funciones medibles
}

\section{Introducci\'on???}

\textbf{ Copiado de acuerdo al orden de las notas manuscritas...parece que falta algo para el contexto...}

\textbf{Habr\'ia que definir $\mathcal{M}$ o referenciar la definici\'on del cap\'itulo  de conjuntos medibles...si existiese...ese  capitulo no esta completo}

\begin{definicion}{}
$f$ es medible si y s\'olo si $f^{-1}((a,\infty)) \in \mathcal{M}$. 
\end{definicion}

\begin{teorema}{}
Si $H$ es boreliano y $f$ es medible, entonces $f^{-1}(H)$ es medible.
\end{teorema}

\begin{demo}{}
Sea $\mathcal{M}^{'}=\{ H: f^{-1}(H)\;\mbox{ es medible}\}$.
$\mathcal{M}^{'}$ es $\sigma$-\'algebra y $\mathscr{I}\subset \mathcal{M}^{'}$.
Por lo tanto, $\mathscr{B}(\rr^n) \subset \mathcal{M}$.
\end{demo}

\begin{definicion}{}
Diremos que $H \subset \overline{R}$ es boreliano de la recta extendida si $H-\{ -\infty, \infty\} \in \mathscr{B}(\rr)$.
\end{definicion}

\begin{teorema}{}
Si $f:\rr^n \to \overline{\rr}$ es medible si y s\'olo si $f^{-1}(H) \in \mathcal{M}$ cada vez que $H$ es boreliano de $\overline{\rr}$.
\end{teorema}

\begin{demo}
$\Leftarrow)$ Inmediata a partir de la definici\'on de funci\'on medible.

$\Rightarrow)$ Se tiene que 
\[
\{f=+\infty\}=\bigcap\limits_{k=1}^{\infty} \{f \geq k\}
\;\;\mbox{ y }\;\;
\{f=-\infty\}=\bigcap\limits_{k=1}^{\infty} \{f \leq -k\}.
\]
Si $H$ es boreliano de $\rr$, supongamos que $H=H^{'}\cup \{+\infty\}$ y $H^{'}$ es conjunto medible de $\rr$.
Luego, $f^{-1}(H)=f^{-1}(H^{'})\cup \{f=+\infty\}$ es medible.
\end{demo}


\section{Funciones medibles sobre una $\sigma$-\'algebra}

\begin{definicion}{}
Si $\Sigma$ es una $\sigma$-\'algebra de $\rr^n$, diremos que $f:\rr^n\to \overline{\rr}$ es $\sigma$-medible si 
\[\{f>a\} \in \Sigma\;\;\;\;\forall a \in \rr.\]
\end{definicion}

\begin{proposicion}{}
$f:\rr^n \to \overline{\rr}$ es $\Sigma$-medible si y s\'olo si
$f^{-1}(H)\in \Sigma$ cuando $H$ es boreliano de $\overline{\rr}$.
\end{proposicion}

Cuando $\Sigma=\mathcal{M}$, decimos que $f$ es \emph{medible}.

Cuando $\Sigma=\mathscr{B}$, llamamos a $f$ \emph{medible Borel} 
o funci\'on \emph{boreliana}.

\begin{ejercicio}{}
Si $f$ es semicontinua inferiormente, entonces $f$ es medible. 
\end{ejercicio}

Ahora, estudiaremos $g\circ f$ cuando 
$\rr^n \xrightarrow{f}\overline{\rr} \xrightarrow {g} \overline{\rr}$.

\begin{definicion}{}
Diremos que $g: \rr^n \to \overline{\rr}$ es boreliana si $g^{-1}(M)$ es boreliano de $\overline{\rr}$ cuando $M$ lo es.
\end{definicion}

Luego, si $f:\rr^n \to \overline{\rr}$ es medible y $g:\overline{\rr} \to \overline{\rr}$ es boreliana, entonces $g\circ f$ es medible.

Tambi\'en, se tiene que si $f$ es medible entonces $|f|$, $|f|^2$, $\log|f|$ y $e^f$ son medibles. 

Si $f$ y $g$ son medibles, entonces $\{f<g\}$ es medible pues
\[
\{f<g\}=\bigcup\limits_{q \in \qq} \{ f<q\} \cap \{ g>q\}.
\]
\begin{teorema}{teo:propiedades-func-medibles}
Si $f$ y $g$ son medibles con respecto a $\Sigma$ y si $c\in \rr$, entonces
$f+g$, $cf$ y $fg$ son medibles.
\end{teorema}

\begin{demo}
Supondremos que todas las funciones son finitas.

Veamos que 
\[
\{
f+g>a\}=\bigcup\limits_{r \in \qq} \{ a>r\} \cap \{q>a-r\}.
\}
\]
Si $f(x)+g(x)>a$ entonces existe $r \in \qq$ tal que 
\[
f(x)>r>a-g(x),\;\;\mbox{ es decir }\;\; 
f(x)>r\;\mbox{ y }\; g(x)>a-r.
\]
Rec\'iprocamente, si para alg\'un $r\in \qq$ se tiene que $f(x)>r$ y $g(x)>a-r$, entonces $f(x)+g(x)>a$.

Si $c\in \rr$, entonces
\[
\{cf>a\}=
\left\{
\begin{array}{ll}
f>\frac{a}{c}&c>0
\medskip
\\
f<\frac{a}{c}&c<0
\medskip
\\
\emptyset\;\mbox{ \'o }\;\rr^n &c=0.
\end{array}
\right.
\]

Ahora, como 
\[
fg=\frac{1}{4}[(f+g)^2-(f-g)^2]
\]
luego $fg$ es medible con respecto a $\Sigma$. 

Para estudiar el cociente entre funciones medibles, denotamos por $\frac{1}{f}$
la funci\'on que toma el valor $\frac{1}{f(x)}$ si $f(x)\neq 0$ y el valor $0$ si $f(x)=0$.

Ahora, como 
\[
\left\{\frac{1}{f}>a\right\}
=
\left \{
\begin{array}{ll}
 \{f>0\} \cap \{f<\frac{1}{a}\}    &  a>0 
 \medskip
 \\
 \{ f>0\} \cup  (\{f<\frac{1}{a}\} \cap \{f<0\}) \cup \{f=0\}
     & a<0, 
\end{array}
\right.
\]
a partir de que $f$ es medible se tiene que  $1/f$ es medible. 
\end{demo}

\section{Sucesiones de funciones medibles}

\begin{proposicion}{prop:inf-y-sup-medibles}
Si $\{f_k\}$ es una sucesi\'on de funciones medibles con respecto a $\Sigma$, 
entonces
\[
g(x)=\inf\limits_{k} f_k(x) \;\;\mbox{ y }\;\; h(x)=\sup\limits_{k} f_k(x) 
\]
son $\Sigma$-medibles. 
\end{proposicion}


La demostraci\'on se deduce de las f\'ormulas
\[
\{h>a \}=\bigcup_{k=1}^{\infty} \{f_k>a\}
\;\;\mbox{ y }\;\;
\{g<a \}=\bigcup_{k=1}^{\infty} \{f_k<a\}.
\]

\begin{proposicion}{prop:liminf-y-limsup-medibles}
Si $\{f_k\}$ es una sucesi\'on de funciones $\Sigma$-medibles, entonces
\[
g(x)=\liminf\limits_{k \to \infty} f_k(x)
\;\;\mbox{ y }
\;\;
h(x)=\limsup\limits_{k \to \infty} f_k(x)
\]
son   medibles con respecto a $\Sigma$.
\end{proposicion}


La prueba resulta de aplicar la Proposici\'on \ref{prop:inf-y-sup-medibles}
a las relaciones 
\[
g(x)=\sup\limits_{j} \inf\limits_{k\geq j} f_k(x)
\;\;\mbox{ y }\;\;
h(x)=\inf\limits_{j}\sup\limits_{k \geq j} f_k(x).
\]

\begin{corolario}{}
Si $f_k(x) \to f(x)$ y $\{f_k\}$ es una sucesi\'on de funciones $\Sigma$-medibles, entonces $f$ es medible con respecto a $\Sigma$.
\end{corolario}

A continuaci\'on, completamos la demostraci\'on del Teorema\ref{teo:propiedades-func-medibles} que se reliz\'o suponiendo
que tanto $f$ como $g$ son finitas. 
Para evitar esa restricci\'on, ahora consideramos la sucesi\'on de funciones $\varphi_k:\overline{\rr} \to \overline{\rr}$ definidas por
\[
\varphi_k(t)=
\left\{
\begin{array}{rl}
   t  &  \mbox{si } |t|\leq k \\
   k  &  \mbox{si } t>k\\
   -k &  \mbox{si } t<-k.
\end{array}
\right.
\]
Cada $\varphi_k$ es una funci\'on boreliana pues la restricci\'on de $\varphi$ a $\rr$ es continua. Adem\'as, para cada $t\in \overline{\rr}$, 
se tiene que $\varphi_k \to t$ cuando $k \to \infty$. Entonces, las funciones
\[
f_k=\varphi_k \circ f \;\;\mbox{ y }
\;\;
g_k=\varphi_k \circ g,
\]
son medibles con respecto a $\Sigma$, finitas y convergen puntualmente a $f$ y $g$ respectivamente, cuando $k  \to \infty$. Luego, las funciones 
\[
f+g=\lim\limits_{k \to \infty} (f_k +g_k)
\;\;\mbox{ y }\;\;
fg=\lim\limits_{k \to \infty} f_kg_k,
\]
resultan medibles a partir de la aplicaci\'on de la Proposici\'on \ref{prop:liminf-y-limsup-medibles}.

\section{Funciones simples}
Definimos la funci\'on caracter\'istica  $\chi_E$ de un conjunto $E \subset \rr^n$ mediante
\[
\chi_E(x)=
\left\{
\begin{array}{ll}
   1  & \mbox{si } x \in E \\
   0  & \mbox{si } x \notin E.
\end{array}
\right.
\]

Se tiene la siguiente propiedad
\begin{itemize}
    \item $\chi_E$ es medible  si y s\'olo si $E$ es medible.
\end{itemize}


\begin{definicion}{defi:funcion-simple}
Una funci\'on medible y finita  $\varphi:\rr^n \to \rr$ 
se llama \emph{simple} si el conjunto de todos sus valores es finito, es decir, si $\varphi$ es medible y la imagen  $\varphi(\rr^n)$ es un subconjunto finito de $\rr$.
\end{definicion}

A partir de la Definici\'on \ref{defi:funcion-simple}, se tiene que si $\varphi,\psi$ son funciones simples y $c \in \rr$, entonces $\varphi + \psi$, $c\varphi$,  $\varphi \psi$ son simples.

Si $\varphi(\rr^n)=\{\alpha_1,\ldots,\alpha_N\}$, entonces $E_i=\varphi^{-1}(\{\alpha_i\})$ son medibles y 
\[
\varphi=\sum\limits_{i=1}^N \alpha_i \chi_{E_i}.
\]
Las funciones simples desempe\~nan un papel muy importante en la teor\'ia de integraci\'on en virtud del siguiente teorema.

\begin{teorema}{teo:simples-convergen-a-medible}
Si $f:\rr^n \to \overline{\rr}$ es una funci\'on medible no negativa, entonces existe una sucesi\'on $\{\varphi_k\}$ de funciones simples tal que
\[
\varphi_1\leq \varphi_2 \leq \varphi_3\leq \ldots \;\;\mbox{ y }\;\;
f(x)=\lim\limits_{k \to \infty} \varphi_k(x)
\]
en cada punto $x \in \rr^n$.
\end{teorema}

\begin{demo}
Para $k \in \nn$, dividimos $[0,2^k)$ en $k2^k$ intervalos disjuntos
\[
\left[\frac{i-1}{2^k},\frac{i}{2^k}\right), \;\;i=1,2,\ldots,k2^k.
\]
Definimos $g_k:\overline{\rr} \to \overline{\rr}$ por
\[
g_k(x)=
\left\{
\begin{array}{ll}
    \frac{i-1}{2^k} & \mbox{si }\; 0\leq t\leq k, \;\; (i-1)/2^k\leq t< i/2^k, 
    \medskip
    \\
    k     & \mbox{si }\;t\geq k
    \medskip
    \\
    0     & \mbox{si }\; t<0.
\end{array}
\right.
\]
Las funciones $g_k$ son borelianas, no negativas y verifican
\[
0\leq g_1\leq g_2\leq \ldots,\;\mbox{ y }\;\;
\lim\limits_{k \to \infty} g_k(t)=t, \;\;\mbox{ en } [0, +\infty].
\]
Las funciones $\varphi_k = g_k \circ f$ son simples y verifican  el teorema.
\end{demo}

\begin{observacion}{}
\begin{enumerate}
    \item Si $f$ es medible con respecto a una $\sigma$-\'algebra $\Sigma$, entonces las funciones $\varphi_k$ del Teorema \ref{teo:simples-convergen-a-medible} son tambi\'en medibles con respecto a $\Sigma$.
    \item Multiplicando a las $\varphi_k$ por $\chi_{B(0,k)}$ se obtiene una sucesi\'on de  funciones $\psi_k$ que verifican las hip\'otesis del Teorema \ref{teo:simples-convergen-a-medible} y que tienen soporte compacto.
    \item Si $f$ es acotada y positiva,  la convergencia es uniforme.
\end{enumerate}
\end{observacion}

\section{Partes positiva y negativa}

Si $f$ es medible, tambi\'en lo son 
\[
f^+=\sup\{0,f \}\;\;\mbox{ y }\;\;f^{-}=\sup\{0,-f\}
\]
llamadas partes positiva y negativa de $f$.

Se verifica que 
\[
f=f^{+}-f^{-}\;\;\mbox{ y }\;\; |f|=f^{+}+f^{-}.
\]

\begin{teorema}{}
Si $f$ es medible y $f=f_1-f_2$, con $f_i\geq 0$ para $i=1,2$, entonces
$f^{+}\leq f_1$ y $f^{-}\leq f_2$.
\end{teorema}

\begin{demo}
Se tiene que $f\leq f_1$ de donde $f^+=\sup\{ f,0\}\leq f_1$. 

Adem\'as, $-f \leq f_2$. Luego, $f^{-}\leq f_2$
\end{demo}

Si $f$ es medible,  aplicando el Teorema \ref{teo:simples-convergen-a-medible} a $f^+$ y $f^{-}$, 
existen funciones simples $\varphi_k,\psi_k$ tales que $\varphi_k \to f^+$
y $\psi_k \to f^{-}$. Luego, 
$\varphi_k - \psi_k \to f$, y adem\'as, $|\varphi_k-\psi_k|\leq \varphi_k+\psi_k \leq f^{+}+f^{-}=|f|$.

\section{Propiedades verdaderas en casi todo punto}