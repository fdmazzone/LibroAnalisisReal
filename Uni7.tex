\chapter{Funciones medibles
}

\section{Introducci\'on???}

\textbf{ Copiado de acuerdo al orden de las notas manuscritas...parece que falta algo para el contexto...o reordenarlo...}

\textbf{Habr\'ia que definir $\mathcal{M}$ o referenciar la definici\'on del cap\'itulo  de conjuntos medibles...si existiese!!!}

\begin{definicion}{}
$f$ es medible si y s\'olo si $f^{-1}((a,\infty)) \in \mathcal{M}$. 
\end{definicion}

\begin{teorema}{}
Si $H$ es boreliano y $f$ es medible, entonces $f^{-1}(H)$ es medible.
\end{teorema}

\begin{demo}{}
Sea $\mathcal{M}^{'}=\{ H: f^{-1}(H)\;\mbox{ es medible}\}$.
$\mathcal{M}^{'}$ es $\sigma$-\'algebra y $\mathscr{I}\subset \mathcal{M}^{'}$.
Por lo tanto, $\mathscr{B}(\rr^n) \subset \mathcal{M}$.
\end{demo}

\begin{definicion}{}
Diremos que $H \subset \overline{R}$ es boreliano de la recta extendida si $H-\{ -\infty, \infty\} \in \mathscr{B}(\rr)$.
\end{definicion}

\begin{teorema}{}
Si $f:\rr^n \to \overline{\rr}$ es medible si y s\'olo si $f^{-1}(H) \in \mathcal{M}$ cada vez que $H$ es boreliano de $\overline{\rr}$.
\end{teorema}

\begin{demo}
$\Leftarrow)$ Inmediata a partir de la definici\'on.

$\Rightarrow)$ Se tiene que 
\[
\{f=+\infty\}=\bigcap\limits_{k=1}^{\infty} \{f \geq k\}
\;\;\mbox{ y }\;\;
\{f=-\infty\}=\bigcap\limits_{k=1}^{\infty} \{f \leq -k\}.
\]
Si $H$ es boreliano de $\rr$, supongamos que $H=H^{'}\cup \{+\infty\}$ y $H^{'}$ es conjunto medible de $\rr$.
Luego, $f^{-1}(H)=f^{-1}(H^{'})\cup \{f=+\infty\}$ es medible.
\end{demo}


\section{Funciones medibles sobre una $\sigma$-\'algebra}

\begin{definicion}{}
Si $\Sigma$ es una $\sigma$-\'algebra de $\rr^n$, diremos que $f:\rr^n\to \overline{\rr}$ es $\sigma$-medible si 
\[\{f>a\} \in \Sigma\;\;\;\;\forall a \in \rr.\]
\end{definicion}

\begin{proposicion}{}
$f:\rr^n \to \overline{\rr}$ es $\Sigma$-medible si y s\'olo si
$f^{-1}(H)\in \Sigma$ cuando $H$ es boreliano de $\overline{\rr}$.
\end{proposicion}

Cuando $\Sigma=\mahtcal{M}$, decimos que $f$ es \emph{medible}.

Cuando $\Sigma=\mahtscr{B}$, llamamos a $f$ \emph{medible Borel} o funci\'on \emph{boreliana}.

\begin{ejercicio}{}
Si $f$ es semicontinua inferiormente, entonces $f$ es medible. 
\end{ejercicio}

Ahora, estudiaremos $g\circ f$ cuando 
$\rr^n \xrightarrow{f}\overline{\rr} \xrightarrow {g} \overline{\rr}$.

\begin{definicion}{}
Diremos que $g: \rr^n \to \overline{\rr}$ es boreliana si $g^{-1}(M)$ es boreliano de $\overline{\rr}$ cuando $M$ lo es.
\end{definicion}

Luego, si $f:\rr^n \to \overline{\rr}$ es medible y $g:\overline{\rr} \to \overline{\rr}$ es boreliana, entonces $g\circ f$ es medible.

Tambi\'en, se tiene que si $f$ es medible entonces $|f|$, $|f|^2$, $\log|f|$ y $e^f$ son medibles. 

Si $f$ y $g$ son medibles, entonces $\{f<g\}$ es medible pues
\[
\{f<g\}=\bigcup\limits_{q \in \qq} \{ f<q\} \cap \{ g>q\}.
\]
\begin{proposicion}{}
Si $f$ y $g$ son medibles con respecto a $\Sigma$ y si $c\in \rr$, entonces
$f+g$, $cf$ y $fg$ son medibles.
\end{proposicion}

\begin{demo}{}
Supondremos que todas las funciones son finitas.

Veamos que 
\[
\{
f+g>a\}=\bigcup\limits_{r \in \qq} \{ a>r\} \cap \{q>a-r\}.
\}
\]
Si $f(x)+g(x)>a$ entonces existe $r \in \qq$ tal que 
\[
f(x)>r>a-g(x),\;\;\mbox{ es decir }\;\; 
f(x)>r\;\mbox{ y }\; g(x)>a-r.
\]
Rec\'iprocamente, si para alg\'un $r\in \qq$ se tiene que $f(x)>r$ y $g(x)>a-r$, entonces $f(x)+g(x)>a$.

Si $c\in \rr$, entonces
\[
\{cf>a\}=
\left\{
\begin{array}{ll}
f>\frac{a}{c}&c>0
\\
f<\frac{a}{c}&c<0
\\
\emptyset\;\'o\;\rr^n &c=0
\end{array}
\right.
\]

Ahora, como 
\[
fg=\frac{1}{4}[(f+g)^2-(f-g)^2]
\]
luego $fg$ es medible con respecto a $\Sigma$. 

Para estudiar el cociente entre funciones medibles, denotamos por $1/f$
la funci\'on que toma el valor $\frac{1}{f(x}$ si $f(x)\neq 0$ y el valor $0$ si $f(x)=0$.

Ahora, como 
\[
\left\{\frac{1}{f}>a\right\}
=
\left \{
\begin{array}{ll}
 \{f>0\} \cap \{f<\frac{1}{a}\}    &  a>0 
 \\
 \{ f>0\} \cup  (\{f<\frac{1}{a}\} \cap \{f<0\}) \cup \{f=0\}
     & a<0, 
\end{array}
\right.
\]
a partir de que $f$ es medible se tiene que  $1/f$ es medible. 

\section{Sucesiones de funciones medibles}

\end{demo}


