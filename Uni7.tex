\chapter{Funciones medibles
}

\section{Introducci\'on???}

\textbf{ Copiado de acuerdo al orden de las notas manuscritas...parece que falta algo para el contexto...}

\textbf{Controlar que se defini\'o  $\mathscr{M}$ o referenciar la definici\'on del cap\'itulo  de conjuntos medibles...si existiese...}

\begin{definicion}{}
$f$ es medible si y s\'olo si $f^{-1}((a,\infty)) \in \mathscr{M}$. 
\end{definicion}

\begin{teorema}{}
Si $H$ es boreliano y $f$ es medible, entonces $f^{-1}(H)$ es medible.
\end{teorema}

\begin{demo}{}
Sea $\mathscr{M}^{'}=\{ H: f^{-1}(H)\;\mbox{ es medible}\}$.
$\mathscr{M}^{'}$ es $\sigma$-\'algebra y $\mathscr{I}\subset \mathscr{M}^{'}$.
Por lo tanto, $\mathscr{B}(\rr^n) \subset \mathscr{M}$.
\end{demo}

\begin{definicion}{}
Diremos que $H \subset \overline{R}$ es boreliano de la recta extendida si $H-\{ -\infty, \infty\} \in \mathscr{B}(\rr)$.
\end{definicion}

\begin{teorema}{}
Si $f:\rr^n \to \overline{\rr}$ es medible si y s\'olo si $f^{-1}(H) \in \mathscr{M}$ cada vez que $H$ es boreliano de $\overline{\rr}$.
\end{teorema}

\begin{demo}
$\Leftarrow)$ Inmediata a partir de la definici\'on de funci\'on medible.

$\Rightarrow)$ Se tiene que 
\[
\{f=+\infty\}=\bigcap\limits_{k=1}^{\infty} \{f \geq k\}
\;\;\mbox{ y }\;\;
\{f=-\infty\}=\bigcap\limits_{k=1}^{\infty} \{f \leq -k\}.
\]
Si $H$ es boreliano de $\rr$, supongamos que $H=H^{'}\cup \{+\infty\}$ y $H^{'}$ es conjunto medible de $\rr$.
Luego, $f^{-1}(H)=f^{-1}(H^{'})\cup \{f=+\infty\}$ es medible.
\end{demo}


\section{Funciones medibles sobre una $\sigma$-\'algebra}

\begin{definicion}{}
Si $\Sigma$ es una $\sigma$-\'algebra de $\rr^n$, diremos que $f:\rr^n\to \overline{\rr}$ es $\sigma$-medible si 
\[\{f>a\} \in \Sigma\;\;\;\;\forall a \in \rr.\]
\end{definicion}

\begin{proposicion}{}
$f:\rr^n \to \overline{\rr}$ es $\Sigma$-medible si y s\'olo si
$f^{-1}(H)\in \Sigma$ cuando $H$ es boreliano de $\overline{\rr}$.
\end{proposicion}

Cuando $\Sigma=\mathscr{M}$, decimos que $f$ es \emph{medible}.

Cuando $\Sigma=\mathscr{B}$, llamamos a $f$ \emph{medible Borel} 
o funci\'on \emph{boreliana}.

\begin{ejercicio}{}
Si $f$ es semicontinua inferiormente, entonces $f$ es medible. 
\end{ejercicio}

Ahora, estudiaremos $g\circ f$ cuando 
$\rr^n \xrightarrow{f}\overline{\rr} \xrightarrow {g} \overline{\rr}$.

\begin{definicion}{}
Diremos que $g: \rr^n \to \overline{\rr}$ es boreliana si $g^{-1}(M)$ es boreliano de $\overline{\rr}$ cuando $M$ lo es.
\end{definicion}

Luego, si $f:\rr^n \to \overline{\rr}$ es medible y $g:\overline{\rr} \to \overline{\rr}$ es boreliana, entonces $g\circ f$ es medible.

Tambi\'en, se tiene que si $f$ es medible entonces $|f|$, $|f|^2$, $\log|f|$ y $e^f$ son medibles. 

Si $f$ y $g$ son medibles, entonces $\{f<g\}$ es medible pues
\[
\{f<g\}=\bigcup\limits_{q \in \qq} \{ f<q\} \cap \{ g>q\}.
\]
\begin{teorema}{teo:propiedades-func-medibles}
Si $f$ y $g$ son medibles con respecto a $\Sigma$ y si $c\in \rr$, entonces
$f+g$, $cf$ y $fg$ son medibles.
\end{teorema}

\begin{demo}
Supondremos que todas las funciones son finitas.

Veamos que 
\[
\{
f+g>a\}=\bigcup\limits_{r \in \qq} \{ a>r\} \cap \{q>a-r\}.
\}
\]
Si $f(x)+g(x)>a$ entonces existe $r \in \qq$ tal que 
\[
f(x)>r>a-g(x),\;\;\mbox{ es decir }\;\; 
f(x)>r\;\mbox{ y }\; g(x)>a-r.
\]
Rec\'iprocamente, si para alg\'un $r\in \qq$ se tiene que $f(x)>r$ y $g(x)>a-r$, entonces $f(x)+g(x)>a$.

Si $c\in \rr$, entonces
\[
\{cf>a\}=
\left\{
\begin{array}{ll}
f>\frac{a}{c}&c>0
\medskip
\\
f<\frac{a}{c}&c<0
\medskip
\\
\emptyset\;\mbox{ \'o }\;\rr^n &c=0.
\end{array}
\right.
\]

Ahora, como 
\[
fg=\frac{1}{4}[(f+g)^2-(f-g)^2]
\]
luego $fg$ es medible con respecto a $\Sigma$. 

Para estudiar el cociente entre funciones medibles, denotamos por $\frac{1}{f}$
la funci\'on que toma el valor $\frac{1}{f(x)}$ si $f(x)\neq 0$ y el valor $0$ si $f(x)=0$.

Ahora, como 
\[
\left\{\frac{1}{f}>a\right\}
=
\left \{
\begin{array}{ll}
 \{f>0\} \cap \{f<\frac{1}{a}\}    &  a>0 
 \medskip
 \\
 \{ f>0\} \cup  (\{f<\frac{1}{a}\} \cap \{f<0\}) \cup \{f=0\}
     & a<0, 
\end{array}
\right.
\]
a partir de que $f$ es medible se tiene que  $1/f$ es medible. 
\end{demo}

\section{Sucesiones de funciones medibles}

\begin{proposicion}{prop:inf-y-sup-medibles}
Si $\{f_k\}$ es una sucesi\'on de funciones medibles con respecto a $\Sigma$, 
entonces
\[
g(x)=\inf\limits_{k} f_k(x) \;\;\mbox{ y }\;\; h(x)=\sup\limits_{k} f_k(x) 
\]
son $\Sigma$-medibles. 
\end{proposicion}


La demostraci\'on se deduce de las f\'ormulas
\[
\{h>a \}=\bigcup_{k=1}^{\infty} \{f_k>a\}
\;\;\mbox{ y }\;\;
\{g<a \}=\bigcup_{k=1}^{\infty} \{f_k<a\}.
\]

\begin{proposicion}{prop:liminf-y-limsup-medibles}
Si $\{f_k\}$ es una sucesi\'on de funciones $\Sigma$-medibles, entonces
\[
g(x)=\liminf\limits_{k \to \infty} f_k(x)
\;\;\mbox{ y }
\;\;
h(x)=\limsup\limits_{k \to \infty} f_k(x)
\]
son   medibles con respecto a $\Sigma$.
\end{proposicion}


La prueba resulta de aplicar la Proposici\'on \ref{prop:inf-y-sup-medibles}
a las relaciones 
\[
g(x)=\sup\limits_{j} \inf\limits_{k\geq j} f_k(x)
\;\;\mbox{ y }\;\;
h(x)=\inf\limits_{j}\sup\limits_{k \geq j} f_k(x).
\]

\begin{corolario}{}
Si $f_k(x) \to f(x)$ y $\{f_k\}$ es una sucesi\'on de funciones $\Sigma$-medibles, entonces $f$ es medible con respecto a $\Sigma$.
\end{corolario}

A continuaci\'on, completamos la demostraci\'on del Teorema \ref{teo:propiedades-func-medibles} que se reliz\'o suponiendo
que tanto $f$ como $g$ son finitas. 
Para evitar esa restricci\'on, ahora consideramos la sucesi\'on de funciones $\varphi_k:\overline{\rr} \to \overline{\rr}$ definidas por
\[
\varphi_k(t)=
\left\{
\begin{array}{rl}
   t  &  \mbox{si } |t|\leq k \\
   k  &  \mbox{si } t>k\\
   -k &  \mbox{si } t<-k.
\end{array}
\right.
\]
Cada $\varphi_k$ es una funci\'on boreliana pues la restricci\'on de $\varphi$ a $\rr$ es continua. Adem\'as, para cada $t\in \overline{\rr}$, 
se tiene que $\varphi_k \to t$ cuando $k \to \infty$. Entonces, las funciones
\[
f_k=\varphi_k \circ f \;\;\mbox{ y }
\;\;
g_k=\varphi_k \circ g,
\]
son medibles con respecto a $\Sigma$, finitas y convergen puntualmente a $f$ y $g$ respectivamente, cuando $k  \to \infty$. Luego, las funciones 
\[
f+g=\lim\limits_{k \to \infty} (f_k +g_k)
\;\;\mbox{ y }\;\;
fg=\lim\limits_{k \to \infty} f_kg_k,
\]
resultan medibles a partir de la aplicaci\'on de la Proposici\'on \ref{prop:liminf-y-limsup-medibles}.

\section{Funciones simples}
Definimos la funci\'on caracter\'istica  $\chi_E$ de un conjunto $E \subset \rr^n$ mediante
\[
\chi_E(x)=
\left\{
\begin{array}{ll}
   1  & \mbox{si } x \in E \\
   0  & \mbox{si } x \notin E.
\end{array}
\right.
\]

Se tiene la siguiente propiedad
\begin{itemize}
    \item $\chi_E$ es medible  si y s\'olo si $E$ es medible.
\end{itemize}


\begin{definicion}{defi:funcion-simple}
Una funci\'on medible y finita  $\varphi:\rr^n \to \rr$ 
se llama \emph{simple} si el conjunto de todos sus valores es finito, es decir, si $\varphi$ es medible y la imagen  $\varphi(\rr^n)$ es un subconjunto finito de $\rr$.
\end{definicion}

A partir de la Definici\'on \ref{defi:funcion-simple}, se tiene que si $\varphi,\psi$ son funciones simples y $c \in \rr$, entonces $\varphi + \psi$, $c\varphi$,  $\varphi \psi$ son simples.

Si $\varphi(\rr^n)=\{\alpha_1,\ldots,\alpha_N\}$, entonces $E_i=\varphi^{-1}(\{\alpha_i\})$ son medibles y 
\[
\varphi=\sum\limits_{i=1}^N \alpha_i \chi_{E_i}.
\]
Las funciones simples desempe\~nan un papel muy importante en la teor\'ia de integraci\'on en virtud del siguiente teorema.

\begin{teorema}{teo:simples-convergen-a-medible}
Si $f:\rr^n \to \overline{\rr}$ es una funci\'on medible no negativa, entonces existe una sucesi\'on $\{\varphi_k\}$ de funciones simples tal que
\[
\varphi_1\leq \varphi_2 \leq \varphi_3\leq \ldots \;\;\mbox{ y }\;\;
f(x)=\lim\limits_{k \to \infty} \varphi_k(x)
\]
en cada punto $x \in \rr^n$.
\end{teorema}

\begin{demo}
Para $k \in \nn$, dividimos $[0,2^k)$ en $k2^k$ intervalos disjuntos
\[
\left[\frac{i-1}{2^k},\frac{i}{2^k}\right), \;\;i=1,2,\ldots,k2^k.
\]
Definimos $g_k:\overline{\rr} \to \overline{\rr}$ por
\[
g_k(x)=
\left\{
\begin{array}{ll}
    \frac{i-1}{2^k} & \mbox{si }\; 0\leq t\leq k, \;\; (i-1)/2^k\leq t< i/2^k, 
    \medskip
    \\
    k     & \mbox{si }\;t\geq k
    \medskip
    \\
    0     & \mbox{si }\; t<0.
\end{array}
\right.
\]
Las funciones $g_k$ son borelianas, no negativas y verifican
\[
0\leq g_1\leq g_2\leq \ldots,\;\mbox{ y }\;\;
\lim\limits_{k \to \infty} g_k(t)=t, \;\;\mbox{ en } [0, +\infty].
\]
Las funciones $\varphi_k = g_k \circ f$ son simples y verifican  el teorema.
\end{demo}

\begin{observacion}{}
\begin{enumerate}
    \item Si $f$ es medible con respecto a una $\sigma$-\'algebra $\Sigma$, entonces las funciones $\varphi_k$ del Teorema \ref{teo:simples-convergen-a-medible} son tambi\'en medibles con respecto a $\Sigma$.
    \item Multiplicando a las $\varphi_k$ por $\chi_{B(0,k)}$ se obtiene una sucesi\'on de  funciones $\psi_k$ que verifican las hip\'otesis del Teorema \ref{teo:simples-convergen-a-medible} y que tienen soporte compacto.
    \item Si $f$ es acotada y positiva,  la convergencia es uniforme.
\end{enumerate}
\end{observacion}

\section{Partes positiva y negativa}

Si $f$ es medible, tambi\'en lo son 
\[
f^+=\sup\{0,f \}\;\;\mbox{ y }\;\;f^{-}=\sup\{0,-f\}
\]
llamadas \emph{parte positiva} y \emph{parte negativa} de $f$.

Se verifica que 
\[
f=f^{+}-f^{-}\;\;\mbox{ y }\;\; |f|=f^{+}+f^{-}.
\]

\begin{teorema}{}
Si $f$ es medible y $f=f_1-f_2$, con $f_i\geq 0$ para $i=1,2$, entonces
$f^{+}\leq f_1$ y $f^{-}\leq f_2$.
\end{teorema}

\begin{demo}
Se tiene que $f\leq f_1$ de donde $f^+=\sup\{ f,0\}\leq f_1$. 

Adem\'as, $-f \leq f_2$. Luego, $f^{-}\leq f_2$
\end{demo}

Si $f$ es medible,  aplicando el Teorema \ref{teo:simples-convergen-a-medible} a $f^+$ y $f^{-}$, 
existen funciones simples $\varphi_k,\psi_k$ tales que $\varphi_k \to f^+$
y $\psi_k \to f^{-}$. Luego, 
$\varphi_k - \psi_k \to f$, y adem\'as, $|\varphi_k-\psi_k|\leq \varphi_k+\psi_k \leq f^{+}+f^{-}=|f|$.

\section{Propiedades verdaderas en casi todo punto}

Si $P$ es una propiedad sobre puntos de $\rr^n$ ($P(x)$) diremos que $P$ es verdadera en casi todo punto si $P(x)$ es verdadera excepto, posiblemente, en un conjunto de medida cero. 

As\'i, por ejemplo:
\begin{enumerate}
    \item Casi todo n\'umero es irracional.
    \item  Si $f$ y $g$ son funciones definidas sobre todo $\rr^n$, diremos que $f=g$ en casi todo punto si $f(x)=g(x)$ para todo  
    $x \notin E$  siendo la  $m(E)=0$.
\end{enumerate}

\begin{teorema}{}
Si $h=0$ en c.t.p., entonces $h$ es medible.
\end{teorema}

\begin{demo}
Sea $Z=\{h \neq 0\}$. 

Si $a \geq 0$, entonces $\{h>a\}\subset Z$ y por tanto $m(\{h>a\})=0.$ Luego, $\{h>a\} \in \mathscr{M}$.

Si $a < 0$, se tiene que  $\{h\leq a\}\subset Z$. Ahora,  como $\{h>a\}=\rr^n-\{h\leq a\}$, entonces  $\{h>a\} \in \mathscr{M}$ por ser complemento de un conjunto medible.
\end{demo}



\begin{corolario}{}
Si $f$ es medible y $f=g$ en c.t.p., entonces $g$ es medible. 
\end{corolario}


Ser\'a frecuente decir que $f_k(x) \to f(x)$ en c.t.p.

\begin{teorema}{}
Si $f_k \to f$ en c.t.p. y las funciones $f_k$ son medibles, entonces 
$f$ es medible.
\end{teorema}

\begin{demo}
La funci\'on $g=\liminf\limits_{k \to \infty} f_k$ es medible y $g=f$ en c.t.p.
\end{demo}

Si $f$ y $g$ son medibles, definimos $f \sim g$ si $f=g$ en c.t.p.

Entonces, $\sim$ es una relaci\'on de equivalencia. Adem\'as, si $f\sim h$ y $g \sim g$, entonces $f+g\sim h+g$ y $fg\sim hg$.

Si $f=g$ en c.t.p., entonces $f$ es esencialmente igual a $g$.

\section{Convergencia en medida}

Si para cada $x\in \rr^n$ tenemos una propiedad, enunciado o afirmaci\'on $P(x)$ que puede ser tildada de  verdadera o falsa, escribiremos $E(P)$ para denotar $E\cap \{x:\,P(x)\}$.

\begin{definicion}{}
Sean $f_k$ y $f$  medibles sobre $E$.\\
Se dice que $f_k$ converge en medida a $f$ si $\forall \delta>0$ se tiene que 
\[
m(E(|f_k-f|\geq \delta))\xrightarrow[k \to \infty]{} 0.
\]
Notaremos: $f_k \xrightarrow[]{m} f$.
\end{definicion}

\begin{teorema}{}
Si $f_k \xrightarrow[]{m}f$ y $f_k \xrightarrow[]{m} g$, entonces
$f=g$ en c.t.p.
\end{teorema}

\begin{demo}
A partir de que 
\[
\{|f-g|\geq \delta\} \subset 
\left\{|f-f_k|\geq\frac{\delta}{2}\right\} \cup \left\{|f_k-g|\geq\frac{\delta}{2}\right\}, 
\]
se tiene que 
\[
m\left(E\left(|f-g|\geq \delta\right)\right)\leq 
m\left(E\left(|f-f_k|\geq \frac{\delta}{2}\right)\right)+
m\left(E\left(|f_k-g|\geq \frac{\delta}{2}\right)\right) \xrightarrow[k \to \infty]{} 0.
\]
Luego, 
\[
m(\{f\neq g\})=
m\left( \bigcup\limits_{k=1}^{\infty} \left\{|f-g|\geq \frac{1}{k}  \right\} \right)=0.
\]
As\'i, $f=g$ en c.t.p.
\end{demo}


\begin{teorema}{teo:conv-puntual-implica-medida}
Si $m(E)<\infty$ y $f_k \to f$ en c.t.p. de $E$, entonces 
$f_k \xrightarrow[]{m}f$.
\end{teorema}


\begin{demo}
Sea $Z$ el conjunto de puntos donde $f_k$ no tiende a $f$. Entonces $m(z)=0$. 

Dado $\delta>0$, definimos 
\[
B_j=\bigcup\limits_{k=j}^{\infty} E(|f_k-f|\geq \delta).
\]
 y se tiene que $ \bigcap\limits_{j=1}^{\infty} B_j=Z.$
 Luego, $m(B_j)\to 0$ cuando $j \to \infty$.
 
 Como si $k \geq j$, se tiene que $E(|f_k-f|\geq \delta )\subset B_j$.
Luego $m(E(|f_k-f|\geq \delta))\to 0$ cuando $k \to \infty$
y por lo tanto $f_k \xrightarrow[]{m}f.$
\end{demo}

\begin{observacion}{}
\begin{enumerate}
    \item Si $m(E)=+\infty$, el Teorema \ref{teo:conv-puntual-implica-medida} no es cierto. 
    Por ejemplo,  $f_k=\chi_{B(0,k)} \to 1$ en c.t.p., mientras que 
    $m(E(|f_k-1|=1 ))=m(\rr^n-B(0,k))= +\infty$\; $\forall k\in \nn$.
    \item La rec\'iproca del Teorema \ref{teo:conv-puntual-implica-medida} no es cierta. 
    Basta tomar $f_{k_n}=\chi_{\left[\frac{k}{2^n}, \frac{k+1}{2^n}\right)}$ con 
    $k=0,1, \ldots, 2^{n-1}$ y $n=1,2,\ldots$
\end{enumerate}
\end{observacion}{}

\begin{definicion}{}
Diremos que $f_k$ es fundamental en medida sobre $E$ si $\forall \delta>0$ se tiene que 
\[
m\left(E\left(\left|f_k-f\right|\geq \delta\right)\right)\xrightarrow[k,j \to \infty]{} 0.
\]
\end{definicion}

\begin{observacion}{}
Si $f_k \xrightarrow[]{m}f$ y $f$ es finita, entonces $f_k$ es fundamental en medida. 
\end{observacion}



\begin{teorema}{}
Si $f_k$ es fundamental en medida sobre $E$, entonces existe una subsucesi\'on $k_j$ y una funci\'on $f$ medible sobre $E$ tal que $f_{k_j}\to f$ en c.t.p. de $E$. Adem\'as, $f$ es finita y $f_{k} \xrightarrow[]{m}f$.
\end{teorema}

\begin{demo}
Para cada $i>0$, existe $k_i\in \nn$ tal que 
\[
m\left(E\left(|f_k-f|\geq \frac{1}{2^i}\right)\right)\leq
\frac{1}{2^i},
\]
para $k_j\geq k_i$.
Podemos suponer que $k_1<k_2<\ldots$. Sea 
\[
E_i=E\left(   |f_{i} -f_{i+1}|\geq \frac{1}{2^i}\right)
\]
y tenemos que $m(E_i)<\frac{1}{2^i}$. 

Sea 
\[ Z=\bigcap\limits_{j=1}^{\infty}\bigcup\limits_{i=j}^{\infty} E_i
=\limsup\limits_{i \to \infty} E_i.
\]
Ahora, $m(Z)=0$. 

Si $x \in E-Z$, existe un $j$ tal que $x \notin E_i$ para $i\geq j$, es decir, 
\[
x\in E\left( \left|f_{k_i}-f_{k_{i+1}}\right|<\frac{1}{2^i}\right).
\]
Luego, la serie
\begin{equation}\label{eq:serie-telesc-conv-med-a-puntual}
f_{k_1}(x)+\left(f_{k_2}(x)-f_{k_1}(x)\right)+\ldots
\end{equation}
converge absolutamente en $E-Z$.

Sea $f(x)$ la suma de \eqref{eq:serie-telesc-conv-med-a-puntual} en $E-Z$ y sea $f(x)=0$ en $Z$. A partir de la definici\'on de $f$ es claro que $f$ es finita. Adem\'as, pasando a sumas parciales, tenemos
\[
f(x)=\lim\limits_{i \to \infty} f_{k_i}(x)\;\mbox{ en c.t.p. de } \,E.
\]

A continuaci\'on, veamos que $f_{k_i} \xrightarrow[]{m} f$.

Sea $\delta>0$ y elijamos $j$ tal que $\frac{1}{2^{j-1}}<\delta$. 
Si  $x\notin Z$, entonces 
\[
f(x)=
f_{k_j}(x)+\left(f_{k_{j+1}}(x)-f_{k_j}(x)\right)+\ldots.
\]
As\'i
\[
E\left(\left|f(x)-f_{k_j}(x)  \right|\geq \delta\right)\subset 
Z\cup \left(\bigcup\limits_{i\geq j} E_i\right)
\]
de donde
\[
m\left(
E\left(\left|f(x)-f_{k_j}(x)  \right|\geq \delta\right)
\right)
\leq \sum\limits_{i\geq j} m(E_i)=\frac{1}{2^{j-1}}.
\]
De este modo, obtenemos $f_{k_j} \xrightarrow[]{m} f$.

Por \'ultimo, a partir de 
\[
E\left(|f_k-f|\right)\subset 
E\left(|f_k -f_{k_j}|\geq \frac{\delta}{2}\right) \cup
E\left(|f_{k_j} -f|\geq \frac{\delta}{2}\right),
\]
tomando $k$ y $k_j$ grandes, deducimos
\[
m\left(E\left(\left|f_k-f\right|\geq \delta\right)\right)< \epsilon
\]
para valores de $k$ grandes. En consecuencia, $f_k \xrightarrow[]{m}f$.
\end{demo}

\section{Funci\'on singular de Cantor}

El conjunto de Cantor se define como 
\[
P=\bigcap\limits_{n=1}^{\infty} F_n,
\]
donde $F_n$ es la uni\'on de $2^n$ intervalos cerrados y disjuntos contenidos en el $[0,1]$.

El conjunto $[0,1]-F_n$ es la uni\'on de $2^n-1$ intervalos abiertos disjuntos. Si los numeramos de izquierda a derecha, formamos los intervalos abiertos $J_{n,i}$, para $i=1,2,\ldots, 2^n-1$ y se tiene la relaci\'on
\[ 
J_{n,i}=J_{n+1,2i}.
\]

Sea $\varphi_n$ la funci\'on que toma los siguientes valores
$\varphi_n(0)=0$, $\varphi_n(1)=1$, $\varphi_n(x)=\frac{i}{2^n}$ en $J_{n,i}$, es lineal entre los $F_n$ y es continua.

Tenemos que $\varphi_{n+1}=\varphi_n$ en  $J_{n,i}$   y adem\'as
\[
\left|\varphi_{n+1} -\varphi_n\right|<\frac{1}{2^{n+1}},
\]
en cada punto de $[0,1]$.
Luego, la serie
\[
\varphi_1+(\varphi_2-\varphi_1)+(\varphi_3-\varphi_2)+\ldots
\]
converge uniformemente a una funci\'on continua $\varphi$ que se llama 
\emph{funci\'on singular de Cantor}. 

Es claro que $\varphi$ es mon\'otona creciente y su restricci\'on a cualquiera de los intervalos $J_{n,i}$ es constante.

