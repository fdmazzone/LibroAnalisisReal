\documentclass{book}
\usepackage{amssymb,amsmath}
\usepackage{polyglossia}
\setmainlanguage{spanish} % Idioma principal
\usepackage{theorem}
\usepackage{times}
\usepackage{array}
\usepackage{graphicx}
\usepackage{hyperref}
\usepackage{multirow}
\usepackage{fancyhdr}
%\usepackage[cp1252]{inputenc}
\usepackage{hhline}
\usepackage{multicol}
\usepackage[a4paper,driver=xetex,top=4.5cm,head=4.5cm, bottom=2cm,%
layouthoffset=0mm, left=2.5cm, right=2.5cm,marginparwidth=0cm]{geometry}
%\usepackage{bm}
%\usepackage{tabular}
\usepackage{fontspec}
 \usepackage[breakable,many]{tcolorbox}
\defaultfontfeatures{Ligatures=TeX}
\usepackage{empheq}
 \setromanfont{Roboto Condensed}
 \usepackage{float}
 \usepackage{mathrsfs} 
%
%
% \renewcommand{\familydefault}{\sfdefault}
%\renewcommand{\familydefault}{\sfdefault}

%%%%%%%%%Estilo de la pagina%%%%%%%%%%%%%%%%%%%%%%%%%%%%%%%%%%%
%%%%%%%%%%%%%%%%%%%%%%%%%%%%%%%%%%%%%%%%%%%%%%%%%%%%%%%%%%%%%%%%%%
% \newcounter{ejer}
% 
% {\theorembodyfont{\normalfont}
% \newtheorem{ejercicio}[ejer]{Ejercicio}}

\newcommand{\rr}{\mathbb{R}}
\newcommand{\qq}{\mathbb{Q}}
\newcommand{\nn}{\mathbb{N}}



\DeclareMathOperator{\atan2}{atan2}
%\DeclareMathOperator{\sen}{sen}
\DeclareMathOperator{\sign}{sign}
\DeclareMathOperator{\sn}{sn}
\DeclareMathOperator{\SO}{SO}
%\DeclareMathOperator{\arcsen}{arcsen}
\DeclareMathOperator{\Or}{O}

\usepackage[framemethod=TikZ]{mdframed}
%%%%%%%%%%%%%%%%%%%%%%%%%%%%%%
%Theorem

%% Ejercicio
\newcounter{ejer} \setcounter{ejer}{0}
\renewcommand{\theejer}{\arabic{ejer}}
\newenvironment{ejer}[2][]{%
\vspace{5pt}
\refstepcounter{ejer}%
\ifstrempty{#1}%
{%
% \mdfsetup{%
% frametitle={%
% \tikz[baseline=(current bounding box.east),outer sep=-0pt]
% \node[anchor=east,rectangle,fill=green!50]
{\noindent\bfseries Ejercicio~\theejer}.}
%
{%
% \mdfsetup{%
% frametitle={%
% \tikz[baseline=(current bounding box.east),outer sep=0pt]
% \node[anchor=east,rectangle,fill=green!50]
{\noindent\bfseries  Ejercicio~\theejer:~#1}.}%
%
%\mdfsetup{innertopmargin=10pt,linecolor=green!50,%
%linewidth=2pt,topline=true,%
%frametitleaboveskip=\dimexpr-\ht\strutbox\relax
%}
%\begin{mdframed}[]
\relax%
\label{#2}}{\vspace{5pt}}%\end{mdframed}}

%Theorem
\newcounter{theo}[chapter] \setcounter{theo}{0}
\renewcommand{\thetheo}{\arabic{section}.\arabic{theo}}
\newenvironment{theo}[2][]{%
\refstepcounter{theo}%
\ifstrempty{#1}%
{\mdfsetup{%
frametitle={%
\tikz[baseline=(current bounding box.east),outer sep=0pt]
\node[anchor=east,rectangle,fill=blue!20]
{\strut Teorema~\thetheo};}}
}%
{\mdfsetup{%
frametitle={%
\tikz[baseline=(current bounding box.east),outer sep=0pt]
\node[anchor=east,rectangle,fill=blue!20]
{\strut Teorema~\thetheo:~#1};}}%
}%
\mdfsetup{innertopmargin=10pt,linecolor=blue!20,%
linewidth=2pt,topline=true,%
frametitleaboveskip=\dimexpr-\ht\strutbox\relax
}
\begin{mdframed}[]\relax%
\label{#2}}{\end{mdframed}}
%%%%%%%%%%%%%%%%%%%%%%%%%%%%%%
%Lemma
\newcounter{lem}[chapter] \setcounter{lem}{0}
\renewcommand{\thelem}{\arabic{section}.\arabic{lem}}
\newenvironment{lem}[2][]{%
\refstepcounter{lem}%
\ifstrempty{#1}%
{\mdfsetup{%
frametitle={%
\tikz[baseline=(current bounding box.east),outer sep=0pt]
\node[anchor=east,rectangle,fill=green!20]
{\strut Lemma~\thelem};}}
}%
{\mdfsetup{%
frametitle={%
\tikz[baseline=(current bounding box.east),outer sep=0pt]
\node[anchor=east,rectangle,fill=green!20]
{\strut Lemma~\thelem:~#1};}}%
}%
\mdfsetup{innertopmargin=10pt,linecolor=green!20,%
linewidth=2pt,topline=true,%
frametitleaboveskip=\dimexpr-\ht\strutbox\relax
}
\begin{mdframed}[]\relax%
\label{#2}}{\end{mdframed}}
%%%%%%%%%%%%%%%%%%%%%%%%%%%%%%
%% Definicion
\newcounter{defini}[chapter] \setcounter{defini}{1}
\renewcommand{\thedefini}{\arabic{section}.\arabic{defini}}
\newenvironment{definicion}[2][]{%
\refstepcounter{defini}%
\ifstrempty{#1}%
{\mdfsetup{%
frametitle={%
\tikz[baseline=(current bounding box.east),outer sep=0pt]
\node[anchor=east,rectangle,fill=green!20]
{\strut Definición~\thedefini};}}
}%
{\mdfsetup{%
frametitle={%
\tikz[baseline=(current bounding box.east),outer sep=0pt]
\node[anchor=east,rectangle,fill=green!20]
{\strut Definición~\thedefini:~#1};}}%
}%
\mdfsetup{innertopmargin=10pt,linecolor=green!20,%
linewidth=2pt,topline=true,%
frametitleaboveskip=\dimexpr-\ht\strutbox\relax
}
\begin{mdframed}[]\relax%
\label{#2}}{\end{mdframed}}

%Proof
\newenvironment{prf}{\noindent\emph{Dem.}}{$\square$ \newline\vspace{5pt}}


%Corolario
\newcounter{cor}[chapter] \setcounter{cor}{0}
\renewcommand{\thecor}{\arabic{section}.\arabic{cor}}
\newenvironment{cor}[2][]{%
\refstepcounter{cor}%
\ifstrempty{#1}%
{\mdfsetup{%
frametitle={%
\tikz[baseline=(current bounding box.east),outer sep=0pt]
\node[anchor=east,rectangle,fill=green!20]
{\strut Corolario~\thelem};}}
}%
{\mdfsetup{%
frametitle={%
\tikz[baseline=(current bounding box.east),outer sep=0pt]
\node[anchor=east,rectangle,fill=green!20]
{\strut Corolario~\thelem:~#1};}}%
}%
\mdfsetup{innertopmargin=10pt,linecolor=green!20,%
linewidth=2pt,topline=true,%
frametitleaboveskip=\dimexpr-\ht\strutbox\relax
}
\begin{mdframed}[]\relax%
\label{#2}}{\end{mdframed}}

\tcbset{highlight math style={enhanced,
  colframe=red!60!black,colback=yellow!50!white,arc=4pt,boxrule=1pt,
  drop fuzzy shadow}}
  
  
  
  
  
  
  


\pagestyle{fancyplain}

 \renewcommand{\sectionmark}[1]
                 {\markright{\thesection\ #1}}


% \lhead[\fancyplain{}{\bfseries\thepage}]
%       {\fancyplain{}{\bfseries\rightmark}}
%
 \rhead[\fancyplain{}{\bfseries\leftmark}]{\fancyplain{}{\bfseries}}




 \lhead[\fancyplain{}{ \includegraphics[scale=.3]{EscudoUNLPam.png}}]{\fancyplain{}{ \includegraphics[scale=.3]{EscudoUNLPam.png}}}

\cfoot{}





  
  
  
  
  
  
  
\begin{document}


\hyphenation{excen-tri-ci-dad}


\begin{large}
\begin{bfseries} % \begin{scshape}
        \noindent Depto de Matem\'atica.\\
        Primer Cuatrimestre de 2022\\                                                                                                                                                                                                                                                                                                                                                
        Teoría de la Medida \\
        Práctica 5: Integral de Lebesgue

%\end{scshape}
\end{bfseries}
\end{large}
\par\noindent\rule{\textwidth}{.5pt}








\begin{ejer}{}
  Mostrar que la función $x^{p-1} e^{-x}$ es integrable sobre $(0,\infty)$ si y sólo si $p>0$.
	\end{ejer}
  
	\begin{ejer}{}
  La función $\frac{\text{sen}\,x}{x}$ no es integrable sobre $(0,\infty)$, aunque existe el límite
\linebreak  $\lim\limits_{R\rightarrow \infty}\int_0^R \frac{\text{sen}\,x}{x}\,dx.$
\end{ejer}
	
	\begin{ejer}{}
	Probar que la integral $\int_0^1 \frac{1}{x}\text{sen} \left(\frac{1}{x}\right)\,dx$ 
	existe como integral impropia de Riemann pero no existe como integral de Lebesgue.
	\end{ejer}

	\begin{ejer}{}
	Sup\'ongase que $f$ es integrable de Riemann sobre un intervalo  infinito (tal integral
	s\'olo puede existir en el sentido impropio). 
	Demostrar que $f$ es integrable de Lebesgue sobre el mismo intervalo si y s\'olo si la integral
	impropia converge absolutamente.
  \end{ejer}
	
	
	\begin{ejer}{}
	Probar, usando el Teorema de la Convergencia Mayorada, la fórmula
  $$\lim\limits_{n\rightarrow \infty} \int_0^1 \frac{n^{\frac{3}{2}} x}{1+n^2 x^2}\,dx=0. $$
	\end{ejer}

%  \item Probar que si $p>0$, entonces 
 % $$\lim_{n \rightarrow \infty}\int_0^n x^{p-1}.\Bigl(1-\frac{x}{n}\Bigr)^n\,dx=
 % \int_0^{\infty} x^{p-1}.e^{-x}\,dx. $$

  \begin{ejer}{}
	Usando integración término a término probar que 
  $$\int_0^{\infty} \frac{x}{e^x-1}\,dx=\sum_{n=1}^{\infty}\frac{1}{n^2}. $$
  \end{ejer}
	
	\begin{ejer}{}
	\begin{enumerate}
	\item Sup\'ongase que $f\geq 0$, integrable de Riemann en $[0,a]$ $\forall a>0$ y que adem\'as tiene integral impropia. 
	Probar que $f$ es integrable de Lebesgue en $\rr$.
	\item Sup\'ongase que $f\geq 0$, integrable de Riemann en $[a+\epsilon,b]$ y que existe la integral impropia.
	Probar que $f$ es integrable de Lebesgue en $[a,b]$.
	\end{enumerate}
	\end{ejer}
	
  \begin{ejer} Sea $f$ una función medible no negativa sobre $\rr$ y sea $(E_k)$ una sucesión creciente de conjuntos
  cuya unión es $E$.
	\begin{enumerate}
  \item Probar que $\int_{E} f=\lim\limits_{k \rightarrow \infty} \int_{E_k} f$.
  \item Extender a cualquier función $f$ integrable sobre $E$.
	\end{enumerate}
	\end{ejer}

  

  \begin{ejer} {}
	Si se considera la sucesión de funciones $f_n(x)=n \chi_n(x)$ en el intervalo $0\leq x\leq 1$,
  donde $\chi_n$ es la función característica del intervalo $\left(0,\frac{1}{n}\right)$, ?`es posible que exista
  una función $g(x)$ integrable en dicho intervalo, tal que $f_n(x) \leq g(x)$ para cualquier $n$ y cualquier
  $x$?
	\end{ejer}


  \begin{ejer}{}
	Si $\varphi(x) f(x)$ es integrable sobre $E$ para cualquier función $f$ integrable sobre $E$, entonces
  existe una constante finita $C$, tal que $|\varphi(x)|\leq C$ en c.t.p $x$ de $E$.
	\end{ejer}
  
  \begin{ejer}{} 
	Sea $f$ una función medible no negativa sobre $\rr^1$ tal que $\int_a^b f(x)\,dx>0$ siempre
  que $a<b$, ?`puede concluirse que $f(x)>0$ en c.t.p $x$?
	\end{ejer}
 
	\begin{ejer}{}
	\textbf{Absoluta continuidad de la integral de Lebesgue.}
	Sup\'ongase que $f$ es integrable de Lebesgue sobre $E$. 
	Probar que para cada $\epsilon>0$ existe $\delta>0$ tal que $\int_E |f|\,dx<\epsilon$ siempre que $m(E)<\delta$.
	\end{ejer}
	
  \begin{ejer}{} Probar que el Teorema de la Convergencia Mayorada se extiende a una familia de funciones medibles $f_t(x)$,\;  
  $a<t<b$, dependiente de un parámetro real $t$, de la manera siguiente:
   \\
   Supongamos que $\tau \in (a,b)$ y que en cada punto de $E$ existe el límite 
   $f(x)=\lim\limits_{t \rightarrow \tau} f_t(x)$.
   Si existe una función $\Phi(x)$ integrable sobre $E$, tal que $|f_t(x)|\leq \Phi(x)$ para $x \in E$ y $a<t<b$;
   entonces $f$ es integrable sobre $E$ y además 
   $$\int_E f(x)\,dx=\lim_{t \rightarrow \tau}\int_E f_t(x)\,dx. $$
	\end{ejer}
	
	\begin{ejer}{}
   {\bf{Derivación de una integral paramétrica}}. Supongamos que la integral 
   $$ \varphi(t)=\int_E f(t,x)\,dx\;\;\;\;\;\;\;(a<t<b),$$ 
   existe para cada $t\in (a,b)$; que $f(t,x)$ es derivable con respecto
   a $t$ y existe una función $g(x)$ integrable sobre $E$, tal que 
   $$\Bigl|\frac{\partial f(t,x)}{\partial t}\Bigr|\leq g(x)$$
   para $x \in E$ y $a<t<b$.
   Probar que $\varphi$ es derivable y además 
   $$\varphi'(t)=\int_E \frac{\partial f(t,x)}{\partial t}\,dx $$ para $a<t<b$.
	\\
   \underline{Sugerencia:} Escribir el cociente $\frac{\varphi(t+h)-\varphi(t)}{h}$, emplear el Teorema del Valor
   Medio del Cálculo Diferencial y el ejercicio anterior.
	\end{ejer}

\begin{ejer}{}
  {\bf{Transformada de Fourier}}. Si $f$ es integrable sobre $\rr$, la función 
   $$g(t)=\int_{-\infty}^{\infty} e^{i t x} f(x)\,dx $$
   es acotada y uniformemente continua.
   \\
   Si $x^k.f(x)$ es integrable, entonces $g$ es de clase $C^k$ y además 
   $$g^{(k)}(t)=\int_{-\infty}^{\infty} e^{i t x} (i x)^k  f(x)\,dx.$$
 \end{ejer}
   
  




	
	
	
	


%¿¿¿¿¿¿¿¿¿¿¿¿¿¿¿¿¿¿¿¿¿¿¿¿¿¿¿¿¿¿¿¿¿¿¿¿¿¿¿


\end{document}
