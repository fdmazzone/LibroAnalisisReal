\chapter{Los números reales}


\section{Axiomas}\label{sec,apendice}
 Hay dos ópticas para introducir los numeros
 reales, las denominaremos axiomática y constructiva.
 Describimos a continuación, y someramente, cada una de ellas.

 Se pueden introducir los números reales a travez de un sistema
 axiomático. Como es conocido, un sistema axiomático consta de
 \emph{términos primitivos}, que son, por decirlo as\'{\i}, los
 objetos iniciales, a travez de los cuales se construyen todos los
 demás objetos de la teor\'{\i}a. Pueden ser términos
 primitivos: conjuntos, operaciones, relaciones, etc. Es importante
 aclarar que los términos primitivos son objetos puramente
 hipotéticos, es decir no se afirma la existencia de estos
 objetos. Los términos primitivos para los números reales son:
 un conjunto, comunmente denotado por $\rr$, dos funciones de
  $\rr\times\rr$ en $\rr$, usualmente denotadas por $+$ y $.$\footnote
  {Estas operaciones se denominan suma y multiplicación,
  usualmente se omite el signo de multiplicación} y una relación
  $\leq$. Para completar el sistema axiomático,
  debemos dar los axiomas, estos son propiedades que se postulan
  para los términos primitivos. Los axiomas para los números
  reales los podemos dividir en cuatro grupos:

  \begin{itemize}
    \item[1)] $\rr$ es un cuerpo, es decir:
        \begin{itemize}
            \item[1.1)] $x+(y+z)=(x+y)+z$;
            \item[1.2)] $x+y=y+x$;
            \item[1.3)] Existe un elemento $0\in\rr$ tal que $x+0=x$;
            \item[1.4)] Para cada $x\in\rr$ existe un $y\in\rr$
            tal que $x+y=0$;
            \item[1.5)] $x(yz)=(xy)z$;
            \item[1.6)] $xy=yx$;
            \item[1.7)] Existe un elemento $1\in\rr$ tal que
            $1.x=x$;
            \item[1.8)] Para cada elemento $0\neq x\in\rr$ existe un
            elemento $y\in\rr$ tal que $xy=1$;
            \item[1.9)] $x(y+z)=xy+xz$;
        \end{itemize}
    \item[2)] $\rr$ es un cuerpo ordenado.
        \begin{itemize}
            \item[2.1)] Si $x\leq y$ e $y\leq z$ entonces $x\leq
            z$;
            \item[2.2)] Si $x\leq y$ e $y\leq x$ entonces $x=y$;
            \item[2.3)] Si $x$ e $y$ pertenecen a $\rr$ entonces
            $x\leq y$ o $y\leq x$;
            \item[2.4)] Si $x\leq y$ entonces $x+z\leq y+z$;
            \item[2.5)] Si $0\leq x$ e $0\leq y$ entonces $0\leq
            xy$;
        \end{itemize}

        Dentro de $\rr$  se puede construir un conjunto, denotado
        por $\mathbb{Z}$. que corresponde a los enteros.
    \item[3)] $\rr$ es un cuerpo ordenado y arquimedeano. Esto es:
    para todo $y\geq 0$ y todo $x>0$ existe un entero $n$ tal que
    $nx\geq y$.

    Por último tenemos el axioma de completitud. Hay varias
    formulaciones equivalentes para este axioma, ver el Ejercicio
     \vref{ejer,completitud} nosotros elegimos
    la siguiente:

    \item[4)] Todo subconjunto $A\subset\rr$ acotado
    superiormente, tiene supremo, es decir existe un $\alpha\in\rr$
    tal que: 1) $\alpha$ es una cota superior de $A$, esto es $\alpha\geq x$
    para todo $x\in A$ y 2) $\alpha$ es la más chica de las
    cotas superiores, esto es si $\beta$ es cota superior entonces
    $\alpha\leq\beta$. Observar que no es necesario que $\alpha\in
    A$.
  \end{itemize}



  Hay tres propiedades que
  ser\'{\i}an deseables que un sistema axiomático tuviera: 1)
  \emph{coherencia}, es decir que los axiomas no se
  ``contradigan'' 2)\emph{independencia}, entendiendo por esto que
  los axiomas no sean redundantes, es decir que ninguno de ellos
  se obtenga a partir de los demás y 3)
  \emph{completitud}\footnote{No confundir este concepto con el de
  completitud de un e.m.}, esto es que toda afirmación de la
  teor\'{\i}a o su negación se pueda deducir.

  Destacamos,
  nuevamente, que los objetos postulados como términos
  primitivos en el sistema axiomatico y que satisfagan los axiomas
  podr\'{\i}an no existir. En particular esto ocurre si el sistema
  axiomático es contradictorio. Obviamente, en ese caso, nuestro
  s\'{\i}stema axiomático no servir\'{\i}a de mucho. Esto no sucede para el
  s\'{\i}stema de axiomas para los números reales. Este
  s\'{\i}stema tiene un modelo, es decir podemos encontrar un
  conjunto $\rr$ y las funciones y relación postuladas de modo
  tal que se satisfagan todos los axiomas. Esto nos lleva a la otra óptica de
  introducción de los números reales, la que denominamos constructiva. Varios
  modelos fueron propuestos por diversos matemáticos, en
  particular Dedekind y Cantor. Estos modelos
  son construidos a partir de los números racionales.\footnote{Es bueno
  decir que también es posible construir los números
  racionales a partir de los naturales y estos a partir de la
  Teor\'{\i}a de Conjuntos; no obstante esto se aparta
  considerablemente de los objetivos de esta materia} A Dedekind
  le debemos el método de cortaduras y a Cantor el método de
  sucesiones fundamentales de Cauchy.

