%==================================================
%CAPITULO II
%=================================================

\chapter{Nociones Básicas de Topología }

\section{Espacios Métricos}

\subsection{Enfoque axiomático de las  estructuras métricas} 
\marginnote{\vspace{-2cm}
\begin{center}
\adjustimage{max size={0.9\linewidth}{0.9\paperheight}}{imagenes/Frechet.jpg}
\end{center}
\small
Maurice René Fréchet; (Maligny, 2 de septiembre de 1878 - París, 4 de junio de 1973) fue un matemático francés. Trabajó en topología, teoría de la probabilidad y la estadística. 
Sus trabajos en análisis funcional lo empujaron a buscar un marco más general que el espacio euclídeo introduciendo la noción de espacio métrico\index[personas]{Frechet}. 
}
Uno de los conceptos fundamentales de la matemática es la noción de \index{Distancia}\emph{distancia}. Esta noción está presente en multitud de actividades humanas, desde el comercio a la descripción del cosmos. En matemática medimos distancias en el plano y en el espacio y en las representaciones algebraicas de ellos  $\rr^2$ y $\rr^3$. Más generalmente  en \index{Espacio Euclideano} \emph{espacios euclideanos} $n$-dimensionales $\rr^n$. Desde comienzos del siglo XX los matemáticos fueron extendiendo la noción de distancia a conjuntos compuestos de los más diversos entes, matrices, funciones, funciones que actúan sobre funciones, etc. Esta ubicuidad y multiplicidad del concepto de distancia justifica un tratamiento axiomático de él.








\begin{definicion}{} Sea $X$ un conjunto y $d:X\times X\rightarrow
\mathbb{R}$ una función. Diremos que $d$ es una \index{Métrica}\emph{métrica} o \emph{distancia}
sobre $X$ si satisface las siguientes propiedades:
	  \begin{itemize}
		   \item[i)]$\forall x \forall y: d(x,y)=0\Leftrightarrow x=y$.
		   \item[ii)] $\forall x \forall y : d(x,y)=d(y,x)$.
		   \item[iii)]$\forall x \forall y \forall z: d(x,z)\leq
				d(x,y)+d(x,z)$.
	  \end{itemize}
Si $d$ es una métrica sobre $X$ diremos, entonces, que el par
$(X,d)$ es un \index{Espacio métrico}\emph{espacio métrico}.
\end{definicion}
\marginnote{
\begin{center}
\adjustimage{max size={0.9\linewidth}{0.9\paperheight}}{imagenes/triag.png}
Desigualdad triangular
\end{center}
}
La desigualdad iii) en la definición anterior se denomina\index{Desigualdad triágular}
\emph{desigualdad triágular}, esto debido a que se la puede
pensar como la relación entre un lado de un triágulo y la suma
de los otros dos, ver figura  en el margen.




Veamos ahora algunos ejemplos de espacios métricos.

\begin{ejemplo}{} La función módulo $|.|:\mathbb{R}\rightarrow
\mathbb{R}$ induce una métrica sobre $\mathbb{R}$, a saber: para
$x,y\in\mathbb{R}$ definimos
\begin{equation}\label{distmod}
	d(x,y)=|x-y|.
\end{equation}
\end{ejemplo}

\begin{ejemplo}{ejem,disteuclidea} Sobre $\mathbb{R}^n$
consideremos la función
distancia $d$ definida por
\begin{equation}\label{meteuclidea}
	d(\mathbf{x},\mathbf{y}):=\sqrt{\sum\limits_{i=1}^{n}(x_i-y_i)^2},
\end{equation}
donde $\mathbf{x}=(x_1,\dots,x_n)$ e $\mathbf{y}=(y_1,\dots,y_n)$.
Dejamos al alumno la demostración de que $d$ es una métrica,
ver Ejercicio \vref{ejmeteuclidea}. Esta métrica es conocida
como \emph{métrica euclidea}\index{Métrica!euclidea} y es la métrica con la que estamos más familiarizados.
\end{ejemplo}



\begin{ejemplo}{} Dado cualquier conjunto no vacío $X$, la
función definida por:
\[
	d(x,y):=\left\{
\begin{array}{ll}
	1, & \hbox{si $x\neq y$;} \\
	0, & \hbox{si $x=y$.} \\
\end{array}
\right.
\]
es una métrica. Esta métrica se denomina \emph{métrica
discreta}.\index{Métrica!
discreta}
\end{ejemplo}

\begin{ejemplo}{ejem,distsobrecont} Dado un conjunto $X$,
definamos $\mathcal{A}(X)$
como el conjunto de todas las funciones acotadas $f:X\rightarrow
\mathbb{R}$. Entonces $(\mathcal{A}(X),d)$ es una métrica,
donde:
\begin{equation}\label{convunifmet}
	d(f,g):=\sup\limits_{x\in X}|f(x)-g(x)|.
\end{equation}
\end{ejemplo}

\begin{ejemplo}{ejem,distsobrecontl1} Sea $\mathcal{C}([0,1])$
el conjunto de funciones
continuas $f:[0,1]\rightarrow\mathbb{R}$. Entonces\linebreak
$(\mathcal{C}([0,1]),d)$ es un espacio métrico, donde:
\begin{equation}\label{l1metint}
	d(f,g):=\int_0^1|f(x)-g(x)|dx.
\end{equation}
\end{ejemplo}
\subsection{Bolas, esferas y diámetro}
Definido lo que es una métrica y un espacio métrico, pasamos a
definir algunas entidades de carácter geométrico, esta son el
concepto de \emph{bola}, \emph{esfera} y \emph{diámetro}.\index{Bola} \index{Esfera}  \index{Diámetro}

\begin{definicion}{} Sea $(X,d)$ un espacio métrico, $x\in X$ y
$r>0$.
\begin{itemize}
\item[a)]Definimos la bola abierta $B(x,r)$, con centro en $x$ y radio
$r$, por:
\[B(x,r):=\{y\in X:d(x,y)<r\}.\]
\item[b)]Definimos la esfera $E(x,r)$, con centro en $x$ y radio
$r$, por:
\[E(x,r):=\{y\in X:d(x,y)=r\}.\]
\end{itemize}
\end{definicion}

Todos tenemos una concepción de lo que entendemos por una bola,
quizas se nos venga a la mente, y de hecho es un ejemplo, un
círculo en $\mathbb{R}^2$. No obstante, debemos proceder con
cuidado. Estamos considerando métricas generales, ocurrirá que
en algunos espacios métricos las  bolas no se parecen a lo
que comunmente entendemos por este concepto. Esto es debido a que
en nuestra vida cotidiana estamos habituados a considerar la
métrica euclidea, pero en este curso trabajaremos con métricas
muy generales.

En $\mathbb{R}$, con la métrica dada por el módulo, la bola
centrada en $x\in\mathbb{R}$ y radio $r$, no es mas que el
intervalo $(x-r,x+r)$. En la figura \vref{ejembolas}  mostramos
varios ejemplos de bolas en diferentes métricas sobre
$\mathbb{R}^2$, las demostraciones las desarrollaremos en la
clase.


Todavía mas curiosas son las bolas respecto a la métrica
discreta. Sea $(X,d)$ un espacio métrico discreto y $x\in X$,
entonces:

\[B(x,r)=\left\{
\begin{array}{ll}
	\{x\}, & \hbox{si $r<1$;} \\
	X, & \hbox{si $r\geq 1$.} \\
\end{array}
\right.
\]

La esfera la podemos pensar como el borde de la bola, que no
está incluida en la bola abierta. También tenemos en este caso
situaciones que, en un primer momento, nos pueden parecer
extra\~nas. Como casi siempre, el mayor ``grado de
extra\~namiento'' se consigue con la métrica discreta. En este
caso, si $(X,d)$ es un espacio métrico discreto, tenemos:
\[E(x,r)=\left\{
\begin{array}{ll}
	\{x\}, & \hbox{si $r=0$;} \\
	X-\{x\}, & \hbox{si $r=1$;} \\
	\emptyset, & \hbox{si $r\neq 0$ y $r\neq 1$.} \\
\end{array}
\right.
\]

Pasamos a definir, ahora, el concepto de diámetro de un
conjunto.

\begin{definicion}{} Sea $(X,d)$ un espacio métrico y $A\subset
X$. Definimos el diámetro del conjunto $A$ por:
\[
	\delta(A):=\sup\limits_{x,y\in A}d(x,y).
\]
Eventualmente, podría ocurrir que $\delta(A)=+\infty$.
\end{definicion}
La figura \vref{diamconj} explica, por si sola, el significado del
concepto de diámetro.




\begin{definicion}{} Un conjunto no vacío $A$ se dirá acotado si
$\delta(A)<\infty$.
\end{definicion}
Es oportuno aclarar que el concepto de acotación depende del
conjunto en si mismo y de la métrica. Así puede ocurrir que
un mismo conjunto sea acotado con una métrica y con otra no.

\begin{ejemplo}{} En el espacio $\mathbb{R}$, con la métrica del
módulo, el conjunto $(0,+\infty)$ es no acotado. En cambio, con
la métrica discreta todo conjunto, y en particular el dado, lo
es.
\end{ejemplo}

También definiremos la distancia de un punto a un conjunto dado.

\begin{definicion}{} En un espacio métrico $(X,d)$ se define la
distancia de $x\in X$ a $A\subset X$ como
\[d(x,A):=\inf\limits_{y\in A}d(x,y).\]
\end{definicion}

Demostremos que
\[\delta(B(x,r))\leq 2r.\]
Efectivamente, dados $z$ e $y$ en la bola $B(x,r)$, tenemos, por
la desigualdad triangular
\[d(y,z)\leq d(y,x)+d(x,z)\leq 2r.\]
Tomando supremo sobre $z$ e $y$ obtenemos la afirmación. Notar
que ya no es cierto que $\delta(B(x,r))=2r$. En efecto, por
ejemplo si $(X,d)$ es un espacio métrico discreto, entonces
$\delta(B(x,1/2))=0$.

Ahora probaremos que la unión de conjuntos acostados es, a la
vez, un conjunto acotado.

\begin{proposicion}{} Sean $(X,d)$ un espacio métrico, $A$ y $B$
subconjuntos acotados de $X$. Entonces $A\cup B$ es acotado.
\end{proposicion}
\begin{demo}  Tenemos que probar que:
	\[\delta(A\cup B)<\infty.\]
Para esto, es suficiente demostrar que $\forall x,y\in A\cup B$
existe una constante $M$, independiente de $x$ e $y$, tal que:
\[d(x,y)\leq M.\]
Sean $z\in A$ y $w\in B$ dos cualesquiera puntos en los conjuntos
indicados. A travez de esta demostración estos puntos estaran
fijos, no importandonos que puntos sean, cualquiera conduce al
mismo argumento. Tomemos, ahora, $x,y\in A\cup B$ cualesquiera,
pero ya no estaran fijos. Si ocurriera que $x$ e $y$ estuvieran
simultaneamente en uno mismo de los conjuntos, supongamos $A$,
entonces tenemos que:
\[d(x,y)\leq \delta(A),\]
de modo que, en este caso, existe una constante $M$ con la
propiedad deseada. Debemos considerar el caso en que $x$ e $y$
esten en ``conjuntos diferentes'', digamos $x\in A$ e $y\in B$.
Entonces tenemos:
\[
	d(x,y)\leq d(x,z)+d(z,w)+d(w,y)\leq \delta(A)+d(z,w)+\delta(B).
\]
El miembro derecho, de la desigualdad anterior, es independiente
de $x$ e $y$, de modo que quedó demostrada la  proposición.
\end{demo}

\subsection{Conjuntos abiertos} Uno de los conceptos
más importantes, sino el más, de la Topología es el de
conjunto abierto.
\begin{definicion}{} Sea $(X,d)$ un e.m\footnote{Abreviación para espacio
métrico}. Diremos que $A\subset X$ es un conjunto abierto si
$\forall x\in A\exists r>0$ tal que:
\[
	B(x,r)\subset X.
\]
\end{definicion}
\marginnote{
\begin{center}
	\adjustimage{max size={0.9\linewidth}{0.9\paperheight}}{imagenes/conjabi.png}
	Conjuntos abiertos y no abiertos.
\end{center}
}
En la figura \vref{conjabi} podemos ver un ejemplo de conjunto
abierto, en $\mathbb{R}^2$ con la métrica euclidea, y otro que
no lo es. La diferencia es que en el conjunto b) el borde (en la
parte recta del conjunto) forma parte del mismo conjunto, entonces
si $x$ está en este borde, toda bola centrada en $x$ contiene
puntos fuera del conjunto.


Un ejemplo, esperable, de conjunto abierto lo constituyen las
bolas abiertas.
\begin{proposicion}{} Toda bola abierta es un conjunto abierto.
\end{proposicion}
\begin{demo} 
\marginnote{
\begin{center}
	\adjustimage{max size={0.9\linewidth}{0.9\paperheight}}{imagenes/grafbol.png}\\
	Construcción de $r'$.
\end{center}
}
Sea $x\in X$ y $r>0$. Consideremos la bola abierta
$B(x,r)$. Para demostrar que la bola es abierta, hay que
encontrar, para todo $y\in B(x,r)$, un $r'>0$ tal que
\begin{equation}\label{inclusion}
	B(y,r')\subset B(x,r).
\end{equation}
Sea, pues, $y\in B(x,r)$. Tomemos:
\[r':=r-d(x,y).\]
Ver la figura \vref{construccionrprima} para un gráfico de la
situación. Este $r'$ es mayor que cero. En efecto, como $y$
está en la bola, tenemos que $d(x,y)<r$.




Ahora, veamos la inclusión \vref{inclusion}. Sea $z\in B(y,r')$,
entonces tenemos, por la desigualdad triangular, que:
\[d(x,z)\leq d(x,y)+d(y,z)< d(x,y)+r'<r.\]
Así $z\in B(x,r)$, que es lo que queríamos demostrar.
\end{demo}

Ahora damos dos propiedades de conjuntos abiertos que tendrán
mucha trascendencia más adelante.
\begin{teorema}{teo,uniointerabiertos} Sea $I$ un conjunto de índices y
$\{A_i\}_{i\in I}$ una familia de conjuntos abiertos. Entonces:
\begin{itemize}
\item[a)] La unión $\bigcup_{i\in I}A_i$ es un conjunto abierto.
\item[b)] Si $I$ es finito, la intersección $\bigcap_{i\in
I}A_i$ es un conjunto abierto.
\end{itemize}
\end{teorema}
\begin{demo} Empecemos por la propiedad a). Sea $x$ un punto en la
unión, es decir existe algún índice $i_0$ tal que $x\in
A_{i_0}$. Como este $A_{i_0}$ es un conjunto abierto, deberá
existir $r>0$ tal que $B(x,r)\subset A_{i_0}$. Claramente la bola
$B(x,r)$, al ser un subconjunto de $A_{i_0}$ es un subconjunto de
la unión de todos los $A_i$, que es lo que teníamos que
probar.

Ahora veamos b). Podemos suponer que, para algún $n\in
\mathbb{N}$, tenemos que $I=\{1,\dots,n\}$. Sea $x$ un punto en la
intersección. En este caso, $x\in A_i$, para todo $i$. Como cada
$A_i$ es abierto, existen radios $r_i$ tales que $B(x,r_i)\subset
A_i$. Definamos:
\[
	r:=\min\{r_1,\dots,r_n\}.
\]
El mínimo existe, y es mayor que cero, pues hay una cantidad
finita de radios. Ahora tenemos que, como $r\leq r_i$,
$B(x,r)\subset B(x,r_i)\subset A_i$, para todo $i\in I$. Por
consiguiente $B(x,r)$ es un subconjunto de la intersección de
todos los $A_i$.
\end{demo}

Es interesante notar que, en un e.m. discreto $(X,d)$, todo
subconjunto $A\subset X$ es abierto. Efectivamente, en un e.m.
discreto $B(x,1/2)=\{x\}$ para todo $x\in X$. En particular, si
$x\in A$ entonces $B(x,1/2)\subset A$.

\subsection{Interior de un conjunto y entornos} Como es costumbre,
empezamos con una definición.
\begin{definicion}{} Sea $(X,d)$ un e.m. y $A\subset X$. Definimos
el interior de $A$, denotaremos este conjunto $A^0$, como el
conjunto de todos los puntos $x\in A$ tales que existe un $r>0$
que satisface $B(x,r)\subset A$.
\end{definicion}

Hay una gran similitud de esta definición con la de conjunto
abierto. De hecho se tiene que un conjunto $A$ es abierto si y
solo si $A=A^0$.

En $\mathbb{R}^2$ con la métrica euclidea podemos visualizar el
interior de un conjunto como la parte del conjunto que no está
sobre el borde de él, ver figura \vref{interiordeconj}.

\begin{figure}
\begin{center}
	\adjustimage{max size={0.9\linewidth}{0.9\paperheight}}{imagenes/interio.png}
	\caption{Interior de un conjunto}\label{interiordeconj}
\end{center}
\end{figure}

Tenemos una caracterización alternativa del interior de un
conjunto.

\begin{teorema}{teo,mayorabierto} El interior de un conjunto $A$, es el mayor abierto
contenido en $A$.
\end{teorema}
\begin{demo} El hecho de que $A^0$ es abierto y está contenido
en $A$, es consecuencia inmediata de la definición y lo dejamos
como ejercicio. Vamos a demostrar que es el mayor de los abiertos
contenido en $A$. Vale decir, hay que demostrar que si $B$ es un
abierto contenido en $A$, entonces $B\subset A^0$. Sea pues $B$
abierto y $B\subset A$. Tomemos $x\in B$. Como $B$ es abierto
existe un $r>0$ tal que $B(x,r)\subset B\subset A$. Así,
necesariamente $x\in A^0$. Lo que demuestra que $B\subset A^0$.
\end{demo}

Daremos algunas propiedades de la operación de tomar el interior
de un conjunto.
\begin{teorema}{} Sea $(X,d)$ un e.m., $A$ y $B$ subconjuntos de
$X$.
\begin{itemize}
\item[a)] $(A^0)^0=A^0$
\item[b)] Si $A\subset B$ entonces $A^0\subset B^0$.
\item[c)] $(A\cap B)^0=A^0\cap B^0$.
\end{itemize}
\end{teorema}
\begin{demo} a) Como dijimos, $A^0$ es abierto, por ende
$(A^0)^0=A^0$.

b)$A^0$ es un abierto y además está contenido en $B$, por
consiguiente $A^0\subset B^0$.

c) Como $A\cap B\subset A$ tenemos que, a acausa de b), $(A\cap
B)^0\subset A^0$. De la misma manera $(A\cap B)^0\subset B^0$. Por
consiguiente $(A\cap B)^0\subset A^0\cap B^0$. Para la otra
inclusión, tener en cuenta que $A^0\cap B^0$ es un abierto
contenido en $A\cap B$, por lo tanto $A^0\cap B^0\subset (A\cap
B)^0$.
\end{demo}
 Introducimos otro concepto.

\begin{definicion}{} En un e.m. el exterior de un conjunto $A$ es el
interior de su complemento. En símbolos ponemos
$\text{Ext}(A)=(A^c)^0$.
\end{definicion}


\begin{definicion}{} Sea $(X,d)$ un e.m. y $x\in X$. Diremos que $V$
es un entorno de $x$ si $x\in V^0$. También denotaremos por
$E(x)$ al conjunto de todos los entornos de $x$.
\end{definicion}

El anterior es otro de los conceptos claves de la topología.
Observemos que un conjunto abierto es entorno de cada uno de sus
puntos. La recíproca es también cierta, es decir si un
conjunto es entorno de cada uno de sus puntos entonces es abierto.

\begin{proposicion}{} La intersección de una cantidad finita de
entornos de un punto $x$ en un e.m. $(X,d)$ es, a su vez, un
entorno de $x$.
\end{proposicion}
\begin{demo} Sean $V_i$, $i=1,...,n$, entornos de $x\in X$. Por
definición $x\in V_i^0$ para todo $i=1,...,n$. Entonces $x\in
V_1^0\cap\dots\cap V_n^0=(V_1\cap\dots\cap V_n)^0$. De modo que
$V_1\cap\dots\cap V_n$ es un entorno de $x$. Así queda
establecida la propiedad que expresa la proposición.
\end{demo}

\subsection{Conjuntos cerrados y clausura de
conjuntos}

Ahora introduciremos el concepto de conjunto cerrado.
\begin{definicion}{} Un conjunto es cerrado si su complemento es
abierto.
\end{definicion}

Esta sencilla definición hace las nociones de conjunto cerrado y
abierto duales\footnote{Dos tipos de conceptos son duales cuando
cualquier afirmación sobre uno de ellos se convierte en una
afirmación sobre el otro. En este proceso de ``transformación
de enunciados'' hay que traducir cada concepto por su dual. Por
ejemplo, en el caso que nos ocupa, un conjunto cerrado muta en
abierto y las intersecciones mutan en uniones y viceverza. Uniones
e intersecciones son duales como consecuencia de las leyes de de
Morgan}, así veremos que cada propiedad de conjuntos abiertos
induce una correspondiente propiedad sobre conjuntos cerrados.
Tener en cuenta esto en la siguiente teorema.

\begin{teorema}{teo,uniointercerrados} Sea $I$ un conjunto de índices y
$\{F_i\}_{i\in I}$ una familia de conjuntos cerrados. Entonces:
\begin{itemize}
\item[a)] La intersección $\bigcap_{i\in I}F_i$ es un conjunto cerrado.
\item[b)] Si $I$ es finito, la unión $\bigcup_{i\in
I}F_i$ es un conjunto cerrado.
\end{itemize}
\end{teorema}
\begin{demo} La afirmaciones a) y b) de este teorema son duales de
las a) y b) del Teorema \vref{teo,uniointerabiertos}. Por ejemplo,
para demostrar a), observemos que, por definición, la siguiente
es una familia de conjuntos abiertos: $\{F_i^c\}_{i\in I}$. De
modo que por a) del Teorema \vref{teo,uniointerabiertos} tenemos
que:
\[\bigcup\limits_{i\in I}F_i^c\]
es un conjunto abierto. De allí que el complemento de este
conjunto es cerrado. Pero el complemento de este conjunto es, en
virtud de las leyes de de Morgan, la intersección de todos los
$F_i$. La propiedad b) se obtiene de la misma manera.
\end{demo}

Ejemplos de conjuntos cerrados son los intervalos cerrados de
$\mathbb{R}$, con la métrica del módulo; las bolas cerradas en
cualquier e.m., es decir los conjuntos de la forma:
\[B'(x,r):=\{y\in X: d(x,y)\leq r\}.\]
Las esferas también resultan ser conjuntos cerrados. Por otra
parte, como en un e.m. discreto todo conjunto es abierto, todo
conjunto, también, es cerrado. La demostración de que los
anteriores son conjuntos cerrados las dejamos como ejercicios. A
lo largo de esta materia veremos varios ejemplos mas de conjuntos
cerrados, encomendamos al estudiante prestar atención a ellos,
puesto que tan importante como aprender las definiciones y
propiedades de determinado concepto, es conocer, y poder construir
ejemplos de ese concepto.

El concepto de interior de un conjunto tiene su dual
correspondiente.
\begin{definicion}{} Sea $(X,d)$ un e.m.. La clausura de un conjunto $A\subset
X$ se define y denota como se ve a continuación:
\[\C{A}:=(\text{Ext}(A))^c=\bigl[(A^c)^0\bigr]^c.\]
\end{definicion}

 En $\rr^2$ con la métrica euclidea podemos visualizar la
clausura de un conjunto  como el conjunto más su ``borde'', ver
la figura \vref{fig,clausura}.

\begin{figure}
\begin{center}
	\adjustimage{max size={0.9\linewidth}{0.9\paperheight}}{imagenes/clausu.png}
	\caption{Clausura de un conjunto}\label{fig,clausura}
\end{center}
\end{figure}

Tenemos la siguiente caracterización alternativa de clausura de
un conjunto.
\begin{proposicion}{pro,clausura} Sea $(X,d)$ un e.m. y $A\subset X$. Son
equivalentes:
\begin{itemize}
\item[a)] $x\in \C{A}$.
\item[b)] $\forall r>0: B(x,r)\cap A\neq\emptyset$.
\end{itemize}
\end{proposicion}
\begin{demo}Veamos primero que a)$\Rightarrow$b). Sea $x\in \C{A}$. Por
definición $x\notin (A^c)^0$. Así, por definición de
conjunto interior, tenemos que para todo $r>0$, $B(x,r)\nsubseteq
A^c$. Es decir que para todo $r>0$ existe $y=y_r\in B(x,r)\cap A$.
Esto prueba b).

Veamos ahora que b)$\Rightarrow$a). Sea, pues, $x$ un punto
satisfaciendo la propiedad b). Toda bola de radio $x$ y centro
$r>0$ corta al conjunto $A$. De modo que no existe una de tales
bolas con la propiedad que este completamente contenida en el
conjunto $A^c$. Esto nos dice, por definicion de conjunto
interior, que $x$ no está en el interior de $A^c$. Dicho de otro
modo $x\in \bigl[(A^c)^0\bigr]^c$.
\end{demo}

Las propiedades del interior tienen propiedades duales
correspondientes para la clausura.

\begin{teorema}{} Sean $(X,d)$ un e.m., $A$ y $B$ subconjuntos de
$X$. Entonces tenemos que:
\begin{itemize}
\item[a)] $A\subset \C{A}$.
\item[b)] El conjunto $\C{A}$ es el menor conjunto cerrado que
contiene a $A$.
\item[c)] $\C{\C{A}}=\C{A}$.
\item[d)] Si $A\subset B$ entonces $\C{A}\subset \C{B}$.
\item[e)]$\C{A\cup B}=\C{A}\cup \C{B}$.
\item[f)]$x\in \C{A}\Leftrightarrow d(x,A)=0$
\end{itemize}
\end{teorema}
\begin{demo} Veamos a) cuya propiedad dual es que $C^0\subset C$.
En efecto, tenemos que:

\[(A^c)^0\subset A^c.\]
Ahora, tomando complementos a ambos miembros\footnote{La
operación de complemento invierte las inclusiones}, obtenemos
que:
\[\C{A}=\bigl[(A^c)^0\bigr]^c\supset (A^c)^c=A.\]
Esto prueba a).

Veamos b). El conjunto $\C{A}$ es cerrado pues es el complemento
del abierto $(A^c)^0$. Sea $F$ un conjunto cerrado que contiene a
$A$, hay que demostrar que $F\supset \C{A}$. Entonces, tomando
complemento, tenemos que $F^c$ es un abierto contenido en $A^c$.
Como $(A^c)^0$ es el mayor abierto contenido en $A^c$, tenemos que
$F^c\subset (A^c)^0$. Ahora tomemos complemento a esta última
inclusión y obtenemos
\[F\supset \bigl[(A^c)^0\bigr]^c=\C{A},\]
que es lo que queríamos demostrar.

Como corolario de b), obtenemos que  $A$ es cerrado si, y solo si,
$\C{A}=A$. A su vez, como corolario de esto, obtemos c) y d).

Veamos e). Tenemos que:

\begin{align}
  \C{A\cup B}&=\bigl[(A\cup B)^c)^0\bigr]^c         &\qquad &\text{Definición clausura}\notag \\
			 &=\bigl[(A^c\cap B^c)^0\bigr]^c    &&\text{Leyes de de         Morgan}\notag\\
			 &=\bigl[(A^c)^0\cap (B^c)^0)\bigr]^c &&\text{Propiedad dual del interior}\notag\\
			 &=\bigl[(A^c)^0\bigr]^c\cup \bigl[(B^c)^0)\bigr]^c &&\text{Leyes de de Morgan}\notag\\
			 &= \C{A}\cup \C{B}                     &&\text{Definición de
			 clausura}\notag
  \end{align}


Que es lo que queríamos demostrar.

Por último demostremos f). Si $x\in \C{A}$ entonces, como
consecuancia de la proposición \vref{pro,clausura}, tenemos que
para todo $n\in\nn$ existe un $y_n\in A$ tal que $d(x,y_n)<1/n$,
ver figura \vref{fig,incf)}.


Tenemos asi que
\[d(x,A)=\inf\limits_{y\in A}d(x,y)\leq d(x,y_n)\leq \frac1n.\]
Y como la desigualdad es válida para todo $n\in\nn$, obtenemos
que $d(x,A)=0$.

Recíprocamente, si $d(x,A)=0$ entonces, por definición del
ínfimo, para todo $r>0$ existe un $y=y_r\in A$ tal que
$d(x,y)<r$. Así tenemos que $B(x,r)\cap A\neq \emptyset$,
para todo $r>0$. Esto, como sabemos, es equivalente a afirmar que
$x\in \C{A}$.
\end{demo}

Por último estamos interesados en definir aquellos puntos que
estan en lo que hemos denominado, sin ninguna precisión, borde
de un conjunto.

\begin{definicion}{} Diremos que $x$ pertenece a la \emph{frontera} de un
conjunto $A$ cuando $x$ está en la clausura de $A$ y en la
clausura de $A^c$. Llamamos al conjunto de todos los puntos
frontera de $A$ la \emph{frontera }de $A$ y denotaremnos este
conjunto por $\partial A$. \end{definicion}


La costumbre de denotar la frontera de un conjunto con el signo de
una derivada proviene, suponemos, del calculo sobre variedades
donde se observa que cierta integral de una ``derivada'' sobre un
conjunto es igual a la integral de la función sobre la frontera
del conjunto. Este resultado se conoce como Teorema de Stokes. El
Teorema fundamenteal del Cálculo es un caso particular de este
teorema. Es en este contexto donde se consigue una conexión
entre derivadas y fronteras.






\subsection{Ejercicios}

\begin{ejercicio}{ejmeteuclidea} Demostrar que los siguientes son espacios métricos.

\begin{itemize}
	\item[a)] $(\mathbb{R},d)$ donde $d$ está definida en
	\vref{distmod}.
	\item[b)]  $(\mathbb{R}^n, d)$ donde $d$ está definida en \vref{meteuclidea}.
	\emph{Ayuda:} Usar
la desigueladad de Cauchy-Schwartz
$\forall\mathbf{x}\in\mathbb{R}^n\forall\mathbf{y}\in\mathbb{R}^n$:
\[
	\sum\limits_{i=1}^{n}x_iy_i\leq
	\sqrt{\sum\limits_{i=1}^{n}|x_i|^2}
	\sqrt{\sum\limits_{i=1}^{n}|y_i|^2}
\]
\item[c)] $(\mathbb{R}^n, d)$, donde $d$ es la función definida
en \vref{l1met}.

\item[d)]$(\mathbb{R}^n, d)$, donde $d$ es la función definida
en \vref{linfmet}.
\item[e)] Probar que la métrica discreta es, valga la
redundancia, una métrica.
\item[f)] Demostrar que las ecuaciones \vref{convunifmet} y
\vref{l1metint} definen métricas.
\end{itemize}
\end{ejercicio}

\begin{ejercicio}{deslipschitz} Sea $(X,d)$ un espacio métrico. Demostrar que
para todos $x$ , $y$ y $z$ en $X$ tenemos que:
\[|d(x,y)-d(x,z)|\leq d(y,z).\]
\end{ejercicio}

\begin{ejercicio}{daconjeslipschitz} Sea $(X,d)$ un espacio
métrico y $A\subset X$. Demostrar que:
\[|d(x,A)-d(y,A)|\leq d(x,y).\]
\end{ejercicio}

\begin{ejercicio}{ejer,distequiv} Sea $(X,d)$ un e.m., probar que las siguientes
funciones son métricas sobre $X$:
\begin{itemize}
\item[a)] $d_1(x,y):=\min\{1,d(x,y)\}$.
\item[b)] $d_2(x,y):=\frac{d(x,y)}{1+d(x,y)}$.
\end{itemize}
\end{ejercicio}

\begin{ejercicio}{} Sea $(X,d)$ un e.m.. Demostrar que $\forall
x,y\in X$, existe entornos $U\in E(x)$ y $V\in E(y)$ tales que
$U\cap V=\emptyset$.
\end{ejercicio}

\begin{ejercicio}{} Sea $(X,d)$ u e.m.. Demostrar las siguientes
propiedades:
\begin{itemize}
	\item[a)] Si $A\subset X$ es finito, entonces $X-A$ es abierto.
	\item[b)] Si $A\subset X$ es abierto, entonces para todo
	conjunto $B$ se tiene que $A\cap \C{B}\subset \C{A\cap B}$.
	\item[c)] Si $A$ es abierto entonces $A\subset (\C{A})^0$.
	\item[d)] Si $A$ es cerrado entonces $(\C{A})^0\subset A$.
	\item[e)]
	$(\C{A})^0=\C{\biggl(\C{\bigl((\C{A})^0\bigr)}\biggr)}$.
	\item[f)]
	$\C{(A^0)}=\C{\biggl(\C{\bigl(\C{(A^0)}\bigr)^0}\biggr)}$.
	\item[g)] $A^0=\bigl(\C{A^c}\bigr)^c$.
	\item[h)] $\partial A=\C{A}-A^0$.
	\item[i)] $\text{Ext}(A)=(\C{A})^c$.
	\item[j)] $\partial A^0\subset \partial A$ y $\partial
	\C{A}\subset \partial A$.
	\item[k)] $\partial (A\cup B)\subset \partial A\cup\partial
	B$, Si $\C{A}\cap\C{B}=\emptyset$ entonces vale la igualdad en
	la anterior inclusión.
	\item[l)] $d(A,B)=d(\C{A},\C{B})$, donde por definición:
	\[
		d(A,B):=\inf\limits_{x\in A,y\in B}d(x,y).
	\]
\end{itemize}
\end{ejercicio}
\begin{ejercicio}{} Dar ejemplos de:
\begin{itemize}
\item[a)] $A$ y $B$ abiertos de $\rr$ tales que los siguientes
conjuntos sean todos diferentes: $\C{A}\cap \C{B}$, $\C{A\cap B}$,
$\C{A}\cap B$, $A\cap \C{B}$.
\item[b)] $A$ y $B$ intervalos de $\rr$ tales que
$A\cap\C{B}\nsubseteq \C{A\cap B}$.
\item[c)] $A\subset \rr^2$ tal que $\partial A=A$.
\item[d)] $A$ y $B$ subconjuntos de $\rr^2$ tales que entre los
siguientes conjuntos no valga ninguna inclusión: $\partial
A\cup\partial B-\partial(A\cap B)$ y $\partial (A\cup B)$.
\end{itemize}
\end{ejercicio}
\begin{ejercicio}{} Demostrar que los siguientes conjuntos de
$\rr^2$ y $\rr^3$, son abiertos con la métrica euclidea:
\begin{itemize}
\item[a)] $\{(x,y)\in\rr^2: m<d\bigl((x,y),(0,0)\bigr)<n\}$, donde
$n,m\in\nn$ y $m<n$.
\item[b)] $\{(x,y)\in\rr^2: 0<x<1,\,\, 0<y<1,\,\,x\neq
\frac1n\,\,\forall n\in\nn\}$.
\item[c)] $\{(x,y,z)\in\rr^3: x,y,z\in\mathbb{Z}\}^c$.
\end{itemize}
\end{ejercicio}
\begin{ejercicio}{} Hallar la frontera y el diámetro del conjunto
$\{\frac1n:n\in\nn\}$.
\end{ejercicio}
\begin{ejercicio}{} Demostrar que el diámetro de la dola unitaria
en $\rr^2$ con la métrica euclidea es 2.
\end{ejercicio}
\begin{ejercicio}{} Un e.m. $(X,d)$ se dice ultramétrico si $d$ verifica la
desigualdad ultramétrica, es decir:
\[
	d(x,y)\leq\max\{d(x,z),d(z,y)\}.
\]
Sea $X$ un e.m. ultramétrico. Demostrar que:
\begin{itemize}
\item[a)] Si $d(x,y)\neq d(y,z)$ entonces
$d(x,z)=\max\{d(x,y),d(y,z)\}$.
\item[b)] Si $y\in B(x,r)$ entonces $B(x,r)=B(y,r)$. Como
consecuencia las bolas abiertas son también conjuntos cerrados.
\item[c)] Si $y\in\C{B(x,r)}$ entonces $\C{B(y,r)}=\C{B(x,r)}$.
Las bolas cerradas son, también, conjuntos abiertos.

\item[d)] Si dos bolas tienen intersección no vacía
entonces una está  contenidad en la otra.

\item[e)] La distancia de dos bolas abiertas distintas de radio
$r$, contenidas en una bola cerrada de radio $r$, es igual a $r$.
\end{itemize}
\end{ejercicio}

 \begin{ejercicio}{} Si $(X,d)$ es un e.m. demostrar que
 \[
	d'(x,y)=\log(1+d(x,y))
 \]
 define una nueva métrica sobre $X$. ?`Que tipo de conjunto son
 las bolas de esta métrica?
 \end{ejercicio}


% 
% %=================================================================
% %seccion II
% %==================================================================
% 


\section{Subespacios de un espacio métrico}

Sea $(X,d)$ un e.m. e $Y\subset X$. La métrica $d$ es una
función definida sobre $X\times X$, luego podemos considerar su
restricción a $Y\times Y$. Esta restricción también
cumplirá, es inmediato verlo,  los axiomas de una métrica. Por
este motivo, el par $(Y,d)$ es un e.m., por una abuso de\marginnote{
\begin{center}
    \adjustimage{max size={0.9\linewidth}{0.9\paperheight}}{imagenes/bolasub.jpg}
    Una bola en un subespacio
\end{center}
}
notación denotaremos la restricción de $d$ al conjunto
$Y\times Y$ por el mismo s\'{\i}mbolo $d$.  Diremos que $Y$ es un
subespacio de $X$.  Observar que la forma de las bolas en un
subespacio puede ser diferente que en el espacio total, como puede
verse en la figura \vref{fig,bolasub}. En este gráfico $X$ es el
espacio ``total'', $Y$ el subespacio y $x$ es un punto sobre la
frontera de $Y$, entonces la bola en $Y$ de centro $x$ y radio $r$
es la parte que quedo cuadriculada en el dibujo. Pongamos
$B_Y(x,r)$ para la bola en $Y$ de centro  $x$ y radio $r$,
entonces tenemos la relación:

\begin{equation}\label{eq,bolasub}
    B_Y(x,r):=\{y\in Y: d(x,y)<r\}=B(x,r)\cap Y,
\end{equation}
donde $B(x,r)$ es la bola en el espacio total.



El siguiente teorema nos dá una relación de los abiertos y
cerrados en $Y$ con los abiertos y cerrados en $X$.

\begin{teorema}{} Sea $(X,d)$ un e.m. e $Y\subset X$. Entonces:
\begin{itemize}
\item[a)] El conjunto $A$ es abierto en $Y$ si y solo si existe un
$G$ abierto en $X$ tal que $A=G\cap Y$.
\item[b)] El conjunto $C$ es cerrado en $Y$ si y solo si existe un
cerrado $F$ en $X$ tal que $C=Y\cap F$.
\end{itemize}
\end{teorema}
\begin{demo} Veamos, primero, la propiedad a). Sea $A$ un abierto
en $Y$. Para cada $x\in A$ existe, de acuerdo a la Ecuación
\vref{eq,bolasub}, un radio $r_x>0$ tal que:
\begin{equation}\label{eq,inclusiondebolas}
    B(x,r_x)\cap Y\subset A.
\end{equation}
Definamos:
\[
    G:=\bigcup\limits_{x\in A}B(x,r_x).
\]
El conjunto $G$ es abierto, pues la unión de conjuntos abiertos
resulta abierto. Además
\[
    G\cap Y=\bigcup\limits_{x\in A}B(x,r_x)\cap Y= A.
\]
La última igualdad es cierta por la ecuación
\vref{eq,inclusiondebolas} y por que cada $x\in A$ está en el
conjunto $B(x,r_x)\cap Y$. De modo que, encontramos el conjunto
que cumple la propiedad a).

Ahora la demostración de b) es sencilla de obtener. Sea $C$
cerrado en $Y$, en particular $C\subset Y$, entonces $Y-C$ es
abierto en $Y$. Por a) existe un abierto $G$ tal que:
\[
    Y-C=G\cap Y.
\]
Entonces
\[C=Y\cap G^c.\]
Como el conjunto $G^c$ es cerrado, obtenemos la tesis con $F=G^c$.
\end{demo}

\begin{proposicion}{pro,entornosub}Sea $(X,d)$ un e.m. e $Y\subset X$ un
subespacio. El conjunto $U\subset Y$ es un entorno de $x\in Y$ en
el espacio $(Y,d)$ si y solo si existe un entorno $V$ de $x$ en el
e.m. $(X,d)$ tal que $U=V\cap Y$.
\end{proposicion}
\begin{demo}
\marginnote{
\begin{center}
   \adjustimage{max size={0.9\linewidth}{0.9\paperheight}}{imagenes/entosub.jpg}
    Demostración de la Proposición
    \ref{pro,entornosub}
\end{center}
}
Si $U$ es un entorno de $x$ en $Y$, entonces $x$
está en el interior de $U$ relativo a $Y$ (pongamos $U^0_Y$ para
este conjunto). Como $U^0_Y$ es un abierto en $Y$, por el teorema
anterior, existe un abierto $W$ tal que $U^0_Y=Y\cap W$. Tomemos
$V=W\cup U$. El conjunto $V$ es un entorno de $x$ en $(X,d)$, pues
contiene al conjunto $W$ que lo es. Además $V\cap Y=U$, lo que
demuestra la aserción.
\end{demo}







\begin{proposicion}{} Sea $(X,d)$ un e.m. e $Y\subset X$ un
subespacio. Supongamos que $A\subset Y$. Entonces la clausura de
$A$ en el subespacio $Y$ (denotemos esto por $\overline{A}^Y$) es
igual a $\overline{A}^Y=\overline{A}\cap Y$.
\end{proposicion}
\begin{demo} El conjunto $\overline{A}\cap Y$ es un cerrado en $Y$
que contiene al conjunto $A$, de modo que
$\overline{A}^Y\subset\overline{A}\cap Y$. Veamos la otra
inclusión. Sea $x\in \overline{A}\cap Y$. Como $x\in
\overline{A}$ entonces para todo entorno $U$ de $x$, tenemos que
$U\cap A\neq\emptyset$. Como $A\subset Y$ tenemos que $(U\cap
Y)\cap A=U\cap A\neq\emptyset$. As\'{\i}, como $U\cap Y$ es un
entorno  arbitrario de $X$ en el subespacio $Y$, tenemos que
$x\in\overline{A}^Y$.
\end{demo}

\subsection{Ejercicios}

\begin{ejercicio}{} Sea $(X,d)$ un e.m., $A\subset X$ y $B\subset
A$.  Demostrar que $B^0\subset B^0_A$. Dar un ejemplo donde
$B^0\neq B^0_A$.
\end{ejercicio}

\begin{ejercicio}{} Sea $(X,d)$ un e.m., $B$ y $C$ subconjuntos de
$X$ y $A\subset C\cap B$. Demostrar que $A$ es abierto (cerrado)
en $B\cup C$ si, y solo si, es abierto (respectivamente cerrado)
en $B$ y $C$.
\end{ejercicio}

\begin{ejercicio}{} Sea $\{G_i\}_{i\in I}$ un cubrimiento por abiertos de un e.m.
$X$. Demostrar que $F\subset X$ es cerrado si, y solo si, $F\cap
G_i$ es cerrado en $G_i$ para todo $i\in I$.
\end{ejercicio}

\begin{ejercicio}{} Dar un ejemplo de un subespacio $A$ de $\rr^2$
tal que exista una bola abierta que es un conjunto cerrado, pero
no una bola cerrada, y una bola cerrada que es un conjunto
abierto, pero no una bola abierta. Ayuda: Considerar $A$ formado
por los puntos $(0,1)$, $(0,-1)$ y por un subconjunto apropiado
del eje $x$.
\end{ejercicio}



\section{Espacios separables}
\begin{definicion}{} Sea $(X,d)$ un e.m.. Un conjunto $A\subset X$
se dirá \emph{denso en }$B\subset X$ si $\overline{A}\supset B$.
Si el conjunto $A$ es denso en $X$ se dirá, brevemente, que $A$
\emph{es denso}.
\end{definicion}
\begin{ejemplo}{} $\mathbb{R}-\{0\}$, $\mathbb{Q}$ son densos en $\mathbb{R}$.
\end{ejemplo}
\begin{ejemplo}{} Sea $(X,d)$ un e.m discreto. Entonces $A$ es denso
si y solo si $A=X$. En efecto, tenemos que $\overline{A}=X$, de
modo que si $x\in X$  todo entorno de $x$ interseca al conjunto
$A$. De modo que  $B(x,1/2)\cap A\neq\emptyset$. Pero, como se
sabe, $B(x,1/2)=\{x\}$, de modo que $x\in A$. Esto demuestra que
$X=A$.
\end{ejemplo}

\begin{definicion}{} Un e.m. $(X,d)$ se dirá separable si tiene un
subconjunto denso y a lo sumo numerable.
\end{definicion}

\begin{ejemplo}{} Como se dijo $\mathbb{Q}$ es un conjunto denso,
además es numerable, por consiguiente $\mathbb{R}$ es separable.
\end{ejemplo}

\begin{ejemplo}{ej,rndenso} $\mathbb{R}^n$ con la métrica euclidea es
separable. Afirmamos que $\mathbb{Q}^n$ es un conjunto denso y
numerable. Para verlo, tomemos $x=(x_1,...,x_n)\in \mathbb{R}^n$ y
veamos que está en $\overline{\mathbb{Q}^n}$. Para ello es
suficiente probar que $B(x,r)\cap \mathbb{Q}^n\neq\emptyset$, para
todo $r>0$. Sea $r>0$ un radio, no es muy dificil demostrar, ver
la figura \vref{fig,rndenso}, la siguiente inclusión de un
``cubo'' en la bola:
\marginnote{
\begin{center}
    \adjustimage{max size={0.9\linewidth}{0.9\paperheight}}{imagenes/rndenso.jpg}
    Construcción del Ejemplo
    \ref{ej,rndenso}
    \end{center}
}
\[
    (x_1-\frac{r}{\sqrt{n}},x_1+\frac{r}{\sqrt{n}})\times\dots\times
    (x_n-\frac{r}{\sqrt{n}},x_n+\frac{r}{\sqrt{n}})\subset B(x,r).
\]
Como $\mathbb{Q}$ es denso en $\mathbb{R}$ existen racionales
$q_i\in (x_i-\frac{r}{\sqrt{n}},x_i+\frac{r}{\sqrt{n}})$,
$i=1,...,n$. En virtud de esto $(q_1,...,q_n)\in\mathbb{Q}^n\cap
B(x,r)$. Y as\'{\i} queda establecida la afirmación.


\end{ejemplo}

\begin{ejemplo}{} Un e.m. discreto $(X,d)$ es separable si y solo si
$X$ es a lo sumo numerable. Como vimos en un ejemplo anterior el
único conjunto denso que hay en un e.m. discreto es el total, de
modo que si el espacio es separable $X$ debe ser a lo sumo
numerable.
\end{ejemplo}

\begin{definicion}{} En un e.m. $(X,d)$, una familia de conjuntos
abiertos $\{G_i\}_{i\in I}$ se dirá \emph{base} si todo abierto
se puede obtener como unión de miembros de la familia. Más
precisamente, si $G$ es un abierto cualquiera existe un
subconjunto de sub\'{\i}ndices $J\subset I$ tal que:
\[
    G=\bigcup_{i\in J}G_i.
\]
\end{definicion}

\begin{ejemplo}{} En cualquier e.m. $(X,d)$ la familia de todas las
bolas es una base. También es una base la familia de todas las
bolas con radio igual a $1/n$ con $n\in \nn$. En efecto, si $G$ es
un abierto cualquiera, para todo $x\in G$ existe un $r_x>0$ tal
que $B(x,r_x)\subset G$. As\'{\i} podemos ver que
\[
    G=\bigcup\limits_{x\in G}B(x,r_x),
\]
lo que demuestra que $G$ lo podemos escribir como unión de
bolas. Para el otro caso elegimos un natural $n_x$ suficientemente
grande para que $1/n_x<r_x$.
\end{ejemplo}

\begin{proposicion}{pro,caracbase} Una familia de abiertos $\{G_i\}_{i\in I}$ es
una base si y solo si para todo $x\in X$ y para todo entorno $U\in
E(x)$, existe un $i\in I$ tal que:
\[
    x\in G_i\subset U.
\]
\end{proposicion}
\begin{demo} $\Rightarrow )$. Sea $x\in X$ y $U\in E(x)$. Como la familia es
base, tenemos que $U^0$ es unión de miembros de la familia.
Además, por definición, tenemos que $x\in U^0$, estos dos
hechos implican la tesis.

$\Leftarrow)$ Sea $G$ un abierto. Por hipótesis, para cada $x\in
G$ encontramos un $i_x\in I$ tal que $x\in G_{i_x}\subset G$.
As\'{\i} tenemos que:
\[
    G=\bigcup\limits_{x\in G}G_{i_x}.
\]
\end{demo}

\begin{teorema}{teo,basenum} Un e.m. es separable si y solo si existe una base
a lo sumo numerable.
\end{teorema}
\begin{demo}$\Leftarrow )$. Sea $\{G_n\}_{n\in\nn}$ una base numerable de
abiertos (si hubiera una base finita el razonamiento es
idéntico). Elijamos $a_n\in G_n$. El conjunto
$D:=\{a_n:n\in\nn\}$ es, entonces, a lo sumo numerable (?` Por
qué?). Además, veamos que es denso. Efectivamente, sea $x\in
X$ un punto arbitrario y $U\in E(x)$. Como consecuencia de la
Proposición \vref{pro,caracbase} existe un $n\in\nn$ tal que
$x\in G_n\subset U$. Ahora tenemos el punto $a_n\in G_n$, y por
ello $U\cap D\neq\emptyset$. Probamos as\'{\i} que todo entorno de
$x$ interseca a $D$, en consecuencia $x\in \overline{D}$. Como el
$x$ es arbitrario, aquello prueba que $D$ es un conjunto denso.


$\Rightarrow )$. Sea $D$ un conjunto denso y a lo sumo numerable.
Definamos la siguiente familia de bolas abiertas:
\[
   \mathcal{A}:= \{B(x,\frac1n):x\in D \wedge n\in\nn\}.
\]
Esta es una familia a lo sumo numerable, pues la siguiente
función
\marginnote{
\begin{center}
    \adjustimage{max size={0.9\linewidth}{0.9\paperheight}}{imagenes/basenum.jpg}
    \small Demostración del Teorema
    \ref{teo,basenum}
\end{center}
}

\begin{eqnarray}
    T:\nn\times D&\longrightarrow\mathcal{A}\nonumber \\
    (n,x)&\longmapsto B(x,\frac1n)\nonumber \\
\end{eqnarray}
es suryectiva.


Veamos que la familia propuesta es una base de abiertos usando la
Proposición \vref{pro,caracbase}. Sea $x\in X$ y $U\in E(x)$.
Como $x\in U^0$, podemos elejir $r>0$ tal que $B(x,r)\subset U$.
Sea, ahora, $n\in\nn$ suficientemente grande, de modo que $2/n<r$.
Como $D$ es denso debe existir un $a\in D$ tal que $a\in
B(x,\frac1n)$. Observesé que tenemos que $x\in B(a,1/n)$, ver
Figura \ref{fig,basenum}. Además, tenemos que $B(a,1/n)\subset
B(x,r)\subset U$. Para demostrarlo, tomemos $y\in B(a,1/n)$.
Entonces
\[
    d(x,y)\leq d(x,a)+d(a,y)<\frac1n+\frac1n=\frac2n<r.
\]
Tenemos as\'{\i} que $x\in B(a,1/n)\subset U$, como $B(a,1/n)$ es
un elemento de la familia propuesta, tenemos probada la propiedad
de la Proposicion \vref{pro,caracbase} y, de este modo, la familia
propuesta resulta una base.
\end{demo}

\begin{corolario}{} Un subespacio de un espacio separable es
separable.
\end{corolario}
\begin{demo} Sea $(X,d)$ un e.m. e $Y\subset X$. Sea $\{G_n\}_{n\in I}$
una base a lo sumo numerable de abiertos. Es fácil demostrar que
la familia $\{G_n\cap Y\}_{n\in I}$ es una base de los abiertos de
$Y$.
\end{demo}

\subsection{Ejercicios}

\begin{ejercicio}{} Sea $(X,d)$ un e.m. y $A\subset X$. Demostrar
que $A\cup \text{Ext}(A)$ es denso en $A$. ?`Será cierto que
$A^0\cup \text{Ext}(A)$ es, siempre, denso?
\end{ejercicio}

\begin{ejercicio}{} Demostrar que $\mathbb{I}:=\rr-\mathbb{Q}$ es
separable. Exhibir un conjunto denso numerable.
\end{ejercicio}

\begin{ejercicio}{} Sea $A\subset \rr$. Definamos $B:=\{x\in
A|\exists y>x: (x,y)\cap A=\emptyset\}$. Demostrar que $B$ es a lo
sumo numerable.
\end{ejercicio}

\begin{ejercicio}{} Sea $(X,d)$ un e.m. y $A\subset X$. Diremos que
$a\in A$ es un \emph{punto aislado} de $A$ si existe un entorno
$U$ de $a$ tal que $U\cap A=\{a\}$. En la Figura
\vref{fig,aislado} el conjunto $A$ consiste de la parte sombreada
y el punto $a$, este último es un punto aislado, pues el entorno
$U$ satisface la definición.




Por otra parte, un punto $a\in X$ es un \emph{punto de
acumulación} de $A$ si, para todo entorno $U$ de $a$ se tiene
que $(U-\{a\})\cap A\neq\emptyset$.

Sea $A$ un conjunto, $B$ el conjunto de puntos de acumulación de
$A$ y $C$ el conjunto de puntos aislados de $A$. Demostrar los
siguientes items:
\begin{itemize}
    \item[a)] $B$ es cerrado, y $\overline{A}=B\cup C$.
    \item[b)] Si $X$ es separable entonces $C$ es numerable.
\end{itemize}
\end{ejercicio}
\marginnote{
\begin{center}
    \adjustimage{max size={0.9\linewidth}{0.9\paperheight}}{imagenes/aislado.jpg}
    \small Punto aislado
\end{center}
}
\begin{ejercicio}{} Demostrar que $(X,d)$ es separable si y solo si
 todo cubrimiento de $X$ por abiertos\footnote{Un cubrimiento por
abiertos de $X$ es una familia de conjuntos abiertos
$\{G_i\}_{i\in I}$ tal que $X=\bigcup_{i\in I}G_i$.} tiene un
subcubrimiento a lo sumo numerable\footnote{Es decir existe una
subfamilia a lo sumo numerable de la familia $\{G_i\}$ que
también es un cubrimiento.}. \emph{Ayuda:} Sea $\{U_i\}_{i\in
I}$ un cubrimiento de $X$ y $\{G_n\}_{n\in\nn}$ una base
numerable. Para cada $n\in\nn$ elegir un $i_n$ tal que $G_n\subset
U_{i_n}$. Luego la familia $\{U_{i_n}\}_{n\in\nn}$ será un
cubrimiento.
\end{ejercicio}



\section{Funciones Continuas}
Vamos a ver que, en el contexto de los espacios métricos,
podemos definir el concepto de que una función sea continua.

\marginnote{
\begin{center}
   \adjustimage{max size={0.9\linewidth}{0.9\paperheight}}{imagenes/funccon.jpg}
    \small Definición de función
    continua
\end{center}
} 
\begin{definicion}{} Sean $(X,d)$, $(Y,d')$ dos e.m, $f:X\rightarrow
Y$ una función y $x\in X$.
Diremos que $f$ \emph{es continua en}
$x$ si para todo entorno  $V\in E(f(x))$, existe un entorno de
$U\in E(x)$ tal que $f(U)\subset V$ (ver Figura
\vref{fig,funccon}). Diremos que $f:X\rightarrow Y$ es
\emph{continua} si es continua en cada punto de $X$.
\end{definicion}



Algunas veces es más práctico emplear las siguientes
equivalencias de la definición de función continua en un
punto.

\begin{proposicion}{pro,caraccontpunto} Sean $(X,d)$, $(Y,d')$ dos e.m, $f:X\rightarrow
Y$ una función y $x\in X$. Entonces son equivalentes:
\begin{itemize}
\item[i)] $f$ es continua en $x$.
\item[ii)] Para todo entorno $V$ de $f(x)$, $f^{-1}(V)$ es un
entorno de $x$.
\item[iii)]Para todo $\epsilon>0$ existe un $\delta>0$ tal que
si $d(x,y)<\delta$ entonces  $d'(f(x),f(y))<\epsilon$.
\end{itemize}
\end{proposicion}
\begin{demo} i)$\Rightarrow$ii). Sea $V$ un entorno  de $f(x)$.
En virtud de la definición, existe un entorno $U$ de $x$ tal que
$f(U)\subset V$. As\'{\i} tenemos que $U\subset f^{-1}(V)$ y como
$U$ es un entorno de $x$, $f^{-1}(V)$ también lo és.

ii)$\Rightarrow$iii). Sea $\epsilon>0$. La bola $B(f(x),\epsilon)$
es un entorno de $f(x)$, as\'{\i}, por ii), el conjunto
$U:=f^{-1}(B(f(x),\epsilon))$ es un entorno de $x$. Entonces $x\in
U^0$, lo que implica que existe un $\delta>0$ tal que
$B(x,\delta)\subset f^{-1}(B(f(x),\epsilon))$. Esta inclusión es
otra forma de afirmar iii).

iii)$\Rightarrow$i). Sea $V$ un entorno de $f(x)$. Entonces existe
$\epsilon>0$ tal que $B(f(x),\epsilon)\subset V$. Por iii), existe
un $\delta>0$ tal que si $d(x,y)<\delta$ entonces
$d(f(x),f(y))<\epsilon$. Esto afirma que $f(B(x,\delta))\subset
B(f(x),\epsilon)$. Como $B(f(x),\epsilon)\subset V$, tenemos que
$f(B(x,\delta))\subset V$. Pero $B(x,\delta)$ es un entorno de
$x$, de modo que hemos establecido que $f$ es continua en $x$.
\end{demo}
\begin{ejemplo}{} Sea $f:X\rightarrow Y$ una función entre
e.m.. Si $(X,d)$ es discreto entonces $f$ es continua. Vale decir
si el dominio de una función es un e.m. discreto la función es
continua, no importa que función sea ni, que sea el codominio.
En efecto, sea $x\in X$ y $V$ un entorno de $f(x)$. Como todo
conjunto en un e.m. es abierto, $f^{-1}(V)$ es un abierto,
además contiene a $x$, de este modo es un entorno de $x$, lo que
demuestra la condición ii) de la Proposición
\vref{pro,caraccontpunto}.
\end{ejemplo}

\begin{ejemplo}{} Sea $(X,d)$ un e.m. e $Y$ un subespacio de $X$. La
\emph{inyección natural} $j:Y\rightarrow X$, definida por
$j(x)=x$ es una función continua, como se puede corroborar
facilmente, quedando esta demostración como ejercicio.
\end{ejemplo}

\begin{ejemplo}{} Las funciones constantes son continuas, es decir:
sea $(X,d)$ y $(Z,d')$ dos e.m. y $f:X\rightarrow Z$ definida por
$f(x)=a$, donde $a$ es un punto de $Z$, entonces $f$ es continua.
La demostración queda como ejercicio.
\end{ejemplo}

\begin{proposicion}{pro,caraccont}Sea $f:X\rightarrow Y$ continua en $x$. Supongamos que $x\in
\overline{A}$ entonces $f(x)\in\overline{f(A)}$.
\end{proposicion}

\begin{demo} Sea $V$ un entorno de $f(x)$, hay que demostrar que
$V\cap f(A)\neq\emptyset$. Pero, como $f$ es continua en $x$,
$f^{-1}(V)$ es un entorno de $x$. Ahora, ya que
$x\in\overline{A}$, tenemos que $f^{-1}(V)\cap A\neq\emptyset$.
Sea, pues, $y\in f^{-1}(V)\cap A$. As\'{\i}, tenemos que $f(y)\in
V\cap f(A)$. Luego $V\cap f(A)\neq\emptyset$.
\end{demo}

Ahora vamos a dar una serie de equivalencias a que una función
sea globalmente continua.

\begin{teorema}{teo,caraccont} Sean $(X,d)$ e $(Y,d')$ e.m. y $f:X\rightarrow Y$
una función. Los siguientes enunciados son equivalentes:
\begin{itemize}
\item[a)] $f$ es continua.
\item[b)]  Si $A\subset Y$ es un abierto de $Y$, entonces $f^{-1}(A)$ es
abierto de $X$.
\item[c)] Si $A\subset Y$ es un cerrado de $Y$, entonces $f^{-1}(A)$ es
un cerrado de $X$.
\item[d)] Para todo subconjunto $A\subset X$ se tiene que
$f(\overline{A})\subset \overline{f(A)}$.
\end{itemize}
\end{teorema}
\marginnote{
\begin{center}
   \adjustimage{max size={0.9\linewidth}{0.9\paperheight}}{imagenes/fclausu.jpg}
    \small Interpretación del inciso d) del Teorema
    \ref{teo,caraccont}
\end{center}
}
\begin{demo} La Proposición \vref{pro,caraccont} establece
a)$\Rightarrow$d). Veamos que d)$\Rightarrow$c). Sea $A$ cerrado
en $Y$ y $A'=f^{-1}(A)$. Entonces

\begin{equation}\label{eq,caraccont}
  \begin{split}
       f(\overline{A'})&\subset\overline{f(A')}\quad\quad\,\,\,\text{Hipótesis}\\
       &\subset\overline{A}\quad\quad\quad\quad\text{definición de $A'$}\\
        &=A\quad\quad\quad\quad\text{$A$ es cerrado}\\
  \end{split}
\end{equation}
Luego

\[
    \begin{split}
        \overline{A'}&\subset f^{-1}(f(\overline{A'}))\quad\text{Propiedad de la
        función imagen}\\
        &\subset f^{-1}(A)\,\,\,\quad\quad\text{Inclusión
        \ref{eq,caraccont}}\\
        &=A'\quad\quad\quad\quad\,\,\,\,\text{Definición de $A'$}
    \end{split}
\]
Por otro lado, como es sabido, $A'\subset \overline{A'}$, luego
$\overline{A'}=A'$, lo que implica que $A'$ es cerrado.

Ahora veamos que c)$\Rightarrow$b). Sea $A$ abierto en $Y$.
Entonces $A^c$ es cerrado en $Y$. Entonces, por c), $f^{-1}(A^c)$
es cerrado en $X$. Pero, $f^{-1}(A^c)=\bigl(f^{-1}(A)\bigr)^c$.

Por último veamos que b)$\Rightarrow$a). Sea $x\in X$ y $V$ un
entorno de $f(x)$, entonces $f(x)\in V^0$. Por hipótesis
$f^{-1}(V^0)$ es un abierto que contiene a $x$. De este modo
$f^{-1}(V^0)$ es un entorno de $x$. Como $f^{-1}(V^0)\subset
f^{-1}(V)$ tenemos que $f^{-1}(V)$ es un entorno de $x$ también.
Lo que prueba que $f$ es continua en $x$.
\end{demo}

La propiedad d) tiene una interpretación gráfica. Expresa el
hecho que si un punto $a$ está ``pegado'' a un conjunto $A$ (en
el sentido que $a\in \overline{A}$) entonces $f(a)$ está
``pegado'' a $f(A)$, ver Figura \ref{fig,interpretacionteo}. Esto
es as\'{\i} pues las funciones continuas aplican ``puntos
próximos'' en ``puntos próximos'', y, al decir que
$a\in\overline{A}$ estamos diciendo que $a$ ``está próximo''
al conjunto $A$.



Ahora veamos que la composición de funciones continuas es
continua.

\begin{proposicion}{} Sean $(X,d)$, $(Y,d')$, $(Z,d'')$ tres e.m.,
$f:X\rightarrow Y$ y $g:Y\rightarrow Z$ funciones tales que $f$ es
continua en $a\in X$ y $g$ es continua en $f(a)\in Y$. Entonces
$g\circ f:X\rightarrow Z$ es continua en $a$.
\end{proposicion}
\begin{demo} Sea $W$ un entorno de $g(f(a))$. Como $g$ es continua
en $f(a)$ entonces $V:=g^{-1}(W)$ es un entorno de $f(a)$. Luego,
como $f$ es continua en $a$, $f^{-1}(V)=f^{-1}(g^{-1}(W))$ es un
entorno de $a$. Esto implica la tesis, pues
$f^{-1}(g^{-1}(W))=(g\circ f)^{-1}(W)$.
\end{demo}

\begin{corolario}{} Sea $f:X\rightarrow Y$ una función continua
en $a$. Supongamos que $Z\subset X$ es un subespacio con $a\in Z$.
Entonces la restricción de $f$ al subespacio $Z$, con la
métrica de subespacio, es continua en $a$.
\end{corolario}
\begin{demo} La susodicha restricción es la composición de $f$
con la inyección natural $j:Z\rightarrow X$. Por lo tanto el
resultado sigue del hecho que la composición de funciones
continuas es continua.
\end{demo}

Otro concepto importante es el de función uniformemente
continua.

\begin{definicion}{} Sea $f$ una función entre dos e.m. $(X,d)$ e $(Y,d')$.
Diremos que $f$ es \emph{uniformemente continua} si para todo
$\epsilon>0$ existe un $\delta>0$ tal que
$d'(f(x),f(y))<\epsilon$, si $d(x,y)<\delta$.
\end{definicion}

No es facil entender la diferencia de esta definición con la que
expresa que $f$ es continua en cada punto de $X$. La diferencia es
que el $\delta$ de esta definición es el mismo para todos los
puntos de $X$. Mientras que decir que $f$ es continua en cada
punto de $X$ implicar\'{\i}a, en principio, la existencia de un
delta  que puede depender del punto. Los siguientes ejemplos
aclararan más esta definición.


\begin{ejemplo}{} Las funciones constantes son uniformemente
continuas. Dado un $\epsilon>0$ podemos tomar cualquier valor de
$\delta$ que seguramente cumplirá la definición.
\end{ejemplo}

\begin{ejemplo}{} Una función puede ser continua en todo punto y,\marginnote{
\begin{center}
   \adjustimage{max size={0.9\linewidth}{0.9\paperheight}}{imagenes/nounco.jpg}
    \small Función no uniformemente
    continua
\end{center}
}
sin embargo, no ser uniformemente continua,, como muestra el
siguiente ejemplo: Sea $f:\rr\rightarrow\rr$ definida por
$f(x)=x^2$. Esta $f$ es continua en todo punto y no uniformemente
continua. En efecto, la diferencia $(a+h)^2-a^2=2ah+h^2$ tiende a
$+\infty$ si $a$ tiende a $+\infty$. De modo que asegurar que
$h<\delta$ no implica que las imagenes de $a+h$ y $a$ esten cerca,
no importando, para ello, cuan chico sea $\delta$. Ver Figura
\ref{fig,nouncon}.
\end{ejemplo}

Si una función es uniformemente continua es continua en cada
punto. La demostración de este hecho es bastante directa y
simple.

\begin{ejemplo}{} Sea $(X,d)$ un e.m. y $A\subset X$. La función:
\[
    \begin{split}
        f:&X\longrightarrow \rr\\
        &x\longmapsto d(x,A).
    \end{split}
\]
es uniformemente continua. Esto es consecuencia de la desigualdad
probada en el Ejercicio \vref{daconjeslipschitz}, a saber:
\[
    |d(x,A)-d(y,A)|\leq d(x,y).
\]
\end{ejemplo}

\subsection{Ejercicios}



\begin{ejercicio}{} Sean $(X,d)$, $(Y,d')$ e.m., $A$ y $B$
subconjuntos de $X$ tales que $A\cup B=X$.
\begin{itemize}
\item[i)] Sea $f:X\rightarrow Y$ una función tal que
$f_{|A}$\footnote{$f_{|A}$denota la restricción de $f$ al
conjunto $A$} y $f_{|B}$ son ambas continuas en $x\in A\cap B$,
probar que $f$ es continua en $x$.
\item[ii)] Dar un ejemplo de función tal que $f_{|A}$, $f_{|B}$ y $f_{|A\cap
B}$sean continuas pero $f$ no lo sea.
\end{itemize}
\end{ejercicio}

\begin{ejercicio}{} Sean $(X,d)$, $(Y,d')$ e.m. y $f:X\rightarrow Y$
una función. Demostrar que son equivalentes:
\begin{itemize}
\item[i)] $f$ es continua.
\item[ii)] Para todo $B\subset Y$: $f^{-1}(B^0)\subset
\bigl[f^{-1}(B)\bigr]^0$.
\item[iii)]Para todo $B\subset Y$: $\overline{f^{-1}(B)}\subset
f^{-1}(\overline{B})$.
\end{itemize}
Dar un ejemplo de función continua donde
$\overline{f^{-1}(B)}\neq f^{-1}(\overline{B})$.
\end{ejercicio}

\begin{ejercicio}{} Sean $(X,d)$, $(Y,d')$ e.m. y $f,g:X\rightarrow Y$
 funciones continuas. Demostrar que:
 \begin{itemize}
    \item[i)]  El conjunto $\{x\in X| f(x)=g(x)\}$ es cerrado.
    \item[ii)] Si $f$ y $g$ coinciden en un conjunto denso
    entonces son iguales.
 \end{itemize}
 \end{ejercicio}

\begin{ejercicio}{} Sean $(X,d)$, $(Y,d')$ e.m.. Demostrar que son equivalentes
\begin{itemize}
    \item[i)] Toda función $f:X\rightarrow Y$ es continua.
    \item[ii)] Todo punto de $X$ es aislado\footnote{Por abuso de
    lenguaje los e.m. con esta propiedad se denominan discretos}.
\end{itemize}
\end{ejercicio}


\section{Homeomorfismos e isometr\'{\i}as}

\begin{definicion}{} Sean $(X,d)$, $(Y,d')$ dos e.m. y
$f:X\rightarrow Y$ una función \emph{biyectiva}. Diremos que $f$
es un \emph{homeomorfismo} si $f$ y $f^{-1}$ son ambas continuas.
Dos e.m. tales que exista un homeomorfismo entre ellos se
denominaran homeomorfos.
\end{definicion}

\begin{ejemplo}{} Dos intervalos abiertos cualesquiera de $\rr$ son
homeomorfos, uno puede construir una función lineal, que son
homeomorfismos, que aplique uno en el otro. Mientras que un
intervalos abierto  cualquiera $(a,b)$ es homeomorfo a $\rr$. Un
homeomorfismo entre ambos es la función:
\[
    \begin{split}
        f:&(a,b)\longrightarrow\rr\\
          &x\longmapsto
          \text{tan}\biggl(\pi\frac{2x-(a+b)}{2(b-a)}\biggr)
    \end{split}
\]
\end{ejemplo}

 \marginnote{
\begin{center}
    \adjustimage{max size={0.9\linewidth}{0.9\paperheight}}{imagenes/graffun.jpg}
    \small Grafico de la función $f(x)=x/(1+|x|)$
\end{center}
}
\begin{ejemplo}{}Un intervalo cerrado ya no es homeomorfo a $\rr$,
esto lo demostraremos más adelante. No obstante podemos definir
la \emph{recta real extendida} que será homeomorfa a los
intervalos cerrados. Más precisamente, sea $f:\rr\rightarrow
(-1,1)$ la función $f(x)=x/(1+|x|)$. No es difícil demostrar
que $f$ es biyectiva, de hecho analizando esta función con las
herramientas aprendidas en Cálculo I vemos que tiene la forma de
la Figura \vref{fig,graffunc}. Definamos el conjunto
$\overline{\rr}$, al que llamaremos \emph{recta extendida}, como
la unión de $\rr$ con dos nuevos elementos, a los que llamaremos
$-\infty$ y $+\infty$. Ahora extendemos $f$ de $\overline{\rr}$ al
$[-1,1]$ por $f(+\infty)=1$ y $f(-\infty)=-1$. Definimos la
función $d:\overline{\rr}\times\overline{\rr}\rightarrow \rr$
por: \begin{equation}\label{eq,defiso}
    d(x,y)=|f(x)-f(y)|.
\end{equation}


La función $d$ es una métrica en $\overline{\rr}$ (la sencilla
demostración la desarrollaremos en clase). Con esta métrica el
conjunto $\rr$ es acotado, de hecho $\delta(\rr)=2$. Además la
función $f$ resulta un homeomorfismo de $\overline{\rr}$ en
$[-1,1]$ (este último con la métrica del módulo). En efecto,
en virtud de la ecuación \vref{eq,defiso}, dado $\epsilon>0$
basta elegir $\delta=\epsilon$ para verificar que $f$ es
uniformemente continua. Si llamamos $g$ a la inversa de $f$ y
reemplazamos $x$ e $y$ en \vref{eq,defiso} por $g(t)$ y $g(s)$
respectivamente, comprobamos que
\begin{equation}\label{eq,defisoinv}
    d(g(t),g(s))=|t-s|.
\end{equation}
Lo cual implica que $g$ es uniformemente continua, por razones
similares a las que invocamos para $f$. En particular $f$ y su
inversa son continuas, de modo que $f$ es un homeomorfismo.


\end{ejemplo}

En el ejemplo anterior las funciones $f$ y $g$ tienen una
propiedad más fuerte que la de ser homeomorfismos, esta
propiedad la definimos a continuación.
\begin{definicion}{} Sean $(X,d)$, $(Y,d')$ dos e.m. y
$f:X\rightarrow Y$ una función biyectiva. Se dirá que $f$ es
una \emph{isometr\'{\i}a} si para todos $x$ e $y$ en $X$ se tiene
que:
\[
    d'(f(x),f(y))=d(x,y).
\]
Si, entre dos e.m. existe una isometr\'{\i}a diremos que los
espacios son \emph{isométricos}.
\end{definicion}

Una isometr\'{\i}a es un homeomorfismo, la idea central de la
demostración de esta afirmación está en el ejemplo anterior.
Igual que en aquel ejemplo, hay que demostrar que la inversa de
una isometr\'{\i}a es, a la vez, una isometr\'{\i}a.

\begin{proposicion}{} Sean $(X,d)$, $(Y,d')$ dos e.m. y
$f:X\rightarrow Y$ una función biyectiva con inversa $g$.
Entonces son equivalentes:
\begin{itemize}
\item[i)] $f$ es un homeomorfismo.
\item[ii)] $A\subset X$ es abierto si, y solo si, $f(A)$ es
abierto.
\end{itemize}
\end{proposicion}
\begin{demo} i)$\Rightarrow$ii). Sea $A\subset X$. Supongamos, en primer lugar,
que $A$ es abierto. Como $g:Y\rightarrow X$ es continua,
$g^{-1}(A)=f(A)$ es abierto. Supongamos, ahora, que $f(A)$ es
abierto. Como $f$ es continua, $f^{-1}(f(A))=g(f(A))=A$ es
abierto. Esto concluye la demostración de la primera
implicación.

ii)$\Rightarrow$i). Tenemos que demostrar que $f$ y $g$ son
continuas. Veamos, primero, que $f$ es continua. Sea $B$ un
abierto de $Y$, hay que demostrar que $f^{-1}(B)=g(B)$ es un
abierto de $X$. Pero $B=f(g(B))$ y $B$ es abierto, entonces, por
ii), $g(B)$ es abierto. Veamos, ahora, que $g$ es continua. Sea
$A$ abierto en $X$. Luego, por ii), $g^{-1}(A)=f(A)$ es abierto en
$Y$, por lo cual, $g$ es continua.
\end{demo}

Este teorema nos dice que si dos espacios son homeomorfos,
entonces existe una correspondencia de los abiertos de uno con los
del otro espacio.

El conjunto formado por todos los conjuntos abiertos, se denomina
\emph{topolog\'{\i}a}. Brevemente, digamos que un \emph{espacio
topológico} es un par $(X,\tau)$, donde $\tau\subset
\mathcal{P}(X)$\footnote{$\mathcal{P}(X)$ es el conjunto de partes
de $X$, es decir $\tau$ es un conjunto cuyos elementos son
subconjuntos de $X$}, que satisface los siguientes axiomas:
\begin{itemize}
    \item[1)] $\emptyset, X\in \tau$
    \item[2)] Si $G_i\in\tau$, para $i\in I$, entonces
    $\bigcup_{i\in I}G_i\in \tau$.
    \item[3)] Si $G_i\in\tau$, para $i\in I$, e $I$ es finito, entonces
    $\bigcap_{i\in I}G_i\in \tau$.
\end{itemize}

En un espacio topológico uno puede construir la nociones, que
hemos construido para e.m., por ejemplo conjunto cerrado,
interior, clausura,  entorno, función continua y espacio
separable. Por tanto, estas propiedades se denominan
topológicas. Las propiedades topológicas son invariantes por
homeomorfismos, por ejemplo si un espacio es separable, cualquier
homeomorfo a él también lo es. Algunas propiedades no son
topológicas, por ejemplo que una función sea uniformemente
continua, puesto que para definir este concepto necesitamos de una
métrica.

Un mismo conjunto $X$, puede tener dos métricas distintas, por
ejemplo en $\rr$ tenemos la métrica euclidea y la discreta.
Podemos plantearnos que estas métricas den origen a una misma
topolog\'{\i}a, si esto sucede diremos que las dos
\emph{métricas son equivalentes}.


\subsection{Ejercicios}


\begin{ejercicio}{} Sean $(X,d)$, $(Y,d')$ e.m. y $f:X\rightarrow Y$
 una función biyectiva. Demostrar que $f$ es un homeomorfismo
 si, y solo si, para todo $A\subset X$ tenemos que
 $f(\overline{A})=\overline{f(A)}$.
 \end{ejercicio}

 \begin{ejercicio}{} Demostrar:
 \begin{itemize}
    \item[i)] que las métricas sobre $\rr^n$
    definidas en los Ejemplos \vref{ejem,disteuclidea},
    \vref{ejem,distl1} y \vref{ejem,distlinf} son todas equivalentes.
    \item[ii)] que las distancias $d$ , $d_1$ y $d_2$ del Ejercicio
    \vref{ejer,distequiv} son equivalentes.
    \item[iii)] Dados dos topolog\'{\i}as $\tau_1$ y $\tau_2$
    sobre el mismo espacio $X$ decimos que $\tau_1$ es más fina
    que $\tau_2$ si $\tau_2\subset \tau_1$. Demostrar que, sobre $C([0,1])$,
    la topolog\'{\i}a que genera la métrica del
    Ejemplo \vref{ejem,distsobrecont} es más fina que la
    topolog\'{\i}a que genera la métrica del Ejemplo
    \vref{ejem,distsobrecont} es más fina que la del
    Ejemplo \vref{ejem,distsobrecontl1}.
 \end{itemize}
 \end{ejercicio}
 
 \section{Completitud}

Una propiedad importante de los espacios métricos es la
completitud. En esta unidad introducimos esta propiedad e
indagamos algunas de sus consecuencias.
\subsection{Sucesiones}
\begin{definicion}{} Una \index{Sucesión}\emph{sucesión} en un e.m. $(X,d)$ es una
función $f:\nn\rightarrow X$.
\end{definicion}

Esta el la definición formal de sucesión, no obstante cuando
se trabaja con sucesiones no se hace alusión expl\'{\i}cita a la
función $f$ de la definición. Normalmente una sucesión se
introduce con el s\'{\i}mbolo $\{a_n\}_{n\in\nn}$ o, brevemente,
$\{a_n\}$, asumiendo que los \'{\i}ndices $n$ son naturales. Claro
está que, \'{\i}mplicitamente, estos s\'{\i}mbolos conllevan la
función $f$. Esta es la función tal que $f(n)=a_n$.

\begin{definicion}{def,sucesionconvergente} Sea $\{a_n\}$ una sucesión en el e.m.
$(X,d)$. Diremos que esta sucesión \emph{converge} al punto
$a\in X$ (denotaremos esto por $a_n\rightarrow a$) si, y solo si,
para todo entorno $U$ de $a$ existe un $n_0=n_0(U)$ tal que cuando
$n\geq n_0$ se tiene que $a_n\in U$. Sinteticamente, dado
cualquier entorno, salvo posiblemente una cantidad finita de
términos de la sucesión todos los términos restantes están
inclu\'{\i}dos en el entorno, ver la Figura
\vref{fig,sucesionconvergente}
\end{definicion}
\marginnote{
\begin{center}
    \adjustimage{max size={0.9\linewidth}{0.9\paperheight}}{imagenes/succonv.jpg}
    \small Definición
    \vref{def,sucesionconvergente}
\end{center}
}


La convergencia es una propiedad topológica. Confiamos en que el
alumno tiene muchos ejemplos de sucesiones convergentes en $\rr$,
esto fué visto en Cálculo I. Vamos a ver que sucede en otros
espacios métricos.

\begin{ejemplo}{} En un e.m. discreto $(X,d)$ si una sucesión
$\{a_n\}$  converge al punto $a$, entonces a partir de un $n_0$ en
adelante se tiene que $a_n=a_{n_0}$. En efecto, esto es
consecuencia de considerar el siguiente entorno:
$U=B(a,1/2)=\{a\}$.
\end{ejemplo}

\begin{ejemplo}{ejem,convfunciones} Consideremos el e.m. $(C([0,1]),d)$, donde
$C([0,1])$ representa al conjunto de funciones continuas
$f:[0,1]\rightarrow\rr$ y $d$ es la métrica definida en el
Ejemplo \vref{ejem,distsobrecontl1}. Consideremos las funciones
definidas por:

\[
    f(x)=\left\{%
\begin{array}{ll}
    nx, & \hbox{si $0\leq x\leq \frac1n$;} \\
    2-nx, & \hbox{si $\frac1n\leq x\leq \frac2n$.} \\
    0,    &\hbox{si $\frac2n\leq x\leq 1.$}
\end{array}%
\right.
\]
En la Figura \vref{fig,convfunciones}, se pueden observar los
gráficos de estas funciones.
\marginnote{
\begin{center}
    \adjustimage{max size={0.9\linewidth}{0.9\paperheight}}{imagenes/funciones0.jpg}
    \small Funciones del Ejemplo
    \ref{ejem,convfunciones} 
\end{center}
}
Es un ejercicio de Cálculo I demostrar que $f_n\rightarrow 0$
con la métrica propuesta. Sin embargo, sobre $C([0,1])$ tenemos
definida otra métrica, a saber: la del Ejemplo
\vref{ejem,distsobrecont}. Con esta métrica la sucesión $f_n$
no converge a ninguna función.
\end{ejemplo}

Es posible caracterizar algunos de los conceptos, que ya hemos
visto, en términos de sucesiones. Por ejemplo, el concepto de
clausura y continuidad.

\begin{proposicion}{} Sea $(X,d)$ un e.m. y $A\subset X$. Entonces
$a\in\C{A}$ si, y solo si, existe una sucesion $a_n\in A$ tal que
$a_n\rightarrow a$.
\end{proposicion}
\begin{demo} Supongamos que $a\in\C{A}$, entonces, para todo
$n\in\nn$ se tiene que $B(a,1/n)\cap A\neq\emptyset$. Sea, pues,
$a_n\in B(a,1/n)\cap A$. Se puede ver, sin dificultad, que
$a_n\rightarrow a$. Rec\'{\i}procamente, supongamos que existe la
sucesión $\{a_n\}$. Si $U$ es un entorno arbitrario de $a$,
entonces, puesto que $a\in\C{A}$, tenemos que, para ciertos $n$,
$a_n\in U$, luego, estos $a_n$, estan en la intersección de $U$
con $A$, lo que implica que esta es no vac\'{\i}a. Eso prueba que
$a\in\C{A}$.
\end{demo}

\subsection{Sucesiones de Cauchy, espacios métricos completos}


\begin{definicion}{}
\begin{itemize}
\item[i)]Dada una sucesión $\{a_n\}$ en un e.m. $(X,d)$, diremos que
$\{a_n\}$ es una \emph{sucesión de Cauchy} si: para todo
$\epsilon>0$ existe un $n_0=n_0(\epsilon)$\footnote{Con
$n_0=n_0(\epsilon)$ queremos decir que el número $n_0$ depende
de $\epsilon$ pero que normalmente no usaremos la notación
$n_0(\epsilon)$ sino, simplemente, $n_0$} tal que para $n,m\geq
n_0$ tenemos que $d(a_n,a_m)<\epsilon$. En otras palabras, para
valores grandes de $n$ los términos $a_n$ están cerca entre
si.
\item[ii)] Un e.m. se dirá \emph{completo} si, y solo si, todo
sucesión de Cauchy en él es convergente.
\end{itemize}
\end{definicion}


Como acabamos de decir, en un e.m. completo toda sucesión de
Cauchy converge. La reciproca de esta afirmación es siempre
cierta, es decir en cualquier e.m. toda sucesión convergente es
de Cauchy. En efecto, sea $\{a_n\}$ una sucesión convergente en
$(X,d)$ al punto $a$ y sea  $\epsilon > 0$. Existe un
$N=N(\epsilon)$ tal que para $n>N$ se tiene que
\[
    d(a_n,a)<\frac{\epsilon}{2}.
\]
Luego, para $n,m>N$ y por la desigualdad triágular, tenemos que
\[
    d(a_n,a_m)\leq d(a_n,a)+d(a,a_m)<\frac{\epsilon}{2}+\frac{\epsilon}{2}=\epsilon.
\]
Esto prueba que la sucesión es de Cauchy, como quer\'{\i}amos.

Un ejemplo importante de e.m. completo es $\rr$. Esta propiedad de
$\rr$ es enunciada, prácticamente, como un axióma. Hablaremos,
mas no sea brevemente, de los ``fundamentos'' de los
 números reales en el apéndice al final de esta unidad, ver
 Sección  \vref{sec,apendice}. Sabiendo que $\rr$ con la
 métrica del módulo es completo podemos demostrar la
 completitud de otros e.m., como veremos más abajo. Antes de ver
 esto demostremos que toda sucesión convergente es de Cauchy

 \begin{ejemplo}{} $\rr^n$ con la métrica euclidea es un e.m.
 completo.\footnote{Observar que, en virtud del Ejercicio
 \vref{ejer,completitudconmetequi} $\rr^n$ con cualquier
 métrica equivalente a la euclidea también resultará
 completo} Vamos a demostrar esta afirmación. Denotemos
 por letras en negritas $\mathbf{x}$, $\mathbf{y}$, $\mathbf{z}$,
 etc $n$-uplas en $\rr^n$, es decir $\mathbf{x}=(x_1,\dots, x_n)$
 con $x_i\in\rr$, $i=1,\dots,n$. Consideremos una sucesión de
 Cauchy  $\{\mathbf{x}_j\}_{j\in\nn}$ en $\rr^n$. Esto nos determina $n$
 sucesiones en $\rr$, puesto que
 $\mathbf{x}_j=(x_1^j,\dots,x_n^j)$. Veamos que, para cada $i$,
 $\{x_i^j\}_{j\in\nn}$ es una sucesión de Cauchy en $\rr$. Se
 tiene que:
\[
    |x_i^j-x_i^k|\leq \sqrt{\sum\limits_{s=1}^n(x_s^j-x_s^k)^2}\leq
    d(\mathbf{x}_j,\mathbf{x}_k).
\]
Como el último miembro se puede hacer tan chico como queramos,
puesto que $\{\mathbf{x}_j\}$ es de Cauchy, podemos conseguir lo
mismo para el primer miembro, esto es $\{x_i^j\}$ es de Cauchy.
As\'{\i}, como $\rr$ es completo, existe un $x_i$, para
$i=1,\dots,n$ tal que $x_i^j\rightarrow x_i$, para
$j\rightarrow\infty$. Definamos, pues,
$\mathbf{x}=(x_1,\dots,x_n)$ y veamos que $\mathbf{x}_j\rightarrow
\mathbf{x}$. Sea $\epsilon>0$, para cada $i=1,\dots,n$ podemos
hallar un $j(\epsilon,i)$, es decir $j$ depende de $\epsilon $ y
de $i$, tal que para $j\geq j(\epsilon, i)$ tenemos que:
\[
    |x_i^j-x_i|<\frac{\epsilon}{\sqrt{n}}.
\]
As\'{\i}, si:
\[
    j\geq\max\limits_{1\leq i\leq n}j(\epsilon,i)
\]
entonces
\[
    d(\mathbf{x}_j-\mathbf{x})=\sqrt{\sum\limits_{i=1}^n(x_i^j-x_i)^2}=
    \sqrt{\sum\limits_{i=1}^n\frac{\epsilon^2}{n}}=\epsilon,
\]
lo que demuestra que $\mathbf{x}_j\rightarrow \mathbf{x}$, como
quer\'{\i}amos.
\end{ejemplo}

\begin{ejemplo}{ejem,emnocompleto} El e.m. $(C([0,1]),d)$, con $d$ como en el Ejemplo
\vref{ejem,convfunciones}, no es completo. Consideremos las
siguientes funciones, para $n/geq 2$:

\[
    f_n(x):=\left\{
                \begin{array}{ll}
                        1, & \hbox{si $0\leq x\leq \frac12$;} \\
                        -nx+\frac{n+2}{2n}, & \hbox{si $\frac12\leq x\leq \frac12+\frac1{n}$;} \\
                        0, & \hbox{si $\frac12\leq x\leq 1$.} \\
\end{array}
\right.
\]
 en la Figura \vref{fig,emnocompleto} graficamos estas
funciones.
\marginnote{
\begin{center}
    \adjustimage{max size={0.9\linewidth}{0.9\paperheight}}{imagenes/funciones1.jpg}
    \small Funciones del Ejemplo
    \ref{ejem,emnocompleto}
\end{center}
}

No es dificil convencerse que esta es una sucesi/ón de Cauchy,
puesto que, para $j,k/geq n$ tenemos que:

\[
    d(f_j,f_k)=\frac12|\frac1j-\frac1k|\leq \frac1n\rightarrow
    0\quad\text{cuando}\quad n\rightarrow \infty.
\]
Sin embargo estas funciones no convergen a ninguna función en
$C([0,1])$. Para ver esto, supongamos que, por el contrario,
existe $f\in C([0,1])$ tal que $f_n\rightarrow f$.\marginnote{
\begin{center}
      \adjustimage{max size={0.9\linewidth}{0.9\paperheight}}{imagenes/demnoco.jpg}
    \small Construcción de la demostración en el Ejemplo
    \ref{ejem,emnocompleto}
\end{center}
} Vamos a
demostrar que, necesariamente, $f$ debe valer 1 en el intervalo
$[0,1/2)$ y debe valer 0 en el intervalo $(1/2,1)$, por tal motivo
no podr\'{\i}a ser continua contradiciendo las hipótesis. Vamos
a demostrar solo que $f$ es 0 en $(1/2,1)$, la otra parte es
similar y aun más facil. Supongamos que exista $1/2<a<1$ tal que
$f(a)\neq 0$, podemos suponer que $f(a)>0$. Elijamos $\delta_1>0$
suficientemente peque\~no de modo que $1/2<a-\delta_1$ y elijamos
$\delta_2>0$ suficientemente peque\~no de modo tal que
$f(x)>f(a)/2$ para $x\in(a-\delta_2,a+\delta+2)$ (esto es posible
pues $f$ es continua en $a$). Ahora, el número
$\delta:=\min\{\delta_1,\delta_2\}$ satisface las dos propiedades
anteriores simultaneamente. Podemos encontrar un $n_0$
suficientemente grande para que $1/2+1/n<a-\delta$, cuando
$n>n_0$, esto implica que $f_n$ es idénticamente cero en el
intervalo $(a-\delta,a+\delta)$, ver la Figura
\vref{fig,dememnocompleto}.

Juntando todas las propiedades vistas en el párrafo anterior,
deducimos que, para $n>n_0$,
\[
    d(f_n,f)=\int_0^1|f_n-f|dx\geq\int_{a-\delta}^{a+\delta}|f_n-f|dx\geq
    \delta f(a)>0.
\]
De modo que $f_n$ no converge a $f$, contradiciendo nuestras
suposiciones.
\end{ejemplo}


Ahora veremos que subespacios, de un e.m. completo, son, a su vez,
completos.

\begin{proposicion}{} Sea $(X,d)$ un e.m. completo e $Y\subset X$.
Son equivalentes:
\begin{itemize}
\item[i)] $(Y,d)$ es un subespacio completo.
\item[ii)] $Y$ es cerrado en $X$.
\end{itemize}
\end{proposicion}


2.4. Teorema de la categoría de Baire
Definición 2.4.1. Sea $(X, d)$ un espacio métrico. Se dice que un subconjunto $Y \subseteq X$ no es denso en ninguna parte si $(\bar{Y})^{\circ}$ está vacío, es decir, $(\bar{Y})^{\circ}$ no contiene punto interior. Se dice que un subconjunto $F \subseteq X$ es de categoría $I$ si es una unión contable de subconjuntos densos en ninguna parte. Los subconjuntos que no son de categoría I se dice que son de categoría II.

Observaciones 2.4.2. (i) Un subconjunto $Y$ de $X$ no es denso en ninguna parte si y solo si el complemento $(\bar{Y})^{c}$ es denso en $X$, o $\overline{(X-\ barra{Y})}=X$. Esto se sigue fácilmente del comentario inmediatamente después de la Definición 2.3.12.
(ii) Si $d$ denota la métrica discreta, el único conjunto denso en ninguna parte es el conjunto nulo.
(iii) La noción de ser denso en ninguna parte no es lo opuesto a ser denso en todas partes, es decir, no ser denso en ninguna parte no implica que el conjunto sea denso en todas partes. Para un ejemplo de un conjunto que no es ninguno de los dos, sea $\mathbf{R}$ la línea real con la métrica usual y considere el conjunto $Y=\{x \in \mathbf{R}: 1<x<2\}$. Luego
$$
(\bar{Y})^{0}=Y \neq \varnothing \text { y }(\bar{Y})^{c}=\{x \in \mathbf{R}: x<1 \text { o } x>2\}=(-\infty, 1) \cup(2, \infty),
$$
que no es denso en $\mathbf{R}$.
(iv) Cada subconjunto debe ser de categoría I o de categoría II.
(v) Está claro que el conjunto nulo es de categoría I. Además, el subconjunto $\mathbf{Q}$ de racionales en $\mathbf{R}$ es un conjunto de categoría I. De hecho, si $x_{1 }, x_{2}, \ldots$ es una enumeración de los racionales, cada $\left\{x_{i}\right\}$ es cerrado y $\left\{x_{i}\right\}^{ \circ}=\emptyset$; se sigue que $\cup\left\{x_{i}\right\}$, el conjunto de todos los racionales en $\mathbf{R}$, es de categoría I.
(vi) Dado que una unión numerable de conjuntos numerables es nuevamente un conjunto numerable, se sigue que, si $Y_{1}, Y_{2}, \ldots$ son cada uno de la categoría I, entonces también debe ser $\bigcup_{i } Y_{i}$.
(vii) Si $X=Y_{1} \cup Y_{2}$ y se sabe que $Y_{1}$ es de categoría I mientras que $X$ es de categoría II, entonces $Y_{2}$ debe ser de categoría II. Porque, si $Y_{2}$ es de categoría I, entonces se sigue de (vi) arriba que $X$ también es de categoría I, lo cual es una contradicción. 







\subsection{Ejercicios}

\begin{ejercicio}{}\label{ejer,testearconv} Consideremos el conjunto $C([0,1])$ donde
tenemos definidas las dos métricas $d_1$ y $d_2$ de los Ejemplos
\vref{convunifmet} y \vref{l1metint} respectivamente. Determinar
si las siguientes sucesiones son convergentes con estas métricas
y si son de Cauchy.
\begin{itemize}
    \item[i)] $f_n(x):=\frac1n\text{sen}(nx)$.
    \item[ii)] $f_n(x):=x^n$.
    \item[iii)]$f_n(x):=nx^n$.
\end{itemize}
\end{ejercicio}
\begin{ejercicio}{} Con la misma notación del ejercicio anterior
demostrar que si $f_n\to f$ con la métrica  $d_1$ entonces lo
mismo ocurre con la métrica $d_2$.
\end{ejercicio}

\begin{ejercicio}{} Sean $(X,d)$ e $(Y,d')$ dos e.m. y $f:X\to Y$ un homeomorfismo.
Demostrar que la sucesión $\{a_n\}$ es convergente en $X$ si, y
solo si, $f(a_n)$ es convergente en $Y$.
\end{ejercicio}



\begin{ejercicio}{} Sea $(X,d)$ un e.m., $a\in A$ y $A\subset X$. Demostrar
que existe un sucesión $\{a_n\}$, con $a_n\in A$, para todo $n$,
y:
\[\lim\limits_{n\to\infty}d(a,a_n)=d(a,A).\]
\end{ejercicio}

\begin{ejercicio}{} Sea $A\subset \rr$ un conjunto acotado
superiormente. Demostrar que existe una sucesión $\{a_n\}$, con
$a_n\in A$, para todo $n$, y además:
\[
    \sup A=\lim\limits_{n\to\infty} a_n.
\]
\end{ejercicio}

\begin{ejercicio}{} Demostrar que $(C([0,1]),d_1)$, con $d_1$ como
en el Ejercicio \ref{ejer,testearconv}, es un e.m. completo.
\end{ejercicio}

\begin{ejercicio}{} Demostrar que un e.m. con una cantidad finita de
elementos es completo.
\end{ejercicio}

\begin{ejercicio}{} Demostrar que una sucesión de Cauchy es
acotada.
\end{ejercicio}

\begin{ejercicio}{} Sea $\{a_n\}$ una sucesión en un e.m. $(X,d)$,
demostrar que cualquierqa de las dos condiciones implica que
$\{a_n\}$ es de Cauchy.
\begin{itemize}
    \item[i)] $d(a_n,a_{n+1})\leq \alpha^n$, con $0<\alpha<1$.
    \item[ii)] La siguiente serie es convergente:
    \[
        \sum\limits_{n=1}^{\infty}d(a_n,a_{n+1}).
    \]
\end{itemize}
\end{ejercicio}

\begin{ejercicio}{ejer,completitudconmetequi} Sean $d$ y $d'$ dos métricas
 uniformemente equivalentes
sobre el mismo espacio $X$. Demostrar que $(X,d)$ es completo si,
y solo si, $(X,d')$ es completo.
\end{ejercicio}

\begin{ejercicio}{} Sea $f:X\to Y$ una función uniformemente
continua entre dos e.m.. Demostrar que si $\{a_n\}$ es de Cauchy
en $X$ entonces $\{f(a_n)\}$ es de Cauchy en $Y$. Dar un
contraejemplo a la afirmación anterior suponiendo, solo, que $f$
es continua. \emph{Ayuda:} Considerar la recta extendida.
\end{ejercicio}


\begin{ejercicio}{ejer,completitud} Demostrar que el axioma de completitud de $\rr$
dado, se puede sustitu\'{\i}r por cualquiera de los siguientes:
\begin{itemize}
    \item[i)] Toda sucesión de Cauchy en $\rr$ converge.
    \item[ii)] \emph{Principio de Encajes de Intervalos}. Sea
    $\{I_n\}_{n\in\nn}$ una sucesión de intervalos cerrados
    tales que $I_{n+1}\subset I_n$, para todo $n$, entonces
    $\bigcap_{n\in\nn}I_n\neq\emptyset$.
\end{itemize}
\end{ejercicio}
% 
% 
\section{Compacidad}

Es quizas con la noción de conjunto compacto donde encontraremos
las diferencias más grandes entre la topología de $\mathbb{R}^n$
y la de un espacio métrico arbitrario. En particular, ya no será
válida la carectización de compacto como cerrado y acotado. Para
obtener una caracterización necesitaremos un concepto más fuerte
que la acotación, este será el de conjunto \textbf{totalmente
acotado} y, a la vez, un concepto más fuerte que el de conjunto
cerrado y en este caso usaremos la de conjunto completo.

Es interesante hacer notar que, en topología, interesan aquellas
propiedades que se preservan por homeomorfismos. En este sentido
vemos que la noción de conjunto cerrado acotado no se preserva
por este tipo de aplicaciones (claro está, los espacios métricos
involucrados deberían ser distintos que $\mathbb{R}^n$ con la
métrica euclidea). Por ejemplo, como ya hemos visto, la identidad
es un homeomorfismo de $(\mathbb{R}^n,d)$, con $d$ la métrica
euclidea, en $(\mathbb{R}^n,d_1)$, con
\[
	d_1(x,y)=\frac{d(x,y)}{1+d(x,y)}.
\]
Ahora bien, como $0\leq d_1<1$ cualquier conjunto de
$\mathbb{R}^n$ tiene diámetro, respecto a $d_1$ menor o igual a
$1$ y, por ende, cualquier conjunto es acotado. Sin embargo, no
todo conjunto es acotado respecto a la métrica euclidea. Por otra
parte $\mathbb{R}^n$ es cerrado en ambas métricas, pues es el
conjunto total. Vemos así que el concepto de conjunto cerrado y
acotado no necesariamente se preserva por homeomorfismos lo que
relativiza su importancia.



\begin{definicion}{}  Diremos que un conjunto $A$ de un e.m. $(X,d)$
es totalmente acotado
si para cada $\epsilon>0$ existe una cantidad finita de conjuntos
de diámetro menor que $\epsilon$ cuya unión contiene a  $A$.
En otras palabras existen conjuntos $A_i$, $i=1,...,n$, con
$\delta(A_i)<\epsilon$ que satisfacen:
\[
	A\subset \bigcup\limits_{i=1}^nA_i.
\]
\end{definicion}


\begin{ejercicio}{ejer,subpreespre} Demostrar que un
subconjunto de un conjunto totalmente acotado es totalmente
acotado.
\end{ejercicio}


Veamos algunos ejemplos de conjuntos totalmente acotados y de
conjuntos que no lo son.

\begin{ejemplo}{} Cualquier intervalo acotado de $\rr$ es
totalmente acotado. Para justificar esta aseveración, tomemos
$\epsilon>0$ y un intervalo cualquiera de extremos $a$ y $b$.
Elijamos $n$ suficientemente grande para que $1/n<\epsilon$.
Entonces los conjuntos
\[
	I_k:=\Big[\frac{k}{n},\frac{k+1}{n}\Big]\quad ,k=0,\ldots,n-1,
\]
satisfacen la definición.
\end{ejemplo}

\begin{ejemplo}{ejem,cubpre} Cualquier conjunto  acotado en el
espacio euclideo $\rr^n$ es totalmente acotado. 
\marginnote{
\begin{center}
	\adjustimage{max size={0.9\linewidth}{0.9\paperheight}}{imagenes/cubpre.jpg}
	\small Construcción del Ejemplo
	\ref{ejem,cubpre}
\end{center}
}
Sea
$A\subset\rr^n$ un conjunto acotado, entonces $A$  está
contenido en un cubo de la forma $C:=[-m,m]\times\dots\times
[-m,m]=[-m,m]^n$. En virtud del Ejercicio \vref{ejer,subpreespre}
es suficiente demostrar que $C$ es totalmente acotado. Sea
$\epsilon>0$. Tomemos $k$ suficientemente grande para que
\begin{equation}\label{eq,defk}
	\frac{2\sqrt{n}m}{\epsilon}<k
\end{equation}
Ahora, partimos cada intervalo $[-m,m]$ en $k$ subintervalos de la
misma longitud $1/k$.



Como puede observarse en la Figura \ref{fig,cubpre}, nos quedan
determinados $k^n$ cubos que cubren el cubo $C$. Cada uno de estos
cubos más chicos tiene diámetro $2\sqrt{n}m/k$, por
consiguiente, por la desigualdad \ref{eq,defk}, el diámetro de
ellos es menor que $\epsilon$.
\end{ejemplo}

No es cierto, en general, que todo conjunto acotado en un e.m. sea
totalmente acotado. Los siguientes ejemplos muestran esto.

\begin{ejemplo}{} Sea $(X,d)$ un e.m. discreto con $X$ infinito. El
conjunto $X$ es acotado, de hecho $\delta(X)=1$; sin embargo no
podemos cubrir $X$ con conjuntos de diámetro menor que 1/2
(cualquier número menor que 1 servir\'{\i}a). Esto ocurre debido
a que si un conjunto en un e.m. discreto tiene más de un
elemento entonces su diámetro es 1. As\'{\i}, si cubrimos $X$
con una cantidad finita de conjuntos, alguno de los conjuntos del
cubrimiento necesariamente tiene más de un elemento, de lo
contrario $X$ ser\'{\i}a finito, por consiguiente el diámetro de
este conjunto es 1 y no puede ser menor que $1/2$.
\end{ejemplo}

\begin{ejemplo}{ejem,contnopre} En $C([0,1])$, con la métrica
 del Ejemplo
\vref{ejem,distsobrecont}, la bola cerrada $K(0,1)$ (0 denota la
función que es constantemente igual a 0) no es un conjunto
totalmente acotado. Para ver esto definimos la siguiente
función:

\[
	f(x):=\left\{%
\begin{array}{ll}
	4(x-\frac12), & \hbox{si $\frac12\leq x\leq \frac34$;} \\
	-4(x-1), & \hbox{si $\frac34\leq x\leq 1$;} \\
	0, & \hbox{para los restantes $x$;} \\
\end{array}
\right.
\]\marginnote{
\begin{center}
	\adjustimage{max size={0.9\linewidth}{0.9\paperheight}}{imagenes/noprecom.jpg}
	\small Funciones del Ejemplo
	\ref{ejem,contnopre}
\end{center}
}
y la siguiente sucesión de funciones $f_n(x):=f(2^nx)$. En la
Figura \ref{fig,contnopre} puede verse las gráficas de algunas
de las funciones de la sucesión.


Puede demostrarse que la distancia de cualquiera de las funciones
de la sucesión a otra es igual a 1 y que $f_n\in K(0,1)$. Sea
$C:=\{f_n:n\in\nn\}$, observemos que como subespacio $C$ resulta
ser un e.m. discreto, as\'{\i}, por el Ejemplo anterior y el
Ejercicio \vref{ejer,subpreespre}, $\C{B(0,1)}$ no puede ser
totalmente acotado.
\end{ejemplo}

\begin{ejemplo}{} Hay una interesante conexión de la total
acotación con la dimensión. Para introducirla, veamos
 cuantas bolas abiertas de radio
$1/2$, en $\mathbb{R}^n$ con la métrica euclidea,
se necesitan, al menos,  para cubrir la bola cerrada $K(0,1)$.
Denotemos por $e_j$ los vectores canónicos
\[
	e_j:=(0,\ldots,1,\ldots,0),
\]
donde el $1$ está en el lugar $j$. Notesé que $e_j\in K(0,1)$ y que:
\[
	d(e_j,e_i)=\sqrt{2}\quad i\neq j.
\]
De modo que, si $i\neq j$ entonces $e_i$ y $e_j$ no pueden estar
en una misma bola de radio $1/2$. De lo contrario, si $e_i,e_j\in
B(x,1/2)$, entonces
\[
	d(e_i,e_j)\leq d(e_i,x)+d(x,e_j)<1<\sqrt{2},
\]
que es una contradicción. De esta manera si cubrimos la bola cerrada
$K(0,1)$ por bolas abiertas de radio $1/2$ necesitaremos, al menos,
$n$ de estas bolas. Es decir, la cantidad de estas bolas crece cuando aumenta
la dimensión $n$. Esta observación nos lleva a conjeturar que si
buscamos un espacio vectorial de dimensión infinita\footnote{Esto significa
que no tiene una base finita} tenemos chances de construir conjuntos acotados,
en particular la bola $K(0,1)$, que no son totalmente acotados.

\begin{ejercicio}{} Demostrar que en $l_2$ la bola cerrada $K(0,1)$
no es totalmente acotada.
\end{ejercicio}
\end{ejemplo}


Recordemos que, en un e.m. $(X,d)$, una familia de conjuntos
abiertos $\{U_i\}_{i\in I}$ es un cubrimiento por abiertos de
$A\subset X$ si
\[
	A\subset\bigcup\limits_{i\in I}U_i.
\]
\begin{definicion}{} Un subconjunto $A$ de un e.m. $(X,d)$ se dirá
\textbf{compacto} si, y solo si, todo cubrimiento por abiertos de
$A$ tiene un subcubrimiento finito. Es decir, si $\{U_i\}_{i\in
I}$ es un cubrimiento  de $A$, existe un conjunto finito $F\subset
I$ tal que $\{U_i\}_{i\in F}$ es un cubrimiento de $A$.
\end{definicion}

\begin{teorema}[Caracterización de compacidad en espacios métricos]{}
 Sea $(X,d)$ un  espacio métrico. Entonces son
equivalentes:

\begin{enumerate}
\item\label{inc,compacto} $X$ es compacto;
\item\label{inc,toacotadoycomp} $X$ es totalmente acotado y
completo.
\item\label{inc,subs} Toda sucesión en $X$ tiene una subsucesión
convergente.
\end{enumerate}
\end{teorema}
\begin{demo} Veamos que
\ref{inc,compacto}$\Rightarrow$\ref{inc,subs}. Por el absurdo
supongamos que existe una sucesión $\{a_n\}$ en $X$ que no tiene
ninguna subsucesión convergente. Definamos $\Gamma$ como la
colección de todos los conjuntos abiertos $G$ de $X$ tales que
$G$ tiene una cantidad finita de elementos de la sucesión, es
decir:
\[
G\in\Gamma\Leftrightarrow \#\{n:a_n\in G\}<\infty.
\]
Vamos a probar que $\Gamma$ es un cubrimiento de $X$. Supongamos
que $x\in X$ y $x\notin G$ para todo $G\in\Gamma$. De modo que,
por definición, cada abierto que contiene a $x$ contiene
infinitos términos de la sucesión $\{a_n\}$. En particular,
podemos encontrar $n_1$ tal que $a_{n_1}\in B(x,1)$. Ahora podemos
encontrar $n_2>n_1$ tal que $a_{n_2}\in B(x,\frac12)$. Y así
continuamos, construímos una subsucesión $a_{n_k}$ tal que
$a_{n_k}\in B(x,\frac1k)$. Lo que implica que $a_{n_k}$ converge a
$x$, contradiciendo nuestra suposición. De esta manera,
$\Gamma$ es un cubrimiento de $X$. Sea $G_i$, $i=1,\ldots,n$, un
subcubrimiento finito de $X$. Es decir
\[
	X=G_1\cup\dots\cup G_n.
\]
Como cada $G_i$, $i=1,\ldots,n$, tiene una cantidad finita de
términos de la sucesión, concluímos que $X$
contiene una cantidad finita de términos de la sucesión, lo que,
claro está, no puede ocurrir. Esto finaliza la demostración de
\ref{inc,compacto}$\Rightarrow$\ref{inc,subs}.

Demostremos ahora que \ref{inc,subs}$\Rightarrow$
\ref{inc,toacotadoycomp} empezando por ver que $X$ es totalmente
acotado. Nuevamente procedemos por el absurdo, suponiendo que $X$
no es totalmente acotado. Esto implica que existe un $\epsilon>0$
tal que $X$ no se puede cubrir con una cantidad finita de
conjuntos de diámetro $\epsilon$. Sea $a_1$ cualquier punto de
$X$. Como $B(a_1,\epsilon)$ no cubre $X$, existe $a_2\in
X-B(a_1,\epsilon)$. Como $B(a_i,\epsilon)$, $i=1,2$, no cubren
$X$, existe un $a_3\in X-\bigl(B(a_1,\epsilon)\cup
B(a_2,\epsilon)\bigr)$. Continuando de esta forma, contruímos una
sucesión $a_n$ tal que
\[
	a_n\in X-\big(B(a_1,\epsilon)\cup\dots\cup
	B(a_{n-1},\epsilon)\big).
\]
De esta forma tendremos que:
\[
	d(a_i,a_j)\geq \epsilon\quad\hbox{para $i\neq j$}.
\]
Por hipótesis la sucesión $a_n$ tiene una subsucesión convergente,
en particular esta subsucesión será  de Cauchy. No obstante la desigualdad
anterior implica que ninguna subsucesión de $\{a_n\}$ puede
ser de Cauchy, contradicción que prueba que $X$ es totalmente acotado.

Veamos ahora que $X$ es completo. Sea $\{a_n\}$ una sucesión de
Cauchy en $X$. Podemos extraer una subsucesión $\{a_{n_k}\}$
convergente a un $a\in X$. Sea $\epsilon>0$. Puesto que $\{a_n\}$
es de Cauchy, pondemos encontrar $N>0$ tal que si $n,m>N$ entonces:
\begin{equation}\label{eq,conddecauchy}
	d(a_n,a_m)<\frac{\epsilon}{2}.
\end{equation}
Como $a_{n_k}$ converge a $a$, podemos encontrar un $n_k$ lo
suficientemente
grande para que $n_k>N$ y:
\[
	d(a_{n_k},a)<\frac{\epsilon}{2}.
\]
Así, usando \eqref{eq,conddecauchy},  tenemos que para $n>N$:
\[
	d(a_n,a)\leq d(a_n,a_{n_k})+d(a_{n_k},a)<\epsilon.
\]
Por último veamos que \ref{inc,toacotadoycomp}
$\Rightarrow$\ref{inc,compacto}. Para este fin elijamos un
cubrimiento $\{G_{\lambda}\}_{\lambda\in L}$ arbitrario de $X$.
Supongamos que este cubrimiento no tiene un subcubrimiento finito.
Como $X$ es totalmente acotado, acorde al Ejercicio
\ref{ejer,precantfinibolas}, podemos cubrir a $X$ por una cantidad
finita de bolas de radio $1$. Alguna de estas bolas no se podrá
cubrir por una cantidad finita de $G_{\lambda}$, de lo contrario,
si todas se cubren por una cantidad finita, como hay una cantidad
finita de estas bolas, podríamos cubrir $X$ por una cantidad
finita de $G_{\lambda}$. Llamemos $B(x_1,1)$ a la bola que no se
cubre por finitos $G_{\lambda}$. Como $B(x_1,1)$ es totalmente
acotado, podemos aplicar la construcción anterior a $B(x_1,1)$ en
lugar de $X$ y con bolas de radio $1/2$, en lugar de $1$,
obteniendo de esta forma una bola $B(x_2,1/2)$ que no se cubre por
una cantidad finita de $G_{\lambda}$. Además podemos suponer que
$B(x_2,1/2)\cap B(x_1,1)\neq \emptyset$, de lo contrario no
hubiera tenido sentido usar la bola $B(x_2,1/2)$ para cubrir
$B(x_1,1)$. Continuamos de esta forma y producimos una sucesión
$B(x_n,1/2^n)$ (aquí no basta que los radios tiendan a cero, sino
que es necesario que la serie de radios converja) de bolas tales
que ninguna de ellas se puede cubrir por una cantidad finita de
$G_{\lambda}$. Veamos que la sucesión $x_n$ es de Cauchy.   Para
esto tomemos
\[
	z_n\in B\Big(x_n,\frac{1}{2^n}\Big)\cap B\Big(x_{n+1}
	,\frac{1}{2^{n+1}}\Big).
\]
Entonces
\[
	d(x_n,x_{n+1})\leq d(x_n,z_n)+d(z_n,x_{n+1})<
	\frac{1}{2^n}+\frac{1}{2^{n+1}}<\frac{1}{2^{n-1}}.
\]
Lo que permite utilizar el criterio de comparación para
la convergencia de series para demostrar que:
\[
	\sum\limits_{n=1}^{\infty}d(x_n,x_{n+1})<\infty.
\]
Acorde a un ejercicio de la práctica, esto implica que $\{x_n\}$ es
de Cauchy, por ende converge a algún $x_0\in X$. Como $G_{\lambda}$
es un cubrimiento, existe $\lambda_0$ tal que $x_0\in G_{\lambda_0}$. Como
$G_{\lambda_0}$ es abierto, existe un $r>0$ tal que
\begin{equation}\label{eq,dalecasillegamos}
	B(x_0,r)\subset G_{\lambda_0}.
\end{equation}
Puesto que $x_n$ converge a $x_0$ y $1/2^{n-1}$ converge a 0,
podemos hallar $n$ lo suficientemente grande para que:
\begin{equation}\label{ufamecanse}
	d(x_n,x_0)<\frac{r}{2}\quad\hbox{y}\quad
	\frac{1}{2^{n-1}}<\frac{r}{2}.
\end{equation}
Veamos que esto, \eqref{ufamecanse} y \eqref{eq,dalecasillegamos}
implica que:
\begin{equation}\label{casicasi}
	B\Big(x_n,\frac{1}{2^n}\Big)\subset B(x_0,r)\subset G_{\lambda_0}.
\end{equation}
En efecto, si:
\[
	 y\in B\Big(x_n,\frac{1}{2^n}\Big),
\]
entonces
\[
	d(y,x_0)\leq d(y,x_n)+d(x_n,x_0)<\frac{1}{2^n}+\frac{r}{2}<r,
\]
lo que prueba \eqref{casicasi}, siendo, además, esta inclusión
una contradicción puesto que estamos cubriendo la bola
$B(x_n,1/2^n)$ por un sólo $G_{\lambda}$, recordemos que estas
bolas no se cubrían por finitos $G_{\lambda}$. (ya terminamos,
!`por fin!)
 \end{demo}

 \subsection{Ejercicios}
 
 

\begin{ejercicio}{} Demostrar que un espacio métrico discreto es compacto
si, y solo si, es finito.
\end{ejercicio}

\begin{ejercicio}{} Demostrar que un conjunto totalmente acotado es
acotado.
\end{ejercicio}



\begin{ejercicio}{ejer,precantfinibolas} Demostrar que $X$ es totalmente acotado si, y
solo si, para cada $\epsilon>0$ podemos cubrir $X$ por una cantidad
finita de bolas de radio $\epsilon$.
\end{ejercicio}

\begin{ejercicio}{} Demostrar que un conjunto compacto es cerrado y
acotado.
\end{ejercicio}

\begin{ejercicio}{} Demostrar que si $f:(X,d)\to (Y,d')$ es continua
y $X$ compacto entonces $f(X)$ es compacto. Como corolario,
demostrar que si $f:X\to\mathbb{R}$ y $X$ es compacto entonces $f$
alcanza un máximo y un mínimo.
\end{ejercicio}

\begin{ejercicio}{ejer,subpreespre} Demostrar que un
subconjunto de un conjunto precompacto es precompacto.
\end{ejercicio}

\begin{ejercicio}{} Sea $(X,d)$ un e.m., $A$ y $B$ subconjuntos compactos de
$X$. Demostrar que
\begin{itemize}
    \item[i)] Existen puntos $x$ e $y$ en $A$ tales que $d(x,y)=\delta(A)$.
    \item[ii)] Existe un $x\in A$ e $y\in B$ tales que
    $d(x,y)=d(A,B)$.
    \item[iii)] Si $A\cap B=\emptyset$, entonces $d(A,B)>0$.
\end{itemize}
\end{ejercicio}
\begin{ejercicio}{} Sea $\{a_n\}$ una sucesi\'on en un e.m. $(X,d)$
tal que $a_n\to a$. Demostrar que el conjunto
$\{a_n:n\in\nn\}\cup\{a\}$ es compacto.
\end{ejercicio}

\begin{ejercicio}{} Sean $(X,d)$ e $(Y,d')$ dos e.m. y $f:X\to Y$
una funci\'on. Demostrar que $f$ es continua si, y solo si,
$f_{|K}:K\to Y$ es continua para cada compacto $K$.
\end{ejercicio}


\begin{ejercicio}{} Como se desprende de la teor\'{\i}a, el intervalo
$(0,1)$ no es cerrado en $\rr$. Encontrar un cubrimiento de
$(0,1)$ que no tenga un subcubrimiento finito.
\end{ejercicio}

\begin{ejercicio}{} Sea $(X,d)$ un e.m. compacto y $f:X\to X$ una
funci\'on continua. Supongamos que, para todo $x\in X$, se tiene
que $f(x)\neq x$. Demostrar que existe $\epsilon>0$ tal que
$d(f(x),x)>\epsilon$.
\end{ejercicio}

\begin{ejercicio}{} Sea $A\subset\rr$ no compacto, demostrar que
existe una funci\'on continua $f:A\to\rr$ que no es acotada.
\emph{Sugerencia} Como $A$ es no compacto y $A\subset \rr$
entonces o $A$ es no acotado o $A$ es no cerrado, considerar estos
dos casos.
\end{ejercicio}


\begin{ejercicio}{} Sea $\{K_n\}$ una sucesi\'on de conjuntos
compactos no vacios, tales que $K_n\supset K_{n+1}$. Demostrar que
$\bigcap_{n\in\nn}K_n\neq\emptyset$.
\end{ejercicio}

\begin{ejercicio}{} Sea $(X,d)$ un e.m. compacto y $\{U_i\}_{i\in
I}$ un cubrimiento por abiertos de $X$. Demostrar que existe un
$\epsilon>0$ tal que toda bola de radio $\epsilon$ est\'a
contenida en, al menos, un $U_i$. \emph{Sugerencia} Para cada
$x\in X$ elegir $r_x$ tal que $B(x,r_x)$ esta contenida en alg\'un
$U_i$. Tomar un subcubrimiento finito de estas bolas y luego
considerar $\epsilon$ como el m\'{\i}nimo de las mitades de los
radios de las bolas del subcubrimiento.
\end{ejercicio}

 
 \begin{ejercicio}{} Utilizar la siguiente ``idea'' para dar una
 demostración alternativa de que compacto implica completo. Tomar
 una suseción de Cauchy $\{a_n\}$ en $X$. De la desigualdad

 \[
	|d(x,a_n)-d(x,a_m)|\leq d(a_n,a_m)\quad n,m\in\mathbb{N}\,\,
	x\in X
 \]
 concluír que $d(x,a_n)$ es una sucesión de Cauchy en
 $\mathbb{R}$ y de esto que la siguiente función esta bien
 definida:
\[
	f(x):=\lim\limits_{n\to\infty}d(x,a_n).
\]
Notar que $f:X\to\mathbb{R}$ y por ende $f$ alcanza un mínimo en
algún $a\in X$. Por último demostrar que $a$ es el límite de
$a_n$.
\end{ejercicio}

\section{Conexión}La definición de conjunto \textbf{conexo},
\textbf{arco conexo} es idéntica a la que ya hemos estudiado. Los
conjuntos conexos, así definidos, satisfacen las mismas
propiedades que ya observamos para la métrica euclidea, con una
excepción. El hecho que valga las mismas propiedades nos permite
definir el concepto de \textbf{componente conexa} de la misma
forma que lo hicimos para $\mathbb{R}^n$.

La propiedad que no continua valiendo, en espacios métricos en
general, es aquella que afirmaba que las componentes conexas de
conjuntos abiertos eran abiertas. Esto es así pues en la
demostración de tal propiedad utilizamos que una bola era un
conjunto conexo y por el siguiente ejercicio:

\begin{ejercicio}{} Encontrar un ejemplo de espacio métrico que
tenga bolas disconexas.
\end{ejercicio}

Para conservar tal propiedad podríamos ``pedir'' la hipótesis que
las bolas sean conexas. No obstante observaremos que en tal demostración
podríamos haber utilizado cualquier entorno conexo que fuera bola o no.
Esto nos lleva a la siguiente definición:

\begin{definicion}{} Un espacio métrico $(X,d)$ se llama
\textbf{localmente conexo} si para todo $x\in X$ y $r>0$
existe un entorno conexo $V$ de $x$ tal que $V\subset B(x,r)\subset X$.
\end{definicion}

$\mathbb{R}^n$ es localmente conexo, podemos utilizar como $V$ la
 misma bola que aparece en la definición anterior. Lo curioso del
 caso es que hay espacios métricos $(X,d)$ tales que $X$ es conexo
 y, sin embargo, $X$ no es localmente conexo.

 \begin{ejemplo}{} Sea $X=\mathbb{Q}\times\mathbb{R}\cup \{(x,0):x\in\mathbb{R}\}$
 y consideremos este $X$ como subespacio de $\mathbb{R}^2$. En la
 figura \ref{noloccon} hemos hecho un bosquejo, está claro que es
 imposible lograr exactitud, del gráfico $X$.
 \begin{center}
\begin{figure}[h]
 \psset{unit=1mm}
  \begin{pspicture}(0,0)(100,60)
 \psset{linewidth=1}
 \multirput{0}(0,0)(2,0){26}{\psline(0,0)(0,50)}
 \rput(0,0){\psline(0,25)(50,25)}
  \psset{linewidth=.5}
 \pscircle(25,40){8}

\rput(40,-40){
 \scalebox{2}{
\psclip{\pscircle(25,40){8}
 }

  \multirput{0}(0,0)(2,0){26}{\psline(0,0)(0,50)}
 \rput(0,0){\psline(0,25)(50,25)}
  \psset{linewidth=.5}
 \pscircle(25,40){8}
\endpsclip  }
}
\psline[linestyle=dashed](24,32)(90,24)
\psline[linestyle=dashed](24,47.5)(88,55)
\rput(116,54){$B(x,r)\cap X$}
 \end{pspicture}
 \caption{El subespacio $X$}\label{noloccon}
 \end{figure}
 \end{center}

Notesé que si tomamos un punto en $X$ pero no sobre el eje horizontal
y consideramos una bola de centro $x$ y un radio suficientemete chico
de modo que la bola no interseque el mencionado eje, entonces
$B(x,r)\cap X$ está compuesto de un conjunto de segmentos verticales
disconexos entre si.  Si ahora buscamos un conjunto conexo $V$ tal que
$V\subset B(x,r)$ notaremos que $V$ debería ser un subconjunto de alguno de los
segmentos verticales, precisamente de aquel segmento que tenga el $x$ dentro
de si, pero tal $V$ no será un entorno. Lo que prueba que el espacio $(X.d)$
no es localmente conexo.

\end{ejemplo}

\subsection{Ejercicios}

\begin{ejercicio}{} Demostrar que los siguientes conjuntos son
disconexos:
\begin{enumerate}
    \item $(0,3)\cup [4,6)$.
    \item $\rr-\mathbb{Q}$.
    \item $\{1/n:n\in\nn\}\cup \{0\}$.
\end{enumerate}
\end{ejercicio}

\begin{ejercicio}{} ?` Cu\'ales de los siguientes conjuntos son
conexos? Justificar la respuesta.
\begin{itemize}
    \item[i)] $\bigcup_{n\in\nn}\{(x,\frac1nx):x\in\rr\}$.
    \item[ii)] $\rr\times\rr-\mathbb{I}\times\mathbb{I}$, donde
    $\mathbb{I}$ son los n\'umeros irracionales.
    \item[iii)] $\{(x,y)\in\rr^2:x\neq 1\}$.
    \item[iv)] $\{(x,y)\in\rr^2:x\neq 1\}\cup\{(1,0)\}$.
\end{itemize}
\end{ejercicio}

\begin{ejercicio}{} Supongamos que  $A$ y $B$ son conjuntos conexos
de un e.m.. Demostrar, dando contraejemplos, que no necesariamente
deben ser conexos los siguientes conjuntos $A\cap B$, $A\cup B$,
$\partial A$ y $A^0$.
\end{ejercicio}

\begin{ejercicio}{} Sea $(X,d)$ un e.m. conexo e $(Y,d)$ un e.m. discreto.
Demostrar que una funci\'on $f:X\to Y$ continua es constante.
\end{ejercicio}

\begin{ejercicio}{} Sean $A$ y $B$ subconjuntos conexos de un e.m.
Demostra que si $\C{A}\cap B\neq\emptyset$ entonces $A\cup B$ es
conexo.
\end{ejercicio}

\begin{ejercicio}{} Probar que todo espacio ultram\'etrico es
totalmente disconexo.
\end{ejercicio}

\begin{ejercicio}{} Sea $\{K_n\}$ una sucesi\'on de conjuntos compactos y conexos
de un e.m., supongamos que la sucesi\'on es decreciente, es decir
$K_n\supset K_{n+1}$. Demostrar que $\bigcap_{n\in\nn} K_n$ es
conexo. Dar un ejemplo de una sucesi\'on como la anterior,
cambiando compacto por cerrado, tal que $\bigcap_{n\in\nn}K_n$ no
sea conexo.
\end{ejercicio}

\begin{ejercicio}{} Dado un conjunto $A$ de un e.m. $(X,d)$
definimos la funci\'on caracter\'{\i}stica del conjunto $A$ por:
\[
    1_A(x):=\left\{%
\begin{array}{ll}
    1, & \hbox{si $x\in A$;} \\
    0, & \hbox{si $x\notin A$.} \\
\end{array}%
\right.
\]
Demostrar que $X$ es conexo si, y solo si, no existe una funci\'on
caracter\'{\i}stica $1_A$, con $A\neq\emptyset$ y $A\neq X$,
continua.
\end{ejercicio}

 


\begin{ejercicio}{} Demostrar que el espacio métrico $(X,d)$ es
localmente conexo si, y sólo si, las componentes conexas de conjuntos
abiertos son abiertas.
\end{ejercicio}

\begin{ejercicio}{} Sea $(X,d)$ un espacio métrico localmente
conexo y compacto. Demostrar que $X$ tiene, a lo sumo, una
cantidad finita de componentes.
\end{ejercicio}

\begin{ejercicio}{} Sea ${X,d}$ un espacio métrico homeomorfo a
$\mathbb{Z}$ demostrar que las componentes conexas de $X$ son
conjuntos unitarios. Este tipo de espacios se llaman totalmente
disconexos. \end{ejercicio}
