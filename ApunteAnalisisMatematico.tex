\documentclass[oneside]{book}

%%%%%%%%%%%%%%%%%%%%%%%%%%%%%%Paquetes%%%%%%%%%%%%%%%%%%%%%%%%%%%%%%%%%%%%%%%%%%%%%%%5
%%%%%%%%%%%%%%%%%%%%%%%%%%%%%%%%%%%%%%%%%%%%%%%%%%%%%%%%%%%%%%%%%%%%%%%%%%%%%%%%%%%%%
\usepackage{empheq}
\usepackage[spanish]{babel}
\usepackage{amssymb,amsmath,amsthm}
\usepackage{enumerate}
\usepackage{verbatim}
%\usepackage{ esint }
\usepackage{array}
%\usepackage{listings}
\usepackage{ wasysym }
\usepackage{hyperref}
\usepackage{color}
%\usepackage{url}
%\usepackage{theorem}
\usepackage{boiboites} %Cajas en teorem
\usepackage[spanish]{varioref} % ESTILO PARA REFERENCIAS
\usepackage{fontspec} %para xelatex
\usepackage[a4paper,driver=xetex,top=2.5cm, bottom=2.7cm,%
layouthoffset=10mm, left=1.5cm, right=6.2cm,marginparwidth=3.5cm,showframe]{geometry}
\usepackage{fancyhdr} %Encabezados mejorados


\usepackage{marginnote} % Notas al margen mejoradas
\usepackage{titlesec} %ni idea
%\usepackage{natbib}
%\usepackage{chapterbib}
\usepackage{fontawesome}
%\usepackage{xcolor}
%\usepackage{appendix}
%\usepackage{chngcntr}
%\usepackage{etoolbox}
%\usepackage{lipsum}
\usepackage{pgf,tikz,pgfplots}
\pgfplotsset{compat=1.15}
%\usepackage{mathrsfs}
\usetikzlibrary{arrows}
\usepackage{imakeidx}
%\usepackage{sympytex}
\usepackage{pythontex}
\usepackage{diagrams}

\defaultfontfeatures{Ligatures=TeX}


%%%%%%%%%%%%Configuración Página
\fancyfoot{}
\fancyhead[RO,LE]{\thepage}
\fancyhead[LO]{\leftmark}
\fancyhead[RE]{\rightmark}


\titleformat{\section}
  {\normalfont\Large\bfseries}{\thesection}{1em}{}[{\titlerule[0.8pt]}]

\AtBeginEnvironment{subappendices}{%
\chapter*{Apéndices}
\addcontentsline{toc}{chapter}{Apéndices}
\counterwithin{figure}{section}
\counterwithin{table}{section}
}



%%%%%%%%%%%%Define colores%%%%%%%%%%%%%%%%%%%%



\definecolor{color1}{rgb}{0.48,0.89,0.84}
\definecolor{color2}{rgb}{0.48,0.89,0.84}
\definecolor{color3}{rgb}{0.28, 0.51, .68}
\definecolor{color4}{rgb}{0.29,0.3,0.57}
\definecolor{color5}{rgb}{0.7,0.24,0.24}
\definecolor{color6}{rgb}{0.72,0.4,0.28}
\definecolor{color7}{rgb}{0.84,0.66,0.21}
\definecolor{color8}{HTML}{8E87C1}


%
%
%
%
%
% %%%%%%%%%%%%%Configuración de fuente para XeLaTeX
% %\setromanfont[Mapping=tex-text]{Oswald-Light}
% %\setsansfont{Roboto Condensed}
% %\setsansfont{Gentium Basic}
% %\setsansfont{FreeMono}
\setromanfont{Oswald-Light}
%\setromanfont{Comfortaa}
 %
%
% \renewcommand{\familydefault}{\sfdefault}
%
%
%
%
% %%%%%%%%%%%%%%%%%%%%%%%%%%Nuevos comandos entornos%%%%%%%%%%%%%%%%%%%%%%%%%%%%%%%%
% %%%%%%%%%%%%%%%%%%%%%%%%%%%%%%%%%%%%%%%%%%%%%%%%%%%%%%%%%%%%%%%%%%%%%%%%
\newenvironment{demo}{\noindent\emph{Dem.}}{\hfill\qed \newline\vspace{5pt}}

\newenvironment{observa}{\noindent\textbf{Observación:}}{}
\renewcommand{\C}{\overline}
\newcommand{\com}{\mathbb{C}}
\newcommand{\rr}{\mathbb{R}}
\newcommand{\nn}{\mathbb{N}}
\renewcommand{\epsilon}{\varepsilon}
\renewcommand{\lim}{\mathop{\rm lím}}
\renewcommand{\inf}{\mathop{\rm ínf}}
\renewcommand{\liminf}{\mathop{\rm líminf}}
\renewcommand{\limsup}{\mathop{\rm límsup}}
\renewcommand{\min}{\mathop{\rm mín}}
\renewcommand{\max}{\mathop{\rm máx}}
\renewcommand{\b}[1]{\boldsymbol{#1}}
\renewenvironment{frame}[1]{}{}
\newcommand{\link}{\reversemarginpar\marginnote\selectfont
\faExternalLink} %
\normalmarginpar
}
\newcommand{\advertencia}{\reversemarginpar\marginnote{\fontsize{16}{16}\selectfont
\faBolt}%
\normalmarginpar }
\newcommand{\lectura}{\reversemarginpar\marginnote{\fontsize{16}{16}\selectfont
\faBook }%
\normalmarginpar}
\newcommand{\actividad}{\reversemarginpar\marginnote{\fontsize{16}{16}\selectfont
\faCogs}%
\normalmarginpar}

%\renewcommand{\lim}{displaystyle\lim}
\DeclareMathOperator{\atan2}{atan2}
\DeclareMathOperator{\sen}{sen}



%%%%%%%%%%Definimos una caja con color
\newlength\mytemplen
\newsavebox\mytempbox
\makeatletter
\newcommand\mybluebox{%
\@ifnextchar[%]
{\@mybluebox}%
{\@mybluebox[0pt]}}
\def\@mybluebox[#1]{%
\@ifnextchar[%]
{\@@mybluebox[#1]}%
{\@@mybluebox[#1][0pt]}}
\def\@@mybluebox[#1][#2]#3{
\sbox\mytempbox{#3}%
\mytemplen\ht\mytempbox
\advance\mytemplen #1\relax
\ht\mytempbox\mytemplen
\mytemplen\dp\mytempbox
\advance\mytemplen #2\relax
\dp\mytempbox\mytemplen
\colorbox{color1}{\hspace{1em}\usebox{\mytempbox}\hspace{1em}}}
\makeatother
\DeclareDocumentCommand\boxedeq{ m g }{%
{\begin{empheq}[box={\mybluebox[2pt][2pt]}]{equation}% #1%
\IfNoValueF {#2} {\label{#2}}%
#1
\end{empheq}
}%
}


%%%%%%%%%%%%%%%%%%
\newboxedtheorem[boxcolor=color1, background=color2, titlebackground=color1,
titleboxcolor = black,thcounter=chapter]{problema}{Problema}{thcounter1}

\newboxedtheorem[boxcolor=color1, background=color2, titlebackground=color1,
titleboxcolor = black,thcounter=chapter]{teorema}{Teorema}{thcounter2}

\newboxedtheorem[boxcolor=color1, background=color2, titlebackground=color1,
titleboxcolor = black,thcounter=chapter]{definicion}{Definici\'on}{thcounter3}

\newboxedtheorem[boxcolor=color1, background=color2, titlebackground=color1,
titleboxcolor = black,thcounter=chapter]{lema}{Lema}{thcounter4}

\newboxedtheorem[boxcolor=color1, background=color2, titlebackground=color1,
titleboxcolor = black,thcounter=chapter]{corolario}{Corolario}{thcounter5}

\newboxedtheorem[boxcolor=color1, background=color2, titlebackground=color1,
titleboxcolor = black,thcounter=chapter]{proposicion}{Proposici\'on}{thcounter6}

\newboxedtheorem[boxcolor=color1, background=color2, titlebackground=color1,
titleboxcolor = black,thcounter=chapter]{codigo}{Función SymPy}{}

\newboxedtheorem[boxcolor=color1, background=color2, titlebackground=color1,
titleboxcolor = black,thcounter=chapter]{ejercicio}{Ejercicio}{}

\newboxedtheorem[boxcolor=color1, background=color2, titlebackground=color1,
titleboxcolor = black,thcounter=chapter]{axioma}{Axioma}{thcounter7}



\newcounter{ejemplo_cont}[chapter]
\setcounter{ejemplo_cont}{1}

\newenvironment{ejemplo}{\noindent\textbf{Ejemplo  \arabic{chapter}.\arabic{ejemplo_cont}.} }{\addtocounter{ejemplo_cont}{1}}


%\renewcommand*{\raggedrightmarginnote}{\centering}

%
%
% %%%%%%%%%%%%%%%%%%%%%%%%%%%%%%%%%%%%%%%%%%%%%%%%%%%%%%%%%%%%%%%%%%%%%%%%%%%%%%%%%%%%%%%%%%%%%%%%%%%%%%%%%%%
% %%%%%%%%%%Para escibir en clase articulo o similar
%








\title{Análisis Real\\ \large Una introducción usando Python}
 

\author{Sonia Acinas y Fernando Mazzone}

%%%%%%%%%%%%%%%%%%%%%%%%%%%%%%%%%%%%%%%%%%%%%%%%%%%%%%%%%%%%%%%%%%%%%%%%%%%%%%%%%%%%%%

\makeindex[title=Indice Conceptos]
\makeindex[name=personas,title=Indice de Personas,columns=3]
\makeindex[name=simbolos,title=Indice Símbolos,columns=3]

\begin{document}



\fontsize{11pt}{11pt}\selectfont

\pagestyle{fancy}

%
 \maketitle
 \tableofcontents
%


\chapter*{Prólogo}

















%
%
%  \bibliographystyle{apalike-url}
%  \bibliography{diferenciales_ecuaciones,diferenciales_ecuaciones_sim}
 
 



\chapter{Sucessiones, series de funciones y sus amigos}
\chapter{Integral de Riemann}

\section{Introducción}

\begin{quotation}
<< Bernard Riemann recibió su doctorado en 1851, su \emph{Habilitación} en 1854. La habilitación confiere el reconocimiento de la capacidad de crear sustanciales contribuciones en la investigación más allá de la tesis doctoral, y es un prerequisito necesario para ocupar un cargo de profesor en una universidad Alemana. Riemann eligió como tema  de habilitación el problema de las series de Fourier. Su tesis fue titulada \emph{\"Uber die Darstellbarkeit einer Function durch eine trigonometrische Reine} (Sobre la representación de una función por series trigonométricas) y respondía la pregunta:  Cuándo una función definida en el intervalo $(-\pi,\pi)$ puede ser respresentada por la serie trigonométrica $a_0/2+\sum_{n=1}^{\infty}[a_n\cos(nx)+b_n\sen(nx)]$? 
\marginpar{\includegraphics[scale=.4]{imagenes/Riemann.jpeg}\\
Bernhard Riemann 1826-1866
} 
En este trabajo  es donde hallamos   la Integral de Riemann, introducida en una sección corta antes del nucleo principal de la tesis, como parte del trabajo preparatorio que él necesitó desarrollar antes de abordar el problema de representabilidad por series trigonométricas. >> 
\end{quotation}
\begin{flushright}
 David M. Bressoud\\
 A Radical Approach to Lebesgue's Theory of Integration.\lectura
\end{flushright}


En este capítulo vamos a desarrollar el concepto de la integral de Riemman. Vamos a exponer la definición de la integral debida a Riemann y la ideada por J. G. Darboux.
Mostraemos la equivalencia de las dos definiciones y discutiremos las propiedades de la intergal, sus alcances y límites. Preparamos así el camino para la introducción de la integral de Lebesgue. 
\marginpar{\includegraphics[scale=.6]{imagenes/Darboux.jpg}\\
Jean G. Darboux  1842-1917
} 

Debemos advertir \advertencia  al alumno que en este curso dejaremos un poco de lado las cuestiones procedimentales de cómo calcular integrales, aspecto que seguramente abordó en cursos anteriores y del cual nos vamos a valer. Tampoco debe esperar que las actividades prácticas se centren en esa dirección.   Nuestro principal objetivo aquí es discutir la materia conceptual ligada a la integral y cómo es previsible las actividades prácticas estarán orientadas con ese propósito.


El concepto de integral encuentra su motivación en diversos problemas. Aparece cuando se busca el centro de masas de un determinado cuerpo, cuando se quieren hallar longitudes de arco, volúmenes, cuando se quiere reconstruir el movimiento de cuerpo conocida su velocidad, etc. La integral es utilizada en incontables otros conceptos matemáticos, como ser el mencionado már arriba relativo a las series de Fourier. 

Quizás el 
problema más simple donde aparece la integral es el que utilizaremos como motivación para introducirla y es el concepto de área.  Vamos a tratar de reconstruir este concepto desde su base, esto es analizando la noción de área de figuras tan simples como rectángulos, triángulos, etc. 



\section{Área de figuras elementales planas} 

  
El cálculo de áreas es necesario en multitud de actividades humanas, por ejemplo con el comercio. La cantidad de muchos productos y servicios se estima en medidas de área, por ejemplo: las telas,  el trabajo de un colocador de pisos,  el precio de la construcción,  el valor de las extensiones de tierra, etc.  
 
 


Por figuras elementales planas nos referimos a rectángulos, triángulos, trapecios, etc. Sin duda el alumno  debe estar  muy familiarizado con las áreas de estas figuras, el área de un rectángulo viene dada por la conocida fórmula $b\times h$, donde $b$ es la base del rectángulo y $h$ su altura.  Ahora bien, ¿Cómo 
se llega a esta fórmula? Porque esta fórmula es apropiada para calcular el precio de un terreno por ejemplo. En esta sección vamos a justificar esta fórmula a partir de algunos hechos elementales.



Vamos a considerar un plano $\mathcal{P}$. En este plano $\mathcal{P}$ supondremos fijada una unidad de longitud.  Pretendemos asignar un área a las figuras, es decir a los subconjuntos, de $\mathcal{P}$. De ahora en más, cómo es usual en esta materia  nos referiremos a \emph{medida}\index{medida} en lugar de área. La medida es un concepto más general  que el concepto de área. No obstante en el contexto en que estamos actualmente son sinónimos.  

Queremos construir pues una función $m$ tal que $m(A)$ reppresente la medida  de  $A\subset\mathcal{P}$. Ahora bien ¿qué podemos usar de guía con ese objetivo? Si, como dijimos,  desconocemos todas las fórmulas previamente aprendidas, sobre que partimos para construir la medida o área. La respuesta es que tomaremos como principio rector  ciertas propiedades que son deseables  que una medida satisfaga. Ellas son las  siguientes. 




\begin{description}
 \item[Positividad.] debería ser una magnitud no negativa.  
 \item[Invariancia por movimientos rígidos.] Si una región es transformada en otra por medio de un movimiento rígido, ambas regiones deberían tener la misma área. Otra manera de expresar esta propiedad es diciendo que dos figuras \emph{congruentes}\index{congruencia} tienen la misma área. 
 \item[Aditividad.] Si una región es la unión de cierta cantidad de regiones más chicas mutuamente disjuntas  
\end{description}

\begin{figure}[h]
\begin{center}
 
\definecolor{xdxdff}{rgb}{0.49,0.49,1}
\definecolor{zzttqq}{rgb}{0.6,0.2,0}
\definecolor{ududff}{rgb}{0.30,0.30,1}
\begin{tikzpicture}[line cap=round,line join=round,x=.9cm,y=.9cm]
\clip(-2.261345665671131,-2.6483801457242535) rectangle (15.749723988011487,4.927708528477943);
\fill[line width=2pt,color=zzttqq,fill=zzttqq,fill opacity=0.10000000149011612] (-1.62,-0.89) -- (-1.62,3.13) -- (1.28,3.15) -- (1.3,-0.89) -- cycle;
\fill[line width=2pt,color=zzttqq,fill=zzttqq,fill opacity=0.10000000149011612] (0.09169246661626662,0.6507662540293884) -- (1.2899992647959808,1.130148511211861) -- (1.28,3.15) -- (-0.3199239037382289,3.1389660420431844) -- cycle;
\fill[line width=2pt,color=zzttqq,fill=zzttqq,fill opacity=0.10000000149011612] (-1.62,1.45) -- (-1.62,3.13) -- (-0.3199239037382289,3.1389660420431844) -- (0.09169246661626662,0.6507662540293884) -- (-0.9454716981132074,0.2358490566037732) -- cycle;
\fill[line width=2pt,color=zzttqq,fill=zzttqq,fill opacity=0.10000000149011612] (-1.62,-0.89) -- (-0.32,-0.89) -- (-1.62,1.45) -- cycle;
\fill[line width=2pt,color=zzttqq,fill=zzttqq,fill opacity=0.10000000149011612] (-0.32,-0.89) -- (-0.9454716981132074,0.2358490566037732) -- (0.09169246661626662,0.6507662540293884) -- (1.2899992647959808,1.130148511211861) -- (1.3,-0.89) -- cycle;
\fill[line width=2pt,color=zzttqq,fill=zzttqq,fill opacity=0.10000000149011612] (8.76,1.77) -- (8.134528301886792,2.8958490566037733) -- (9.171692466616268,3.3107662540293887) -- (10.36999926479598,3.790148511211861) -- (10.38,1.77) -- cycle;
\fill[line width=2pt,color=zzttqq,fill=zzttqq,fill opacity=0.10000000149011612] (3.8264466094067267,-1.199878066911803) -- (2.6385072170133266,-0.011938674518402692) -- (3.551459891609421,0.9136938983359697) -- (5.601939561385962,-0.5546723179904587) -- (5.16194451123264,-1.5814488960049211) -- cycle;
\fill[line width=2pt,color=zzttqq,fill=zzttqq,fill opacity=0.10000000149011612] (5.6888024763961145,1.8814084921516754) -- (6.19715889449669,3.0677137999207305) -- (4.761837661840736,4.4888939366884495) -- (3.638322806619573,3.3497747084781038) -- cycle;
\fill[line width=2pt,color=zzttqq,fill=zzttqq,fill opacity=0.10000000149011612] (7.14,0.65) -- (5.84,0.65) -- (7.14,-1.69) -- cycle;
\draw [line width=2pt,color=zzttqq] (-1.62,-0.89)-- (-1.62,3.13);
\draw [line width=2pt,color=zzttqq] (-1.62,3.13)-- (1.28,3.15);
\draw [line width=2pt,color=zzttqq] (1.28,3.15)-- (1.3,-0.89);
\draw [line width=2pt,color=zzttqq] (1.3,-0.89)-- (-1.62,-0.89);
\draw [line width=2pt] (-1.62,1.45)-- (-0.32,-0.89);
\draw [line width=2pt] (-0.9454716981132074,0.2358490566037732)-- (1.2899992647959808,1.130148511211861);
\draw [line width=2pt] (-0.3199239037382289,3.1389660420431844)-- (0.09169246661626662,0.6507662540293884);
\draw [line width=2pt,color=zzttqq] (0.09169246661626662,0.6507662540293884)-- (1.2899992647959808,1.130148511211861);
\draw [line width=2pt,color=zzttqq] (1.2899992647959808,1.130148511211861)-- (1.28,3.15);
\draw [line width=2pt,color=zzttqq] (1.28,3.15)-- (-0.3199239037382289,3.1389660420431844);
\draw [line width=2pt,color=zzttqq] (-0.3199239037382289,3.1389660420431844)-- (0.09169246661626662,0.6507662540293884);
\draw [line width=2pt,color=zzttqq] (-1.62,1.45)-- (-1.62,3.13);
\draw [line width=2pt,color=zzttqq] (-1.62,3.13)-- (-0.3199239037382289,3.1389660420431844);
\draw [line width=2pt,color=zzttqq] (-0.3199239037382289,3.1389660420431844)-- (0.09169246661626662,0.6507662540293884);
\draw [line width=2pt,color=zzttqq] (0.09169246661626662,0.6507662540293884)-- (-0.9454716981132074,0.2358490566037732);
\draw [line width=2pt,color=zzttqq] (-0.9454716981132074,0.2358490566037732)-- (-1.62,1.45);
\draw [line width=2pt,color=zzttqq] (-1.62,-0.89)-- (-0.32,-0.89);
\draw [line width=2pt,color=zzttqq] (-0.32,-0.89)-- (-1.62,1.45);
\draw [line width=2pt,color=zzttqq] (-1.62,1.45)-- (-1.62,-0.89);
\draw [line width=2pt,color=zzttqq] (-0.32,-0.89)-- (-0.9454716981132074,0.2358490566037732);
\draw [line width=2pt,color=zzttqq] (-0.9454716981132074,0.2358490566037732)-- (0.09169246661626662,0.6507662540293884);
\draw [line width=2pt,color=zzttqq] (0.09169246661626662,0.6507662540293884)-- (1.2899992647959808,1.130148511211861);
\draw [line width=2pt,color=zzttqq] (1.2899992647959808,1.130148511211861)-- (1.3,-0.89);
\draw [line width=2pt,color=zzttqq] (1.3,-0.89)-- (-0.32,-0.89);
\draw [line width=2pt,color=zzttqq] (8.76,1.77)-- (8.134528301886792,2.8958490566037733);
\draw [line width=2pt,color=zzttqq] (8.134528301886792,2.8958490566037733)-- (9.171692466616268,3.3107662540293887);
\draw [line width=2pt,color=zzttqq] (9.171692466616268,3.3107662540293887)-- (10.36999926479598,3.790148511211861);
\draw [line width=2pt,color=zzttqq] (10.36999926479598,3.790148511211861)-- (10.38,1.77);
\draw [line width=2pt,color=zzttqq] (10.38,1.77)-- (8.76,1.77);
\draw [line width=2pt,color=zzttqq] (3.8264466094067267,-1.199878066911803)-- (2.6385072170133266,-0.011938674518402692);
\draw [line width=2pt,color=zzttqq] (2.6385072170133266,-0.011938674518402692)-- (3.551459891609421,0.9136938983359697);
\draw [line width=2pt,color=zzttqq] (3.551459891609421,0.9136938983359697)-- (5.601939561385962,-0.5546723179904587);
\draw [line width=2pt,color=zzttqq] (5.601939561385962,-0.5546723179904587)-- (5.16194451123264,-1.5814488960049211);
\draw [line width=2pt,color=zzttqq] (5.16194451123264,-1.5814488960049211)-- (3.8264466094067267,-1.199878066911803);
\draw [line width=2pt,color=zzttqq] (5.6888024763961145,1.8814084921516754)-- (6.19715889449669,3.0677137999207305);
\draw [line width=2pt,color=zzttqq] (6.19715889449669,3.0677137999207305)-- (4.761837661840736,4.4888939366884495);
\draw [line width=2pt,color=zzttqq] (4.761837661840736,4.4888939366884495)-- (3.638322806619573,3.3497747084781038);
\draw [line width=2pt,color=zzttqq] (3.638322806619573,3.3497747084781038)-- (5.6888024763961145,1.8814084921516754);
\draw [line width=2pt,color=zzttqq] (7.14,0.65)-- (5.84,0.65);
\draw [line width=2pt,color=zzttqq] (5.84,0.65)-- (7.14,-1.69);
\draw [line width=2pt,color=zzttqq] (7.14,-1.69)-- (7.14,0.65);
\draw (-1.1356538123159674,2.2366014415507594) node[anchor=north west] {$A_1$};
\draw (3.9651373981996176,0.03798454046645946) node[anchor=north west] {$A_1$};
\draw (0.23628313396063827,2.5883801457242477) node[anchor=north west] {$A_2$};
\draw (4.703872676963944,3.83719454554013) node[anchor=north west] {$A_2$};
\draw (0.09557165229124281,0.5304747263093427) node[anchor=north west] {$A_3$};
\draw (9.101106479132552,3.1160482019844795) node[anchor=north west] {$A_3$};
\draw (-1.5753771925328282,0.31940750380524985) node[anchor=north west] {$A_4$};
\draw (6.445177262622713,0.5480636615180171) node[anchor=north west] {$A_4$};
\begin{scriptsize}
\draw [fill=ududff] (-1.62,-0.89) circle (2.5pt);
\draw [fill=ududff] (-1.62,3.13) circle (2.5pt);
\draw [fill=ududff] (1.28,3.15) circle (2.5pt);
\draw [fill=ududff] (1.3,-0.89) circle (2.5pt);
\draw [fill=xdxdff] (-1.62,1.45) circle (2.5pt);
\draw [fill=xdxdff] (-0.32,-0.89) circle (2.5pt);
\draw [fill=xdxdff] (-0.9454716981132074,0.2358490566037732) circle (2.5pt);
\draw [fill=xdxdff] (1.2899992647959808,1.130148511211861) circle (2.5pt);
\draw [fill=xdxdff] (-0.3199239037382289,3.1389660420431844) circle (2.5pt);
\draw [fill=xdxdff] (0.09169246661626662,0.6507662540293884) circle (2.5pt);
\draw [fill=xdxdff] (8.76,1.77) circle (2.5pt);
\draw [fill=xdxdff] (8.134528301886792,2.8958490566037733) circle (2.5pt);
\draw [fill=xdxdff] (10.36999926479598,3.790148511211861) circle (2.5pt);
\draw [fill=ududff] (10.38,1.77) circle (2.5pt);
\draw [fill=xdxdff] (3.8264466094067267,-1.199878066911803) circle (2.5pt);
\draw [fill=ududff] (2.6385072170133266,-0.011938674518402692) circle (2.5pt);
\draw [fill=xdxdff] (3.551459891609421,0.9136938983359697) circle (2.5pt);
\draw [fill=xdxdff] (5.601939561385962,-0.5546723179904587) circle (2.5pt);
\draw [fill=xdxdff] (5.16194451123264,-1.5814488960049211) circle (2.5pt);
\draw [fill=xdxdff] (5.6888024763961145,1.8814084921516754) circle (2.5pt);
\draw [fill=xdxdff] (6.19715889449669,3.0677137999207305) circle (2.5pt);
\draw [fill=ududff] (4.761837661840736,4.4888939366884495) circle (2.5pt);
\draw [fill=xdxdff] (3.638322806619573,3.3497747084781038) circle (2.5pt);
\draw [fill=ududff] (7.14,0.65) circle (2.5pt);
\draw [fill=xdxdff] (5.84,0.65) circle (2.5pt);
\draw [fill=xdxdff] (7.14,-1.69) circle (2.5pt);
\end{scriptsize}
\end{tikzpicture}


 \caption{El área del rectángulo es la suma de sus partes}\label{fig:rect_descop} 
\end{center}
\end{figure}

Utilizando la segunda y tercer propiedad se pueden relacionar el área del rectángulo de la figura \ref{fig:rect_descop} con las cuatro regiones en la que es dividido.

Como veremos a lo largo de la materia la propiedad de aditividad debe ser estudiada con cuidado, esto ocurre por las intrincadas maneras en que una región puede ser unión de otras regiones. A lo largo de esta materia elaboraremos una  teoría que nos dará una descripción  precisa de a que conjuntos podemos asignarle una medida de modo que las propiedades previas sean ciertas. 

Por el momento veamos como las propiedades anteriores determinan practicamente de manera unívoca la medida de regiones elementales planas.  


Hablando de propiedades de la medida, supongamos que $A$ y $B$ son dos regiones con $A\subset B$. Entonces como $B=A\cup (B-A)$ y por la propiedad de aditividad y positividad

\[
 m(B)=m(A)+m(B-A)\geq m(A).
\]

Descubrimos así que nuestra medida deberá tener adicionalmente la siguiente propiedad:
\begin{description}
 \item[Monotonía.] Si $A\subset B$ entonces $m(A)\leq m(B)$. 
\end{description}

Es claro que si logramos construir una medida que satisfaga las propiedades anteriores cualquier multiplo por un número real positivo  de ella seguirá cumpliendo las propiedades. Esto es una manera de expresar el hecho que podemos usar diferentes unidades de medición. Esta cuestión se sortea proponiendo la unidad de medida. Esta unidad es completamente arbitraria, ud. podría elegir su figura plana preferida como unidad de área. \marginpar{ Podríamos por ejemplo elegir el círculo de radio uno como unidad de área. Así ya no tendríamos el problema de ese número raro $\pi$ que aparece en la fórmula del área del círculo. ¡El área de cualquier círculo sería igual a su radio al cuadrado! Claro que aparecería $\pi$  en la fórmula del área del cuadrado de lado 1. Nos tapamos los pies y se destapa el cuerpo.}  Cómo es habitual, elijamos el cuadrado cuyos lados miden la unidad de longitud previamente fijada. 


Supongamos ahora que tenemos un rectángulo de un lado igual a la unidad y el otro de lado un racional $n/m$, $n,m\in\mathbb{N}$. Veamos que la aditividad, la invariancia por movimientos rígidos y el hecho que decidimos que el cuadrado de lados igual a la unidad determinan el área de este rectángulo. Primero observar que si dividimos el lado de cuadrado unidad en $m$ segmentos iguales de longitud. \marginpar{
 
\definecolor{xdxdff}{rgb}{0.49,0.49,1}
\definecolor{zzttqq}{rgb}{0.6,0.2,0}
\definecolor{ududff}{rgb}{0.30,0.30,1}
\begin{tikzpicture}[x=2.1cm,y=2.1cm]
\clip(-0.07,0) rectangle (1.5,1.5);
\draw [line width=2pt,color=zzttqq] (0,0) -- (1,0) -- (1,1) -- (0,1) -- cycle;
\draw [line width=1pt,color=zzttqq, dashed] (0,.2)--(1,.2);
\draw [line width=1pt,color=zzttqq, dashed] (0,.4)--(1,.4);
\draw [line width=1pt,color=zzttqq, dashed] (0,.6)--(1,.6);
\draw [line width=1pt,color=zzttqq, dashed] (0,.8)--(1,.8);
\draw (0,1) node[anchor=south] {$Q$};
\draw (0.5,.1) node[anchor=center] {$R_1$};
\draw (0.5,.3) node[anchor=center] {$R_2$};
\draw (0.5,.5) node[anchor=center] {$R_3$};
\draw (0.5,.7) node[anchor=center] {$R_4$};
\draw (0.5,.9) node[anchor=center] {$R_5$};



\end{tikzpicture}
\\
 }
Queda dividido el cuadrado en $m$ rectángulos $R_1,\ldots,R_m$ (ver figura en el margen), todos ellos  congruentes entre si, de modo que todos tienen la misma medida, digamos $m(R_1)$. La unión de ellos es el cuadrado que por convención dijimos que tiene medida 1. De modo que por la aditividad debe ocurrir que $m(R_1)=\cdots =m(R_m))=1/m$. Recordemos nuestra pretención de inferir la medida de un rectángulo $R$ de lado 1 y otro $n/m$. Este rectángulo esta compuesto de $n$ rectángulos congruentes a los $R_i$, $i=1,\ldots,m$, nuevamente por la aditividad inferimos que $m(R)=n/m$. 

Sea ahora una rectángulo $R$ con un lado unidad y el otro un real cualquiera $l>0$. Existen sendas sucesiones $0<q_k,p_k\in\mathbb{Q}$, $k\in\mathbb{N}$, tales que $q_1\leq q_2\leq\cdots \leq l \leq \cdots\leq p_2\leq p_1$ y $\lim_{k\to\infty}q_k =\lim_{k\to\infty} p_k=l$. Consideremos una dos sucesiones de rectángulos $R_k$ y $S_k$ que comparten el lado de $R$ igual a la unidad, mientras que el otro lado de $R_k$ y $S_k$ es igual a $q_k$ y $p_k$ respectivamente. Luego por la monotonía
\[
 q_k=m(R_k)\leq m(R) \leq m(S_k)\leq p_k.
\]
Tomando límite cuando $k\to\infty$ inferimos que $m(R)=l$. 



\begin{figure}[h]
 \begin{center}
 \input{imagenes/ParalTria.tikz} 
 \end{center}
 \caption{Áreas de otras figuras elementales.}\label{fig:paral-trig}
\end{figure}


A partir de las propiedades fundamentales que postulamos para la medida o área inferimos la famosa fórmula del área de un rectángulo en el caso que uno de los lados sea igual a la unidad. Para un  rectángulo arbitrario. En la figura \ref{fig:paral-trig} se muestra como relacionar el área de un paralelepípedo con la de un rectángulo y la de un triángulo con la de un paralelepípedo para inferir las conocidas fórmulas para estas figuras.



\section{Integral de Riemann}

En esta sección abordaremos el problema del área de regiones planas. Vamos a contextualizarnos dentro del marco conceptual que nos brinda la geometría analítica. Mediante coordenadas cartesianas ortogonales los puntos del plano se identifican con pares ordenados $(x,y)\in\mathbb{R}^2$ y el plano con el conjunto $\rr^2$.  Nuestro propósito es entonces definir la medida de subconjuntos de $\mathbb{R}^2$. La geometría analítica abre así nuevas posibilidades para abordar el problema del área. 

Nuestra primera aproximación será la que propuso Bernhard Riemann en 1854, pero seguiremos  el enfoque de Jean Darboux. En esta parte de nuestra aproximación consideraremos subconjuntos de $\mathbb{R}^2$ de un tipo especial, concretamente a conjuntos que quedan encerrados entre la gráfica de una función y del eje coordenadas $x$.   
 


\chapter{Sucesiones, series de funciones y sus amigos}


\section{Sucesiones de funciones}
Sea $K$ un espacio métrico, usualmente $K\subset \mathbb{R}^n$ para algún $n$. 

Una colección $f_n:K\to \mathbb{R}$ para $n=1,2,3,\ldots$ se llama sucesión de funciones.

Dada una sucesión de funciones $f_n$, $n=1,2,3,\ldots$  y $x\in K$, $f_n(x)$
es una sucesión de números reales y como tal puede o no converger a cierto límite.

La mayor diferencia entre una sucesión de números reales y una sucesión de funciones es
el hecho que en una sucesión de funciones los términos de la sucesión cambian cuando la variable
$x$ cambia. 
Por lo tanto el límite también puede cambiar, en caso de existir, y por consiguiente el límite
también es una función de $x$.
De manera que es necesario tener presente que cuando una sucesión de funciones es evaluada en
un valor de  $x$ particular resulta una sucesión de números reales.

Supongamos que para todo $x \in K$ la sucesión de números reales $f_n(x)$ converge,  
es decir que existe el $\lim\limits_{n \to \infty} f_n(x)$ y lo denotaremos $f(x)$. 
En este caso diremos que $f_n$ converge puntualmente a $f$.
 
\begin{ejemplo}{ej:sucesion-conv-puntual}
La sucesión $f_n(x)=\frac{1}{1+nx^2}$ converge puntualmente 
\[f(x)=\left\{\begin{array}{ll}
0&x\neq 0
\\
1&x=0
\end{array}
\right.\]
\textbf{Justificación:}
\\
Claramente si $x=0$ tenemos que $f_n(0)=1$ para todo $n \in \mathbb{N}$ y entonces $\lim\limits_{n \to \infty}f_n(0)=1$.

Si $x\neq 0$ entonces $nx^2\to \infty$ cuando $n\to \infty$  y por lo tanto $\lim\limits_{n \to \infty} \frac{1}{1+nx^2}=0$.

Como vemos, la determinación de la convergencia puntual suele reducirse al cálculo de un límite. 
Para este propósito es lícito usar todas las técnicas estudiadas en cursos anteriores como puede ser la Regla de L'H\^opital.
\end{ejemplo}

\begin{ejemplo}{ej:sucesion-conv-unif-1}
La sucesión $f_n(x)=\frac{n^2x-n^2}{1+n^2x}$ converge a $f(x)=\frac{x-1}{x}$ si $x \neq 0$.

Si $x=0$ no converge.

Es necesario ser cuidadoso con la justificación. Por ejemplo, la Regla de L'H\^opital  sólo puede usarse en casos de indeterminaciones.

Si $x=0$ no hay indeterminación pues $f_n(0)=-n^2 $ y 
$\lim\limits_{n \to \infty}f_n(0)=\lim\limits_{n \to \infty}(-n^2)=-\infty$. 

Es lícito decir $\lim\limits_{n \to \infty}f_n(0)=-\infty$ en lugar de que $f_n$ no converge en $x=0$.

Cuando $x=1$ tampoco hay indeterminación pues $f_n(1)=\frac{0}{1+n^2}$ y por tanto $\lim\limits_{n \to \infty} f_n(1)=0$. 
 
Si $x\neq 0$ y $x \neq 1$ podemos usar la Regla de L'H\^opital dado que se tiene la indeterminación $\frac{\infty}{\infty}$. 
En efecto, 
$\lim\limits_{n \to \infty}f_n(x)=
\lim\limits_{n \to \infty}\frac{n^2x-n^2}{1+n^2x}=
\lim\limits_{n \to \infty}\frac{2nx-2n}{2nx}=
\lim\limits_{n \to \infty}\frac{2x-2}{2x}=\frac{1-x}{x}.
$
Observemos que si $f(x)=\frac{1-x}{x}$ entonces $f(1)=0$ y por tanto $\lim\limits_{n\to \infty}f_n(x)=f(x)$ $\forall x\neq 0$.

Suele ser útil graficar algunas funciones de la sucesión y la función límite, ya sea empleando los procedimientos aprendidos 
en materias anteriores o usando sympy.
\end{ejemplo}

Para el Ejemplo \ref{ej:sucesion-conv-puntual} APARECE MAL LA REFERENCIA DEL EJEMPLO!!!!!!!!!!

 \begin{tcolorbox}[breakable, size=fbox, boxrule=1pt, pad at break*=1mm,colback=cellbackground, colframe=cellborder]
\prompt{In}{incolor}{1}{\hspace{4pt}}
\begin{Verbatim}[commandchars=\\\{\}]
\PY{k+kn}{from} \PY{n+nn}{sympy} \PY{k+kn}{import} \PY{o}{*}
\PY{n}{init\PYZus{}printing}\PY{p}{(}\PY{p}{)}
\end{Verbatim}
\end{tcolorbox}

    %Ejemplo \(f_n(x)=\frac{1}{1+nx^2}\)

    \begin{tcolorbox}[breakable, size=fbox, boxrule=1pt, pad at break*=1mm,colback=cellbackground, colframe=cellborder]
\prompt{In}{incolor}{2}{\hspace{4pt}}
\begin{Verbatim}[commandchars=\\\{\}]
\PY{n}{x}\PY{p}{,}\PY{n}{n}\PY{o}{=}\PY{n}{symbols}\PY{p}{(}\PY{l+s+s1}{\PYZsq{}}\PY{l+s+s1}{x,n}\PY{l+s+s1}{\PYZsq{}}\PY{p}{)}
\PY{n}{fn}\PY{o}{=}\PY{l+m+mi}{1}\PY{o}{/}\PY{p}{(}\PY{l+m+mi}{1}\PY{o}{+}\PY{n}{n}\PY{o}{*}\PY{n}{x}\PY{o}{*}\PY{o}{*}\PY{l+m+mi}{2}\PY{p}{)}
\PY{n}{p}\PY{o}{=}\PY{n}{plot}\PY{p}{(}\PY{n}{fn}\PY{o}{.}\PY{n}{subs}\PY{p}{(}\PY{n}{n}\PY{p}{,}\PY{l+m+mi}{1}\PY{p}{)}\PY{p}{,} \PY{p}{(}\PY{n}{x}\PY{p}{,}\PY{o}{\PYZhy{}}\PY{l+m+mi}{5}\PY{p}{,}\PY{l+m+mi}{5}\PY{p}{)}\PY{p}{,}\PY{n}{show}\PY{o}{=}\PY{n}{false}\PY{p}{)}
\PY{k}{for} \PY{n}{k} \PY{o+ow}{in} \PY{n+nb}{range}\PY{p}{(}\PY{l+m+mi}{2}\PY{p}{,}\PY{l+m+mi}{100}\PY{p}{)}\PY{p}{:}
    \PY{n}{p}\PY{o}{.}\PY{n}{append}\PY{p}{(}\PY{n}{plot}\PY{p}{(}\PY{n}{fn}\PY{o}{.}\PY{n}{subs}\PY{p}{(}\PY{n}{n}\PY{p}{,}\PY{n}{k}\PY{p}{)}\PY{p}{,} \PY{p}{(}\PY{n}{x}\PY{p}{,}\PY{o}{\PYZhy{}}\PY{l+m+mi}{5}\PY{p}{,}\PY{l+m+mi}{5}\PY{p}{)}\PY{p}{,}\PY{n}{show}\PY{o}{=}\PY{n}{false}\PY{p}{)}\PY{p}{[}\PY{l+m+mi}{0}\PY{p}{]}\PY{p}{)}
\PY{n}{p}\PY{o}{.}\PY{n}{show}\PY{p}{(}\PY{p}{)}
\end{Verbatim}
\end{tcolorbox}

    \begin{center}
    \adjustimage{max size={0.9\linewidth}{0.9\paperheight}}{python/uni3/output_3_0.png}
    \end{center}
    { \hspace*{\fill} \\}

Para el Ejemplo \ref{ej:sucesion-conv-unif-1}


    \begin{tcolorbox}[breakable, size=fbox, boxrule=1pt, pad at break*=1mm,colback=cellbackground, colframe=cellborder]
\prompt{In}{incolor}{3}{\hspace{4pt}}
\begin{Verbatim}[commandchars=\\\{\}]
\PY{n}{x}\PY{p}{,}\PY{n}{n}\PY{o}{=}\PY{n}{symbols}\PY{p}{(}\PY{l+s+s1}{\PYZsq{}}\PY{l+s+s1}{x,n}\PY{l+s+s1}{\PYZsq{}}\PY{p}{)}
\PY{n}{fn}\PY{o}{=}\PY{p}{(}\PY{n}{n}\PY{o}{*}\PY{o}{*}\PY{l+m+mi}{2}\PY{o}{*}\PY{n}{x}\PY{o}{\PYZhy{}}\PY{n}{n}\PY{o}{*}\PY{o}{*}\PY{l+m+mi}{2}\PY{p}{)}\PY{o}{/}\PY{p}{(}\PY{l+m+mi}{1}\PY{o}{+}\PY{n}{n}\PY{o}{*}\PY{n}{x}\PY{o}{*}\PY{o}{*}\PY{l+m+mi}{2}\PY{p}{)}
\PY{n}{p}\PY{o}{=}\PY{n}{plot}\PY{p}{(}\PY{n}{fn}\PY{o}{.}\PY{n}{subs}\PY{p}{(}\PY{n}{n}\PY{p}{,}\PY{l+m+mi}{1}\PY{p}{)}\PY{p}{,} \PY{p}{(}\PY{n}{x}\PY{p}{,}\PY{o}{\PYZhy{}}\PY{l+m+mi}{5}\PY{p}{,}\PY{l+m+mi}{5}\PY{p}{)}\PY{p}{,}\PY{n}{show}\PY{o}{=}\PY{n}{false}\PY{p}{,}\PY{n}{ylim}\PY{o}{=}\PY{p}{(}\PY{o}{\PYZhy{}}\PY{l+m+mi}{20}\PY{p}{,}\PY{l+m+mi}{10}\PY{p}{)}\PY{p}{)}
\PY{k}{for} \PY{n}{k} \PY{o+ow}{in} \PY{n+nb}{range}\PY{p}{(}\PY{l+m+mi}{2}\PY{p}{,}\PY{l+m+mi}{10}\PY{p}{)}\PY{p}{:}
    \PY{n}{p}\PY{o}{.}\PY{n}{append}\PY{p}{(}\PY{n}{plot}\PY{p}{(}\PY{n}{fn}\PY{o}{.}\PY{n}{subs}\PY{p}{(}\PY{n}{n}\PY{p}{,}\PY{n}{k}\PY{p}{)}\PY{p}{,} \PY{p}{(}\PY{n}{x}\PY{p}{,}\PY{o}{\PYZhy{}}\PY{l+m+mi}{5}\PY{p}{,}\PY{l+m+mi}{5}\PY{p}{)}\PY{p}{,}\PY{n}{show}\PY{o}{=}\PY{n}{false}\PY{p}{)}\PY{p}{[}\PY{l+m+mi}{0}\PY{p}{]}\PY{p}{)}
\PY{n}{p}\PY{o}{.}\PY{n}{show}\PY{p}{(}\PY{p}{)}
\end{Verbatim}
\end{tcolorbox}

    \begin{center}
    \adjustimage{max size={0.9\linewidth}{0.9\paperheight}}{python/uni3/output_5_0.png}
    \end{center}
    { \hspace*{\fill} \\}		
En los ejemplos anteriores se forma una ``montaña'' alrededor de un punto \emph{fijo} ($x=0$). 
Pero, puede ocurrir otro comportamiento que observaremos los siguientes ejemplos.

\begin{ejemplo}{}
Si  $f_n(x)=\left\{
\begin{array}{ll}
1&n\leq x\leq n+1
\\
0&\mbox{en otro caso}
\end{array}
\right.$
entonces $\lim\limits_{n \to \infty} f_n(x)=0$. 
\end{ejemplo} 

GRAFICAR CON SYMPY!!!

A partir del gráfico vemos que los términos de la sucesión $f_n(x)$ son ``montañas móviles'' de altura 1.


\begin{ejemplo}{ej:montaña-movil-cte}
Si  $f_n(x)=\frac{nx}{1+n^2x^2}$ en $[0,\infty)$
entonces 
$\lim\limits_{n \to \infty} \frac{nx}{1+n^2x^2}=
\lim\limits_{n\to \infty} \frac{x}{x}\frac{x}{\frac{1}{n}+nx^2}=
\lim\limits_{n \to \infty}\frac{x}{\frac{1}{n}+nx^2}=0.$ 
\end{ejemplo}


    \begin{tcolorbox}[breakable, size=fbox, boxrule=1pt, pad at break*=1mm,colback=cellbackground, colframe=cellborder]
\prompt{In}{incolor}{4}{\hspace{4pt}}
\begin{Verbatim}[commandchars=\\\{\}]
\PY{n}{x}\PY{p}{,}\PY{n}{n}\PY{o}{=}\PY{n}{symbols}\PY{p}{(}\PY{l+s+s1}{\PYZsq{}}\PY{l+s+s1}{x,n}\PY{l+s+s1}{\PYZsq{}}\PY{p}{)}
\PY{n}{fn}\PY{o}{=}\PY{n}{n}\PY{o}{*}\PY{n}{x}\PY{o}{/}\PY{p}{(}\PY{l+m+mi}{1}\PY{o}{+}\PY{n}{n}\PY{o}{*}\PY{o}{*}\PY{l+m+mi}{2}\PY{o}{*}\PY{n}{x}\PY{o}{*}\PY{o}{*}\PY{l+m+mi}{2}\PY{p}{)}
\PY{n}{p}\PY{o}{=}\PY{n}{plot}\PY{p}{(}\PY{n}{fn}\PY{o}{.}\PY{n}{subs}\PY{p}{(}\PY{n}{n}\PY{p}{,}\PY{l+m+mi}{1}\PY{p}{)}\PY{p}{,} \PY{p}{(}\PY{n}{x}\PY{p}{,}\PY{l+m+mi}{0}\PY{p}{,}\PY{l+m+mi}{5}\PY{p}{)}\PY{p}{,}\PY{n}{show}\PY{o}{=}\PY{n}{false}\PY{p}{)}
\PY{k}{for} \PY{n}{k} \PY{o+ow}{in} \PY{n+nb}{range}\PY{p}{(}\PY{l+m+mi}{2}\PY{p}{,}\PY{l+m+mi}{10}\PY{p}{)}\PY{p}{:}
    \PY{n}{p}\PY{o}{.}\PY{n}{append}\PY{p}{(}\PY{n}{plot}\PY{p}{(}\PY{n}{fn}\PY{o}{.}\PY{n}{subs}\PY{p}{(}\PY{n}{n}\PY{p}{,}\PY{n}{k}\PY{p}{)}\PY{p}{,} \PY{p}{(}\PY{n}{x}\PY{p}{,}\PY{l+m+mi}{0}\PY{p}{,}\PY{l+m+mi}{5}\PY{p}{)}\PY{p}{,}\PY{n}{show}\PY{o}{=}\PY{n}{false}\PY{p}{)}\PY{p}{[}\PY{l+m+mi}{0}\PY{p}{]}\PY{p}{)}
\PY{n}{p}\PY{o}{.}\PY{n}{show}\PY{p}{(}\PY{p}{)}
\end{Verbatim}
\end{tcolorbox}

    \begin{center}
    \adjustimage{max size={0.9\linewidth}{0.9\paperheight}}{python/uni3/output_7_0.png}
    \end{center}
    { \hspace*{\fill} \\}
En este caso también se observa una montaña móvil. 

HACER EL ANÁLISIS CON LA DERIVADA!!!

\begin{ejemplo}{}
Si $f_n(x)=\sqrt{x^2+\frac{1}{n^2}}$, $x \in \mathbb{R}$ luego 
$f_n^{'}(x)=\frac{x}{\sqrt{x^2+\frac{1}{n^2}}}$. 
Entonces $f_n^{'}(x)>0$ en $(0,+\infty)$ y $f_n^{'}(x)<0$ en $(0,+\infty)$, de donde  $(0,+\infty)$ es intervalo de
crecimiento para cada $f_n(x)$ y $(-\infty,0)$ es intervalo de decrecimiento para cada $f_n(x)$.
Luego cada $f_n(x)$ tiene un mínimo en $x=0$ y el valor mínimo es $f_n(0)=\frac{1}{n}.$

Por otra parte, $\lim\limits_{n \to \infty} f_n(x)=\lim\limits_{n \to \infty} \sqrt{x^2+\frac{1}{n^2}}=\sqrt{x^2}=|x|$.
\end{ejemplo}

 \begin{tcolorbox}[breakable, size=fbox, boxrule=1pt, pad at break*=1mm,colback=cellbackground, colframe=cellborder]
\prompt{In}{incolor}{5}{\hspace{4pt}}
\begin{Verbatim}[commandchars=\\\{\}]
\PY{k}{def} \PY{n+nf}{grafica}\PY{p}{(}\PY{n}{f}\PY{p}{,}\PY{n}{x1}\PY{p}{,}\PY{n}{x2}\PY{p}{,}\PY{n}{m}\PY{p}{)}\PY{p}{:}
    \PY{n}{p}\PY{o}{=}\PY{n}{plot}\PY{p}{(}\PY{n}{f}\PY{o}{.}\PY{n}{subs}\PY{p}{(}\PY{n}{n}\PY{p}{,}\PY{l+m+mi}{1}\PY{p}{)}\PY{p}{,} \PY{p}{(}\PY{n}{x}\PY{p}{,}\PY{n}{x1}\PY{p}{,}\PY{n}{x2}\PY{p}{)}\PY{p}{,}\PY{n}{show}\PY{o}{=}\PY{n}{false}\PY{p}{)}
    \PY{k}{for} \PY{n}{k} \PY{o+ow}{in} \PY{n+nb}{range}\PY{p}{(}\PY{l+m+mi}{2}\PY{p}{,}\PY{n}{m}\PY{p}{)}\PY{p}{:}
        \PY{n}{p}\PY{o}{.}\PY{n}{append}\PY{p}{(}\PY{n}{plot}\PY{p}{(}\PY{n}{f}\PY{o}{.}\PY{n}{subs}\PY{p}{(}\PY{n}{n}\PY{p}{,}\PY{n}{k}\PY{p}{)}\PY{p}{,} \PY{p}{(}\PY{n}{x}\PY{p}{,}\PY{n}{x1}\PY{p}{,}\PY{n}{x2}\PY{p}{)}\PY{p}{,}\PY{n}{show}\PY{o}{=}\PY{n}{false}\PY{p}{)}\PY{p}{[}\PY{l+m+mi}{0}\PY{p}{]}\PY{p}{)}
    \PY{n}{p}\PY{o}{.}\PY{n}{show}\PY{p}{(}\PY{p}{)}
\PY{n}{f}\PY{o}{=}\PY{n}{sqrt}\PY{p}{(}\PY{n}{x}\PY{o}{*}\PY{o}{*}\PY{l+m+mi}{2}\PY{o}{+}\PY{l+m+mf}{1.0}\PY{o}{/}\PY{n}{n}\PY{o}{*}\PY{o}{*}\PY{l+m+mi}{2}\PY{p}{)}
\PY{n}{grafica}\PY{p}{(}\PY{n}{f}\PY{p}{,}\PY{o}{\PYZhy{}}\PY{l+m+mi}{5}\PY{p}{,}\PY{l+m+mi}{5}\PY{p}{,}\PY{l+m+mi}{10}\PY{p}{)}
\end{Verbatim}
\end{tcolorbox}

    \begin{center}
    \adjustimage{max size={0.9\linewidth}{0.9\paperheight}}{python/uni3/output_9_0.png}
    \end{center}
    { \hspace*{\fill} \\}

En el Análisis Matemático, además de límites tenemos conceptos como continuidad, derivadas, integrales, etc.
Es común operar expresiones conjugando varios de ellos y queremos contar con relaciones entre ellos que permitan 
transformar las expresiones. 

Por ejemplo, ?`es importante el orden en que se realizan  las operaciones?
?`Es lo mismo tomar límite y luego derivar que hacerlo en el orden inverso? 
Si se tienen dos límites, ?`se pueden permutar?

\begin{ejemplo}{ej:sucesion-conv-2 puntos}
Si $f_n(x)=\sen(nx)$ para $x\in [0,\pi]$ entonces $f_n(0)=f_n(\pi)=0$ y la sucesión converge en estos valores.

Veamos que la sucesión de funciones dada no converge en ningún otro valor. 

Supongamos que $x\neq0$, $x\neq \pi$ y $\lim\limits_{n \to \infty}\sen(nx)=\alpha$. 

Si se tuviese $\alpha\neq 0$ entonces
\[
1=\lim\limits_{n \to \infty}\frac{\sen(2nx)}{\sen(x)}=2\lim\limits_{n \to \infty}\cos(nx)
\]
de donde $\lim\limits_{n \to \infty} \cos(nx)=\frac{1}{2}$.
Sin embargo, 
\[
\lim\limits_{n \to \infty}\cos(2nx)=
\lim\limits_{n \to \infty}2\cos^2(nx)-1=-\frac{1}{2}
\]
lo que nos lleva a una contradicción. 

Por lo tanto, $\lim\limits_{n \to \infty}\sen(nx)=0$.
Entonces 
$\lim\limits_{n \to \infty}|\cos(2nx)|=1
$
y 
\[
0=\lim\limits_{n \to \infty} |\sen[(n+1)x]|=
\lim\limits_{n \to \infty} |\sen(nx)\cos x+\cos(nx)\sen x|=|\sen x|
\]
y necesariamente  $x=0$ \'o $x=\pi$.

 \begin{tcolorbox}[breakable, size=fbox, boxrule=1pt, pad at break*=1mm,colback=cellbackground, colframe=cellborder]
\prompt{In}{incolor}{6}{\hspace{4pt}}
\begin{Verbatim}[commandchars=\\\{\}]
\PY{n}{f}\PY{o}{=}\PY{n}{sin}\PY{p}{(}\PY{n}{n}\PY{o}{*}\PY{n}{x}\PY{p}{)}
\PY{n}{grafica}\PY{p}{(}\PY{n}{f}\PY{p}{,}\PY{l+m+mi}{0}\PY{p}{,}\PY{n}{pi}\PY{p}{,}\PY{l+m+mi}{10}\PY{p}{)}
\end{Verbatim}
\end{tcolorbox}

    \begin{center}
    \adjustimage{max size={0.9\linewidth}{0.9\paperheight}}{python/uni3/output_11_0.png}
    \end{center}
    { \hspace*{\fill} \\}
\end{ejemplo}


\begin{ejemplo}{}
En el Ejemplo \ref{ej:sucesion-conv-puntual} 0.1 OJO CON LA REFERENCIA!!!!
vimos que  la sucesión 
$f_n(x)=\frac{1}{1+nx^2}$ converge puntualmente 
\[f(x)=\left\{\begin{array}{ll}
0&x\neq 0
\\
1&x=0
\end{array}
\right.\]

Si calculamos 
\[
\lim\limits_{n\to \infty}\lim\limits_{x \to 0}f_n(x)=\lim\limits_{n \to \infty}1=1
\]
y a continuación permutamos los límites obtenemos
\[
\lim\limits_{x\to 0}\lim\limits_{n \to \infty}f (x)=\lim\limits_{x \to 0}=0
\]
Por lo tanto, la permutación de los límites produce resultados \textbf{distintos}.

También vemos que la función límite es discontinua a pesar de que cada $f_n(x)$ es continua para cada $n$.
\end{ejemplo}

\begin{ejemplo}{}
Con las funciones del Ejemplo \ref{ej:montaña-movil-cte} 3 tenemos
\[
\int_{-\infty}^{+\infty} \lim\limits_{n \to  \infty} f_n(x)\,dx=\int_{-\infty}^{\infty} 0\,dx=0
\]
mientras que 
\[
\lim\limits_{n \to \infty} \int_{-\infty}^{\infty} f_n(x)\,dx=\lim\limits_{n \to \infty} \int_{n}^{n+1}dx=1
\]
En este caso, la permutación entre la operación de integración y la de límite también produce resultados \textbf{distintos}.
\end{ejemplo}

\begin{ejemplo}{}
Cada $f_n(x)=\sqrt{x^2+\frac{1}{n^2}}$ del Ejemplo \ref{ej:sucesion-conv-2 puntos} 4 ó 5 OJO CON LA REFERENCIA!!!!
es derivable y las derivadas son
$f_n^{'}(x)=\frac{x}{\sqrt{x^2+\frac{1}{n^2}}}$. 
Si computamos
\[
\lim\limits_{n \to \infty} f_n^{'}(x)=\frac{x}{|x|}=
\left\{
\begin{array}{ll}
1&x>0
\\
-1&x<0
\end{array}
\right.\]
cuando $x\neq 0$.
Entonces la funci\'on límite $f(x)=\frac{x}{|x|}$ no es derivable en 0. Luego  
\[
0=\lim\limits_{n \to \infty}f_n^{'}(0)\neq \frac{d}{dx}\left(\lim\limits_{n \to \infty}f_n(x)\right)\left. \right|_{x=0}
\]
pues ni siquiera tiene sentido el miembro de la derecha.
\end{ejemplo}

Es así que tenemos  
\begin{mdframed}[style=MiEstilo]\relax%
Encontrar condiciones que permitan permutar las operaciones anteriores.
\end{mdframed}



Antes de atacar este problema vamos a presentar varios ejemplo \textit{famosos} de sucesiones.


OJO!!!! LO QUE SIGUE EN EL APUNTE NO TIENE EJEMPLOS FAMOSOS, VIENEN LAS SERIES!!!!!

\section{Series de funciones}
Dada una sucesión de funciones $f_n(x)$, $n=1,2,\ldots$ podemos formar otra sucesión tomando las sumas acumuladas o
sumas parciales
\[
\begin{split}
s_1(x)&=f_1(x),
\\
s_2(x)&=f_1(x)+f_2(x),
\\
&\vdots
\\
s_n(x)&=f_1(x)+f_2(x)+\ldots+f_n(x).
\end{split}
\]

Si la nueva sucesión $\{s_n(x)\}$ converge a $f$ se dice que la serie $\sum\limits_{n=1}^{\infty} f_n(x)$
converge a $f$ ó que 
\[
\sum\limits_{n=1}^{\infty} f_n(x)=f(x).
\]

En pocas ocasiones se puede determinar que una serie converge hallando una expresión simple para $s_n(x)$ y
calculando su límite.

\begin{ejemplo}{}
Si $f_n(x)=c$ con $c \in \mathbb{R}$ un número independiente de $n$, entonces
\[
\begin{split}
s_1(x)&=f_1(x)=c,
\\
s_2(x)&=f_1(x)+f_2(x)=2c,
\\
&\vdots
\\
s_n(x)&=f_1(x)+f_2(x)+\ldots+f_n(x)=nc.
\end{split}
\]
Entonces
\[
\lim\limits_{n \to \infty} s_n(x)=
\left\{
\begin{array}{lll}
0&si&c=0,
\\
\infty&si&c\neq 0.
\end{array}
\right.
\]
y por lo tanto la series converge sólo cuando $c=0$.
\end{ejemplo}

\begin{ejemplo}{}
Si $f_n(z)=z^n$ para $n=0,1,2,\ldots$, luego
 
$s_n(z)=f_0(z)+\ldots+f_n(z)=1+z+\ldots+z^n$ y $zs_n(z)=z+z^2+\ldots+z^n+z^{n+1}$ 

entonces
$zs_n(z)-s_n(z)=z^{n+1}-1$ y por tanto $s_n(z)=\frac{z^{n+1}-1}{z-1}$.

De este modo, logramos expresar $s_n(z)$ en una fórmula relativamente sencilla. Ahora, 
\[
\lim\limits_{n \to \infty} s_n(z)=
\lim\limits_{n \to \infty} \frac{z^{n+1}-1}{z-1}=
\left\{\begin{array}{ll}
\frac{1}{1-z}&|z|<1,
\\
\mbox{no converge}&|z|\geq 1.
\end{array}
\right.
\]
\end{ejemplo}

Es interesante ver qué ocurre en $|z|=1$. 

Si $|z|=1$ entonces $z=e^{i\theta}=\cos \theta +i \sen \theta$ y 
\[
\begin{split}
s_n(z)&=
\frac{z^{n+1}-1}{z-1}=
\frac{z^{n+1}}{z-1}-\frac{1}{z-1}
\\
&=\frac{z^{n+1}(\overline z-1)}{|z-1|^2}-\frac{1}{z-1}
\\
&=\frac{ [\cos(n+1)\theta+i\sen(n+1)\theta](\overline z-1)}{|z-1|^2}-\frac{1}{z-1}
\\
&=\cos(n+1)\theta \frac{\overline z-1}{|z-1|^2}+i \sen(n+1)\theta \frac{(\overline z-1)}{|z-1|^2}-\frac{1}{z-1}
\end{split}
\]
son funciones oscilantes como en el Ejemplo \ref{ej:sucesion-conv-2 puntos} 5 y 1/2. VER GRÁFICOS!!!

En los ejemplos anteriores pudimos justificar la convergencia calculando explícitamente el límite. 
Ésto es posible las menos de las veces. En materias anteriores se estudiaron criterios para la 
convergencia de series  numéricas. Estos criterios establecen condiciones, algunas necesarias, otras 
suficientes y algunas necesarias y suficientes para que la serie converja. Recordaremos algunos de ellos.

\begin{teorema}[Criterio del Resto]{}
Si  $\sum\limits_{n=1}^{\infty} a_n$ converge entonces $\lim\limits_{n \to \infty} a_n=0$.
\end{teorema}

Como es una condición  necesaria sólo sirve para decir cuándo una serie no converge.

\begin{ejemplo}{}
Si $f_n(x)= \sen(nx)$ para $x \in [0,\pi]$. 

Como ya vimos, $\sen(nx)$ no converge excepto para $x=0$ \'o $x=\pi$. 

Luego, $\sum\limits_{n=1}^{\infty} \sen(nx)$ no converge.
\end{ejemplo}

Como veremos más adelante, la serie $\sum\limits_{n=1}^{\infty}\frac{1}{n}$ no converge y sin embargo
$\lim\limits_{n \to \infty}\frac{1}{n}=0$. 

Entonces el \textit{Criterio del Resto} no sirve para determinar
la convergencia de una serie.

\begin{teorema}[Convergencia Absoluta]{}
Si$\sum\limits_{n=1}^{\infty} |a_n|$ converge entonces $\sum\limits_{n=1}^{\infty} a_n$ converge.
\end{teorema}

\begin{ejemplo}{}
$\sum \limits_{n=1}^{\infty} (-1)^n z^n$ converge cuando $|z|<1$.
\end{ejemplo}

\begin{teorema}
[Criterio de Comparación]Si $0\leq a_n\leq b_n$ y $\sum\limits_{n=1}^{\infty} b_n$ converge entonces $\sum\limits_{n=1}^{\infty}a_n$ converge.
Dicho de otro modo, si $\sum\limits_{n=1}^{\infty} a_n$ diverge (su suma es $+\infty$) entonces 
$\sum\limits_{n=1}^{\infty}$ diverge.
\end{teorema}

\begin{ejemplo}{}
\begin{enumerate}
\item $\sum\limits_{n=1}^{\infty} \frac{1}{2^n} \sen(nx)$ converge pues $|\frac{1}{2^n} \sen(nx)|\leq \frac{1}{2^n}$.
\item $\sum\limits_{n=1}^{\infty} \frac{1}{n^2} \sin(nx)$ converge pues para $n>1$ tenemos
\[
\left|\frac{1}{n^2}\sin(nx)\right|\leq \frac{1}{n^2}\leq \frac{1}{(n-1)n}=\frac{1}{n-1}-\frac{1}{n}
\]
y
\[
\sum\limits_{n=2}^{m}\frac{1}{n-1}-\frac{1}{n}=1-\frac{1}{m}\to 1\mbox{  cuando  } m\to \infty.
\]
Luego $\sum\limits_{n=1}^{\infty} \frac{1}{n^2}$ converge y 
$\sum\limits_{n=1}^{\infty} \frac{1}{n^2} \sen(nx)$ también.
\end{enumerate}
\end{ejemplo}

\begin{teorema}[Criterio del Cociente]{}
Si $0<a_n$, $\lim\limits_{n \to \infty} \frac{a_{n+1}}{a_n}$ existe y es igual a $r$ entonces: 
\begin{enumerate}
\item si  $r<1$ entonces $\sum\limits_{n=1}^{\infty} a_n$ converge;
\item si $r>1$ entonces $\sum\limits_{n=1}^{\infty} a_n$ diverge;
\item\label{it:r=1} si $r=1$ no se sabe nada sobre la convergencia o divergencia de $\sum\limits_{n=1}^{\infty} a_n$.
\end{enumerate}
\end{teorema}

La situación planteada por el item \ref{it:r=1} ocurre en una cantidad exasperante de casos.

\begin{ejemplo}{}
Consideramos la serie $\sum\limits_{n=1}^{\infty} z^n$ con $z \in \mathbb{C}$ y calculamos
\[
\lim\limits_{n \to \infty} \frac{|n||z|^{n+1}}{|n+1||z|^n}=|z|\lim\limits_{n \to \infty} \frac{n}{n+1}=|z|.
\]
La serie converge cuando $|z|<1$, diverge cuando $|z|>1$ y nada se sabe cuando $|z|=1$.
\end{ejemplo}

\begin{teorema}[Criterio del Cociente]{}
Si $a_n\geq a_{n+1}\geq 0$ para todo $n$ y $\lim\limits_{n \to \infty} a_n=0$, entonces
$\sum\limits_{n=1}^{\infty}(-1)^n a_n$ converge.
\end{teorema}

\begin{ejemplo}{}
La series $f(z)=\sum\limits_{n=1}^{\infty}\frac{z^n}{n}$ converge en $z=-1$ pues 
$f(-1)=\sum\limits_{n=1}^{\infty} \frac{(-1)^n}{n}$, $\frac{1}{n}>\frac{1}{n+1}$ para todo $n\geq 1$ y $\lim\limits_{n \to \infty} \frac{1}{n}=0$.
\end{ejemplo}


\section{Series de Potencias}

Las series de la forma 
\[
\sum\limits_{n=0}^{\infty} a_n (z-z_0)^n,\quad a_n\in \mathbb{C},\quad z\in \mathbb{C},\quad z_0 \in \mathbb{C},
\]
se llaman series de potencias.

Veremos para qué valores de $z$ esta serie converge.
Por simplicidad asumiremos que $z_0=0$.

\begin{lema}{}
Si $\sum\limits_{n=0}^{\infty} a_n z^n$ converge en $z_1\in \mathbb{C}$ entonces 
la serie converge uniformemente para todo $z$ tal que $|z|<|z_1|$.
\end{lema}

\begin{proof}
Como $\sum\limits_{n=0}^{\infty} a_nz_1^n$ converge entonces $\lim\limits_{n\to \infty} a_nz_1^n=0$.
En particular, existe $M>0$ tal que $|a_nz_1^n|\geq M$.
Sea $|z|<|z_1|$ luego
\[|a_n z^n|<|a_n z_1^n|\left|\frac{z}{z_1}\right|^n\leq M \left|\frac{z}{z_1}\right|^n. \]
Como $|\frac{z}{z_1}|<1$ entonces $\sum\limits_{n=0}^{\infty} (\frac{z}{z_1})^n$ converge y por el 
Criterio de Comparación obtenemos que $\sum\limits_{n=0}^{\infty} |a_nz^n|$ converge siempre que 
$|z|<|z_1|$.
\end{proof}

\begin{corolario}{}
Dada una serie de funciones $\sum\limits_{n=0}^{\infty} a_n (z-z_0)^n$ existe $R\geq 0$ tal que la serie
converge en $|z-z_0|<R$ y no converge en $|z-z_0|>R$. 
\end{corolario}

El criterio del cociente suele ser útil para determinar el valor de $R$ que se denomina \emph{radio de convergencia}.

\begin{ejemplo}{}
La serie $\sum\limits_{n=1}^{\infty} \frac{z^n}{n}$ converge si 
\[
1>\lim\limits_{n \to \infty} \frac{\frac{|z|^{n+1}}{n+1}}{\frac{|z|^n}{n}}=
|z|\lim\limits_{n \to \infty} \frac{n }{n+1}=|z|,
\]
y no converge si $|z|>1$. Luego, el radio de convergencia es $1$.

?`Qué pasa en el borde $|z|=1$?

Si  $|z|=1$ $\Leftrightarrow$ $z=\cos \theta +i\sen \theta$ y $z^n=\cos(n \theta)+i \sen(n \theta)$. 
Luego
\[
\sum\limits_{n=1}^{\infty} \frac{z^n}{n}=
\sum\limits_{n=1}^{\infty} \frac{\cos(n \theta)}{n}+i \sum \limits_{n=1}^{\infty} \frac{\sen (n \theta)}{n}.
\]
Esto nos lleva a considerar otras series.
\end{ejemplo}


\section{Series de Fourier}
Una serie de Fourier es una expresión de la forma
\[
f(x)=\frac{a_0}{2}+\sum\limits_{n=1}^{\infty} a_n \cos(n x)+b_n \sen(nx).
\]

Las series de Fourier tiene la capacidad de aproximar funciones $2\pi$-periódicas. 
En primer lugar,   es necesario saber elegir los coeficientes y para ello se usa la propiedad que se presenta a continuación.

\begin{lema}{}
\[
\begin{split}
&\int_{-\pi}^{\pi} \cos(nx) \cos(mx)\,dx=
	\left\{
	\begin{array}{ll}
0&n\neq m
\\
\pi&n=m\neq0
\\
2\pi&n=m=0,
	\end{array}
	\right.
\\
&
\int_{-\pi}^{\pi} \sen(nx) \sen(mx)\,dx=
\left\{
\begin{array}{ll}
0&n\neq m
\\
\pi&n=m\neq0
\\
2\pi&n=m=0,
\end{array}
\right.
\\
&\int_{-\pi}^{\pi} \cos(nx) \sen(mx)\,dx=0.
\end{split}
\]
\end{lema}


%\[\sum\limits_{k=0}^n {n \choose k} x^k (1-x)^{n-k}\]


Luego, si nos permitimos permutar integrales son sumas de series, tenemos
\[
\begin{split}
&\int_{-\pi}^{\pi} f(x)\sen(kx)\,dx
\\
&=\frac{a_0}{2}\int_{-\pi}^{\pi} \sen(kx)\,dx
\\
&+ 
\sum\limits_{n=1}^{\infty} 
a_n \int_{-\pi}^{\pi} \cos(nx)\sen(kx)\,dx
\\
&+
b_n \int_{-\pi}^{\pi}  \sen(nx) \sen(kx)\,dx
\\
&=\pi b_k 
\end{split}\]

\begin{definicion}{}
Dada una función $2\pi$-periódica $f(x)$ definimos los coeficientes de Fourier por
\begin{equation}
a_n=\frac{1}{\pi} \int_{-\pi}{\pi} f(x)\cos(nx)\,dx,\quad n\geq 0
\end{equation}
y 
\begin{equation}
b_n=\frac{1}{\pi} \int_{-\pi}{\pi} f(x)\sen(nx)\,dx,\quad n\geq 1.
\end{equation}
\end{definicion}

En la notebook de Sympy indagamos las facultades aproximativas de la serie de Fourier.
    \begin{tcolorbox}[breakable, size=fbox, boxrule=1pt, pad at break*=1mm,colback=cellbackground, colframe=cellborder]
\prompt{In}{incolor}{10}{\hspace{4pt}}
\begin{Verbatim}[commandchars=\\\{\}]
\PY{n}{x}\PY{o}{=}\PY{n}{symbols}\PY{p}{(}\PY{l+s+s1}{\PYZsq{}}\PY{l+s+s1}{x}\PY{l+s+s1}{\PYZsq{}}\PY{p}{,}\PY{n}{real}\PY{o}{=}\PY{n+nb+bp}{True}\PY{p}{)}
\PY{n}{n}\PY{p}{,}\PY{n}{m}\PY{o}{=}\PY{n}{symbols}\PY{p}{(}\PY{l+s+s1}{\PYZsq{}}\PY{l+s+s1}{n,m}\PY{l+s+s1}{\PYZsq{}}\PY{p}{,}\PY{n}{integer}\PY{o}{=}\PY{n+nb+bp}{True}\PY{p}{,}\PY{n}{positive}\PY{o}{=}\PY{n+nb+bp}{True}\PY{p}{)}
\PY{n}{Integral}\PY{p}{(}\PY{n}{cos}\PY{p}{(}\PY{n}{n}\PY{o}{*}\PY{n}{x}\PY{p}{)}\PY{o}{*}\PY{n}{cos}\PY{p}{(}\PY{n}{m}\PY{o}{*}\PY{n}{x}\PY{p}{)}\PY{p}{,}\PY{p}{(}\PY{n}{x}\PY{p}{,}\PY{o}{\PYZhy{}}\PY{n}{pi}\PY{p}{,}\PY{n}{pi}\PY{p}{)}\PY{p}{)}\PY{o}{.}\PY{n}{doit}\PY{p}{(}\PY{p}{)}
\end{Verbatim}
\end{tcolorbox}
 
            
\prompt{Out}{outcolor}{10}{}
    
    $$\begin{cases} 0 & \text{for}\: m \neq n \\\pi & \text{otherwise} \end{cases}$$

    

    \begin{tcolorbox}[breakable, size=fbox, boxrule=1pt, pad at break*=1mm,colback=cellbackground, colframe=cellborder]
\prompt{In}{incolor}{11}{\hspace{4pt}}
\begin{Verbatim}[commandchars=\\\{\}]
\PY{n}{Integral}\PY{p}{(}\PY{n}{sin}\PY{p}{(}\PY{n}{n}\PY{o}{*}\PY{n}{x}\PY{p}{)}\PY{o}{*}\PY{n}{sin}\PY{p}{(}\PY{n}{m}\PY{o}{*}\PY{n}{x}\PY{p}{)}\PY{p}{,}\PY{p}{(}\PY{n}{x}\PY{p}{,}\PY{o}{\PYZhy{}}\PY{n}{pi}\PY{p}{,}\PY{n}{pi}\PY{p}{)}\PY{p}{)}\PY{o}{.}\PY{n}{doit}\PY{p}{(}\PY{p}{)}
\end{Verbatim}
\end{tcolorbox}
 
            
\prompt{Out}{outcolor}{11}{}
    
    $$\begin{cases} 0 & \text{for}\: m \neq n \\\pi & \text{otherwise} \end{cases}$$

    

    \begin{tcolorbox}[breakable, size=fbox, boxrule=1pt, pad at break*=1mm,colback=cellbackground, colframe=cellborder]
\prompt{In}{incolor}{12}{\hspace{4pt}}
\begin{Verbatim}[commandchars=\\\{\}]
\PY{n}{Integral}\PY{p}{(}\PY{n}{cos}\PY{p}{(}\PY{n}{n}\PY{o}{*}\PY{n}{x}\PY{p}{)}\PY{o}{*}\PY{n}{sin}\PY{p}{(}\PY{n}{m}\PY{o}{*}\PY{n}{x}\PY{p}{)}\PY{p}{,}\PY{p}{(}\PY{n}{x}\PY{p}{,}\PY{o}{\PYZhy{}}\PY{n}{pi}\PY{p}{,}\PY{n}{pi}\PY{p}{)}\PY{p}{)}\PY{o}{.}\PY{n}{doit}\PY{p}{(}\PY{p}{)}
\end{Verbatim}
\end{tcolorbox}
 
            
\prompt{Out}{outcolor}{12}{}
    
    $$0$$

    

    \begin{tcolorbox}[breakable, size=fbox, boxrule=1pt, pad at break*=1mm,colback=cellbackground, colframe=cellborder]
\prompt{In}{incolor}{13}{\hspace{4pt}}
\begin{Verbatim}[commandchars=\\\{\}]
\PY{n}{S}\PY{o}{=}\PY{n+nb}{sum}\PY{p}{(}\PY{p}{[}\PY{n}{sin}\PY{p}{(}\PY{n}{n}\PY{o}{*}\PY{n}{theta}\PY{p}{)}\PY{o}{/}\PY{n}{n} \PY{k}{for} \PY{n}{n} \PY{o+ow}{in} \PY{n+nb}{range}\PY{p}{(}\PY{l+m+mi}{1}\PY{p}{,}\PY{l+m+mi}{50}\PY{p}{)}\PY{p}{]}\PY{p}{)}
\PY{n}{plot}\PY{p}{(}\PY{n}{S}\PY{p}{,}\PY{p}{(}\PY{n}{theta}\PY{p}{,}\PY{o}{\PYZhy{}}\PY{l+m+mi}{4}\PY{o}{*}\PY{n}{pi}\PY{p}{,}\PY{l+m+mi}{4}\PY{o}{*}\PY{n}{pi}\PY{p}{)}\PY{p}{)}
\end{Verbatim}
\end{tcolorbox}

    \begin{center}
    \adjustimage{max size={0.9\linewidth}{0.9\paperheight}}{python/uni3/output_20_0.png}
    \end{center}
    { \hspace*{\fill} \\}
    
            \begin{tcolorbox}[breakable, boxrule=.5pt, size=fbox, pad at break*=1mm, opacityfill=0]
\prompt{Out}{outcolor}{13}{\hspace{3.5pt}}
\begin{Verbatim}[commandchars=\\\{\}]
<sympy.plotting.plot.Plot at 0x7f413f16afd0>
\end{Verbatim}
\end{tcolorbox}
        
    Ejemplo
\[f(x)=\left\{\begin{array}{cc} \frac{\pi-x}{2}, & \text{ si } x\in [0,\pi]\\
    -\frac{\pi+x}{2}, & \text{ si } x\in [-\pi,0) \end{array}    \right.\]

    \begin{tcolorbox}[breakable, size=fbox, boxrule=1pt, pad at break*=1mm,colback=cellbackground, colframe=cellborder]
\prompt{In}{incolor}{14}{\hspace{4pt}}
\begin{Verbatim}[commandchars=\\\{\}]
\PY{n}{f}\PY{o}{=}\PY{n}{Piecewise}\PY{p}{(}\PY{p}{(}\PY{p}{(}\PY{n}{pi}\PY{o}{\PYZhy{}}\PY{n}{x}\PY{p}{)}\PY{o}{/}\PY{l+m+mi}{2}\PY{p}{,} \PY{n}{x}\PY{o}{\PYZgt{}}\PY{o}{=}\PY{l+m+mi}{0}  \PY{p}{)}\PY{p}{,}\PY{p}{(}\PY{o}{\PYZhy{}}\PY{p}{(}\PY{n}{pi}\PY{o}{+}\PY{n}{x}\PY{p}{)}\PY{o}{/}\PY{l+m+mi}{2}\PY{p}{,}\PY{n}{x}\PY{o}{\PYZlt{}}\PY{l+m+mi}{0} \PY{p}{)}\PY{p}{)}
\PY{n}{plot}\PY{p}{(}\PY{n}{f}\PY{p}{,}\PY{p}{(}\PY{n}{x}\PY{p}{,}\PY{o}{\PYZhy{}}\PY{l+m+mi}{4}\PY{p}{,}\PY{l+m+mi}{4}\PY{p}{)}\PY{p}{)}
\end{Verbatim}
\end{tcolorbox}

    \begin{center}
    \adjustimage{max size={0.9\linewidth}{0.9\paperheight}}{python/uni3/output_22_0.png}
    \end{center}
    { \hspace*{\fill} \\}
    
            \begin{tcolorbox}[breakable, boxrule=.5pt, size=fbox, pad at break*=1mm, opacityfill=0]
\prompt{Out}{outcolor}{14}{\hspace{3.5pt}}
\begin{Verbatim}[commandchars=\\\{\}]
<sympy.plotting.plot.Plot at 0x7f413f019b50>
\end{Verbatim}
\end{tcolorbox}
        
    \begin{tcolorbox}[breakable, size=fbox, boxrule=1pt, pad at break*=1mm,colback=cellbackground, colframe=cellborder]
\prompt{In}{incolor}{15}{\hspace{4pt}}
\begin{Verbatim}[commandchars=\\\{\}]
\PY{k}{def} \PY{n+nf}{a}\PY{p}{(}\PY{n}{g}\PY{p}{,}\PY{n}{k}\PY{p}{)}\PY{p}{:}
    \PY{k}{return} \PY{l+m+mi}{1}\PY{o}{/}\PY{n}{pi}\PY{o}{*}\PY{n}{Integral}\PY{p}{(}\PY{n}{g}\PY{o}{*}\PY{n}{cos}\PY{p}{(}\PY{n}{k}\PY{o}{*}\PY{n}{x}\PY{p}{)}\PY{p}{,}\PY{p}{(}\PY{n}{x}\PY{p}{,}\PY{o}{\PYZhy{}}\PY{n}{pi}\PY{p}{,}\PY{n}{pi}\PY{p}{)}\PY{p}{)}\PY{o}{.}\PY{n}{doit}\PY{p}{(}\PY{p}{)}
\PY{k}{def} \PY{n+nf}{b}\PY{p}{(}\PY{n}{g}\PY{p}{,}\PY{n}{k}\PY{p}{)}\PY{p}{:}
    \PY{k}{return} \PY{l+m+mi}{1}\PY{o}{/}\PY{n}{pi}\PY{o}{*}\PY{n}{Integral}\PY{p}{(}\PY{n}{g}\PY{o}{*}\PY{n}{sin}\PY{p}{(}\PY{n}{k}\PY{o}{*}\PY{n}{x}\PY{p}{)}\PY{p}{,}\PY{p}{(}\PY{n}{x}\PY{p}{,}\PY{o}{\PYZhy{}}\PY{n}{pi}\PY{p}{,}\PY{n}{pi}\PY{p}{)}\PY{p}{)}\PY{o}{.}\PY{n}{doit}\PY{p}{(}\PY{p}{)}
\end{Verbatim}
\end{tcolorbox}

    \begin{tcolorbox}[breakable, size=fbox, boxrule=1pt, pad at break*=1mm,colback=cellbackground, colframe=cellborder]
\prompt{In}{incolor}{16}{\hspace{4pt}}
\begin{Verbatim}[commandchars=\\\{\}]
\PY{p}{[}\PY{n}{a}\PY{p}{(}\PY{n}{f}\PY{p}{,}\PY{n}{k}\PY{p}{)} \PY{k}{for} \PY{n}{k} \PY{o+ow}{in} \PY{n+nb}{range}\PY{p}{(}\PY{l+m+mi}{1}\PY{p}{,}\PY{l+m+mi}{10}\PY{p}{)}\PY{p}{]}
\end{Verbatim}
\end{tcolorbox}
 
            
\prompt{Out}{outcolor}{16}{}
    
    $$\left [ 0, \quad 0, \quad 0, \quad 0, \quad 0, \quad 0, \quad 0, \quad 0, \quad 0\right ]$$

    

    \begin{tcolorbox}[breakable, size=fbox, boxrule=1pt, pad at break*=1mm,colback=cellbackground, colframe=cellborder]
\prompt{In}{incolor}{17}{\hspace{4pt}}
\begin{Verbatim}[commandchars=\\\{\}]
\PY{p}{[}\PY{n}{b}\PY{p}{(}\PY{n}{f}\PY{p}{,}\PY{n}{k}\PY{p}{)} \PY{k}{for} \PY{n}{k} \PY{o+ow}{in} \PY{n+nb}{range}\PY{p}{(}\PY{l+m+mi}{1}\PY{p}{,}\PY{l+m+mi}{10}\PY{p}{)}\PY{p}{]}
\end{Verbatim}
\end{tcolorbox}
 
            
\prompt{Out}{outcolor}{17}{}
    
    $$\left [ 1, \quad \frac{1}{2}, \quad \frac{1}{3}, \quad \frac{1}{4}, \quad \frac{1}{5}, \quad \frac{1}{6}, \quad \frac{1}{7}, \quad \frac{1}{8}, \quad \frac{1}{9}\right ]$$

    

    \begin{tcolorbox}[breakable, size=fbox, boxrule=1pt, pad at break*=1mm,colback=cellbackground, colframe=cellborder]
\prompt{In}{incolor}{18}{\hspace{4pt}}
\begin{Verbatim}[commandchars=\\\{\}]
\PY{n}{S}\PY{o}{=}\PY{n+nb}{sum}\PY{p}{(}\PY{p}{[}\PY{n}{b}\PY{p}{(}\PY{n}{f}\PY{p}{,}\PY{n}{k}\PY{p}{)}\PY{o}{*}\PY{n}{sin}\PY{p}{(}\PY{n}{k}\PY{o}{*}\PY{n}{x}\PY{p}{)} \PY{k}{for} \PY{n}{k} \PY{o+ow}{in} \PY{n+nb}{range}\PY{p}{(}\PY{l+m+mi}{1}\PY{p}{,}\PY{l+m+mi}{20}\PY{p}{)}\PY{p}{]}\PY{p}{)}
\PY{n}{S}
\end{Verbatim}
\end{tcolorbox}
 
            
\prompt{Out}{outcolor}{18}{}
    
    %$$
		\[\begin{split}
		&\sin{\left (x \right )} + \frac{\sin{\left (2 x \right )}}{2} + \frac{\sin{\left (3 x \right )}}{3} + \frac{\sin{\left (4 x \right )}}{4}
		\\&+\frac{\sin{\left (5 x \right )}}{5} + \frac{\sin{\left (6 x \right )}}{6} + \frac{\sin{\left (7 x \right )}}{7} +\frac{\sin{\left (8 x \right )}}{8} \\
		&+ \frac{\sin{\left (9 x \right )}}{9} + \frac{\sin{\left (10 x \right )}}{10} + \frac{\sin{\left (11 x \right )}}{11} 
		+ \frac{\sin{\left (12 x \right )}}{12} \\
		&+ \frac{\sin{\left (13 x \right )}}{13} + \frac{\sin{\left (14 x \right )}}{14} + \frac{\sin{\left (15 x \right )}}{15} + \frac{\sin{\left (16 x \right )}}{16} \\
		&+ \frac{\sin{\left (17 x \right )}}{17} + \frac{\sin{\left (18 x \right )}}{18} + \frac{\sin{\left (19 x \right )}}{19}
		\end{split}\]
		%$$

    

    \begin{tcolorbox}[breakable, size=fbox, boxrule=1pt, pad at break*=1mm,colback=cellbackground, colframe=cellborder]
\prompt{In}{incolor}{19}{\hspace{4pt}}
\begin{Verbatim}[commandchars=\\\{\}]
\PY{n}{S}\PY{o}{=}\PY{n+nb}{sum}\PY{p}{(}\PY{p}{[}\PY{n}{b}\PY{p}{(}\PY{n}{f}\PY{p}{,}\PY{n}{k}\PY{p}{)}\PY{o}{*}\PY{n}{sin}\PY{p}{(}\PY{n}{k}\PY{o}{*}\PY{n}{x}\PY{p}{)} \PY{k}{for} \PY{n}{k} \PY{o+ow}{in} \PY{n+nb}{range}\PY{p}{(}\PY{l+m+mi}{1}\PY{p}{,}\PY{l+m+mi}{20}\PY{p}{)}\PY{p}{]}\PY{p}{)}
\PY{n}{plot}\PY{p}{(}\PY{n}{f}\PY{p}{,}\PY{n}{S}\PY{p}{,} \PY{p}{(}\PY{n}{x}\PY{p}{,}\PY{o}{\PYZhy{}}\PY{n}{pi}\PY{p}{,}\PY{n}{pi}\PY{p}{)}\PY{p}{)}
\end{Verbatim}
\end{tcolorbox}

    \begin{center}
    \adjustimage{max size={0.9\linewidth}{0.9\paperheight}}{python/uni3/output_27_0.png}
    \end{center}
    { \hspace*{\fill} \\}
    
            \begin{tcolorbox}[breakable, boxrule=.5pt, size=fbox, pad at break*=1mm, opacityfill=0]
\prompt{Out}{outcolor}{19}{\hspace{3.5pt}}
\begin{Verbatim}[commandchars=\\\{\}]
<sympy.plotting.plot.Plot at 0x7f413f1ae210>
\end{Verbatim}
\end{tcolorbox}


Deberíamos justificar el uso de propiedades como 
\[
\int_a^b \sum\limits_{n=1}^{\infty} f_n(x)\,dx=\sum\limits_{n=1}^{\infty} \int_a^b f_n(x)\,dx,
\]
que se justificar\'ia  si 
\[
\begin{split}
&\int_a^b \sum\limits_{n=1}^{\infty} f_n(x)= \int_a^b \lim\limits_{N \to \infty} \sum\limits_{n=1}^{N}f_n(x)
\\
\underset{?}{=}
\lim\limits_{N \to \infty} &\int_a^b \sum\limits_{n=1}^{N} f_n(x)\,dx=
\lim\limits_{N \to \infty} \sum\limits_{n=1}^N \int_a^b f_n(x)\,dx
\end{split}
\]
Pero, ya hemos visto ejemplos de que éste no es siempre el caso. 
Vamos a identificar otro modo de convergencia que hace esta regla posible. 

Si $f_n(x)$ converge puntualmente a $f$ entonces
\[
\forall x \forall \epsilon>0 \exists N=N(\epsilon, x)>0:
n\geq N\Rightarrow|f_n(x)-f(x)|<\epsilon.
\]

\begin{definicion}{}
$f_n$ converge uniformemente a $f$ si y sólo si
\[
\forall \epsilon>0 \exists N=N(\epsilon)>0 \forall x:
n\geq N\Rightarrow|f_n(x)-f(x)|<\epsilon.
\]
\end{definicion}
IDEA GRÁFICA 

\begin{ejemplo}{}
\begin{enumerate}
\item
$f_n(x)=\frac{1}{n}xe^{-n^2x^2}$ converge uniformemente a cero. 

Se tiene que 
$f^{'}_n(x)=\frac{1}{n}e^{-n^2x^2}-n2x^2 e^{-n^2x^2}$ y 
\[f^{'}_n(x)=0 \Longleftrightarrow 0=e^{-n^2x^2}\left(\frac{1}{n}-2nx^2\right) \Leftrightarrow x=\pm\frac{1}{\sqrt{2}n}.\]
Luego $f^{'}_n(x)>0$ en $|x|<\frac{1}{\sqrt{2}n}$ y $f^{'}_n(x)<0$ en $|x|>\frac{1}{\sqrt{2}n}$.
Y, $f\left(\pm\frac{1}{\sqrt{2}n}\right)=\frac{1}{n}(\pm\frac{1}{\sqrt{2}n} )e^{-\frac{1}{2}}=
\pm \frac{1}{\sqrt{2}}e^{-\frac{1}{2}} \frac{1}{n^2}.$

Luego, dado $\epsilon>0$ existe $N\geq \left(\frac{e^{\frac{1}{2}}}{\sqrt{2}\epsilon}\right)^{\frac{1}{2}}$.

    \begin{tcolorbox}[breakable, size=fbox, boxrule=1pt, pad at break*=1mm,colback=cellbackground, colframe=cellborder]
\prompt{In}{incolor}{20}{\hspace{4pt}}
\begin{Verbatim}[commandchars=\\\{\}]
\PY{n}{x}\PY{p}{,}\PY{n}{n}\PY{o}{=}\PY{n}{symbols}\PY{p}{(}\PY{l+s+s1}{\PYZsq{}}\PY{l+s+s1}{x,n}\PY{l+s+s1}{\PYZsq{}}\PY{p}{)}
\PY{n}{fn}\PY{o}{=}\PY{l+m+mi}{1}\PY{o}{/}\PY{n}{n}\PY{o}{*}\PY{n}{exp}\PY{p}{(}\PY{o}{\PYZhy{}}\PY{n}{n}\PY{o}{*}\PY{o}{*}\PY{l+m+mi}{2}\PY{o}{*}\PY{n}{x}\PY{o}{*}\PY{o}{*}\PY{l+m+mi}{2}\PY{p}{)}\PY{o}{*}\PY{n}{x}
\PY{n}{p}\PY{o}{=}\PY{n}{plot}\PY{p}{(}\PY{n}{fn}\PY{o}{.}\PY{n}{subs}\PY{p}{(}\PY{n}{n}\PY{p}{,}\PY{l+m+mi}{1}\PY{p}{)}\PY{p}{,} \PY{p}{(}\PY{n}{x}\PY{p}{,}\PY{o}{\PYZhy{}}\PY{l+m+mi}{5}\PY{p}{,}\PY{l+m+mi}{5}\PY{p}{)}\PY{p}{,}\PY{n}{show}\PY{o}{=}\PY{n}{false}\PY{p}{)}
\PY{k}{for} \PY{n}{k} \PY{o+ow}{in} \PY{n+nb}{range}\PY{p}{(}\PY{l+m+mi}{2}\PY{p}{,}\PY{l+m+mi}{10}\PY{p}{)}\PY{p}{:}
    \PY{n}{p}\PY{o}{.}\PY{n}{append}\PY{p}{(}\PY{n}{plot}\PY{p}{(}\PY{n}{fn}\PY{o}{.}\PY{n}{subs}\PY{p}{(}\PY{n}{n}\PY{p}{,}\PY{n}{k}\PY{p}{)}\PY{p}{,} \PY{p}{(}\PY{n}{x}\PY{p}{,}\PY{o}{\PYZhy{}}\PY{l+m+mi}{5}\PY{p}{,}\PY{l+m+mi}{5}\PY{p}{)}\PY{p}{,}\PY{n}{show}\PY{o}{=}\PY{n}{false}\PY{p}{)}\PY{p}{[}\PY{l+m+mi}{0}\PY{p}{]}\PY{p}{)}
\PY{n}{p}\PY{o}{.}\PY{n}{show}\PY{p}{(}\PY{p}{)}
\end{Verbatim}
\end{tcolorbox}

    \begin{center}
    \adjustimage{max size={0.9\linewidth}{0.9\paperheight}}{python/uni3/output_29_0.png}
    \end{center}
    { \hspace*{\fill} \\}
\item $f_n(x)=nxe^{-n^2x^2}$ converge puntualmente pero no uniformemente a cero.

Si $x\neq 0$, $\lim\limits_{n\to \infty} \frac{nx}{e^{n^2x^2}}=\lim\limits_{n \to \infty} \frac{x}{2ne^{n^2x^2}}=0$.

Por otra parte, 
\[
0=f^{'}_n(x)=-n^2e^{-n^2 x^2}2x^2+e^{-n^2x^2}\Leftrightarrow 1-2n^2x^2=0 \Leftrightarrow x=\pm \frac{1}{\sqrt{2}n}.
\]
Y, $f\left(\pm\frac{1}{\sqrt{2}n}\right)=\pm e^{\frac{1}{2}}\frac{1}{\sqrt{2}}$.

    \begin{tcolorbox}[breakable, size=fbox, boxrule=1pt, pad at break*=1mm,colback=cellbackground, colframe=cellborder]
\prompt{In}{incolor}{21}{\hspace{4pt}}
\begin{Verbatim}[commandchars=\\\{\}]
\PY{n}{fn}\PY{o}{=}\PY{n}{n}\PY{o}{*}\PY{n}{exp}\PY{p}{(}\PY{o}{\PYZhy{}}\PY{n}{n}\PY{o}{*}\PY{o}{*}\PY{l+m+mi}{2}\PY{o}{*}\PY{n}{x}\PY{o}{*}\PY{o}{*}\PY{l+m+mi}{2}\PY{p}{)}\PY{o}{*}\PY{n}{x}
\PY{n}{p}\PY{o}{=}\PY{n}{plot}\PY{p}{(}\PY{n}{fn}\PY{o}{.}\PY{n}{subs}\PY{p}{(}\PY{n}{n}\PY{p}{,}\PY{l+m+mi}{1}\PY{p}{)}\PY{p}{,} \PY{p}{(}\PY{n}{x}\PY{p}{,}\PY{o}{\PYZhy{}}\PY{l+m+mi}{5}\PY{p}{,}\PY{l+m+mi}{5}\PY{p}{)}\PY{p}{,}\PY{n}{show}\PY{o}{=}\PY{n}{false}\PY{p}{)}
\PY{k}{for} \PY{n}{k} \PY{o+ow}{in} \PY{n+nb}{range}\PY{p}{(}\PY{l+m+mi}{2}\PY{p}{,}\PY{l+m+mi}{10}\PY{p}{)}\PY{p}{:}
    \PY{n}{p}\PY{o}{.}\PY{n}{append}\PY{p}{(}\PY{n}{plot}\PY{p}{(}\PY{n}{fn}\PY{o}{.}\PY{n}{subs}\PY{p}{(}\PY{n}{n}\PY{p}{,}\PY{n}{k}\PY{p}{)}\PY{p}{,} \PY{p}{(}\PY{n}{x}\PY{p}{,}\PY{o}{\PYZhy{}}\PY{l+m+mi}{5}\PY{p}{,}\PY{l+m+mi}{5}\PY{p}{)}\PY{p}{,}\PY{n}{show}\PY{o}{=}\PY{n}{false}\PY{p}{)}\PY{p}{[}\PY{l+m+mi}{0}\PY{p}{]}\PY{p}{)}
\PY{n}{p}\PY{o}{.}\PY{n}{show}\PY{p}{(}\PY{p}{)}
\end{Verbatim}
\end{tcolorbox}

    \begin{center}
    \adjustimage{max size={0.9\linewidth}{0.9\paperheight}}{python/uni3/output_31_0.png}
    \end{center}
    { \hspace*{\fill} \\}
\end{enumerate}
\end{ejemplo}


\begin{teorema}{}
Si $f_n: A\to B$ (aquí $A,B \subset \mathbb{R}^n$ ó $A,B \subset \mathbb{C}$) 
son continuas y convergen uniformemente a $f$, entonces $f$ es continua y 
\[
\lim\limits_{x \to a}\lim\limits_{n \to \infty} f_n(x)=
\lim\limits_{n \to \infty} f_n(a)=
\lim\limits_{n \to \infty} \lim \limits_{x \to a} f_n(x).
\]
\end{teorema}

\begin{proof}
Sea $f(x)=\lim\limits_{n \to \infty} f_n(x)$ y $a \in A$. 

Sea $\epsilon>0$, entonces $\forall x \exists N: n\geq N\Rightarrow|f_n(x)-f(x)|<\frac{\epsilon}{3} $. 

Fijamos un $n$ cualquiera que satisfaga la desigualdad anterior.

Como $f_n$ es continua en $a$ $\exists \delta=\delta(\epsilon,a)$ tal que 
$|x-a|<\delta \Rightarrow |f_n(x)-f_n(a)|<\frac{\epsilon}{3}$. 

Entonces 
\[
|f(x)-f(a)|\leq |f(x)-f_n(x)|+|f_n(x)-f_n(a)|+|f_n(a)-f(a)|<\epsilon.
\]
\end{proof}

\begin{ejemplo}{}
$x^n$ converge puntualmente en $[0,1]$ pero no uniformemente.
\end{ejemplo}

\begin{teorema}{}
Si $f_n$ converge uniformemente a $f$ en $[a,b]$ entonces
\[
\lim\limits_{n \to \infty} \int_a^b f_n(x)\,dx=\int_a^b f(x)\,dx.
\]
\end{teorema}

\begin{proof}
La prueba se deja como ejercicio.
\end{proof}

\begin{teorema}[M-test Weiertrass]{}
Si $f_n:A\subset \mathbb{R}^n \to \mathbb{R}$ y $|f_n(x)|\leq M_n$ independientemente de $x$
y $\sum\limits_{n=1}^{\infty} M_n<\infty$ entonces la serie $\sum\limits_{n=1}^{\infty}f_n(x)$
converge uniformemente en $A$.
\end{teorema}

\begin{proof}
La serie converge puntualmente por aplicación del \textit{Criterio de Comparación}.

Sea $f(x)=\sum\limits_{n=1}^{\infty} f_n(x)$ y sea $\epsilon>0$.
Entonces $\exists N>0 $ tal que $\sum\limits_{n=N+1}^{\infty} M_n<\epsilon$. 

Luego, 
\[
\begin{split}
&\left|f(x)-\sum\limits_{n=1}^{N} f_n(x)\right|=\left|\sum\limits_{n=N+1}^{\infty} f_n(x)\right|
\\
=&\left|\lim\limits_{M\to \infty} \sum\limits_{n=N+1}^{M} f_n(x)\right|
\leq \lim\limits_{M\to \infty} \sum\limits_{n=N+1}^{M} M_n
<\epsilon.
\end{split}\]
\end{proof}

\begin{corolario}{}
Si la serie de potencias $\sum\limits_{n=0}^{\infty} a_n (z-z_0)^n$ tiene radio de convergencia $R>0$, 
entonces converge uniformemente en $|z-z_0|\leq r$ $\forall r<R$.
\end{corolario}

\begin{proof}
Supongamos $z_0=0$.

Sea $0<r<R$ entonces la serie converge abasolutamente en $|z|=r$ y por tanto 
\[
\sum\limits_{n=0}^{\infty}|a_n|r^n
<\infty.\]
Luego si $|z|<r$ entonces $|a_nz^n|\leq |a_n|r^n$ y se verifican las hipótesis del M-Test de Weierstrass.
\end{proof}

\begin{ejemplo}{}
Analizar con Sympy la convergencia de $\sum\limits_{n=0}^{\infty} x^n$.
\end{ejemplo}

\begin{corolario}{}
Si $\sum\limits_{n=0}^{\infty}|a_n|+|b_n|<\infty$ entonces la serie de Fourier 
\[
\frac{a_0}{2}+\sum\limits_{n=1}^{\infty} a_n \cos(nx)+b_n \sen(nx),
\]
converge uniformemente a una función $f$ en $[-\pi,\pi]$ y 
\[
\begin{split}
a_n=\frac{1}{\pi}\int_{-\pi}^{\pi} f(x)\cos(nx)\,dx,\\
b_n=\frac{1}{\pi}\int_{-\pi}^{\pi} f(x)\sen(nx)\,dx.
\end{split}
\]
\end{corolario}

\textbf{Problema:} Hallar $\sum\limits_{n=1}^{\infty} \frac{1}{n^2}$.


\section{Productos infinitos}

Si un polinomio
\[
p(x)=a_nx^n+a_{n-1}x^{n-1}+\ldots+a_1x+1,
\]
tiene raíces reales $x_1,x_2,\ldots,x_n$ entonces
\[
x_1^{-1}+x_2^{-1}+\ldots+x_n^{-1}=-a_1.
\]

\begin{proof}
Tenemos que $p(x)=a_n(x-x_1)(x-x_2)\ldots(x-x_n)$ entonces $1=p(0)=(-1)^n x_1\ldots x_n$
y \[p'(0)=a_1=a_n (-1)^{n-1}(x_1x_3\ldots x_n+x_2x_3\ldots x_n+\ldots +x_1x_2\ldots x_{n-1}).\]

Luego
\[
\begin{split}
a_1&=\frac{p'(0)}{p(0)}
\\
&=\frac{a_n (-1)^{n-1}(x_1x_3\ldots x_n+x_2x_3\ldots x_n+\ldots +x_1x_2\ldots x_{n-1})}{(-1)^n x_1\ldots x_n}
\\&=
-\left(\frac{1}{x_1}+\frac{1}{x_2}+\ldots+\frac{1}{x_n}\right).
\end{split}
\]
Además 
\[
p(x)=(-1)^n x_1x_2\ldots x_n \left(1-\frac{x}{x_1}\right) \left(1-\frac{x}{x_2}\right)\ldots \left(1-\frac{x}{x_n}\right).
\]
\end{proof}

\begin{teorema}[Euler(1748)]{}
\[\sen x=
x \prod\limits_{k=1}^{\infty}
\left(1-\frac{x^2}{k^2 \pi^2}\right).
\]
\end{teorema}

\begin{proof}
\begin{equation}\label{eq:previa-euler-a-wallis}
\begin{split}
\frac{\sen x}{x}=&\frac{x-\frac{x^3}{3!}+\frac{x^5}{5!}+\ldots+(-1)^{n+1}\frac{x^{2n-1}}{(2n-1)!}+\ldots}{x}
\\
=&\left(1-\frac{x}{\pi}\right)\left(1+\frac{x}{\pi}\right)\left(1-\frac{x}{2\pi}\right)\left(1+\frac{x}{2\pi}\right)\ldots
\\
=& \left(1-\frac{x^2}{\pi^2}\right) \left(1-\frac{x^2}{4\pi^2}\right)\ldots \left(1-\frac{x^2}{k^2 \pi^2}\right)\ldots
\end{split}
\end{equation}
\end{proof}
Tomando $x^2=y$  en \eqref{eq:previa-euler-a-wallis} llegamos a 
\[1-\frac{y}{3!}+\frac{y^2}{5!}+\frac{y^4}{7!}+\ldots=\left(1-\frac{y}{\pi^2}\right)\ldots\left(1-\frac{y}{k^2\pi^2}\right).
\]
Luego
\[
\frac{1}{\pi^2}+\frac{1}{4\pi^2}+\ldots=\frac{1}{6}
\]
y 
\[
1+\frac{1}{4}+\ldots=\frac{\pi^2}{6}.
\]
Evaluando en $x=\frac{\pi}{2}$ obtenemos
\[
\begin{split}
\frac{2}{\pi}&=
\left(1-\frac{(\frac{\pi}{2})^2}{\pi^2}\right) \left(1-\frac{(\frac{\pi}{2})^2}{4\pi^2}\right)\ldots
\left(1-\frac{(\frac{\pi}{2})^2}{k^2\pi^2}\right)\ldots
\\
&=\prod\limits_{k=1}^{\infty} \left(1-\frac{1}{4k^2}\right)=\prod\limits_{k=1}^{\infty}\frac{4k^2-1}{4k^2}
\\
&=\prod\limits_{k=1}^{\infty} \frac{2k-1}{2k}\frac{2k+1}{2k}=\frac{1\cdot3}{2\cdot 2}\frac{3 \cdot 5}{4 \cdot 4}\ldots
\end{split}
\]
y por lo tanto obtenemos la fórmula de Wallis
\[
\frac{\pi}{2}=\frac{2\cdot 2}{1 \cdot 3}\frac{4 \cdot 4}{3\cdot 5}\ldots
\]

\section{Aproximación de funciones}

\begin{teorema}[Weierstrass]{}
Si $f$ es continua en $[0,1]$ entonces $f$ es límite uniforme de polinomios.
\end{teorema}

\begin{definicion}{}
Si $f$ es una función se define su polinomio de Berstein de grado n por
\[
B_n(f)=\sum\limits_{k=0}^{n} {n \choose k} f\left(\frac{k}{n}\right) x^k (1-x)^{n-k}. 
\]
\end{definicion}

Si $f \equiv 1$ entonces 
\[
B_n(f)=\sum\limits_{k=0}^{n} {n \choose k} x^k (1-x)^{n-k} =[x+(1-x)]^n=1.
\]


Si $f\equiv x$ entonces 
\[\begin{split}
B_n(f)&=\sum\limits_{k=0}^{n} {n \choose k} \frac{k}{n} x^k (1-x)^{n-k} 
\\&=
x \sum\limits_{k=1}^{n} \frac{(n-1)!}{(k-1)![n-1-(k-1)]!}  x^{k-1} (1-x)^{[n-1-(k-1)]}
\\&=
x B_{n-1}(1)=x.
\end{split}
\]


Si $f\equiv x^2$ entonces 
\[
\begin{split}
B_n(f)
&=\sum\limits_{k=0}^{n} {n \choose k} \frac{k^2}{n^2} x^k (1-x)^{n-k} 
\\
&=
x \sum\limits_{k=1}^{n}  \frac{(n-1)!} {(k-1)![n-1-(k-1)]!} \frac{k}{n}  x^{k-1} (1-x)^{n-k}
\\
&=
x \sum\limits_{k=1}^{n}  \frac{k-1}{n} {{n-1} \choose{k-1}}  x^{k-1} (1-x)^{n-k}
+
\frac{x}{n} \sum\limits_{k=1}^{n} {{n-1 }\choose{k-1}}   x^{k-1} (1-x)^{n-k}
\\
&=
x \frac{n-1}{n} B_{n-1}(x) +  \frac{x}{n} B_{n-1}(1)=\frac{n-1}{n}x^2+\frac{x}{n}.
\end{split}
\]
Luego, tenemos que $B_n(x^2) \to x^2$ uniformemente.

Dado que $f$ es continua, para cada $\epsilon>0$  existe $\delta>0$ tal  que si $|x-y|<\delta$
entonces $|f(x)-f(y)|<\epsilon$.

Sean $I_{x}=\{ k| \left|\frac{k}{n}-x\right|<\delta \}$ y  $J_x=\{ 1,\dots,n\}-I_x$, luego
\[
\begin{split}
&|f(x)-B_n(f)(x)|
\\
=&\left|\sum\limits_{k=0}^n f(x){n \choose k} x^k (1-x)^{n-k}-
\sum\limits_{k=0}^n f\left(\frac{k}{n}\right){n \choose k} x^k (1-x)^{n-k}\right|
\\
\leq&
\sum\limits_{k=0}^n \left|f(x)-f\left(\frac{k}{n}\right)\right|{n \choose k} x^k (1-x)^{n-k}
\\
\leq 
&\sum\limits_{I_x}+\sum\limits_{J_x}<\epsilon+2M \sum\limits_{k=0}^n {n \choose k} x^k (1-x)^{n-k}
\end{split}
\]
siendo $M=\sup|f|$.

Ahora, 
\[
\begin{split}
\sum\limits_{k=0}^n {n \choose k} x^k (1-x)^{n-k}
\leq 
&\frac{1}{\delta^2} \sum\limits_{k=0}^n \left(x-\frac{k}{n}\right)^2 {n \choose k} x^k (1-x)^{n-k}
\\
=&
\frac{1}{\delta^2} \left(x^2-2x^2+\frac{n-1}{n}x^2+\frac{x}{n}\right).
\end{split}
\]

Luego
\[
\left|\sum\limits_{J_x} {n \choose k} x^k (1-x)^{n-k}
\right|
\leq \frac{1}{\delta^2}\left\{\frac{|x|^2+|x|}{n}\right\}\leq \frac{1}{n\delta^2},
\]
y podemos elegir $n$ tal que $\frac{1}{n \delta^2}<\epsilon$.

\newpage
 Ejemplos unidad 2

    \begin{tcolorbox}[breakable, size=fbox, boxrule=1pt, pad at break*=1mm,colback=cellbackground, colframe=cellborder]
\prompt{In}{incolor}{1}{\hspace{4pt}}
\begin{Verbatim}[commandchars=\\\{\}]
\PY{k+kn}{from} \PY{n+nn}{sympy} \PY{k+kn}{import} \PY{o}{*}
\PY{n}{init\PYZus{}printing}\PY{p}{(}\PY{p}{)}
\end{Verbatim}
\end{tcolorbox}

    Ejemplo \(f_n(x)=\frac{1}{1+nx^2}\) En este ejemplo 

    \begin{tcolorbox}[breakable, size=fbox, boxrule=1pt, pad at break*=1mm,colback=cellbackground, colframe=cellborder]
\prompt{In}{incolor}{1}{\hspace{4pt}}
\begin{Verbatim}[commandchars=\\\{\}]
\PY{n}{x}\PY{p}{,}\PY{n}{n}\PY{o}{=}\PY{n}{symbols}\PY{p}{(}\PY{l+s+s1}{\PYZsq{}}\PY{l+s+s1}{x,n}\PY{l+s+s1}{\PYZsq{}}\PY{p}{)}
\PY{n}{fn}\PY{o}{=}\PY{l+m+mi}{1}\PY{o}{/}\PY{p}{(}\PY{l+m+mi}{1}\PY{o}{+}\PY{n}{n}\PY{o}{*}\PY{n}{x}\PY{o}{*}\PY{o}{*}\PY{l+m+mi}{2}\PY{p}{)}
\PY{n}{p}\PY{o}{=}\PY{n}{plot}\PY{p}{(}\PY{n}{fn}\PY{o}{.}\PY{n}{subs}\PY{p}{(}\PY{n}{n}\PY{p}{,}\PY{l+m+mi}{1}\PY{p}{)}\PY{p}{,} \PY{p}{(}\PY{n}{x}\PY{p}{,}\PY{o}{\PYZhy{}}\PY{l+m+mi}{5}\PY{p}{,}\PY{l+m+mi}{5}\PY{p}{)}\PY{p}{,}\PY{n}{show}\PY{o}{=}\PY{n}{false}\PY{p}{)}
\PY{k}{for} \PY{n}{k} \PY{o+ow}{in} \PY{n+nb}{range}\PY{p}{(}\PY{l+m+mi}{2}\PY{p}{,}\PY{l+m+mi}{100}\PY{p}{)}\PY{p}{:}
    \PY{n}{p}\PY{o}{.}\PY{n}{append}\PY{p}{(}\PY{n}{plot}\PY{p}{(}\PY{n}{fn}\PY{o}{.}\PY{n}{subs}\PY{p}{(}\PY{n}{n}\PY{p}{,}\PY{n}{k}\PY{p}{)}\PY{p}{,} \PY{p}{(}\PY{n}{x}\PY{p}{,}\PY{o}{\PYZhy{}}\PY{l+m+mi}{5}\PY{p}{,}\PY{l+m+mi}{5}\PY{p}{)}\PY{p}{,}\PY{n}{show}\PY{o}{=}\PY{n}{false}\PY{p}{)}\PY{p}{[}\PY{l+m+mi}{0}\PY{p}{]}\PY{p}{)}
\PY{n}{p}\PY{o}{.}\PY{n}{show}\PY{p}{(}\PY{p}{)}
\end{Verbatim}
\end{tcolorbox}

    \begin{center}
    \adjustimage{max size={0.9\linewidth}{0.9\paperheight}}{python/uni3/output_3_0.png}
    \end{center}
    { \hspace*{\fill} \\}
    
    Ejemplo \(f_n(x)=\frac{n^2x-n^2}{1+nx^2}\)

    \begin{tcolorbox}[breakable, size=fbox, boxrule=1pt, pad at break*=1mm,colback=cellbackground, colframe=cellborder]
\prompt{In}{incolor}{3}{\hspace{4pt}}
\begin{Verbatim}[commandchars=\\\{\}]
\PY{n}{x}\PY{p}{,}\PY{n}{n}\PY{o}{=}\PY{n}{symbols}\PY{p}{(}\PY{l+s+s1}{\PYZsq{}}\PY{l+s+s1}{x,n}\PY{l+s+s1}{\PYZsq{}}\PY{p}{)}
\PY{n}{fn}\PY{o}{=}\PY{p}{(}\PY{n}{n}\PY{o}{*}\PY{o}{*}\PY{l+m+mi}{2}\PY{o}{*}\PY{n}{x}\PY{o}{\PYZhy{}}\PY{n}{n}\PY{o}{*}\PY{o}{*}\PY{l+m+mi}{2}\PY{p}{)}\PY{o}{/}\PY{p}{(}\PY{l+m+mi}{1}\PY{o}{+}\PY{n}{n}\PY{o}{*}\PY{n}{x}\PY{o}{*}\PY{o}{*}\PY{l+m+mi}{2}\PY{p}{)}
\PY{n}{p}\PY{o}{=}\PY{n}{plot}\PY{p}{(}\PY{n}{fn}\PY{o}{.}\PY{n}{subs}\PY{p}{(}\PY{n}{n}\PY{p}{,}\PY{l+m+mi}{1}\PY{p}{)}\PY{p}{,} \PY{p}{(}\PY{n}{x}\PY{p}{,}\PY{o}{\PYZhy{}}\PY{l+m+mi}{5}\PY{p}{,}\PY{l+m+mi}{5}\PY{p}{)}\PY{p}{,}\PY{n}{show}\PY{o}{=}\PY{n}{false}\PY{p}{,}\PY{n}{ylim}\PY{o}{=}\PY{p}{(}\PY{o}{\PYZhy{}}\PY{l+m+mi}{20}\PY{p}{,}\PY{l+m+mi}{10}\PY{p}{)}\PY{p}{)}
\PY{k}{for} \PY{n}{k} \PY{o+ow}{in} \PY{n+nb}{range}\PY{p}{(}\PY{l+m+mi}{2}\PY{p}{,}\PY{l+m+mi}{10}\PY{p}{)}\PY{p}{:}
    \PY{n}{p}\PY{o}{.}\PY{n}{append}\PY{p}{(}\PY{n}{plot}\PY{p}{(}\PY{n}{fn}\PY{o}{.}\PY{n}{subs}\PY{p}{(}\PY{n}{n}\PY{p}{,}\PY{n}{k}\PY{p}{)}\PY{p}{,} \PY{p}{(}\PY{n}{x}\PY{p}{,}\PY{o}{\PYZhy{}}\PY{l+m+mi}{5}\PY{p}{,}\PY{l+m+mi}{5}\PY{p}{)}\PY{p}{,}\PY{n}{show}\PY{o}{=}\PY{n}{false}\PY{p}{)}\PY{p}{[}\PY{l+m+mi}{0}\PY{p}{]}\PY{p}{)}
\PY{n}{p}\PY{o}{.}\PY{n}{show}\PY{p}{(}\PY{p}{)}
\end{Verbatim}
\end{tcolorbox}

    \begin{center}
    \adjustimage{max size={0.9\linewidth}{0.9\paperheight}}{python/uni3/output_5_0.png}
    \end{center}
    { \hspace*{\fill} \\}
    
    Ejemplo \(f_n(x)=\frac{nx}{1+n^2x^2}\)

    \begin{tcolorbox}[breakable, size=fbox, boxrule=1pt, pad at break*=1mm,colback=cellbackground, colframe=cellborder]
\prompt{In}{incolor}{4}{\hspace{4pt}}
\begin{Verbatim}[commandchars=\\\{\}]
\PY{n}{x}\PY{p}{,}\PY{n}{n}\PY{o}{=}\PY{n}{symbols}\PY{p}{(}\PY{l+s+s1}{\PYZsq{}}\PY{l+s+s1}{x,n}\PY{l+s+s1}{\PYZsq{}}\PY{p}{)}
\PY{n}{fn}\PY{o}{=}\PY{n}{n}\PY{o}{*}\PY{n}{x}\PY{o}{/}\PY{p}{(}\PY{l+m+mi}{1}\PY{o}{+}\PY{n}{n}\PY{o}{*}\PY{o}{*}\PY{l+m+mi}{2}\PY{o}{*}\PY{n}{x}\PY{o}{*}\PY{o}{*}\PY{l+m+mi}{2}\PY{p}{)}
\PY{n}{p}\PY{o}{=}\PY{n}{plot}\PY{p}{(}\PY{n}{fn}\PY{o}{.}\PY{n}{subs}\PY{p}{(}\PY{n}{n}\PY{p}{,}\PY{l+m+mi}{1}\PY{p}{)}\PY{p}{,} \PY{p}{(}\PY{n}{x}\PY{p}{,}\PY{l+m+mi}{0}\PY{p}{,}\PY{l+m+mi}{5}\PY{p}{)}\PY{p}{,}\PY{n}{show}\PY{o}{=}\PY{n}{false}\PY{p}{)}
\PY{k}{for} \PY{n}{k} \PY{o+ow}{in} \PY{n+nb}{range}\PY{p}{(}\PY{l+m+mi}{2}\PY{p}{,}\PY{l+m+mi}{10}\PY{p}{)}\PY{p}{:}
    \PY{n}{p}\PY{o}{.}\PY{n}{append}\PY{p}{(}\PY{n}{plot}\PY{p}{(}\PY{n}{fn}\PY{o}{.}\PY{n}{subs}\PY{p}{(}\PY{n}{n}\PY{p}{,}\PY{n}{k}\PY{p}{)}\PY{p}{,} \PY{p}{(}\PY{n}{x}\PY{p}{,}\PY{l+m+mi}{0}\PY{p}{,}\PY{l+m+mi}{5}\PY{p}{)}\PY{p}{,}\PY{n}{show}\PY{o}{=}\PY{n}{false}\PY{p}{)}\PY{p}{[}\PY{l+m+mi}{0}\PY{p}{]}\PY{p}{)}
\PY{n}{p}\PY{o}{.}\PY{n}{show}\PY{p}{(}\PY{p}{)}
\end{Verbatim}
\end{tcolorbox}

    \begin{center}
    \adjustimage{max size={0.9\linewidth}{0.9\paperheight}}{python/uni3/output_7_0.png}
    \end{center}
    { \hspace*{\fill} \\}
    
    Ejemplo \(f_n(x)=\sqrt{x^2+\frac{1}{n^2}}\)

    \begin{tcolorbox}[breakable, size=fbox, boxrule=1pt, pad at break*=1mm,colback=cellbackground, colframe=cellborder]
\prompt{In}{incolor}{5}{\hspace{4pt}}
\begin{Verbatim}[commandchars=\\\{\}]
\PY{k}{def} \PY{n+nf}{grafica}\PY{p}{(}\PY{n}{f}\PY{p}{,}\PY{n}{x1}\PY{p}{,}\PY{n}{x2}\PY{p}{,}\PY{n}{m}\PY{p}{)}\PY{p}{:}
    \PY{n}{p}\PY{o}{=}\PY{n}{plot}\PY{p}{(}\PY{n}{f}\PY{o}{.}\PY{n}{subs}\PY{p}{(}\PY{n}{n}\PY{p}{,}\PY{l+m+mi}{1}\PY{p}{)}\PY{p}{,} \PY{p}{(}\PY{n}{x}\PY{p}{,}\PY{n}{x1}\PY{p}{,}\PY{n}{x2}\PY{p}{)}\PY{p}{,}\PY{n}{show}\PY{o}{=}\PY{n}{false}\PY{p}{)}
    \PY{k}{for} \PY{n}{k} \PY{o+ow}{in} \PY{n+nb}{range}\PY{p}{(}\PY{l+m+mi}{2}\PY{p}{,}\PY{n}{m}\PY{p}{)}\PY{p}{:}
        \PY{n}{p}\PY{o}{.}\PY{n}{append}\PY{p}{(}\PY{n}{plot}\PY{p}{(}\PY{n}{f}\PY{o}{.}\PY{n}{subs}\PY{p}{(}\PY{n}{n}\PY{p}{,}\PY{n}{k}\PY{p}{)}\PY{p}{,} \PY{p}{(}\PY{n}{x}\PY{p}{,}\PY{n}{x1}\PY{p}{,}\PY{n}{x2}\PY{p}{)}\PY{p}{,}\PY{n}{show}\PY{o}{=}\PY{n}{false}\PY{p}{)}\PY{p}{[}\PY{l+m+mi}{0}\PY{p}{]}\PY{p}{)}
    \PY{n}{p}\PY{o}{.}\PY{n}{show}\PY{p}{(}\PY{p}{)}
\PY{n}{f}\PY{o}{=}\PY{n}{sqrt}\PY{p}{(}\PY{n}{x}\PY{o}{*}\PY{o}{*}\PY{l+m+mi}{2}\PY{o}{+}\PY{l+m+mf}{1.0}\PY{o}{/}\PY{n}{n}\PY{o}{*}\PY{o}{*}\PY{l+m+mi}{2}\PY{p}{)}
\PY{n}{grafica}\PY{p}{(}\PY{n}{f}\PY{p}{,}\PY{o}{\PYZhy{}}\PY{l+m+mi}{5}\PY{p}{,}\PY{l+m+mi}{5}\PY{p}{,}\PY{l+m+mi}{10}\PY{p}{)}
\end{Verbatim}
\end{tcolorbox}

    \begin{center}
    \adjustimage{max size={0.9\linewidth}{0.9\paperheight}}{python/uni3/output_9_0.png}
    \end{center}
    { \hspace*{\fill} \\}
    
    Ejemplo \(f_n(x)=\sin(nx)\)

    \begin{tcolorbox}[breakable, size=fbox, boxrule=1pt, pad at break*=1mm,colback=cellbackground, colframe=cellborder]
\prompt{In}{incolor}{6}{\hspace{4pt}}
\begin{Verbatim}[commandchars=\\\{\}]
\PY{n}{f}\PY{o}{=}\PY{n}{sin}\PY{p}{(}\PY{n}{n}\PY{o}{*}\PY{n}{x}\PY{p}{)}
\PY{n}{grafica}\PY{p}{(}\PY{n}{f}\PY{p}{,}\PY{l+m+mi}{0}\PY{p}{,}\PY{n}{pi}\PY{p}{,}\PY{l+m+mi}{10}\PY{p}{)}
\end{Verbatim}
\end{tcolorbox}

    \begin{center}
    \adjustimage{max size={0.9\linewidth}{0.9\paperheight}}{python/uni3/output_11_0.png}
    \end{center}
    { \hspace*{\fill} \\}
    
    Series de Potencias

Ejemplo \(S(z)=\sum\limits_{n=1}^{\infty}z^n\)

Serie geométrica converge \(|z|<1\). Ponemos \(z=r e^{i\theta}\). Luego
\(z^j=r^je^{j\theta i}\).

    \begin{tcolorbox}[breakable, size=fbox, boxrule=1pt, pad at break*=1mm,colback=cellbackground, colframe=cellborder]
\prompt{In}{incolor}{7}{\hspace{4pt}}
\begin{Verbatim}[commandchars=\\\{\}]
\PY{n}{n}\PY{p}{,}\PY{n}{theta}\PY{p}{,}\PY{n}{r}\PY{o}{=}\PY{n}{symbols}\PY{p}{(}\PY{l+s+s1}{\PYZsq{}}\PY{l+s+s1}{n,theta,r}\PY{l+s+s1}{\PYZsq{}}\PY{p}{,}\PY{n}{real}\PY{o}{=}\PY{n+nb+bp}{True}\PY{p}{)}

\PY{n}{S}\PY{o}{=}\PY{n+nb}{sum}\PY{p}{(}\PY{p}{[}\PY{n}{r}\PY{o}{*}\PY{o}{*}\PY{n}{j}\PY{o}{*}\PY{n}{exp}\PY{p}{(}\PY{n}{j}\PY{o}{*}\PY{n}{theta}\PY{o}{*}\PY{n}{I}\PY{p}{)} \PY{k}{for} \PY{n}{j} \PY{o+ow}{in} \PY{n+nb}{range}\PY{p}{(}\PY{l+m+mi}{1}\PY{p}{,}\PY{l+m+mi}{5}\PY{p}{)}\PY{p}{]}\PY{p}{)}
\PY{n}{SS}\PY{o}{=}\PY{n}{re}\PY{p}{(}\PY{n}{S}\PY{p}{)}
\end{Verbatim}
\end{tcolorbox}

    \begin{tcolorbox}[breakable, size=fbox, boxrule=1pt, pad at break*=1mm,colback=cellbackground, colframe=cellborder]
\prompt{In}{incolor}{8}{\hspace{4pt}}
\begin{Verbatim}[commandchars=\\\{\}]
\PY{k+kn}{from} \PY{n+nn}{sympy.plotting} \PY{k+kn}{import} \PY{n}{plot3d\PYZus{}parametric\PYZus{}surface}
\end{Verbatim}
\end{tcolorbox}

    \begin{tcolorbox}[breakable, size=fbox, boxrule=1pt, pad at break*=1mm,colback=cellbackground, colframe=cellborder]
\prompt{In}{incolor}{9}{\hspace{4pt}}
\begin{Verbatim}[commandchars=\\\{\}]
\PY{n}{plot3d\PYZus{}parametric\PYZus{}surface}\PY{p}{(}\PY{n}{r}\PY{o}{*}\PY{n}{cos}\PY{p}{(}\PY{n}{theta}\PY{p}{)}\PY{p}{,}\PY{n}{r}\PY{o}{*}\PY{n}{sin}\PY{p}{(}\PY{n}{theta}\PY{p}{)}\PY{p}{,}\PY{n}{SS}\PY{p}{,}\PY{p}{(}\PY{n}{theta}\PY{p}{,}\PY{o}{\PYZhy{}}\PY{n}{pi}\PY{p}{,}\PY{n}{pi}\PY{p}{)}\PY{p}{,}\PY{p}{(}\PY{n}{r}\PY{p}{,}\PY{l+m+mi}{0}\PY{p}{,}\PY{l+m+mi}{1}\PY{p}{)}\PY{p}{)}
\end{Verbatim}
\end{tcolorbox}

    \begin{center}
    \adjustimage{max size={0.9\linewidth}{0.9\paperheight}}{python/uni3/output_15_0.png}
    \end{center}
    { \hspace*{\fill} \\}
    
            \begin{tcolorbox}[breakable, boxrule=.5pt, size=fbox, pad at break*=1mm, opacityfill=0]
\prompt{Out}{outcolor}{9}{\hspace{3.5pt}}
\begin{Verbatim}[commandchars=\\\{\}]
<sympy.plotting.plot.Plot at 0x7f413ec776d0>
\end{Verbatim}
\end{tcolorbox}
        
    Series de Fourier 

    \begin{tcolorbox}[breakable, size=fbox, boxrule=1pt, pad at break*=1mm,colback=cellbackground, colframe=cellborder]
\prompt{In}{incolor}{10}{\hspace{4pt}}
\begin{Verbatim}[commandchars=\\\{\}]
\PY{n}{x}\PY{o}{=}\PY{n}{symbols}\PY{p}{(}\PY{l+s+s1}{\PYZsq{}}\PY{l+s+s1}{x}\PY{l+s+s1}{\PYZsq{}}\PY{p}{,}\PY{n}{real}\PY{o}{=}\PY{n+nb+bp}{True}\PY{p}{)}
\PY{n}{n}\PY{p}{,}\PY{n}{m}\PY{o}{=}\PY{n}{symbols}\PY{p}{(}\PY{l+s+s1}{\PYZsq{}}\PY{l+s+s1}{n,m}\PY{l+s+s1}{\PYZsq{}}\PY{p}{,}\PY{n}{integer}\PY{o}{=}\PY{n+nb+bp}{True}\PY{p}{,}\PY{n}{positive}\PY{o}{=}\PY{n+nb+bp}{True}\PY{p}{)}
\PY{n}{Integral}\PY{p}{(}\PY{n}{cos}\PY{p}{(}\PY{n}{n}\PY{o}{*}\PY{n}{x}\PY{p}{)}\PY{o}{*}\PY{n}{cos}\PY{p}{(}\PY{n}{m}\PY{o}{*}\PY{n}{x}\PY{p}{)}\PY{p}{,}\PY{p}{(}\PY{n}{x}\PY{p}{,}\PY{o}{\PYZhy{}}\PY{n}{pi}\PY{p}{,}\PY{n}{pi}\PY{p}{)}\PY{p}{)}\PY{o}{.}\PY{n}{doit}\PY{p}{(}\PY{p}{)}
\end{Verbatim}
\end{tcolorbox}
 
            
\prompt{Out}{outcolor}{10}{}
    
    $$\begin{cases} 0 & \text{for}\: m \neq n \\\pi & \text{otherwise} \end{cases}$$

    

    \begin{tcolorbox}[breakable, size=fbox, boxrule=1pt, pad at break*=1mm,colback=cellbackground, colframe=cellborder]
\prompt{In}{incolor}{11}{\hspace{4pt}}
\begin{Verbatim}[commandchars=\\\{\}]
\PY{n}{Integral}\PY{p}{(}\PY{n}{sin}\PY{p}{(}\PY{n}{n}\PY{o}{*}\PY{n}{x}\PY{p}{)}\PY{o}{*}\PY{n}{sin}\PY{p}{(}\PY{n}{m}\PY{o}{*}\PY{n}{x}\PY{p}{)}\PY{p}{,}\PY{p}{(}\PY{n}{x}\PY{p}{,}\PY{o}{\PYZhy{}}\PY{n}{pi}\PY{p}{,}\PY{n}{pi}\PY{p}{)}\PY{p}{)}\PY{o}{.}\PY{n}{doit}\PY{p}{(}\PY{p}{)}
\end{Verbatim}
\end{tcolorbox}
 
            
\prompt{Out}{outcolor}{11}{}
    
    $$\begin{cases} 0 & \text{for}\: m \neq n \\\pi & \text{otherwise} \end{cases}$$

    

    \begin{tcolorbox}[breakable, size=fbox, boxrule=1pt, pad at break*=1mm,colback=cellbackground, colframe=cellborder]
\prompt{In}{incolor}{12}{\hspace{4pt}}
\begin{Verbatim}[commandchars=\\\{\}]
\PY{n}{Integral}\PY{p}{(}\PY{n}{cos}\PY{p}{(}\PY{n}{n}\PY{o}{*}\PY{n}{x}\PY{p}{)}\PY{o}{*}\PY{n}{sin}\PY{p}{(}\PY{n}{m}\PY{o}{*}\PY{n}{x}\PY{p}{)}\PY{p}{,}\PY{p}{(}\PY{n}{x}\PY{p}{,}\PY{o}{\PYZhy{}}\PY{n}{pi}\PY{p}{,}\PY{n}{pi}\PY{p}{)}\PY{p}{)}\PY{o}{.}\PY{n}{doit}\PY{p}{(}\PY{p}{)}
\end{Verbatim}
\end{tcolorbox}
 
            
\prompt{Out}{outcolor}{12}{}
    
    $$0$$

    

    \begin{tcolorbox}[breakable, size=fbox, boxrule=1pt, pad at break*=1mm,colback=cellbackground, colframe=cellborder]
\prompt{In}{incolor}{13}{\hspace{4pt}}
\begin{Verbatim}[commandchars=\\\{\}]
\PY{n}{S}\PY{o}{=}\PY{n+nb}{sum}\PY{p}{(}\PY{p}{[}\PY{n}{sin}\PY{p}{(}\PY{n}{n}\PY{o}{*}\PY{n}{theta}\PY{p}{)}\PY{o}{/}\PY{n}{n} \PY{k}{for} \PY{n}{n} \PY{o+ow}{in} \PY{n+nb}{range}\PY{p}{(}\PY{l+m+mi}{1}\PY{p}{,}\PY{l+m+mi}{50}\PY{p}{)}\PY{p}{]}\PY{p}{)}
\PY{n}{plot}\PY{p}{(}\PY{n}{S}\PY{p}{,}\PY{p}{(}\PY{n}{theta}\PY{p}{,}\PY{o}{\PYZhy{}}\PY{l+m+mi}{4}\PY{o}{*}\PY{n}{pi}\PY{p}{,}\PY{l+m+mi}{4}\PY{o}{*}\PY{n}{pi}\PY{p}{)}\PY{p}{)}
\end{Verbatim}
\end{tcolorbox}

    \begin{center}
    \adjustimage{max size={0.9\linewidth}{0.9\paperheight}}{python/uni3/output_20_0.png}
    \end{center}
    { \hspace*{\fill} \\}
    
            \begin{tcolorbox}[breakable, boxrule=.5pt, size=fbox, pad at break*=1mm, opacityfill=0]
\prompt{Out}{outcolor}{13}{\hspace{3.5pt}}
\begin{Verbatim}[commandchars=\\\{\}]
<sympy.plotting.plot.Plot at 0x7f413f16afd0>
\end{Verbatim}
\end{tcolorbox}
        
    Ejemplo
\[f(x)=\left\{\begin{array}{cc} \frac{\pi-x}{2}, & \text{ si } x\in [0,\pi]\\
    -\frac{\pi+x}{2}, & \text{ si } x\in [-\pi,0) \end{array}    \right.\]

    \begin{tcolorbox}[breakable, size=fbox, boxrule=1pt, pad at break*=1mm,colback=cellbackground, colframe=cellborder]
\prompt{In}{incolor}{14}{\hspace{4pt}}
\begin{Verbatim}[commandchars=\\\{\}]
\PY{n}{f}\PY{o}{=}\PY{n}{Piecewise}\PY{p}{(}\PY{p}{(}\PY{p}{(}\PY{n}{pi}\PY{o}{\PYZhy{}}\PY{n}{x}\PY{p}{)}\PY{o}{/}\PY{l+m+mi}{2}\PY{p}{,} \PY{n}{x}\PY{o}{\PYZgt{}}\PY{o}{=}\PY{l+m+mi}{0}  \PY{p}{)}\PY{p}{,}\PY{p}{(}\PY{o}{\PYZhy{}}\PY{p}{(}\PY{n}{pi}\PY{o}{+}\PY{n}{x}\PY{p}{)}\PY{o}{/}\PY{l+m+mi}{2}\PY{p}{,}\PY{n}{x}\PY{o}{\PYZlt{}}\PY{l+m+mi}{0} \PY{p}{)}\PY{p}{)}
\PY{n}{plot}\PY{p}{(}\PY{n}{f}\PY{p}{,}\PY{p}{(}\PY{n}{x}\PY{p}{,}\PY{o}{\PYZhy{}}\PY{l+m+mi}{4}\PY{p}{,}\PY{l+m+mi}{4}\PY{p}{)}\PY{p}{)}
\end{Verbatim}
\end{tcolorbox}

    \begin{center}
    \adjustimage{max size={0.9\linewidth}{0.9\paperheight}}{python/uni3/output_22_0.png}
    \end{center}
    { \hspace*{\fill} \\}
    
            \begin{tcolorbox}[breakable, boxrule=.5pt, size=fbox, pad at break*=1mm, opacityfill=0]
\prompt{Out}{outcolor}{14}{\hspace{3.5pt}}
\begin{Verbatim}[commandchars=\\\{\}]
<sympy.plotting.plot.Plot at 0x7f413f019b50>
\end{Verbatim}
\end{tcolorbox}
        
    \begin{tcolorbox}[breakable, size=fbox, boxrule=1pt, pad at break*=1mm,colback=cellbackground, colframe=cellborder]
\prompt{In}{incolor}{15}{\hspace{4pt}}
\begin{Verbatim}[commandchars=\\\{\}]
\PY{k}{def} \PY{n+nf}{a}\PY{p}{(}\PY{n}{g}\PY{p}{,}\PY{n}{k}\PY{p}{)}\PY{p}{:}
    \PY{k}{return} \PY{l+m+mi}{1}\PY{o}{/}\PY{n}{pi}\PY{o}{*}\PY{n}{Integral}\PY{p}{(}\PY{n}{g}\PY{o}{*}\PY{n}{cos}\PY{p}{(}\PY{n}{k}\PY{o}{*}\PY{n}{x}\PY{p}{)}\PY{p}{,}\PY{p}{(}\PY{n}{x}\PY{p}{,}\PY{o}{\PYZhy{}}\PY{n}{pi}\PY{p}{,}\PY{n}{pi}\PY{p}{)}\PY{p}{)}\PY{o}{.}\PY{n}{doit}\PY{p}{(}\PY{p}{)}
\PY{k}{def} \PY{n+nf}{b}\PY{p}{(}\PY{n}{g}\PY{p}{,}\PY{n}{k}\PY{p}{)}\PY{p}{:}
    \PY{k}{return} \PY{l+m+mi}{1}\PY{o}{/}\PY{n}{pi}\PY{o}{*}\PY{n}{Integral}\PY{p}{(}\PY{n}{g}\PY{o}{*}\PY{n}{sin}\PY{p}{(}\PY{n}{k}\PY{o}{*}\PY{n}{x}\PY{p}{)}\PY{p}{,}\PY{p}{(}\PY{n}{x}\PY{p}{,}\PY{o}{\PYZhy{}}\PY{n}{pi}\PY{p}{,}\PY{n}{pi}\PY{p}{)}\PY{p}{)}\PY{o}{.}\PY{n}{doit}\PY{p}{(}\PY{p}{)}
\end{Verbatim}
\end{tcolorbox}

    \begin{tcolorbox}[breakable, size=fbox, boxrule=1pt, pad at break*=1mm,colback=cellbackground, colframe=cellborder]
\prompt{In}{incolor}{16}{\hspace{4pt}}
\begin{Verbatim}[commandchars=\\\{\}]
\PY{p}{[}\PY{n}{a}\PY{p}{(}\PY{n}{f}\PY{p}{,}\PY{n}{k}\PY{p}{)} \PY{k}{for} \PY{n}{k} \PY{o+ow}{in} \PY{n+nb}{range}\PY{p}{(}\PY{l+m+mi}{1}\PY{p}{,}\PY{l+m+mi}{10}\PY{p}{)}\PY{p}{]}
\end{Verbatim}
\end{tcolorbox}
 
            
\prompt{Out}{outcolor}{16}{}
    
    $$\left [ 0, \quad 0, \quad 0, \quad 0, \quad 0, \quad 0, \quad 0, \quad 0, \quad 0\right ]$$

    

    \begin{tcolorbox}[breakable, size=fbox, boxrule=1pt, pad at break*=1mm,colback=cellbackground, colframe=cellborder]
\prompt{In}{incolor}{17}{\hspace{4pt}}
\begin{Verbatim}[commandchars=\\\{\}]
\PY{p}{[}\PY{n}{b}\PY{p}{(}\PY{n}{f}\PY{p}{,}\PY{n}{k}\PY{p}{)} \PY{k}{for} \PY{n}{k} \PY{o+ow}{in} \PY{n+nb}{range}\PY{p}{(}\PY{l+m+mi}{1}\PY{p}{,}\PY{l+m+mi}{10}\PY{p}{)}\PY{p}{]}
\end{Verbatim}
\end{tcolorbox}
 
            
\prompt{Out}{outcolor}{17}{}
    
    $$\left [ 1, \quad \frac{1}{2}, \quad \frac{1}{3}, \quad \frac{1}{4}, \quad \frac{1}{5}, \quad \frac{1}{6}, \quad \frac{1}{7}, \quad \frac{1}{8}, \quad \frac{1}{9}\right ]$$

    

    \begin{tcolorbox}[breakable, size=fbox, boxrule=1pt, pad at break*=1mm,colback=cellbackground, colframe=cellborder]
\prompt{In}{incolor}{18}{\hspace{4pt}}
\begin{Verbatim}[commandchars=\\\{\}]
\PY{n}{S}\PY{o}{=}\PY{n+nb}{sum}\PY{p}{(}\PY{p}{[}\PY{n}{b}\PY{p}{(}\PY{n}{f}\PY{p}{,}\PY{n}{k}\PY{p}{)}\PY{o}{*}\PY{n}{sin}\PY{p}{(}\PY{n}{k}\PY{o}{*}\PY{n}{x}\PY{p}{)} \PY{k}{for} \PY{n}{k} \PY{o+ow}{in} \PY{n+nb}{range}\PY{p}{(}\PY{l+m+mi}{1}\PY{p}{,}\PY{l+m+mi}{20}\PY{p}{)}\PY{p}{]}\PY{p}{)}
\PY{n}{S}
\end{Verbatim}
\end{tcolorbox}
 
            
\prompt{Out}{outcolor}{18}{}
    
    $$\sin{\left (x \right )} + \frac{\sin{\left (2 x \right )}}{2} + \frac{\sin{\left (3 x \right )}}{3} + \frac{\sin{\left (4 x \right )}}{4} + \frac{\sin{\left (5 x \right )}}{5} + \frac{\sin{\left (6 x \right )}}{6} + \frac{\sin{\left (7 x \right )}}{7} + \frac{\sin{\left (8 x \right )}}{8} + \frac{\sin{\left (9 x \right )}}{9} + \frac{\sin{\left (10 x \right )}}{10} + \frac{\sin{\left (11 x \right )}}{11} + \frac{\sin{\left (12 x \right )}}{12} + \frac{\sin{\left (13 x \right )}}{13} + \frac{\sin{\left (14 x \right )}}{14} + \frac{\sin{\left (15 x \right )}}{15} + \frac{\sin{\left (16 x \right )}}{16} + \frac{\sin{\left (17 x \right )}}{17} + \frac{\sin{\left (18 x \right )}}{18} + \frac{\sin{\left (19 x \right )}}{19}$$

    

    \begin{tcolorbox}[breakable, size=fbox, boxrule=1pt, pad at break*=1mm,colback=cellbackground, colframe=cellborder]
\prompt{In}{incolor}{19}{\hspace{4pt}}
\begin{Verbatim}[commandchars=\\\{\}]
\PY{n}{S}\PY{o}{=}\PY{n+nb}{sum}\PY{p}{(}\PY{p}{[}\PY{n}{b}\PY{p}{(}\PY{n}{f}\PY{p}{,}\PY{n}{k}\PY{p}{)}\PY{o}{*}\PY{n}{sin}\PY{p}{(}\PY{n}{k}\PY{o}{*}\PY{n}{x}\PY{p}{)} \PY{k}{for} \PY{n}{k} \PY{o+ow}{in} \PY{n+nb}{range}\PY{p}{(}\PY{l+m+mi}{1}\PY{p}{,}\PY{l+m+mi}{20}\PY{p}{)}\PY{p}{]}\PY{p}{)}
\PY{n}{plot}\PY{p}{(}\PY{n}{f}\PY{p}{,}\PY{n}{S}\PY{p}{,} \PY{p}{(}\PY{n}{x}\PY{p}{,}\PY{o}{\PYZhy{}}\PY{n}{pi}\PY{p}{,}\PY{n}{pi}\PY{p}{)}\PY{p}{)}
\end{Verbatim}
\end{tcolorbox}

    \begin{center}
    \adjustimage{max size={0.9\linewidth}{0.9\paperheight}}{python/uni3/output_27_0.png}
    \end{center}
    { \hspace*{\fill} \\}
    
            \begin{tcolorbox}[breakable, boxrule=.5pt, size=fbox, pad at break*=1mm, opacityfill=0]
\prompt{Out}{outcolor}{19}{\hspace{3.5pt}}
\begin{Verbatim}[commandchars=\\\{\}]
<sympy.plotting.plot.Plot at 0x7f413f1ae210>
\end{Verbatim}
\end{tcolorbox}
        
    Convergencia uniforme

Ejemplo $ f\_n(x)=\frac{1}{n}e^{-n2x^2}x $

    \begin{tcolorbox}[breakable, size=fbox, boxrule=1pt, pad at break*=1mm,colback=cellbackground, colframe=cellborder]
\prompt{In}{incolor}{20}{\hspace{4pt}}
\begin{Verbatim}[commandchars=\\\{\}]
\PY{n}{x}\PY{p}{,}\PY{n}{n}\PY{o}{=}\PY{n}{symbols}\PY{p}{(}\PY{l+s+s1}{\PYZsq{}}\PY{l+s+s1}{x,n}\PY{l+s+s1}{\PYZsq{}}\PY{p}{)}
\PY{n}{fn}\PY{o}{=}\PY{l+m+mi}{1}\PY{o}{/}\PY{n}{n}\PY{o}{*}\PY{n}{exp}\PY{p}{(}\PY{o}{\PYZhy{}}\PY{n}{n}\PY{o}{*}\PY{o}{*}\PY{l+m+mi}{2}\PY{o}{*}\PY{n}{x}\PY{o}{*}\PY{o}{*}\PY{l+m+mi}{2}\PY{p}{)}\PY{o}{*}\PY{n}{x}
\PY{n}{p}\PY{o}{=}\PY{n}{plot}\PY{p}{(}\PY{n}{fn}\PY{o}{.}\PY{n}{subs}\PY{p}{(}\PY{n}{n}\PY{p}{,}\PY{l+m+mi}{1}\PY{p}{)}\PY{p}{,} \PY{p}{(}\PY{n}{x}\PY{p}{,}\PY{o}{\PYZhy{}}\PY{l+m+mi}{5}\PY{p}{,}\PY{l+m+mi}{5}\PY{p}{)}\PY{p}{,}\PY{n}{show}\PY{o}{=}\PY{n}{false}\PY{p}{)}
\PY{k}{for} \PY{n}{k} \PY{o+ow}{in} \PY{n+nb}{range}\PY{p}{(}\PY{l+m+mi}{2}\PY{p}{,}\PY{l+m+mi}{10}\PY{p}{)}\PY{p}{:}
    \PY{n}{p}\PY{o}{.}\PY{n}{append}\PY{p}{(}\PY{n}{plot}\PY{p}{(}\PY{n}{fn}\PY{o}{.}\PY{n}{subs}\PY{p}{(}\PY{n}{n}\PY{p}{,}\PY{n}{k}\PY{p}{)}\PY{p}{,} \PY{p}{(}\PY{n}{x}\PY{p}{,}\PY{o}{\PYZhy{}}\PY{l+m+mi}{5}\PY{p}{,}\PY{l+m+mi}{5}\PY{p}{)}\PY{p}{,}\PY{n}{show}\PY{o}{=}\PY{n}{false}\PY{p}{)}\PY{p}{[}\PY{l+m+mi}{0}\PY{p}{]}\PY{p}{)}
\PY{n}{p}\PY{o}{.}\PY{n}{show}\PY{p}{(}\PY{p}{)}
\end{Verbatim}
\end{tcolorbox}

    \begin{center}
    \adjustimage{max size={0.9\linewidth}{0.9\paperheight}}{python/uni3/output_29_0.png}
    \end{center}
    { \hspace*{\fill} \\}
    
    Ejemplo \$ f\_n(x)=ne\textsuperscript{\{-n}2x\^{}2\}x \$

    \begin{tcolorbox}[breakable, size=fbox, boxrule=1pt, pad at break*=1mm,colback=cellbackground, colframe=cellborder]
\prompt{In}{incolor}{21}{\hspace{4pt}}
\begin{Verbatim}[commandchars=\\\{\}]
\PY{n}{fn}\PY{o}{=}\PY{n}{n}\PY{o}{*}\PY{n}{exp}\PY{p}{(}\PY{o}{\PYZhy{}}\PY{n}{n}\PY{o}{*}\PY{o}{*}\PY{l+m+mi}{2}\PY{o}{*}\PY{n}{x}\PY{o}{*}\PY{o}{*}\PY{l+m+mi}{2}\PY{p}{)}\PY{o}{*}\PY{n}{x}
\PY{n}{p}\PY{o}{=}\PY{n}{plot}\PY{p}{(}\PY{n}{fn}\PY{o}{.}\PY{n}{subs}\PY{p}{(}\PY{n}{n}\PY{p}{,}\PY{l+m+mi}{1}\PY{p}{)}\PY{p}{,} \PY{p}{(}\PY{n}{x}\PY{p}{,}\PY{o}{\PYZhy{}}\PY{l+m+mi}{5}\PY{p}{,}\PY{l+m+mi}{5}\PY{p}{)}\PY{p}{,}\PY{n}{show}\PY{o}{=}\PY{n}{false}\PY{p}{)}
\PY{k}{for} \PY{n}{k} \PY{o+ow}{in} \PY{n+nb}{range}\PY{p}{(}\PY{l+m+mi}{2}\PY{p}{,}\PY{l+m+mi}{10}\PY{p}{)}\PY{p}{:}
    \PY{n}{p}\PY{o}{.}\PY{n}{append}\PY{p}{(}\PY{n}{plot}\PY{p}{(}\PY{n}{fn}\PY{o}{.}\PY{n}{subs}\PY{p}{(}\PY{n}{n}\PY{p}{,}\PY{n}{k}\PY{p}{)}\PY{p}{,} \PY{p}{(}\PY{n}{x}\PY{p}{,}\PY{o}{\PYZhy{}}\PY{l+m+mi}{5}\PY{p}{,}\PY{l+m+mi}{5}\PY{p}{)}\PY{p}{,}\PY{n}{show}\PY{o}{=}\PY{n}{false}\PY{p}{)}\PY{p}{[}\PY{l+m+mi}{0}\PY{p}{]}\PY{p}{)}
\PY{n}{p}\PY{o}{.}\PY{n}{show}\PY{p}{(}\PY{p}{)}
\end{Verbatim}
\end{tcolorbox}

    \begin{center}
    \adjustimage{max size={0.9\linewidth}{0.9\paperheight}}{python/uni3/output_31_0.png}
    \end{center}
    { \hspace*{\fill} \\}
    

    % Add a bibliography block to the postdoc
    
\chapter{Integral de Riemann}

\section{Introducción}

\begin{quotation}
\marginnote{\adjustimage{max size={0.9\linewidth}{0.9\paperheight}}{imagenes/Riemann.jpeg}\\
Bernhard Riemann 1826-1866
} 
<< Bernard Riemann recibió su doctorado en 1851, su \emph{Habilitación} en 1854. La habilitación confiere el reconocimiento de la capacidad de crear sustanciales contribuciones en la investigación más allá de la tesis doctoral, y es un prerequisito necesario para ocupar un cargo de profesor en una universidad Alemana. Riemann eligió como tema  de habilitación el problema de las series de Fourier. Su tesis fue titulada \emph{\"Uber die Darstellbarkeit einer Function durch eine trigonometrische Reine} (Sobre la representación de una función por series trigonométricas) y respondía la pregunta:  Cuándo una función definida en el intervalo $(-\pi,\pi)$ puede ser respresentada por la serie trigonométrica $a_0/2+\sum_{n=1}^{\infty}[a_n\cos(nx)+b_n\sen(nx)]$? 
En este trabajo  es donde hallamos   la Integral de Riemann, introducida en una sección corta antes del nucleo principal de la tesis, como parte del trabajo preparatorio que él necesitó desarrollar antes de abordar el problema de representabilidad por series trigonométricas. >> 
\end{quotation}
\begin{flushright}
 David M. Bressoud\\
 A Radical Approach to Lebesgue's Theory of Integration.
\end{flushright}


En este capítulo vamos a desarrollar el concepto de la integral de Riemman. Vamos a exponer la definición de la integral debida a Riemann y la ideada por J. G. Darboux.
Mostraemos la equivalencia de las dos definiciones y discutiremos las propiedades de la intergal, sus alcances y límites. Preparamos así el camino para la introducción de la integral de Lebesgue. 
\marginnote{\adjustimage{max size={0.9\linewidth}{0.9\paperheight}}{imagenes/Darboux.jpg}\\
Jean G. Darboux  1842-1917
} 

Debemos advertir  al alumno que en este curso dejaremos un poco de lado las cuestiones procedimentales de cómo calcular integrales, aspecto que seguramente abordó en cursos anteriores y del cual nos vamos a valer. Tampoco debe esperar que las actividades prácticas se centren en esa dirección.   Nuestro principal objetivo aquí es discutir la materia conceptual ligada a la integral y cómo es previsible las actividades prácticas estarán orientadas con ese propósito.


El concepto de integral encuentra su motivación en diversos problemas. Aparece cuando se busca el centro de masas de un determinado cuerpo, cuando se quieren hallar longitudes de arco, volúmenes, cuando se quiere reconstruir el movimiento de cuerpo conocida su velocidad, etc. La integral es utilizada en incontables otros conceptos matemáticos, como ser el mencionado már arriba relativo a las series de Fourier. 

Quizás el 
problema más simple donde aparece la integral es el que utilizaremos como motivación para introducirla y es el concepto de área.  Vamos a tratar de reconstruir este concepto desde su base, esto es analizando la noción de área de figuras tan simples como rectángulos, triángulos, etc. 



\section{Área de figuras elementales planas}\label{sec:area_elem}

  
El cálculo de áreas es necesario en multitud de actividades humanas, por ejemplo con el comercio. La cantidad de muchos productos y servicios se estima en medidas de área, por ejemplo: las telas,  el trabajo de un colocador de pisos,  el precio de la construcción,  el valor de las extensiones de tierra, etc.  
 
 


Por figuras elementales planas nos referimos a rectángulos, triángulos, trapecios, etc. Sin duda el alumno  debe estar  muy familiarizado con las áreas de estas figuras, el área de un rectángulo viene dada por la conocida fórmula $b\times h$, donde $b$ es la base del rectángulo y $h$ su altura.  Ahora bien, ¿Cómo 
se llega a esta fórmula? Porque esta fórmula es apropiada para calcular el precio de un terreno por ejemplo. En esta sección vamos a justificar esta fórmula a partir de algunos hechos elementales.



Vamos a considerar un plano $\mathcal{P}$. En este plano $\mathcal{P}$ supondremos fijada una unidad de longitud.  Pretendemos asignar un área a las figuras, es decir a los subconjuntos, de $\mathcal{P}$. De ahora en más, cómo es usual en esta materia  nos referiremos a \emph{medida}\index{medida} en lugar de área. La medida es un concepto más general  que el concepto de área. No obstante en el contexto en que estamos actualmente son sinónimos.  

Queremos construir pues una función $m$ tal que $m(A)$ reppresente la medida  de  $A\subset\mathcal{P}$. Ahora bien ¿qué podemos usar de guía con ese objetivo? Si, como dijimos,  desconocemos todas las fórmulas previamente aprendidas, sobre que partimos para construir la medida o área. La respuesta es que tomaremos como principio rector  ciertas propiedades que son deseables  que una medida satisfaga. Ellas son las  siguientes. 




\begin{description}
 \item[Positividad.] debería ser una magnitud no negativa.  
 \item[Invariancia por movimientos rígidos.] Si una región es transformada en otra por medio de un movimiento rígido, ambas regiones deberían tener la misma área. Otra manera de expresar esta propiedad es diciendo que dos figuras \emph{congruentes}\index{congruencia} tienen la misma área. 
 \item[Aditividad.] Si una región es la unión de cierta cantidad de regiones más chicas mutuamente disjuntas  
\end{description}

\begin{figure}[h]
\begin{center}
 
\definecolor{xdxdff}{rgb}{0.49,0.49,1}
\definecolor{zzttqq}{rgb}{0.6,0.2,0}
\definecolor{ududff}{rgb}{0.30,0.30,1}
\begin{tikzpicture}[line cap=round,line join=round,x=.9cm,y=.9cm]
\clip(-2.261345665671131,-2.6483801457242535) rectangle (15.749723988011487,4.927708528477943);
\fill[line width=2pt,color=zzttqq,fill=zzttqq,fill opacity=0.10000000149011612] (-1.62,-0.89) -- (-1.62,3.13) -- (1.28,3.15) -- (1.3,-0.89) -- cycle;
\fill[line width=2pt,color=zzttqq,fill=zzttqq,fill opacity=0.10000000149011612] (0.09169246661626662,0.6507662540293884) -- (1.2899992647959808,1.130148511211861) -- (1.28,3.15) -- (-0.3199239037382289,3.1389660420431844) -- cycle;
\fill[line width=2pt,color=zzttqq,fill=zzttqq,fill opacity=0.10000000149011612] (-1.62,1.45) -- (-1.62,3.13) -- (-0.3199239037382289,3.1389660420431844) -- (0.09169246661626662,0.6507662540293884) -- (-0.9454716981132074,0.2358490566037732) -- cycle;
\fill[line width=2pt,color=zzttqq,fill=zzttqq,fill opacity=0.10000000149011612] (-1.62,-0.89) -- (-0.32,-0.89) -- (-1.62,1.45) -- cycle;
\fill[line width=2pt,color=zzttqq,fill=zzttqq,fill opacity=0.10000000149011612] (-0.32,-0.89) -- (-0.9454716981132074,0.2358490566037732) -- (0.09169246661626662,0.6507662540293884) -- (1.2899992647959808,1.130148511211861) -- (1.3,-0.89) -- cycle;
\fill[line width=2pt,color=zzttqq,fill=zzttqq,fill opacity=0.10000000149011612] (8.76,1.77) -- (8.134528301886792,2.8958490566037733) -- (9.171692466616268,3.3107662540293887) -- (10.36999926479598,3.790148511211861) -- (10.38,1.77) -- cycle;
\fill[line width=2pt,color=zzttqq,fill=zzttqq,fill opacity=0.10000000149011612] (3.8264466094067267,-1.199878066911803) -- (2.6385072170133266,-0.011938674518402692) -- (3.551459891609421,0.9136938983359697) -- (5.601939561385962,-0.5546723179904587) -- (5.16194451123264,-1.5814488960049211) -- cycle;
\fill[line width=2pt,color=zzttqq,fill=zzttqq,fill opacity=0.10000000149011612] (5.6888024763961145,1.8814084921516754) -- (6.19715889449669,3.0677137999207305) -- (4.761837661840736,4.4888939366884495) -- (3.638322806619573,3.3497747084781038) -- cycle;
\fill[line width=2pt,color=zzttqq,fill=zzttqq,fill opacity=0.10000000149011612] (7.14,0.65) -- (5.84,0.65) -- (7.14,-1.69) -- cycle;
\draw [line width=2pt,color=zzttqq] (-1.62,-0.89)-- (-1.62,3.13);
\draw [line width=2pt,color=zzttqq] (-1.62,3.13)-- (1.28,3.15);
\draw [line width=2pt,color=zzttqq] (1.28,3.15)-- (1.3,-0.89);
\draw [line width=2pt,color=zzttqq] (1.3,-0.89)-- (-1.62,-0.89);
\draw [line width=2pt] (-1.62,1.45)-- (-0.32,-0.89);
\draw [line width=2pt] (-0.9454716981132074,0.2358490566037732)-- (1.2899992647959808,1.130148511211861);
\draw [line width=2pt] (-0.3199239037382289,3.1389660420431844)-- (0.09169246661626662,0.6507662540293884);
\draw [line width=2pt,color=zzttqq] (0.09169246661626662,0.6507662540293884)-- (1.2899992647959808,1.130148511211861);
\draw [line width=2pt,color=zzttqq] (1.2899992647959808,1.130148511211861)-- (1.28,3.15);
\draw [line width=2pt,color=zzttqq] (1.28,3.15)-- (-0.3199239037382289,3.1389660420431844);
\draw [line width=2pt,color=zzttqq] (-0.3199239037382289,3.1389660420431844)-- (0.09169246661626662,0.6507662540293884);
\draw [line width=2pt,color=zzttqq] (-1.62,1.45)-- (-1.62,3.13);
\draw [line width=2pt,color=zzttqq] (-1.62,3.13)-- (-0.3199239037382289,3.1389660420431844);
\draw [line width=2pt,color=zzttqq] (-0.3199239037382289,3.1389660420431844)-- (0.09169246661626662,0.6507662540293884);
\draw [line width=2pt,color=zzttqq] (0.09169246661626662,0.6507662540293884)-- (-0.9454716981132074,0.2358490566037732);
\draw [line width=2pt,color=zzttqq] (-0.9454716981132074,0.2358490566037732)-- (-1.62,1.45);
\draw [line width=2pt,color=zzttqq] (-1.62,-0.89)-- (-0.32,-0.89);
\draw [line width=2pt,color=zzttqq] (-0.32,-0.89)-- (-1.62,1.45);
\draw [line width=2pt,color=zzttqq] (-1.62,1.45)-- (-1.62,-0.89);
\draw [line width=2pt,color=zzttqq] (-0.32,-0.89)-- (-0.9454716981132074,0.2358490566037732);
\draw [line width=2pt,color=zzttqq] (-0.9454716981132074,0.2358490566037732)-- (0.09169246661626662,0.6507662540293884);
\draw [line width=2pt,color=zzttqq] (0.09169246661626662,0.6507662540293884)-- (1.2899992647959808,1.130148511211861);
\draw [line width=2pt,color=zzttqq] (1.2899992647959808,1.130148511211861)-- (1.3,-0.89);
\draw [line width=2pt,color=zzttqq] (1.3,-0.89)-- (-0.32,-0.89);
\draw [line width=2pt,color=zzttqq] (8.76,1.77)-- (8.134528301886792,2.8958490566037733);
\draw [line width=2pt,color=zzttqq] (8.134528301886792,2.8958490566037733)-- (9.171692466616268,3.3107662540293887);
\draw [line width=2pt,color=zzttqq] (9.171692466616268,3.3107662540293887)-- (10.36999926479598,3.790148511211861);
\draw [line width=2pt,color=zzttqq] (10.36999926479598,3.790148511211861)-- (10.38,1.77);
\draw [line width=2pt,color=zzttqq] (10.38,1.77)-- (8.76,1.77);
\draw [line width=2pt,color=zzttqq] (3.8264466094067267,-1.199878066911803)-- (2.6385072170133266,-0.011938674518402692);
\draw [line width=2pt,color=zzttqq] (2.6385072170133266,-0.011938674518402692)-- (3.551459891609421,0.9136938983359697);
\draw [line width=2pt,color=zzttqq] (3.551459891609421,0.9136938983359697)-- (5.601939561385962,-0.5546723179904587);
\draw [line width=2pt,color=zzttqq] (5.601939561385962,-0.5546723179904587)-- (5.16194451123264,-1.5814488960049211);
\draw [line width=2pt,color=zzttqq] (5.16194451123264,-1.5814488960049211)-- (3.8264466094067267,-1.199878066911803);
\draw [line width=2pt,color=zzttqq] (5.6888024763961145,1.8814084921516754)-- (6.19715889449669,3.0677137999207305);
\draw [line width=2pt,color=zzttqq] (6.19715889449669,3.0677137999207305)-- (4.761837661840736,4.4888939366884495);
\draw [line width=2pt,color=zzttqq] (4.761837661840736,4.4888939366884495)-- (3.638322806619573,3.3497747084781038);
\draw [line width=2pt,color=zzttqq] (3.638322806619573,3.3497747084781038)-- (5.6888024763961145,1.8814084921516754);
\draw [line width=2pt,color=zzttqq] (7.14,0.65)-- (5.84,0.65);
\draw [line width=2pt,color=zzttqq] (5.84,0.65)-- (7.14,-1.69);
\draw [line width=2pt,color=zzttqq] (7.14,-1.69)-- (7.14,0.65);
\draw (-1.1356538123159674,2.2366014415507594) node[anchor=north west] {$A_1$};
\draw (3.9651373981996176,0.03798454046645946) node[anchor=north west] {$A_1$};
\draw (0.23628313396063827,2.5883801457242477) node[anchor=north west] {$A_2$};
\draw (4.703872676963944,3.83719454554013) node[anchor=north west] {$A_2$};
\draw (0.09557165229124281,0.5304747263093427) node[anchor=north west] {$A_3$};
\draw (9.101106479132552,3.1160482019844795) node[anchor=north west] {$A_3$};
\draw (-1.5753771925328282,0.31940750380524985) node[anchor=north west] {$A_4$};
\draw (6.445177262622713,0.5480636615180171) node[anchor=north west] {$A_4$};
\begin{scriptsize}
\draw [fill=ududff] (-1.62,-0.89) circle (2.5pt);
\draw [fill=ududff] (-1.62,3.13) circle (2.5pt);
\draw [fill=ududff] (1.28,3.15) circle (2.5pt);
\draw [fill=ududff] (1.3,-0.89) circle (2.5pt);
\draw [fill=xdxdff] (-1.62,1.45) circle (2.5pt);
\draw [fill=xdxdff] (-0.32,-0.89) circle (2.5pt);
\draw [fill=xdxdff] (-0.9454716981132074,0.2358490566037732) circle (2.5pt);
\draw [fill=xdxdff] (1.2899992647959808,1.130148511211861) circle (2.5pt);
\draw [fill=xdxdff] (-0.3199239037382289,3.1389660420431844) circle (2.5pt);
\draw [fill=xdxdff] (0.09169246661626662,0.6507662540293884) circle (2.5pt);
\draw [fill=xdxdff] (8.76,1.77) circle (2.5pt);
\draw [fill=xdxdff] (8.134528301886792,2.8958490566037733) circle (2.5pt);
\draw [fill=xdxdff] (10.36999926479598,3.790148511211861) circle (2.5pt);
\draw [fill=ududff] (10.38,1.77) circle (2.5pt);
\draw [fill=xdxdff] (3.8264466094067267,-1.199878066911803) circle (2.5pt);
\draw [fill=ududff] (2.6385072170133266,-0.011938674518402692) circle (2.5pt);
\draw [fill=xdxdff] (3.551459891609421,0.9136938983359697) circle (2.5pt);
\draw [fill=xdxdff] (5.601939561385962,-0.5546723179904587) circle (2.5pt);
\draw [fill=xdxdff] (5.16194451123264,-1.5814488960049211) circle (2.5pt);
\draw [fill=xdxdff] (5.6888024763961145,1.8814084921516754) circle (2.5pt);
\draw [fill=xdxdff] (6.19715889449669,3.0677137999207305) circle (2.5pt);
\draw [fill=ududff] (4.761837661840736,4.4888939366884495) circle (2.5pt);
\draw [fill=xdxdff] (3.638322806619573,3.3497747084781038) circle (2.5pt);
\draw [fill=ududff] (7.14,0.65) circle (2.5pt);
\draw [fill=xdxdff] (5.84,0.65) circle (2.5pt);
\draw [fill=xdxdff] (7.14,-1.69) circle (2.5pt);
\end{scriptsize}
\end{tikzpicture}


 \caption{El área del rectángulo es la suma de sus partes}\label{fig:rect_descop} 
\end{center}
\end{figure}

Utilizando la segunda y tercer propiedad se pueden relacionar el área del rectángulo de la figura \ref{fig:rect_descop} con las cuatro regiones en la que es dividido.

Como veremos a lo largo de la materia la propiedad de aditividad debe ser estudiada con cuidado, esto ocurre por las intrincadas maneras en que una región puede ser unión de otras regiones. A lo largo de esta materia elaboraremos una  teoría que nos dará una descripción  precisa de a que conjuntos podemos asignarle una medida de modo que las propiedades previas sean ciertas. 

Por el momento veamos como las propiedades anteriores determinan practicamente de manera unívoca la medida de regiones elementales planas.  


Hablando de propiedades de la medida, supongamos que $A$ y $B$ son dos regiones con $A\subset B$. Entonces como $B=A\cup (B-A)$ y por la propiedad de aditividad y positividad

\[
 m(B)=m(A)+m(B-A)\geq m(A).
\]

Descubrimos así que nuestra medida deberá tener adicionalmente la siguiente propiedad:
\begin{description}
 \item[Monotonía.] Si $A\subset B$ entonces $m(A)\leq m(B)$. 
\end{description}
\marginpar{ Podríamos por ejemplo elegir el círculo de radio uno como unidad de área. Así ya no tendríamos el problema de ese número raro $\pi$ que aparece en la fórmula del área del círculo. ¡El área de cualquier círculo sería igual a su radio al cuadrado! Claro que aparecería $\pi$  en la fórmula del área del cuadrado de lado 1. Nos tapamos los pies y se destapa el cuerpo.}
Es claro que si logramos construir una medida que satisfaga las propiedades anteriores cualquier multiplo por un número real positivo  de ella seguirá cumpliendo las propiedades. Esto es una manera de expresar el hecho que podemos usar diferentes unidades de medición. Esta cuestión se sortea proponiendo la unidad de medida. Esta unidad es completamente arbitraria, ud. podría elegir su figura plana preferida como unidad de área.   Cómo es habitual, elijamos el cuadrado cuyos lados miden la unidad de longitud previamente fijada. 


Supongamos ahora que tenemos un rectángulo de un lado igual a la unidad y el otro de lado un racional $n/m$, $n,m\in\mathbb{N}$. Veamos que la aditividad, la invariancia por movimientos rígidos y el hecho que decidimos que el cuadrado de lados igual a la unidad determinan el área de este rectángulo. Primero observar que si dividimos el lado de cuadrado unidad en $m$ segmentos iguales de longitud. \marginnote{
 
\definecolor{xdxdff}{rgb}{0.49,0.49,1}
\definecolor{zzttqq}{rgb}{0.6,0.2,0}
\definecolor{ududff}{rgb}{0.30,0.30,1}
\begin{tikzpicture}[x=2.1cm,y=2.1cm]
\clip(-0.07,0) rectangle (1.5,1.5);
\draw [line width=2pt,color=zzttqq] (0,0) -- (1,0) -- (1,1) -- (0,1) -- cycle;
\draw [line width=1pt,color=zzttqq, dashed] (0,.2)--(1,.2);
\draw [line width=1pt,color=zzttqq, dashed] (0,.4)--(1,.4);
\draw [line width=1pt,color=zzttqq, dashed] (0,.6)--(1,.6);
\draw [line width=1pt,color=zzttqq, dashed] (0,.8)--(1,.8);
\draw (0,1) node[anchor=south] {$Q$};
\draw (0.5,.1) node[anchor=center] {$R_1$};
\draw (0.5,.3) node[anchor=center] {$R_2$};
\draw (0.5,.5) node[anchor=center] {$R_3$};
\draw (0.5,.7) node[anchor=center] {$R_4$};
\draw (0.5,.9) node[anchor=center] {$R_5$};



\end{tikzpicture}
\\
 Descomposición rectángulo $R$
 }
Queda dividido el cuadrado en $m$ rectángulos $R_1,\ldots,R_m$ (ver figura en el margen), todos ellos  congruentes entre si, de modo que todos tienen la misma medida, digamos $m(R_1)$. La unión de ellos es el cuadrado que por convención dijimos que tiene medida 1. De modo que por la aditividad debe ocurrir que $m(R_1)=\cdots =m(R_m))=1/m$. Recordemos nuestra pretención de inferir la medida de un rectángulo $R$ de lado 1 y otro $n/m$. Este rectángulo esta compuesto de $n$ rectángulos congruentes a los $R_i$, $i=1,\ldots,m$, nuevamente por la aditividad inferimos que $m(R)=n/m$. 

Sea ahora una rectángulo $R$ con un lado unidad y el otro un real cualquiera $l>0$. Existen sendas sucesiones $0<q_k,p_k\in\mathbb{Q}$, $k\in\mathbb{N}$, tales que $q_1\leq q_2\leq\cdots \leq l \leq \cdots\leq p_2\leq p_1$ y $\lim_{k\to\infty}q_k =\lim_{k\to\infty} p_k=l$. Consideremos una dos sucesiones de rectángulos $R_k$ y $S_k$ que comparten el lado de $R$ igual a la unidad, mientras que el otro lado de $R_k$ y $S_k$ es igual a $q_k$ y $p_k$ respectivamente. Luego por la monotonía
\[
 q_k=m(R_k)\leq m(R) \leq m(S_k)\leq p_k.
\]
Tomando límite cuando $k\to\infty$ inferimos que $m(R)=l$. 



\begin{figure}[h]
 \begin{center}
 \input{imagenes/ParalTria.tikz} 
 \end{center}
 \caption{Áreas de otras figuras elementales.}\label{fig:paral-trig}
\end{figure}


A partir de las propiedades fundamentales que postulamos para la medida o área inferimos la famosa fórmula del área de un rectángulo en el caso que uno de los lados sea igual a la unidad. Para un  rectángulo arbitrario. En la figura \ref{fig:paral-trig} se muestra como relacionar el área de un paralelepípedo con la de un rectángulo y la de un triángulo con la de un paralelepípedo para inferir las conocidas fórmulas para estas figuras.



\section{Integral de Riemann}

En esta sección abordaremos el problema del área de regiones planas. Vamos a contextualizarnos dentro del marco conceptual que nos brinda la geometría analítica. Mediante coordenadas cartesianas ortogonales los puntos del plano se identifican con pares ordenados $(x,y)\in\mathbb{R}^2$ y el plano con el conjunto $\rr^2$.  Nuestro propósito es entonces definir la medida de subconjuntos de $\mathbb{R}^2$. La geometría analítica abre así nuevas posibilidades para abordar el problema del área. 

Nuestra primera aproximación será la que propuso Bernhard Riemann en 1854, pero seguiremos  el enfoque de Jean Darboux. En esta parte de nuestra exposición consideraremos subconjuntos de $\mathbb{R}^2$ de un tipo especial, concretamente a conjuntos que quedan encerrados entre la gráfica de una función y del eje coordenadas $x$. Esto nos lleva alconcepto de integral. 


\begin{definicion}[Partición]{} Sea $[a,b]$ un intervalo. Una {\em partición}\index{Partición} $P$ es un conjunto ordenado y finito de puntos, donde el primer elemento es $a$ y el último $b$. Es decir $P=\{x_0,x_1,\ldots,x_n\}$, donde $a=x_0<x_1<\cdots<x_n=b$. 
 
\end{definicion}




\begin{definicion}[Sumas de Darboux]{} Sea $f:[a,b]\to\mathbb{R}$ una función acotada y $P=\{x_0,x_1,\ldots,x_n\}$ una partición de $[a,b]$. Consideremos las siguientes magnitudes
\[
 \begin{split}
    m_i&:=\inf\{f(x)| x\in [x_{i-1},x_i]\}\\
    M_i&:=\sup\{f(x)| x\in [x_{i-1},x_i]\}\\
 \end{split}
\]

Definimos la \emph{Suma superior de Darboux}\index{Suma superior} como
\[
 \overline{S}(P,f)=\sum_{i=1}^nM_i(x_i-x_{i-1}),
\]
y la \emph{Suma inferior de Darboux}\index{Suma inferiorr} como
\[
 \underline{S}(P,f)=\sum_{i=1}^nm_i(x_i-x_{i-1}),
\] 
\end{definicion}
\marginpar{
  \begin{center}
    \begin{tikzpicture}[scale=0.45]
\begin{axis}[
    xtick={0,...,5},ytick={5,10,15,20,25},
    y=0.3cm, xmax=5.4,ymax=26.9,ymin=0,xmin=0,
    enlargelimits=true,
    axis lines=middle,
    clip=false,yticklabels=\empty,xticklabels=\empty,
    ]
\addplot+[color=red,fill=red!10!white,const plot, mark=none]
    coordinates {(0,2) (1,5) (2,10) (3,17) (4,26) (5,26)}\closedcycle;
\addplot+[color=green,fill=green!10!white,const plot, mark=none]
    coordinates {(0,1) (1,2) (2,5) (3,10) (4.0,17) (5,17)}\closedcycle;
\addplot[smooth, thick,domain=0:5]{1+x^2};
\addplot[const plot,domain=0:5,color=red] coordinates {(1,0) (1,2)};
\addplot[const plot,domain=0:5,color=red] coordinates {(2,0) (2,5)};
\addplot[const plot,domain=0:5,color=red] coordinates {(3,0) (3,10)};
\addplot[const plot,domain=0:5,color=red] coordinates {(4,0) (4,17)};
\addplot[const plot,domain=0:5,color=red] coordinates {(5,0) (5,26)};
\addplot[color=black] coordinates {(0,-0.8)} node {$a$};
\addplot[color=black] coordinates {(5,-0.8)} node {$b$};

\end{axis}
\end{tikzpicture}
    Sumas de Darboux. 
  \end{center}
}

\begin{lema}[Monotonía sumas de Darboux]{}  Sea $f:[a,b]\to\mathbb{R}$ una función acotada y $P=\{x_0,x_1,\ldots,x_n\}$ una partición de $[a,b]$. Supongamos que $P'$ es otra partición que tiene un pnto más que $P$. Entoces
 \[
  \underline{S}(P',f)\geq \underline{S}(P,f)\quad\text{y}\quad \overline{S}(P',f)\leq \overline{S}(P,f)
 \]
\end{lema}

\begin{ejercicio}{} Sea $f:[a,b]\to\mathbb{R}$ una función acotada y $P,P'$ particiones de $[a,b]$ con $P\subset P'$. Demostrar que 
  \[
  \underline{S}(P,f)\leq \underline{S}(P',f)\quad\text{y}\quad \overline{S}(P',f)\leq \overline{S}(P,f).
 \]
 Inferir que para cualesquiera $P,P'$ (sin importar que una este o no contenida dentro de la otra)
   \[
\underline{S}(P,f)\leq \overline{S}(P,f).
 \]
 
\end{ejercicio}

\begin{definicion}[Funciones integrables]{} Sea $f:[a,b]\to\mathbb{R}$ una función acotada. Diremos que $f$ es {\em integrable Riemann} \index{Integrable Riemann} si 
\begin{equation}\label{eq:integrable}
 \sup\left\{\underline{S}(P,f)| P \text{partición de }[a,b]\right\}=\inf\left\{\overline{S}(P,f)| P \text{partición de }[a,b]\right\}
\end{equation}
En caso que $f$ sea integrable llamamos {\em integral} \index{Integral} entre $a$ y $b$ de $f$ al valor de los dos miembros de \eqref{eq:integrable} y este número se denota
\[
\int_a^bf(x)dx. 
\]
 
\end{definicion}
 
\begin{teorema}[Primer criterio de integrabilidad]{}  Sea $f:[a,b]\to\mathbb{R}$ una función acotada. Entonces  $f$ sea integrable si para todo $\epsilon>0$ existe una partición $P$ tal que 
\begin{equation}\label{eq:Crit1Int}
 \overline{S}(P;f)-\underline{S}(P;f)<\epsilon.
\end{equation}
 
\end{teorema}

\begin{demo} La demo
 
\end{demo}


\begin{ejemplo}{} Sea $0\leq a<b$ veamos que 
\[
 \int_a^b x dx=\frac{b^2}{2}-\frac{a^2}{2}.
\]
\end{ejemplo}


\begin{ejercicio}{} Sea $0\leq a<b$ veamos que 
\[
 \int_a^b x^2 dx=\frac{b^3}{3}-\frac{a^3}{3}.
\]
{\em Ayuda:} Usar particiones uniformes y la fórmula $\sum_{i=1}^nn^2= n(n+1)(2n+1)/6$.
\end{ejercicio}

\begin{ejemplo}{} Sea $0\leq a<b$ y $n\in\mathbb{N}$,  veamos que 
\[
 \int_a^b x^n dx=\frac{b^{n+1}}{n+1}-\frac{a^{n+1}}{n+1}.
\]
Usamos particiones no uniformes
\end{ejemplo}

\begin{ejercicio}{} Sea $0\leq a<b$ y $n$ un entero negativo, veamos que
\[
 \int_a^b x^n dx=\begin{cases}
                  \frac{b^{n+1}}{n+1}-\frac{a^{n+1}}{n+1} & \text{ si } n\neq -1\\
                  \ln(b)-\ln(a) & \text{ si } n=-1\\
                 \end{cases}
\]
\end{ejercicio}


\begin{ejemplo}{} Sea $0\leq a<b\leq \pi/2$, veamos que
\[
 \int_a^b \sen x dx=-(\cos(b)-\cos(a)).
\]
\end{ejemplo}



\begin{ejercicio}{} Sea $0\leq a<b\leq \pi$, veamos que
\[
 \int_a^b \sen x dx=-(\cos(b)-\cos(a)).
\]
\end{ejercicio}


\begin{ejemplo}{} Usamos SymPy y sumas de Darboux aproximar el valor de $\pi$. Utilizamos el hecho que $\pi/4$ es el área de un cuarto de círculo de radio $1$. Entonces 
$$\pi=4\int_0^1\sqrt{1-x^2}dx.$$

\begin{sympyblock}
from sympy import *
N=1000.0
lim=int(N+1)
x=symbols('x')
f=sqrt(1-x**2)
Sinf=sum([ f.subs(x,i/N)*1/N for i in range(1,lim)])
Ssup=sum([f.subs(x,(i-1)/N)*1/N for i in range(1,lim)])
\end{sympyblock}
Encontramos la estimación
\[\sympy{4*Sinf}\leq \pi \leq \sympy{4*Ssup}\]
\end{ejemplo}

\begin{ejercicio}{} Usando SymPy estimar las siguientes integrales 
$$\int_1^2\frac{1}{x}dx,$$
 comparar con $\ln(2)$,

$$\int_{-1}^{1}x^2dx$$
¿A qué parece aproximarse las sumas inferiores y superiores?

$$\int_0^{\frac{1}{2}}\frac{1}{\sqrt{1-x^2}}dx,$$
¿Por qué el resultado puede usarse para aproximar $\pi$ ?

 
\end{ejercicio}




\begin{teorema}[Propiedades elementales de la integral]{} Sean $f,g:[a,b]\to\mathbb{R}$ integrables, $\alpha,\beta\in\mathbb{R}$ y $c\in (a,b)$. Entonces
\begin{description}
 \item[Linealidad] $\alpha f+\beta g$ es integrable y 
 \[
  \int_a^b\alpha f(x)+\beta g(x)dx=\alpha \int_a^bf(x)dx+\beta\int_a^b g(x)dx.
 \]
 \item[Monotonía] Si $f(x)\leq g(x)$  para $x\in [a,b]$ entonces 
 \[
  \int_a^b f(x)dx\leq \int_a^b g(x)dx.
 \]
 \item[Aditividad del Intervalo]  \[
  \int_a^b\alpha f(x)dx= \int_a^cf(x)dx+\int_c^b f(x)dx.
 \]
\end{description}


\end{teorema}

\begin{observa} Las propiedadades anteriores son compatibles con las propiedades que habíamos propuesto para el concepto de área en la sección  \ref{sec:area_elem}.
\end{observa}

\section{Integrabilidad y continuidad}


 
\begin{teorema}[Segundo criterio de integrabilidad]{}  Sea $f:[a,b]\to\mathbb{R}$ una función acotada. Entonces  $f$ sea integrable si para todo $\epsilon>0$ existe un $\delta>0$ tal que para cualquier partición $P$ que satisface
\[\max_i\{x_i-x_{i-1}\}<\delta,\]
se tiene que
\begin{equation}\label{eq:Crit1Int}
 \overline{S}(P;f)-\underline{S}(P;f)<\epsilon.
\end{equation}
 
\end{teorema}
\begin{demo} Agarrate catalina
 
\end{demo}

\begin{teorema}[Continuidad implica integrabilidad]{}  Si $f:[a,b]\to\mathbb{R}$ es una función continua entonces es integrable.
\end{teorema}
\begin{demo} hacer
 \end{demo}

 
¿Qué ocurre con las funciones discontinuas? 

\begin{ejemplo}[Función de Heavside]{} Es la función
\[
 H(x)=\begin{cases}0 & \text{ si } x<0\\1 & \text{ si } x\geq 0\end{cases}.
\]
Es discontinua en $[-1,1]$ pero integrable.
 
\end{ejemplo}



\marginnote{\adjustimage{max size={0.9\linewidth}{0.9\paperheight}}{imagenes/dirichlet.png}\\
Función de Dirichlet}
\begin{ejemplo}[Función de Dirichlet]{} Es la función $f:[0,1]\to\mathbb{R}$ definida por 
\[
 f(x)=\begin{cases} 1 & \text{ si } x\in\mathbb{Q}\\0 & \text{ si }   x\notin\mathbb{Q}\\
\end{cases}
\]
Veamos que $f$ es discontinua en todo punto y no integrable.
\end{ejemplo}





\begin{ejemplo}[Función de Thomae]{} Es la función $f:[0,1]\to\mathbb{R}$ definida por 
\[
 f(x)=\begin{cases} \frac{1}{q} & \text{ si } x=\frac{p}{q},p,q\in\mathbb{Z}, \text{m.c.d}(p,q)=1
 \\0 & \text{ si }   x\notin\mathbb{Q}\\
\end{cases}
\]
\end{ejemplo}
\marginnote{\adjustimage{max size={0.9\linewidth}{0.9\paperheight}}{imagenes/thomae.png}\\
Función de Thomae}

Para graficarla

\begin{sympyverbatim}
from matplotlib import pyplot as plt
total=500
q=[]
f=[]
for i in range(1,total):
    for j in range(1,i):
        if gcd(j,i)==1:
            q.append(Rational(j,i))
            f.append(1.0/i)
plt.plot(q,f,'.',markersize=12)
\end{sympyverbatim}

Veamos que es discontinua en todo punto racional y es integrable integrable.

\marginnote{\adjustimage{max size={0.9\linewidth}{0.9\paperheight}}{imagenes/escalera.png}\\
Función creciente y discontinua en $\mathbb{Q}\cap [0,1]$}
\begin{ejemplo}[Escalera discontinua]{} Sea $\mathbb{Q}\cap [0,1]=\{q_1,q_2,\ldots\}$ una numeración de los racionales del $[0,1]$. Definamos $f:[0,1]\to\mathbb{R}$ como
 \[
  f(x)=\sum_{n=1}^{\infty}H(x-q_n),
 \]
donde $H$ es la función de \emph{Heavside}\index{Heavside}. 
\begin{sympyverbatim}
q=[]
f=[]
total=20
for i in range(1,total):
    for j in range(1,i):
        if gcd(j,i)==1:
            q.append(float(Rational(j,i)))
x=symbols('x')
Heavside=Piecewise((0,x<0),(1,x>=0))
f=sum([Heavside.subs(x,x-q[n])/2**n for n in range(len(q))])
plot(f,(x,0,1))
\end{sympyverbatim}

Veamos que $f$ es monotona no decreciente y discontinua en todo punto de $[0,1]\cap \mathbb{Q}$. Además $f$ es integrable.


\end{ejemplo}
\begin{definicion}[Oscilación sobre un intervalo]{} Sea $f:[a,b]\to\mathbb{R}$ acotada y $I=[\alpha,\beta]\subset [a,b]$. Definimos la \emph{oscilación}\index{oscilación} de $f$ en $I$ por 
\[
 w(f,I)=\sup\{f(x)| x\in I\}-\inf\{f(x)| x\in I\}.
\]
\end{definicion}

\begin{ejemplo}{} 
\begin{enumerate}
 \item Para la función de Dirichlet $w(f,I)=1$ para todo $I$ con interior no vacío.
 \item Para la función de Heavside e $I=[\alpha,\beta]$
 \[
  w(f,I)=\begin{cases}
          1 & \text{ si } 0\in (\alpha,\beta]\\
          0 & \text{ si } 0\notin (\alpha,\beta]\\
         \end{cases}
 \]
  \item Si $I^o\neq\emptyset$, $f$ la función de Thomae e $I\subset [0,1]$ entonces 
     $ w(f,I)=1/q^*$, donde $q^*$ es el mínimo valor de $q$ para el que existe $p\leq q$ tal que $p/q\in I$.
  \item Para la función escalera discontinua e $I\subset [0,1]$
  \[
   w(f,I)=\sum_{q_n\in I}\frac{1}{2^n}.
  \]
\end{enumerate}

 
\end{ejemplo}



\begin{definicion}{} Sea $f:[a,b]\to\mathbb{R}$ acotada, $\sigma>0$ y $P=\{x_0,x_1,\ldots,x_n\}$ una partición. Definimos
\[
 I_{\sigma}:=\{i\in\{1,\ldots,n\}|w(f,[x_{i-1},x_i])>\sigma\}.
\]
y
\[
 R(P,f,\sigma)=\sum_{i\in I_{\sigma}}(x_i-x_{i-1}).
\]
\end{definicion}





\begin{proposicion} Si $f$ es continua en $[a,b]$ para todo $\sigma>0$ existe $\delta>0$ tal que 
\[
\max_i(x_i-x_{i-1})<\delta\Rightarrow I_{\sigma}=\emptyset\Rightarrow R(P,f,\sigma)=0. 
\]

 
\end{proposicion}

\begin{ejemplo}{} Para la función de Dirichlet y para todo $0<\sigma<1$ y para toda partición de $[0,1]$ tenemos $I_{\sigma}=\{x_1,\ldots,x_n\}$ y $R(P,f,\sigma)=[0,1]$
 
\end{ejemplo}

\begin{ejemplo}{} Para la función de Heavside,   para todo $0<\sigma<1$ y para toda partición de $[0,1]$ tenemos $I_{\sigma}=i$, donde $i$ es el índice para el que $i\in (x_{i-1},x_i]$ y $R(P,F,\sigma)=x_i-x_{i-1}$.
 
\end{ejemplo}

\begin{teorema}[Criterio de integrabilidad de Riemann]{} Sea $f$ acotada en $[a,b]$ entonces $f$ es integrable si y sólo si  para todo $\epsilon>0$ y $\sigma>0$ existe $\delta>0$ talque $R(P,f,\sigma)<\epsilon$.
\end{teorema}

\begin{ejemplo}{} Discutir los ejemplos Dirichlet, Heavside, Continuas, escalera discontinua
 
\end{ejemplo}

\begin{ejemplo}[Función de Riemann]{} Definimos
\[
 ((x))=x-[x+0.5]
\]

\marginnote{\adjustimage{max size={0.9\linewidth}{0.9\paperheight}}{imagenes/serrucho.pdf}\\
Función serrucho}
\begin{sympyverbatim}
x=symbols('x')
g=x-floor(x+.5)
plot(g,(x,-5,5))
\end{sympyverbatim}
Definimos la función de Riemann Porque
\[
 f(x)=\sum_{n=1}^{\infty}\frac{((x))}{n^2}.
\]

\begin{sympyverbatim}
f=sum([g.subs(x,n*x)/n**2 for n in range(1,20)])
plot(f,(x,0,1))
\end{sympyverbatim}
\marginnote{\adjustimage{max size={0.9\linewidth}{0.9\paperheight}}{imagenes/funcRiemann.png}\\
Función de Riemann}
 
Demostramos que la función de Riemann es discontinua en los racionales $p/q$ donde $\text{m.c.d}(p,q)=1$ y $q$ par. Es integrable en $[0,1]$.  
 
\end{ejemplo}


\begin{definicion}[Oscilación de una función en un punto]{}\index{oscilación} 
Sea $f:[a,b]\to\mathbb{R}$ y $x\in [a,b]$ 
acotada definimos la {\em oscilación de $f$ en $x$} como 
\[
 w(f;x)=\inf\limits_{x\in I^o}w(f,I),
\]
donde el ínfimo se toma sobre todos los intervalos que contienen a $x$ en su interior.
\end{definicion}

\begin{ejercicio}{} $f$ es continua en $x$ si y solo si $w(f;x)=0$.
 \end{ejercicio}

 \begin{definicion}[Contenido exterior]{}\index{Contenido exterior} Sea $S\subset\mathbb{R}$. Un \emph{cubrimiento finito}\index{cubrimiento finito} de $S$ es una colección de intervalos $\left\{ [x_{i-1},x_i]\right\}_{i=1,\ldots,n}$ tal que $S\subset \cup_{i=1}^n[x_{i-1},x_i]$.
 
 El \emph{contenido exterior} de $S$ se define por 
 \[
  c_e(S)=\inf \sum_{i=1}^n (x_i-x_{i-1}),
 \]
donde el ínfimo es tomado sobre todos los cubrimientos finitos de $S$.
  
 \end{definicion}
 
 \begin{teorema}[Criterio de integrabilidad de Hankel]{}  Sea $f$ acotada en $[a,b]$ entonces $f$ es integrable si y sólo si para todo $\sigma>0$ el conjunto $S_{\sigma}:=\{x\in [a,b]| w(f,x)>\sigma\}$ tiene contenido exterior igual a $0$ ($c_e(S_{\sigma})=0)$.
   \end{teorema}


\section{Teorema Fundamental de Cálculo}


\section{Función de Volterra}

\begin{pyblock}
import numpy as np
import scipy.optimize
from matplotlib import pyplot as plt
\end{pyblock}

Consideramos  la función $f(x)=x^2\sen(1/x)$. 

\begin{pyblock}
def G(x):
    return x**2*np.sin(1/x)
x=np.arange(0,.15,0.0000001)
y=G(x)
plt.plot(x,y)
\end{pyblock}


 \begin{figure}[h]
 \begin{center}
\adjustimage{max size={0.9\linewidth}{0.9\paperheight}}{imagenes/VolterraPrecursora.png}
 \caption{Función precursora de Volterra}
\end{center}
 \end{figure}





\begin{pyblock}
def F(x):
    return 2*x*np.sin(1/x) - np.cos(1/x)
x = scipy.optimize.broyden2(F, .13, f_tol=1e-14)
x,1-x, G(x)
\end{pyblock}

Se alcanza un máximo en $x=\py{np.float32(x)}$ y toma el valor $G(x)=\py{np.float32(G(x))}$. Hay que utilizar el punto simétrico a $x$, es decir $1-x=\py{np.float32(1-x)}$.


Definimos la función ``madre''.



\begin{pyblock}
def f0(x):
    x1=x[x<=0]
    x2=x[(x<=0.13163877)*(x>0)]
    x3=x[(x>0.13163877)*(x<0.868361226)]
    x4=x[(x>=0.868361226)*(x<1)]
    x5=x[x>=1]
    y1=np.zeros(np.shape(x1))
    y2=x2**2*np.sin(1/x2)
    y3=0.01675771541054875*np.ones(np.shape(x3))
    y4=(1-x4)**2*np.sin(1/(1-x4))
    y5=np.zeros(np.shape(x5))
    return np.concatenate((y1,y2,y3,y4,y5), axis=None)
\end{pyblock}

Definimos la función de Volterra
\begin{pyverbatim}
def volterra(x,n,a=0,b=1):
    if n == 0:
        return 0
    
    a1,b1 = 2.*a/3. + b/3., a/3. + 2.*b/3.
    pto_med = .5*(a+b)
    return volterra(x,n-1,a,a1) + (b1-a1)*f0((x-a1)/(b1-a1))\
    + volterra(x,n-1,b1,b)
\end{pyverbatim}

Graficamos

\begin{pyverbatim}
x=np.arange(0,1,0.0000001)
y=volterra(x,12)
plt.plot(x,y)
\end{pyverbatim}


 \begin{figure}[h]
 \begin{center}
\adjustimage{max size={0.9\linewidth}{0.9\paperheight}}{imagenes/volterra.png}
 \caption{Función de Volterra}
\end{center}
 \end{figure}


\section{Integral de Riemann y pasos al límite}


\chapter{Medida de Lebesgue en $\mathbb{R}$}

\section{Longitud de intervalos}

En el present


\definecolor{cffff00}{RGB}{255,255,0}


\begin{tikzpicture}[y=0.80pt, x=0.80pt, yscale=-1.000000, xscale=1.000000, inner sep=0pt, outer sep=0pt]
  \shade[left color=cffff00,right color=white,rounded corners=0.0000cm] (163.5714,519.5051) rectangle
    (405.0000,820.9336);


    
\end{tikzpicture}


swdasd


\chapter*{Apéndice}

\section{Topología}

\begin{teorema}[Principio de encaje de intervalos\index{Intervalos encajados}]{}  Sea $I_n=[a_n,b_n]\subset\mathbb{R}$ una sucessión de intervalos con las siguientes propiedades
\begin{enumerate}
 \item $\forall n\in\mathbb{N}: I_n\subset I_{n+1},$
 \item $\lim\limits_{n\to\infty}(b_n-a_n)=0.$
\end{enumerate}
Entonces $\bigcap_{n=1}^{\infty}I_n$ consiste de uno, y solo un, punto $x\in\mathbb{R}$.
 
\end{teorema}

\begin{proof}
 
\end{proof}





\begin{teorema}[Heine-Borel]{}\index[personas]{Heine}\index[personas]{Borel} Toda sucesión acotada de $\mathbb{R}$ 
tiene una subsucesión convergente.
 
\end{teorema}

\begin{proof} Uso encajes de intervalos.
 
\end{proof}


\begin{definicion}[Continuidad uniforme \index{Continuidad uniforme}]{} Sea $f:A\subset\mathbb{R}\to\mathbb{R}$ una función. Diremos que $f$ es uniformemente continua si 
\[
 \forall \epsilon>0\exists \delta>0 \forall x,y\in A:|x-y|<\delta\Rightarrow |f(x)-f(y)|<\epsilon.
 \]

\end{definicion}

\begin{ejemplo} Varios ilustrando diferencia con continuidad
 
\end{ejemplo}


\begin{teorema}{} Sea $f:[a,b]\to\mathbb{R}$ continua. Entonces $f$ es uniformemente continua. 
 \end{teorema}
\begin{proof} Uso Heine-Borel
 \end{proof}


\printindex
\printindex[personas]
\printindex[simbolos]
\end{document}
