\documentclass[oneside]{book}

%%%%%%%%%%%%%%%%%%%%%%%%%%%%%%Paquetes%%%%%%%%%%%%%%%%%%%%%%%%%%%%%%%%%%%%%%%%%%%%%%%5
%%%%%%%%%%%%%%%%%%%%%%%%%%%%%%%%%%%%%%%%%%%%%%%%%%%%%%%%%%%%%%%%%%%%%%%%%%%%%%%%%%%%%
\usepackage{empheq}
\usepackage[spanish]{babel}
\usepackage{amssymb,amsmath,amsthm}
\usepackage{enumerate}
\usepackage{verbatim}
\usepackage{ esint }
\usepackage{array}
%\usepackage{listings}
\usepackage{ wasysym }
\usepackage{hyperref}
\usepackage{color}
\usepackage{url}
\usepackage{theorem}
\usepackage{boiboites}
\usepackage[spanish]{varioref}
 \usepackage{fontspec}
\usepackage[a4paper,driver=xetex,top=2.5cm, bottom=2.7cm,%
layouthoffset=10mm, left=1.5cm, right=6.2cm,marginparwidth=3.5cm,showframe]{geometry}
\usepackage{fancyhdr}
\usepackage{marginnote}
\usepackage{titlesec}
\usepackage{natbib}
\usepackage{chapterbib}
\usepackage{fontawesome}
\usepackage{xcolor}
\usepackage{appendix}
\usepackage{chngcntr}
\usepackage{etoolbox}
\usepackage{lipsum}
\usepackage{pgf,tikz,pgfplots}
\pgfplotsset{compat=1.15}
\usepackage{mathrsfs}
\usetikzlibrary{arrows}
\usepackage{imakeidx}
%\usepackage{sympytex}
\usepackage{pythontex}

\defaultfontfeatures{Ligatures=TeX}


%%%%%%%%%%%%Configuración Página
\fancyfoot{}
\fancyhead[RO,LE]{\thepage}
\fancyhead[LO]{\leftmark}
\fancyhead[RE]{\rightmark}


\titleformat{\section}
  {\normalfont\Large\bfseries}{\thesection}{1em}{}[{\titlerule[0.8pt]}]

\AtBeginEnvironment{subappendices}{%
\chapter*{Apéndices}
\addcontentsline{toc}{chapter}{Apéndices}
\counterwithin{figure}{section}
\counterwithin{table}{section}
}



%%%%%%%%%%%%Define colores%%%%%%%%%%%%%%%%%%%%



\definecolor{color1}{rgb}{0.48,0.89,0.84}
\definecolor{color2}{rgb}{0.48,0.89,0.84}
\definecolor{color3}{rgb}{0.28, 0.51, .68}
\definecolor{color4}{rgb}{0.29,0.3,0.57}
\definecolor{color5}{rgb}{0.7,0.24,0.24}
\definecolor{color6}{rgb}{0.72,0.4,0.28}
\definecolor{color7}{rgb}{0.84,0.66,0.21}
\definecolor{color8}{HTML}{8E87C1}

 
% 
% 
% 
% 
% 
% %%%%%%%%%%%%%Configuración de fuente para XeLaTeX
% %\setromanfont[Mapping=tex-text]{Oswald-Light}
% %\setsansfont{Roboto Condensed}
% %\setsansfont{Gentium Basic}
% %\setsansfont{FreeMono}
 \setsansfont{Oswald-Light}
% 
% 
 \renewcommand{\familydefault}{\sfdefault}
% 
% 
% 
% 
% %%%%%%%%%%%%%%%%%%%%%%%%%%Nuevos comandos entornos%%%%%%%%%%%%%%%%%%%%%%%%%%%%%%%%
% %%%%%%%%%%%%%%%%%%%%%%%%%%%%%%%%%%%%%%%%%%%%%%%%%%%%%%%%%%%%%%%%%%%%%%%%
\newenvironment{demo}{\noindent\emph{Dem.}}{$\square$ \newline\vspace{5pt}}

\newenvironment{observa}{\noindent\textbf{Observación:}}{}

\newcommand{\com}{\mathbb{C}}
\newcommand{\rr}{\mathbb{R}}
\renewcommand{\epsilon}{\varepsilon}
\renewcommand{\lim}{\mathop{\rm lím}}
\renewcommand{\inf}{\mathop{\rm ínf}}
\renewcommand{\liminf}{\mathop{\rm líminf}}
\renewcommand{\limsup}{\mathop{\rm límsup}}
\renewcommand{\min}{\mathop{\rm mín}}
\renewcommand{\max}{\mathop{\rm máx}}
\renewcommand{\b}[1]{\boldsymbol{#1}}
\renewenvironment{frame}[1]{}{}
\newcommand{\link}{\reversemarginpar\marginnote{%\fontsize{24}{24                                                                                                                                                                                                                                                                                                                                                                                                                                                                                                                                                                                                                                                                                                                                                                                                                                                                                                                                                                       
%}\selectfont
\faExternalLink} %
\normalmarginpar
}
\newcommand{\advertencia}{\reversemarginpar\marginnote{\fontsize{16}{16}\selectfont
\faBolt}%
\normalmarginpar }
\newcommand{\lectura}{\reversemarginpar\marginnote{\fontsize{16}{16}\selectfont
\faBook }%
\normalmarginpar}
\newcommand{\actividad}{\reversemarginpar\marginnote{\fontsize{16}{16}\selectfont
\faCogs}%
\normalmarginpar}

%\renewcommand{\lim}{displaystyle\lim}
\DeclareMathOperator{\atan2}{atan2}
\DeclareMathOperator{\sen}{sen}



%%%%%%%%%%Definimos una caja con color
\newlength\mytemplen
\newsavebox\mytempbox
\makeatletter
\newcommand\mybluebox{%
\@ifnextchar[%]
{\@mybluebox}%
{\@mybluebox[0pt]}}
\def\@mybluebox[#1]{%
\@ifnextchar[%]
{\@@mybluebox[#1]}%
{\@@mybluebox[#1][0pt]}}
\def\@@mybluebox[#1][#2]#3{
\sbox\mytempbox{#3}%
\mytemplen\ht\mytempbox
\advance\mytemplen #1\relax
\ht\mytempbox\mytemplen
\mytemplen\dp\mytempbox
\advance\mytemplen #2\relax
\dp\mytempbox\mytemplen
\colorbox{color1}{\hspace{1em}\usebox{\mytempbox}\hspace{1em}}}
\makeatother
\DeclareDocumentCommand\boxedeq{ m g }{%
{\begin{empheq}[box={\mybluebox[2pt][2pt]}]{equation}% #1%
\IfNoValueF {#2} {\label{#2}}%
#1
\end{empheq}
}%
}


%%%%%%%%%%%%%%%%%%
\newboxedtheorem[boxcolor=color1, background=color2, titlebackground=color1,
titleboxcolor = black,thcounter=section]{problema}{Problema}{thcounter1}

\newboxedtheorem[boxcolor=color1, background=color2, titlebackground=color1,
titleboxcolor = black,thcounter=section]{teorema}{Teorema}{thcounter2}

\newboxedtheorem[boxcolor=color1, background=color2, titlebackground=color1,
titleboxcolor = black,thcounter=section]{definicion}{Definici\'on}{thcounter3}

\newboxedtheorem[boxcolor=color1, background=color2, titlebackground=color1,
titleboxcolor = black,thcounter=section]{lema}{Lema}{thcounter4}

\newboxedtheorem[boxcolor=color1, background=color2, titlebackground=color1,
titleboxcolor = black,thcounter=section]{corolario}{Corolario}{thcounter5}

\newboxedtheorem[boxcolor=color1, background=color2, titlebackground=color1,
titleboxcolor = black,thcounter=section]{proposicion}{Proposici\'on}{thcounter6}

\newboxedtheorem[boxcolor=color1, background=color2, titlebackground=color1,
titleboxcolor = black,thcounter=section]{codigo}{Función SymPy}{}

\newboxedtheorem[boxcolor=color1, background=color2, titlebackground=color1,
titleboxcolor = black,thcounter=section]{ejercicio}{Ejercicio}{}





\newcounter{ejemplo_cont}[chapter]
\setcounter{ejemplo_cont}{1}

\newenvironment{ejemplo}{\noindent\textbf{Ejemplo  \arabic{chapter}.\arabic{ejemplo_cont}.} }{\addtocounter{ejemplo_cont}{1}}


\renewcommand*{\raggedrightmarginnote}{\centering}

% 
% 
% %%%%%%%%%%%%%%%%%%%%%%%%%%%%%%%%%%%%%%%%%%%%%%%%%%%%%%%%%%%%%%%%%%%%%%%%%%%%%%%%%%%%%%%%%%%%%%%%%%%%%%%%%%%
% %%%%%%%%%%Para escibir en clase articulo o similar
% 








\title{Análisis Real}

\author{Sonia Acinas y Fernando Mazzone}

%%%%%%%%%%%%%%%%%%%%%%%%%%%%%%%%%%%%%%%%%%%%%%%%%%%%%%%%%%%%%%%%%%%%%%%%%%%%%%%%%%%%%%

\makeindex


\begin{document}



\fontsize{11pt}{11pt}\selectfont

\pagestyle{fancy}
 
%  
 \maketitle
 \tableofcontents
%   


\chapter*{Prólogo}

















%
%
%  \bibliographystyle{apalike-url}
%  \bibliography{diferenciales_ecuaciones,diferenciales_ecuaciones_sim}
 
 




\chapter{Breve introducci\'on a Python y SymPy}




\section{Descripción}
\href{https://www.python.org/}{Python} es un lenguaje de programación interpretado, 
abierto, facil de aprender, potente y portátil. Es utilizado en proyectos de todo tipo, 
no sólo aplicaciones científicas.
\marginnote{
%\begin{tabular}{b{.7in} b{.7in}}
 \includegraphics[scale=.25]{imagenes/python-logo.png}
% &\begin{pspicture}(.7in,.7in)
%        \psbarcode{https://www.python.org/}{}{qrcode}
%    \end{pspicture}\\
%    &
% {\tiny https://www.python.org/}
% \end{tabular}
}



\href{http://www.scipy.org/}{SciPy}, 
Python científico, es un conjunto de módulos de python para distintos tipos de cálculos. 
Está integrado por los módulos, SymPy (para cálculos simbólicos), 
numpy (cálculos numéricos), matplotlib (gráficos) entre otros.  
En este curso sólo usaremos SymPy.
\marginnote{
%\begin{tabular}{b{.7in} b{.7in}}
 \includegraphics[scale=.5]{imagenes/scipy_logo.png}
%&\begin{pspicture}(.7in,.7in)
%       \psbarcode{http://www.scipy.org/}{}{qrcode}
%   \end{pspicture}\\
%   &
%{\tiny http://www.scipy.org/}
%\end{tabular}
}


\href{http://www.sympy.org/}{SymPy}
es una biblioteca de Python para matemática simbólica. Su objetivo es convertirse en 
un sistema de álgebra computacional (SAC) completo, manteniendo el código lo más simple 
posible para que sea comprensible y fácilmente extensible. SymPy está escrito enteramente 
en Python y no requiere de ninguna biblioteca externa.
\marginnote{
 % \begin{tabular}{b{.7in} b{.7in}}
    \includegraphics[scale=.2]{imagenes/sympy_logo.png}
% 	&
% 	  \begin{pspicture}(.7in,.7in)
% 	    \psbarcode{http://www.sympy.org/}{}{qrcode}
% 	  \end{pspicture}
% \\
% 	&
% 	  {\tiny http://www.sympy.org/}
%   \end{tabular}
}
% 
% 



\href{http://matplotlib.org/}{Matplotlib} es una biblioteca de trazado de gráficos de Python que produce figuras de calidad de publicación en una variedad de formatos impresos y entornos interactivos a través de plataformas. Matplotlib se puede utilizar en scripts Python, en el shell Python e IPython, el portátil jupyter, servidores de aplicaciones web.
\marginnote{
  \includegraphics[scale=.1]{imagenes/matplotlib.jpg}
}






% 
% \section{Local y online} 
% 
% Se pueden usar todos los recursos anteriores de dos formas
% \begin{enumerate}
% \item Instalando el software necesario en una computadora. Nos referiremos a este modo como de acceso local.
% 
% \item A traves de trasacciones en línea que permiten usar una computadora remota que ejecuta las instrucciones y programas que se tipean en una página web con la que se interactúa usualmente por medio de un navegador.  Hay varios sitios que ofrecen este servicio. 
%Sugerimos la \href{https://cloud.sagemath.com/}{SageMathCloud}. El usuario debe registrase.
% \end{enumerate}
% 


\section{Instalación}


Son muchas las componentes requeridas para poder ejecutar los programas con los que trabajaremos 
en esta asignatura. Hay que instalar un interprete de python, los módulos que utilizaremos 
(sympy, matplotlib), es útil utilizar entornos integrados de desarrollo (IDE), que facilitan al usuario
editores de código fuente (especializados con la sintáxis de python), consolas de comandos
mejoradas (ipython, qt, etc). Otro recurso que se dispone son las notebooks, de las cuales 
hablaremos más adelante. Sería engorroso instalar todas estas componentes, que muchas veces 
tienen orígenes en desarrolladores diferentes, de manera independiente. Para nuestra fortuna
existen, las así llamadas, \emph{distribuciones}. Estas en algunos casos son archivos ejecutables que instalan todas 
las componentes necesarias, o al menos muchas  de ellas, de un determinada aplicación.
Recomendamos las siguientes distribuciones.  

\subsection{\href{https://www.continuum.io/downloads}{Anaconda}} 
La versión de código abierto de Anaconda es una distribución de alto rendimiento de Python y R 
e incluye más de 100 de los paquetes científicos más populares asociados a estos lenguajes.
%\reversemarginpar\marginpar{\includegraphics[scale=.12]{imagenes/library.png} } 
Además, se puede acceder a más de 720 paquetes que pueden ser fácilmente instalados con Conda, 
 un programa incluído en Anaconda para la gestión de paquetes.
 Anaconda tiene licencia BSD que da permiso para utilizar Anaconda comercialmente 
 y para su redistribución. Al día que se escriben estas líneas, anaconda parece la opción más 
 sencilla y completa para instalar todos los recursos necesarios para desarrollar los contenidos de 
 estas notas. Existen versiones para linux, OS X y Windows. 
\marginnote{
    \includegraphics[scale=.3]{imagenes/anaconda.png}
}



\subsection{Windows} Hay distribuciones específicas para distintos sistemas operativos. La distribución  \href{https://code.google.com/p/pythonxy/}{python(x,y)}  instala el interprete de python y todos los módulos de scipy. Además el entorno de desarrollo integrado (IDE) spyder.
\marginnote{\includegraphics[scale=.07]{imagenes/windows-logo.png}}


\subsection{linux} Aquí todo es más sencillo, el interprete de python suele venir con 
la distribución del SO y se pueden instalar los módulos, SymPy, NumPy, etc, 
recurriendo al administrador de paquetes o tipeando la sentencia adecuada en la línea 
de comandos.  
\marginnote{\includegraphics[scale=.4]{imagenes/linux.jpeg}}

\subsection{Android} \href{http://qpython.com/}{Qpython} es una aplicación que permite ejecutar código python y una versión básica de sympy desde tablets y smartphones. Se descarga desde la plataforma \href{https://play.google.com/store/apps/details?id=com.hipipal.qpyplus}{google play}.
\marginnote{\includegraphics[scale=.1]{imagenes/android.jpg}}. 

\subsection{\href{https://es.wikipedia.org/wiki/Computación_en_la_nube}{Computación en la nube}}

En los últimos tiempos se ha popularizado el uso de la computación en la nube. Esto se trata de servidores que algunas empresas o asociaciones sin fines de lucro facilitan en la web para ejecutar programas en diversos lenguajes. Citamos como ejemplo \href{http://www.cocalc.com}{cocalc}. Una vez registrado en el sitio se pueden subir o crear notebooks de jupyter. Soporta varios lenguajes, incluído Python y sus librerías.  Como uno utiliza los recursos instalados en el servidor, no se necesita tener instalado ningún interprete de los lenguajes. Sólo se necesita un navegador web actualizado. De este modo pueden ejecutarse programas desde un smartphone o tablet. 



\section{Forma de trabajo: por medio de scripts e interactiva}


Se puede trabajar de tres formas

\begin{enumerate}
\item Interactivamente, ingresando sentencias, de a una por vez, en la línea de comandos y obteniendo respuestas. Se requiere una consola.

\item Haciendo un script (programa) donde se guardan todas las sentencias que se desea ejecutar. Posteriormente este script se puede ejecutar, ya sea desde la línea de comandos o desde un IDE (spyder) oprimiendo un botón de ejecución.

\item En una notebook. Se hacen celdas que contienen porciones de código que pueden ejecutarse.

\end{enumerate}





\section{Características sobresalientes del lenguaje}

Seguiremos en esta exposición a \cite{wiki_python} de manera cercana. \link
Las principales características del lenguaje son:

\normalmarginpar
\begin{itemize}
\item Interpretado. Es necesario un conjunto de programas, 
el interprete, que entienda el código python y ejecute las acciones contenidas en él.
\item Implementa  tipos dinámicos.
\item  Multiparadigma, ya que soporta orientación a objetos, programación imperativa y, en menor medida, programación funcional.
\item Multiplataforma.

\item Es comprendido  con facilidad. Usa  palabras donde otros lenguajes utilizarían símbolos. Por ejemplo, los operadores lógicos \verb~!, || y \&\&~ en Python se escriben not, or y and, respectivamente.


\item  El contenido de los bloques de código (bucles, funciones, clases, etc.) es delimitado mediante espacios o tabuladores.

\item Empieza a contar desde cero (elementos en listas, vectores, etc).



\end{itemize}




\section{Elementos del Lenguaje}

\subsection{Comentarios}

Hay dos formas de producir comentarios, texto que el interprete  no ejecuta y que sirve para entender un programa.

Para comentarios largos se utilizan las tildes: \linebreak\verb~''' comentario '''~ .


 La segunda notación utiliza el símbolo \verb~#~, no necesita símbolo de finalización 
 pues se extiende hasta el final de la línea.

 \begin{pyverbatim}
'''
Comentario  largo en un script de Python
'''
print "Hola mundo" # Comentario corto
\end{pyverbatim}




El intérprete no tiene en cuenta los comentarios, lo cual es útil si deseamos poner información adicional en nuestro código como, por ejemplo, una explicación sobre el comportamiento de una sección del programa.








\subsection{Variables}
Las variables se definen de forma dinámica, lo que significa que no se tiene que especificar cuál es su tipo de antemano y que una variable puede tomar distintos valores en distintos momentos de un programa, incluso puede tomar
 un tipo diferente al que tenía previamente. \emph{Se usa el símbolo = para asignar valores a variables.}
\advertencia Es importante distinguir este = (de asignación) con el igual que es utilizado para definir igualdades en sympy, para ecuaciones por ejemplo.



\begin{pyverbatim}
x = 1
x = "texto" 
\end{pyverbatim}
Esto es posible porque los tipos son asignados
dinamicamente

\subsection{Tipo de datos}

Python implementa diferentes tipos de datos. Para la noción de \emph{tipos de datos} en 
general 
ver \cite{wiki:tipo_dato}\link. A continuación describimos sumariamente algunos de los tipos 
más comunes presentes en Python. Cuando se utilizan módulos específicos (p. ej. sympy) la 
diversidad  de tipos de datos se expande, con la incorporación de tipos con significación
matemática, p.ej. matrices,  expresiones algebraicas, etc. 


\includegraphics[scale=.4]{imagenes/tipo_datos.jpg}

Se clasifican en:
\begin{description}
 \item[Mutable] si su contenido puede cambiarse.
 \item[Inmutable] si su contenido no puede cambiarse.
\end{description}

Se usa el comando \verb~\type~ para averiguar que tipo de dato contiene una variable

 \begin{pyconsole}
x=1
type(x)
x='Ecuaciones'
type(x)
\end{pyconsole}





\subsection{Listas y tuplas}


\begin{itemize}

\item Es una estructura de dato, que contiene, como su nombre lo indica, listas de otros datos en cierto orden. Listas y tuplas son muy similares.

\item Para declarar una lista se usan los corchetes [], en cambio, para declarar una tupla se usan los paréntesis (). En ambos casos los elementos se separan por comas, y en el caso de las tuplas es necesario que tengan como mínimo una coma.

\item    Tanto las listas como las tuplas pueden contener elementos de diferentes tipos. No obstante las listas suelen usarse para elementos del mismo tipo en cantidad variable mientras que las tuplas se reservan para elementos distintos en cantidad fija.
    
\item Para acceder a los elementos de una lista o tupla se utiliza un índice entero (empezando por "0", no por "1"). Se pueden utilizar índices negativos para acceder elementos a partir del final.


\item Las listas se caracterizan por ser mutables, mientras que las tuplas son inmutables.

\end{itemize}






\begin{pyconsole}
lista = ["abc", 42, 3.1415]
lista[0] # Acceder a un elemento por su indice
lista[-1] # Acceder a un elemento usando un indice negativo
lista.append(True) # Agregar un elemento al final de la lista
lista
del lista[3] # Borra un elemento de la lista usando un indice
lista[0] = "xyz" # Re-asignar el valor del primer elemento
lista[0:2]#elementos del indice "0" al "1" 
lista_anidada = [lista, [True, 42L]] #Es posible anidar listas
lista_anidada
lista_anidada[1][0] #accede lista dentro de otra lista
\end{pyconsole}

\begin{pyconsole}
tupla = ("abc", 42, 3.1415)
tupla[0] # Acceder a un elemento por su indice
del tupla[0] # No es posible borrar ni agregar
tupla[0] = "xyz" # Tampoco es posible re-asignar
tupla[0:2] # elementos del indice "0" al "2" sin incluir
tupla_anidada = (tupla, (True, 3.1415)) # es posible anidar
1, 2, 3, "abc" # Esto tambien es una tupla
(1) #  no es una tupla, ya que no posee al menos una coma
(1,) # si es una tupla
(1, 2) # Con mas de un elemento no es necesaria la coma final
(1, 2,) # Aunque agregarla no modifica el resultado
\end{pyconsole}

\subsection{Diccionarios}
\begin{itemize}
\item    Para declarar un diccionario se usan las llaves \verb~{}~. Contienen elementos separados por comas, donde cada elemento está formado por un par clave:valor (el símbolo : separa la clave de su valor correspondiente).
 \item   Los diccionarios son mutables, es decir, se puede cambiar el contenido de un valor en tiempo de ejecución.
\item    En cambio, las claves de un diccionario deben ser inmutables. Esto quiere decir, por ejemplo, que no podremos usar ni listas ni diccionarios como claves.
\item    El valor asociado a una clave puede ser de cualquier tipo de dato, incluso un diccionario.

\end{itemize}






\begin{pyconsole}
dicci = {"cadena": "abc", "numero": 42, "lista": [True, 42L]}
dicci["cadena"] # Usando una clave, se accede a su valor
dicci["lista"][0]
dicci["cadena"] = "xyz" # Re-asignar el valor de una clave
dicci["cadena"]
dicci["decimal"] = 3.1415927 # nuevo elemento clave:valor
dicci["decimal"]
dicci_mixto = {"tupla": (True, 3.1415), "diccionario": dicci}
dicci_mixto["diccionario"]["lista"][1]
dicci = {("abc",): 42} # tupla puede ser clave 
dicci = {["abc"]: 42} # una clave no puede ser lista
\end{pyconsole}



\subsection{Listas por comprensión}
Una lista por comprensión es una expresión compacta para definir listas. Al igual que el operador lambda, aparece en lenguajes funcionales. Ejemplos:

\pyv{range(n)} devuelve una lista, empezando en 0 y terminando en $n-1$.

\begin{pyconsole}
range(5) #  
[i*i for i in range(5)]
lista = [(i, i + 2) for i in range(5)]
lista
\end{pyconsole}



\subsection{Funciones}
\begin{itemize}

  \item  Las funciones se definen con la palabra clave \verb~def~, seguida del nombre de
  la función y sus parámetros. 
  Otra forma de escribir funciones, aunque menos utilizada, es con la palabra clave \verb~lambda~ (que aparece en lenguajes funcionales como Lisp). Generalemente esta forma es apropiada para funciones que es posible definir en una sola línea.
  
  
  \item  El valor devuelto en las funciones con \verb~def~ será el dado con la instrucción \verb~return~.
  \end{itemize}


\begin{pyconsole}
def suma(x, y = 2): #el argumento y tiene un valor por defecto
    return x + y # Retornar la suma

suma(4) # La variable "y" no se modifica, siendo su valor: 2
suma(4, 10) # La variable "y" si se modifica
\end{pyconsole}


\begin{pyconsole}
suma = lambda x, y = 2: x + y
suma(4) # La variable "y" no se modifica
suma(4, 10) # La variable "y" si se modifica
\end{pyconsole}

\subsection{Condicionales}
 Una sentencia condicional (\pyv{if} \verb~condicion~) ejecuta su bloque de código interno 
 sólo si \verb~condicion~ tiene el valor booleano \pyv{True}.  Condiciones adicionales, si las hay, se introducen usando \pyv{elif} seguida de la condición y su bloque de código. Todas las condiciones se evalúan secuencialmente hasta encontrar la primera que sea verdadera, y su bloque de código asociado es el único que se ejecuta. Opcionalmente, puede haber un bloque final (la palabra clave \pyv{else} seguida de un bloque de código) que se ejecuta sólo cuando todas las condiciones fueron falsas.



\begin{pyconsole}
verdadero = True
if verdadero: # No es necesario poner "verdadero == True"
    print "Verdadero"

else:
    print "Falso"

lenguaje = "Python"
if lenguaje == "C": 
    print "Lenguaje de programacion: C"
elif lenguaje == "Python": # tantos "elif" como se quiera
    print "Lenguaje de programacion: Python"
else: 
    print "Lenguaje de programacion: indefinido"

if verdadero and lenguaje == "Python": 
    print "Verdadero y Lenguaje de programacion: Python"

\end{pyconsole}






\subsection{Bucles}
El bucle \pyv{for} es similar a  otros lenguajes. Recorre un objeto \emph{iterable},
esto es  una lista o una tupla, y por cada elemento del iterable 
ejecuta el bloque de código interno. 
Se define con la palabra clave \pyv{for} seguida de un nombre de variable, 
seguido de \pyv{in} seguido del iterable, y finalmente el bloque de código interno. 
En cada iteración, el elemento siguiente del iterable se asigna al nombre de variable 
especificado:

\begin{pyconsole}
lista = ["a", "b", "c"]
for i in lista: # Iteramos sobre una lista, que es iterable
    print i

cadena = "abc"
for i in cadena: # Iteramos sobre una cadena, que es iterable
    print i # una coma al final evita un salto de linea

\end{pyconsole}



El bucle \pyv{while} evalúa una condición y, si es verdadera, ejecuta el bloque
de código interno. Continúa evaluando y ejecutando mientras la condición sea verdadera.
Se define con la palabra clave \pyv{while} seguida de la condición, y a continuación 
el bloque de código interno:
\begin{pyconsole}
numero = 0
while numero < 3:
    print numero
    numero += 1  

\end{pyconsole}


% 
%  \bibliographystyle{plain}
%  \bibliography{biblio}
% 
% 
% \end{document}

\bibliographystyle{plain}%{apalike-url}
\bibliography{biblio,diferenciales_ecuaciones,diferenciales_ecuaciones_sim}




\chapter{Sucessiones, series de funciones y sus amigos}
\chapter{Integral de Riemann}

\section{Introducción}

\begin{quotation}
<< Bernard Riemann recibió su doctorado en 1851, su \emph{Habilitación} en 1854. La habilitación confiere el reconocimiento de la capacidad de crear sustanciales contribuciones en la investigación más allá de la tesis doctoral, y es un prerequisito necesario para ocupar un cargo de profesor en una universidad Alemana. Riemann eligió como tema  de habilitación el problema de las series de Fourier. Su tesis fue titulada \emph{\"Uber die Darstellbarkeit einer Function durch eine trigonometrische Reine} (Sobre la representación de una función por series trigonométricas) y respondía la pregunta:  Cuándo una función definida en el intervalo $(-\pi,\pi)$ puede ser respresentada por la serie trigonométrica $a_0/2+\sum_{n=1}^{\infty}[a_n\cos(nx)+b_n\sen(nx)]$? 
\marginpar{\includegraphics[scale=.4]{imagenes/Riemann.jpeg}\\
Bernhard Riemann 1826-1866
} 
En este trabajo  es donde hallamos   la Integral de Riemann, introducida en una sección corta antes del nucleo principal de la tesis, como parte del trabajo preparatorio que él necesitó desarrollar antes de abordar el problema de representabilidad por series trigonométricas. >> 
\end{quotation}
\begin{flushright}
 David M. Bressoud\\
 A Radical Approach to Lebesgue's Theory of Integration.\lectura
\end{flushright}


En este capítulo vamos a desarrollar el concepto de la integral de Riemman. Vamos a exponer la definición de la integral debida a Riemann y la ideada por J. G. Darboux.
Mostraemos la equivalencia de las dos definiciones y discutiremos las propiedades de la intergal, sus alcances y límites. Preparamos así el camino para la introducción de la integral de Lebesgue. 
\marginpar{\includegraphics[scale=.6]{imagenes/Darboux.jpg}\\
Jean G. Darboux  1842-1917
} 

Debemos advertir \advertencia  al alumno que en este curso dejaremos un poco de lado las cuestiones procedimentales de cómo calcular integrales, aspecto que seguramente abordó en cursos anteriores y del cual nos vamos a valer. Tampoco debe esperar que las actividades prácticas se centren en esa dirección.   Nuestro principal objetivo aquí es discutir la materia conceptual ligada a la integral y cómo es previsible las actividades prácticas estarán orientadas con ese propósito.


El concepto de integral encuentra su motivación en diversos problemas. Aparece cuando se busca el centro de masas de un determinado cuerpo, cuando se quieren hallar longitudes de arco, volúmenes, cuando se quiere reconstruir el movimiento de cuerpo conocida su velocidad, etc. La integral es utilizada en incontables otros conceptos matemáticos, como ser el mencionado már arriba relativo a las series de Fourier. 

Quizás el 
problema más simple donde aparece la integral es el que utilizaremos como motivación para introducirla y es el concepto de área.  Vamos a tratar de reconstruir este concepto desde su base, esto es analizando la noción de área de figuras tan simples como rectángulos, triángulos, etc. 



\section{Área de figuras elementales planas} 

  
El cálculo de áreas es necesario en multitud de actividades humanas, por ejemplo con el comercio. La cantidad de muchos productos y servicios se estima en medidas de área, por ejemplo: las telas,  el trabajo de un colocador de pisos,  el precio de la construcción,  el valor de las extensiones de tierra, etc.  
 
 


Por figuras elementales planas nos referimos a rectángulos, triángulos, trapecios, etc. Sin duda el alumno  debe estar  muy familiarizado con las áreas de estas figuras, el área de un rectángulo viene dada por la conocida fórmula $b\times h$, donde $b$ es la base del rectángulo y $h$ su altura.  Ahora bien, ¿Cómo 
se llega a esta fórmula? Porque esta fórmula es apropiada para calcular el precio de un terreno por ejemplo. En esta sección vamos a justificar esta fórmula a partir de algunos hechos elementales.



Vamos a considerar un plano $\mathcal{P}$. En este plano $\mathcal{P}$ supondremos fijada una unidad de longitud.  Pretendemos asignar un área a las figuras, es decir a los subconjuntos, de $\mathcal{P}$. De ahora en más, cómo es usual en esta materia  nos referiremos a \emph{medida}\index{medida} en lugar de área. La medida es un concepto más general  que el concepto de área. No obstante en el contexto en que estamos actualmente son sinónimos.  

Queremos construir pues una función $m$ tal que $m(A)$ reppresente la medida  de  $A\subset\mathcal{P}$. Ahora bien ¿qué podemos usar de guía con ese objetivo? Si, como dijimos,  desconocemos todas las fórmulas previamente aprendidas, sobre que partimos para construir la medida o área. La respuesta es que tomaremos como principio rector  ciertas propiedades que son deseables  que una medida satisfaga. Ellas son las  siguientes. 




\begin{description}
 \item[Positividad.] debería ser una magnitud no negativa.  
 \item[Invariancia por movimientos rígidos.] Si una región es transformada en otra por medio de un movimiento rígido, ambas regiones deberían tener la misma área. Otra manera de expresar esta propiedad es diciendo que dos figuras \emph{congruentes}\index{congruencia} tienen la misma área. 
 \item[Aditividad.] Si una región es la unión de cierta cantidad de regiones más chicas mutuamente disjuntas  
\end{description}

\begin{figure}[h]
\begin{center}
 
\definecolor{xdxdff}{rgb}{0.49,0.49,1}
\definecolor{zzttqq}{rgb}{0.6,0.2,0}
\definecolor{ududff}{rgb}{0.30,0.30,1}
\begin{tikzpicture}[line cap=round,line join=round,x=.9cm,y=.9cm]
\clip(-2.261345665671131,-2.6483801457242535) rectangle (15.749723988011487,4.927708528477943);
\fill[line width=2pt,color=zzttqq,fill=zzttqq,fill opacity=0.10000000149011612] (-1.62,-0.89) -- (-1.62,3.13) -- (1.28,3.15) -- (1.3,-0.89) -- cycle;
\fill[line width=2pt,color=zzttqq,fill=zzttqq,fill opacity=0.10000000149011612] (0.09169246661626662,0.6507662540293884) -- (1.2899992647959808,1.130148511211861) -- (1.28,3.15) -- (-0.3199239037382289,3.1389660420431844) -- cycle;
\fill[line width=2pt,color=zzttqq,fill=zzttqq,fill opacity=0.10000000149011612] (-1.62,1.45) -- (-1.62,3.13) -- (-0.3199239037382289,3.1389660420431844) -- (0.09169246661626662,0.6507662540293884) -- (-0.9454716981132074,0.2358490566037732) -- cycle;
\fill[line width=2pt,color=zzttqq,fill=zzttqq,fill opacity=0.10000000149011612] (-1.62,-0.89) -- (-0.32,-0.89) -- (-1.62,1.45) -- cycle;
\fill[line width=2pt,color=zzttqq,fill=zzttqq,fill opacity=0.10000000149011612] (-0.32,-0.89) -- (-0.9454716981132074,0.2358490566037732) -- (0.09169246661626662,0.6507662540293884) -- (1.2899992647959808,1.130148511211861) -- (1.3,-0.89) -- cycle;
\fill[line width=2pt,color=zzttqq,fill=zzttqq,fill opacity=0.10000000149011612] (8.76,1.77) -- (8.134528301886792,2.8958490566037733) -- (9.171692466616268,3.3107662540293887) -- (10.36999926479598,3.790148511211861) -- (10.38,1.77) -- cycle;
\fill[line width=2pt,color=zzttqq,fill=zzttqq,fill opacity=0.10000000149011612] (3.8264466094067267,-1.199878066911803) -- (2.6385072170133266,-0.011938674518402692) -- (3.551459891609421,0.9136938983359697) -- (5.601939561385962,-0.5546723179904587) -- (5.16194451123264,-1.5814488960049211) -- cycle;
\fill[line width=2pt,color=zzttqq,fill=zzttqq,fill opacity=0.10000000149011612] (5.6888024763961145,1.8814084921516754) -- (6.19715889449669,3.0677137999207305) -- (4.761837661840736,4.4888939366884495) -- (3.638322806619573,3.3497747084781038) -- cycle;
\fill[line width=2pt,color=zzttqq,fill=zzttqq,fill opacity=0.10000000149011612] (7.14,0.65) -- (5.84,0.65) -- (7.14,-1.69) -- cycle;
\draw [line width=2pt,color=zzttqq] (-1.62,-0.89)-- (-1.62,3.13);
\draw [line width=2pt,color=zzttqq] (-1.62,3.13)-- (1.28,3.15);
\draw [line width=2pt,color=zzttqq] (1.28,3.15)-- (1.3,-0.89);
\draw [line width=2pt,color=zzttqq] (1.3,-0.89)-- (-1.62,-0.89);
\draw [line width=2pt] (-1.62,1.45)-- (-0.32,-0.89);
\draw [line width=2pt] (-0.9454716981132074,0.2358490566037732)-- (1.2899992647959808,1.130148511211861);
\draw [line width=2pt] (-0.3199239037382289,3.1389660420431844)-- (0.09169246661626662,0.6507662540293884);
\draw [line width=2pt,color=zzttqq] (0.09169246661626662,0.6507662540293884)-- (1.2899992647959808,1.130148511211861);
\draw [line width=2pt,color=zzttqq] (1.2899992647959808,1.130148511211861)-- (1.28,3.15);
\draw [line width=2pt,color=zzttqq] (1.28,3.15)-- (-0.3199239037382289,3.1389660420431844);
\draw [line width=2pt,color=zzttqq] (-0.3199239037382289,3.1389660420431844)-- (0.09169246661626662,0.6507662540293884);
\draw [line width=2pt,color=zzttqq] (-1.62,1.45)-- (-1.62,3.13);
\draw [line width=2pt,color=zzttqq] (-1.62,3.13)-- (-0.3199239037382289,3.1389660420431844);
\draw [line width=2pt,color=zzttqq] (-0.3199239037382289,3.1389660420431844)-- (0.09169246661626662,0.6507662540293884);
\draw [line width=2pt,color=zzttqq] (0.09169246661626662,0.6507662540293884)-- (-0.9454716981132074,0.2358490566037732);
\draw [line width=2pt,color=zzttqq] (-0.9454716981132074,0.2358490566037732)-- (-1.62,1.45);
\draw [line width=2pt,color=zzttqq] (-1.62,-0.89)-- (-0.32,-0.89);
\draw [line width=2pt,color=zzttqq] (-0.32,-0.89)-- (-1.62,1.45);
\draw [line width=2pt,color=zzttqq] (-1.62,1.45)-- (-1.62,-0.89);
\draw [line width=2pt,color=zzttqq] (-0.32,-0.89)-- (-0.9454716981132074,0.2358490566037732);
\draw [line width=2pt,color=zzttqq] (-0.9454716981132074,0.2358490566037732)-- (0.09169246661626662,0.6507662540293884);
\draw [line width=2pt,color=zzttqq] (0.09169246661626662,0.6507662540293884)-- (1.2899992647959808,1.130148511211861);
\draw [line width=2pt,color=zzttqq] (1.2899992647959808,1.130148511211861)-- (1.3,-0.89);
\draw [line width=2pt,color=zzttqq] (1.3,-0.89)-- (-0.32,-0.89);
\draw [line width=2pt,color=zzttqq] (8.76,1.77)-- (8.134528301886792,2.8958490566037733);
\draw [line width=2pt,color=zzttqq] (8.134528301886792,2.8958490566037733)-- (9.171692466616268,3.3107662540293887);
\draw [line width=2pt,color=zzttqq] (9.171692466616268,3.3107662540293887)-- (10.36999926479598,3.790148511211861);
\draw [line width=2pt,color=zzttqq] (10.36999926479598,3.790148511211861)-- (10.38,1.77);
\draw [line width=2pt,color=zzttqq] (10.38,1.77)-- (8.76,1.77);
\draw [line width=2pt,color=zzttqq] (3.8264466094067267,-1.199878066911803)-- (2.6385072170133266,-0.011938674518402692);
\draw [line width=2pt,color=zzttqq] (2.6385072170133266,-0.011938674518402692)-- (3.551459891609421,0.9136938983359697);
\draw [line width=2pt,color=zzttqq] (3.551459891609421,0.9136938983359697)-- (5.601939561385962,-0.5546723179904587);
\draw [line width=2pt,color=zzttqq] (5.601939561385962,-0.5546723179904587)-- (5.16194451123264,-1.5814488960049211);
\draw [line width=2pt,color=zzttqq] (5.16194451123264,-1.5814488960049211)-- (3.8264466094067267,-1.199878066911803);
\draw [line width=2pt,color=zzttqq] (5.6888024763961145,1.8814084921516754)-- (6.19715889449669,3.0677137999207305);
\draw [line width=2pt,color=zzttqq] (6.19715889449669,3.0677137999207305)-- (4.761837661840736,4.4888939366884495);
\draw [line width=2pt,color=zzttqq] (4.761837661840736,4.4888939366884495)-- (3.638322806619573,3.3497747084781038);
\draw [line width=2pt,color=zzttqq] (3.638322806619573,3.3497747084781038)-- (5.6888024763961145,1.8814084921516754);
\draw [line width=2pt,color=zzttqq] (7.14,0.65)-- (5.84,0.65);
\draw [line width=2pt,color=zzttqq] (5.84,0.65)-- (7.14,-1.69);
\draw [line width=2pt,color=zzttqq] (7.14,-1.69)-- (7.14,0.65);
\draw (-1.1356538123159674,2.2366014415507594) node[anchor=north west] {$A_1$};
\draw (3.9651373981996176,0.03798454046645946) node[anchor=north west] {$A_1$};
\draw (0.23628313396063827,2.5883801457242477) node[anchor=north west] {$A_2$};
\draw (4.703872676963944,3.83719454554013) node[anchor=north west] {$A_2$};
\draw (0.09557165229124281,0.5304747263093427) node[anchor=north west] {$A_3$};
\draw (9.101106479132552,3.1160482019844795) node[anchor=north west] {$A_3$};
\draw (-1.5753771925328282,0.31940750380524985) node[anchor=north west] {$A_4$};
\draw (6.445177262622713,0.5480636615180171) node[anchor=north west] {$A_4$};
\begin{scriptsize}
\draw [fill=ududff] (-1.62,-0.89) circle (2.5pt);
\draw [fill=ududff] (-1.62,3.13) circle (2.5pt);
\draw [fill=ududff] (1.28,3.15) circle (2.5pt);
\draw [fill=ududff] (1.3,-0.89) circle (2.5pt);
\draw [fill=xdxdff] (-1.62,1.45) circle (2.5pt);
\draw [fill=xdxdff] (-0.32,-0.89) circle (2.5pt);
\draw [fill=xdxdff] (-0.9454716981132074,0.2358490566037732) circle (2.5pt);
\draw [fill=xdxdff] (1.2899992647959808,1.130148511211861) circle (2.5pt);
\draw [fill=xdxdff] (-0.3199239037382289,3.1389660420431844) circle (2.5pt);
\draw [fill=xdxdff] (0.09169246661626662,0.6507662540293884) circle (2.5pt);
\draw [fill=xdxdff] (8.76,1.77) circle (2.5pt);
\draw [fill=xdxdff] (8.134528301886792,2.8958490566037733) circle (2.5pt);
\draw [fill=xdxdff] (10.36999926479598,3.790148511211861) circle (2.5pt);
\draw [fill=ududff] (10.38,1.77) circle (2.5pt);
\draw [fill=xdxdff] (3.8264466094067267,-1.199878066911803) circle (2.5pt);
\draw [fill=ududff] (2.6385072170133266,-0.011938674518402692) circle (2.5pt);
\draw [fill=xdxdff] (3.551459891609421,0.9136938983359697) circle (2.5pt);
\draw [fill=xdxdff] (5.601939561385962,-0.5546723179904587) circle (2.5pt);
\draw [fill=xdxdff] (5.16194451123264,-1.5814488960049211) circle (2.5pt);
\draw [fill=xdxdff] (5.6888024763961145,1.8814084921516754) circle (2.5pt);
\draw [fill=xdxdff] (6.19715889449669,3.0677137999207305) circle (2.5pt);
\draw [fill=ududff] (4.761837661840736,4.4888939366884495) circle (2.5pt);
\draw [fill=xdxdff] (3.638322806619573,3.3497747084781038) circle (2.5pt);
\draw [fill=ududff] (7.14,0.65) circle (2.5pt);
\draw [fill=xdxdff] (5.84,0.65) circle (2.5pt);
\draw [fill=xdxdff] (7.14,-1.69) circle (2.5pt);
\end{scriptsize}
\end{tikzpicture}


 \caption{El área del rectángulo es la suma de sus partes}\label{fig:rect_descop} 
\end{center}
\end{figure}

Utilizando la segunda y tercer propiedad se pueden relacionar el área del rectángulo de la figura \ref{fig:rect_descop} con las cuatro regiones en la que es dividido.

Como veremos a lo largo de la materia la propiedad de aditividad debe ser estudiada con cuidado, esto ocurre por las intrincadas maneras en que una región puede ser unión de otras regiones. A lo largo de esta materia elaboraremos una  teoría que nos dará una descripción  precisa de a que conjuntos podemos asignarle una medida de modo que las propiedades previas sean ciertas. 

Por el momento veamos como las propiedades anteriores determinan practicamente de manera unívoca la medida de regiones elementales planas.  


Hablando de propiedades de la medida, supongamos que $A$ y $B$ son dos regiones con $A\subset B$. Entonces como $B=A\cup (B-A)$ y por la propiedad de aditividad y positividad

\[
 m(B)=m(A)+m(B-A)\geq m(A).
\]

Descubrimos así que nuestra medida deberá tener adicionalmente la siguiente propiedad:
\begin{description}
 \item[Monotonía.] Si $A\subset B$ entonces $m(A)\leq m(B)$. 
\end{description}

Es claro que si logramos construir una medida que satisfaga las propiedades anteriores cualquier multiplo por un número real positivo  de ella seguirá cumpliendo las propiedades. Esto es una manera de expresar el hecho que podemos usar diferentes unidades de medición. Esta cuestión se sortea proponiendo la unidad de medida. Esta unidad es completamente arbitraria, ud. podría elegir su figura plana preferida como unidad de área. \marginpar{ Podríamos por ejemplo elegir el círculo de radio uno como unidad de área. Así ya no tendríamos el problema de ese número raro $\pi$ que aparece en la fórmula del área del círculo. ¡El área de cualquier círculo sería igual a su radio al cuadrado! Claro que aparecería $\pi$  en la fórmula del área del cuadrado de lado 1. Nos tapamos los pies y se destapa el cuerpo.}  Cómo es habitual, elijamos el cuadrado cuyos lados miden la unidad de longitud previamente fijada. 


Supongamos ahora que tenemos un rectángulo de un lado igual a la unidad y el otro de lado un racional $n/m$, $n,m\in\mathbb{N}$. Veamos que la aditividad, la invariancia por movimientos rígidos y el hecho que decidimos que el cuadrado de lados igual a la unidad determinan el área de este rectángulo. Primero observar que si dividimos el lado de cuadrado unidad en $m$ segmentos iguales de longitud. \marginpar{
 
\definecolor{xdxdff}{rgb}{0.49,0.49,1}
\definecolor{zzttqq}{rgb}{0.6,0.2,0}
\definecolor{ududff}{rgb}{0.30,0.30,1}
\begin{tikzpicture}[x=2.1cm,y=2.1cm]
\clip(-0.07,0) rectangle (1.5,1.5);
\draw [line width=2pt,color=zzttqq] (0,0) -- (1,0) -- (1,1) -- (0,1) -- cycle;
\draw [line width=1pt,color=zzttqq, dashed] (0,.2)--(1,.2);
\draw [line width=1pt,color=zzttqq, dashed] (0,.4)--(1,.4);
\draw [line width=1pt,color=zzttqq, dashed] (0,.6)--(1,.6);
\draw [line width=1pt,color=zzttqq, dashed] (0,.8)--(1,.8);
\draw (0,1) node[anchor=south] {$Q$};
\draw (0.5,.1) node[anchor=center] {$R_1$};
\draw (0.5,.3) node[anchor=center] {$R_2$};
\draw (0.5,.5) node[anchor=center] {$R_3$};
\draw (0.5,.7) node[anchor=center] {$R_4$};
\draw (0.5,.9) node[anchor=center] {$R_5$};



\end{tikzpicture}
\\
 }
Queda dividido el cuadrado en $m$ rectángulos $R_1,\ldots,R_m$ (ver figura en el margen), todos ellos  congruentes entre si, de modo que todos tienen la misma medida, digamos $m(R_1)$. La unión de ellos es el cuadrado que por convención dijimos que tiene medida 1. De modo que por la aditividad debe ocurrir que $m(R_1)=\cdots =m(R_m))=1/m$. Recordemos nuestra pretención de inferir la medida de un rectángulo $R$ de lado 1 y otro $n/m$. Este rectángulo esta compuesto de $n$ rectángulos congruentes a los $R_i$, $i=1,\ldots,m$, nuevamente por la aditividad inferimos que $m(R)=n/m$. 

Sea ahora una rectángulo $R$ con un lado unidad y el otro un real cualquiera $l>0$. Existen sendas sucesiones $0<q_k,p_k\in\mathbb{Q}$, $k\in\mathbb{N}$, tales que $q_1\leq q_2\leq\cdots \leq l \leq \cdots\leq p_2\leq p_1$ y $\lim_{k\to\infty}q_k =\lim_{k\to\infty} p_k=l$. Consideremos una dos sucesiones de rectángulos $R_k$ y $S_k$ que comparten el lado de $R$ igual a la unidad, mientras que el otro lado de $R_k$ y $S_k$ es igual a $q_k$ y $p_k$ respectivamente. Luego por la monotonía
\[
 q_k=m(R_k)\leq m(R) \leq m(S_k)\leq p_k.
\]
Tomando límite cuando $k\to\infty$ inferimos que $m(R)=l$. 



\begin{figure}[h]
 \begin{center}
 \input{imagenes/ParalTria.tikz} 
 \end{center}
 \caption{Áreas de otras figuras elementales.}\label{fig:paral-trig}
\end{figure}


A partir de las propiedades fundamentales que postulamos para la medida o área inferimos la famosa fórmula del área de un rectángulo en el caso que uno de los lados sea igual a la unidad. Para un  rectángulo arbitrario. En la figura \ref{fig:paral-trig} se muestra como relacionar el área de un paralelepípedo con la de un rectángulo y la de un triángulo con la de un paralelepípedo para inferir las conocidas fórmulas para estas figuras.



\section{Integral de Riemann}

En esta sección abordaremos el problema del área de regiones planas. Vamos a contextualizarnos dentro del marco conceptual que nos brinda la geometría analítica. Mediante coordenadas cartesianas ortogonales los puntos del plano se identifican con pares ordenados $(x,y)\in\mathbb{R}^2$ y el plano con el conjunto $\rr^2$.  Nuestro propósito es entonces definir la medida de subconjuntos de $\mathbb{R}^2$. La geometría analítica abre así nuevas posibilidades para abordar el problema del área. 

Nuestra primera aproximación será la que propuso Bernhard Riemann en 1854, pero seguiremos  el enfoque de Jean Darboux. En esta parte de nuestra aproximación consideraremos subconjuntos de $\mathbb{R}^2$ de un tipo especial, concretamente a conjuntos que quedan encerrados entre la gráfica de una función y del eje coordenadas $x$.   
 


%\chapter*{Apéndice}

\section{Topología}

\begin{teorema}[Principio de encaje de intervalos\index{Intervalos encajados}]{}  Sea $I_n=[a_n,b_n]\subset\mathbb{R}$ una sucessión de intervalos con las siguientes propiedades
\begin{enumerate}
 \item $\forall n\in\mathbb{N}: I_n\subset I_{n+1},$
 \item $\lim\limits_{n\to\infty}(b_n-a_n)=0.$
\end{enumerate}
Entonces $\bigcap_{n=1}^{\infty}I_n$ consiste de uno, y solo un, punto $x\in\mathbb{R}$.
 
\end{teorema}

\begin{proof}
 
\end{proof}





\begin{teorema}[Heine-Borel]{}\index[personas]{Heine}\index[personas]{Borel} Toda sucesión acotada de $\mathbb{R}$ 
tiene una subsucesión convergente.
 
\end{teorema}

\begin{proof} Uso encajes de intervalos.
 
\end{proof}


\begin{definicion}[Continuidad uniforme \index{Continuidad uniforme}]{} Sea $f:A\subset\mathbb{R}\to\mathbb{R}$ una función. Diremos que $f$ es uniformemente continua si 
\[
 \forall \epsilon>0\exists \delta>0 \forall x,y\in A:|x-y|<\delta\Rightarrow |f(x)-f(y)|<\epsilon.
 \]

\end{definicion}

\begin{ejemplo} Varios ilustrando diferencia con continuidad
 
\end{ejemplo}


\begin{teorema}{} Sea $f:[a,b]\to\mathbb{R}$ continua. Entonces $f$ es uniformemente continua. 
 \end{teorema}
\begin{proof} Uso Heine-Borel
 \end{proof}


\printindex
\end{document}
