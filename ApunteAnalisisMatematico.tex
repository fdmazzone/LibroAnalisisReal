\documentclass[oneside]{book}

%%%%%%%%%%%%%%%%%%%%%%%%%%%%%%Paquetes%%%%%%%%%%%%%%%%%%%%%%%%%%%%%%%%%%%%%%%%%%%%%%%5
%%%%%%%%%%%%%%%%%%%%%%%%%%%%%%%%%%%%%%%%%%%%%%%%%%%%%%%%%%%%%%%%%%%%%%%%%%%%%%%%%%%%%
%\usepackage{empheq}
\usepackage[spanish]{babel}
\usepackage{amssymb,amsmath,amsthm}
\usepackage{enumerate}
\usepackage{verbatim}
\usepackage{array}
\usepackage{ wasysym }%simbolos especiales
\usepackage{hyperref}
\usepackage{color}
\usepackage[spanish]{varioref} % ESTILO PARA REFERENCIAS
\usepackage{fontspec} %para xelatex
\usepackage[a4paper,driver=xetex,top=2.5cm, bottom=2.7cm,%
layouthoffset=10mm, left=1.5cm, right=6.2cm,marginparwidth=3.5cm]{geometry}
\usepackage{fancyhdr} %Encabezados mejorados
\usepackage{marginnote} % Notas al margen mejoradas
\usepackage{titlesec} %para encabezados
\usepackage{pst-all} %pstricks graficos
\let\clipbox\relax
\usepackage{pgf,tikz,pgfplots} %tikz graficos
\pgfplotsset{compat=1.15}
\usetikzlibrary{arrows}
\usepackage{imakeidx} %supongo que para indices multiples
\usepackage{pythontex}%Ejecutar python dentro latex
\usepackage{diagrams}%diagramas comutativos
\usepackage{hyperref}%hipertexto
\defaultfontfeatures{Ligatures=TeX}
\usepackage[framemethod=TikZ]{mdframed}
\usepackage[breakable,many]{tcolorbox}
\tcbset{nobeforeafter} % prevents tcolorboxes being placing in paragraphs
\usepackage{float}
\floatplacement{figure}{H} % forces figures to be placed at the correct location
\usepackage{graphicx}
    % We will generate all images so they have a width \maxwidth. This means
    % that they will get their normal width if they fit onto the page, but
    % are scaled down if they would overflow the margins.
% \makeatletter
% \def\maxwidth{\ifdim\Gin@nat@width>\linewidth\linewidth
% \else\Gin@nat@width\fi}
% \makeatother
% \let\Oldincludegraphics\includegraphics
% Set max figure width to be 80% of text width, for now hardcoded.
% \renewcommand{\includegraphics}[1]{\Oldincludegraphics[width=.8\maxwidth]{#1}}
    % Ensure that by default, figures have no caption (until we provide a
    % proper Figure object with a Caption API and a way to capture that
    % in the conversion process - todo).
%     \usepackage{caption}
%     \DeclareCaptionLabelFormat{nolabel}{}
%     \captionsetup{labelformat=nolabel}
\usepackage{adjustbox} % Used to constrain images to a maximum size 
\usepackage{textcomp} % defines textquotesingle
    % Hack from http://tex.stackexchange.com/a/47451/13684:
\AtBeginDocument{%
\def\PYZsq{\textquotesingle}% Upright quotes in Pygmentized code
    }
\usepackage{upquote} % Upright quotes for verbatim code
\usepackage{eurosym} % defines \euro
\usepackage[mathletters]{ucs} % Extended unicode (utf-8) support
\usepackage{fancyvrb} % verbatim replacement that allows latex
\usepackage{grffile} % extends the file name processing of package graphics 
                         % to support a larger range 
    % The hyperref package gives us a pdf with properly built
    % internal navigation ('pdf bookmarks' for the table of contents,
    % internal cross-reference links, web links for URLs, etc.)
    
%\usepackage{longtable} % longtable support required by pandoc >1.10
\usepackage{booktabs}  % table support for pandoc > 1.12.2
\usepackage[inline]{enumitem} % IRkernel/repr support (it uses the enumerate* environment)
\usepackage[normalem]{ulem} % ulem is needed to support strikethroughs (\sout)
                                % normalem makes italics be italics, not underlines
\usepackage{mathrsfs}

\usepackage{bigints}

%%%%%%%%%% Nuevos Entornos-Comandos%%%%%%%%%%%%%%%%%%%%%%%%%%%%%%%%%%%5
%\input{entornos_comandos.tex}

% %%%%%%%%%%%%%%%%%%%%%%%%%%Nuevos comandos entornos%%%%%%%%%%%%%%%%%%%%%%%%%%%%%%%%
% %%%%%%%%%%%%%%%%%%%%%%%%%%%%%%%%%%%%%%%%%%%%%%%%%%%%%%%%%%%%%%%%%%%%%%%%
\newenvironment{demo}{\noindent\emph{Dem.}}{\hfill\qed \newline\vspace{5pt}}

\newenvironment{observa}{\noindent\textbf{Observación:}}{}
\renewcommand{\C}{\overline{C}}
\newcommand{\com}{\mathbb{C}}
\newcommand{\rr}{\mathbb{R}}
\newcommand{\nn}{\mathbb{N}}
\newcommand{\qq}{\mathbb{R}}
\newcommand{\R}{\text{(R)}}
\renewcommand{\epsilon}{\varepsilon}
\renewcommand{\lim}{\mathop{\rm lím}}
\renewcommand{\inf}{\mathop{\rm ínf}}
\renewcommand{\liminf}{\mathop{\rm líminf}}
\renewcommand{\limsup}{\mathop{\rm límsup}}
\renewcommand{\min}{\mathop{\rm mín}}
\renewcommand{\max}{\mathop{\rm máx}}
\renewcommand{\b}[1]{\boldsymbol{#1}}
\renewenvironment{frame}[1]{}{}

%%%%%%%%%%%%%%%% Funcion característica %%%%%%%%%%%5555555

\DeclareRobustCommand{\rchi}{{\mathpalette\irchi\relax}}
\newcommand{\irchi}[2]{\raisebox{\depth}{$#1\chi$}} % inner command, used by \rchi


% 
% 
% 


%\renewcommand{\lim}{displaystyle\lim}
\DeclareMathOperator{\atan2}{atan2}
\DeclareMathOperator{\sen}{sen}
\DeclareMathOperator{\sgn}{sgn}
\DeclareMathOperator{\diametro}{diam}


\pgfdeclareverticalshading{exersicebackground}{100bp}
  {color(0bp)=(black!40);color(50bp)=(black!0)}

\mdfdefinestyle{MiEstilo}{innertopmargin=10pt,linecolor=white!100,%
linewidth=2pt,topline=true,tikzsetting={shading=exersicebackground}}  


%%%%%%%%%%%%  Teorema %%%%%%%%%%%%%%%%%%%%%%%%%%55

\newcounter{teorema}[chapter] \setcounter{teorema}{0}
\renewcommand{\theteorema}{\arabic{chapter}.\arabic{section}.\arabic{teorema}}
\newenvironment{teorema}[2][]{%
\refstepcounter{teorema}%
\mdfsetup{style=MiEstilo%
}
\ifstrempty{#1}
{
\begin{mdframed}[]\relax%
\strut \textbf{Teorema~\theteorema}\label{#2}
}
{
\begin{mdframed}[]\relax%
\strut \textbf{Teorema~\theteorema~(#1)}\label{#2}
}
}{\end{mdframed}}
%%%%%%%%%%%%%%%%%%%%%%%%%%%%%
%Lemma


\newcounter{lema}[chapter] \setcounter{lema}{0}
\renewcommand{\thelema}{\arabic{chapter}.\arabic{section}.\arabic{lema}}
\newenvironment{lema}[2][]{%
\refstepcounter{lema}%
\mdfsetup{style=MiEstilo%
}
\ifstrempty{#1}
{
\begin{mdframed}[]\relax%
\strut \textbf{Lema~\thelema}\label{#2}
}
{
\begin{mdframed}[]\relax%
\strut \textbf{Lema~\thelema~(#1)}\label{#2}
}
}{\end{mdframed}}
%%%%%%%%%%%%%%%%%%%%%%%%%%%%%


%% Definicion
\newcounter{definicion}[chapter] \setcounter{definicion}{0}
\renewcommand{\thedefinicion}{\arabic{chapter}.\arabic{section}.\arabic{definicion}}
\newenvironment{definicion}[2][]{%
\refstepcounter{definicion}%
\mdfsetup{style=MiEstilo%
}
\ifstrempty{#1}
{
\begin{mdframed}[]\relax%
\strut \textbf{Definición~\thedefinicion}\label{#2}
}
{
\begin{mdframed}[]\relax%
\strut \textbf{Definición~\thedefinicion~(#1)}\label{#2}
}}{\end{mdframed}}
%%%%%%%%%%%%%%%%%%%%%%%%%%%%%
%%%%%%%%%%%%%%%%%%%%%%%%%%%%%%

%%%%%%%%%%%%%%%%% Demostracion

\newenvironment{prf}{\noindent\emph{Dem.}}{$\square$ \newline\vspace{5pt}}


%Corolario
\newcounter{corolario}[chapter] \setcounter{corolario}{0}
\renewcommand{\thecorolario}{\arabic{chapter}.\arabic{section}.\arabic{corolario}}
\newenvironment{corolario}[2][]{%
\refstepcounter{corolario}%
\mdfsetup{style=MiEstilo%
}
\ifstrempty{#1}
{
\begin{mdframed}[]\relax%
\strut \textbf{Corolario~\thecorolario}\label{#2}
}
{
\begin{mdframed}[]\relax%
\strut \textbf{Corolario~\thecorolario~(#1)}\label{#2}
}}{\end{mdframed}}
%%%%%%%%%%%%%%%%%%%%%%%%%%%%%




%%%%%%%%%%%%%%%%%%%%%%%%%%%%%%
%% Ejercicio
\newcounter{ejercicio}[chapter] \setcounter{ejercicio}{0}
\renewcommand{\theejercicio}{\arabic{chapter}.\arabic{section}.\arabic{ejercicio}}
\newenvironment{ejercicio}[2][]{%
\refstepcounter{ejercicio}%
\mdfsetup{style=MiEstilo%
}
\ifstrempty{#1}
{
\begin{mdframed}[]\relax%
\strut \textbf{Ejercicio~\theejercicio}\label{#2}
}
{
\begin{mdframed}[]\relax%
\strut \textbf{Ejercicio~\theejercicio~(#1)}\label{#2}
}}{\end{mdframed}}
%%%%%%%%%%%%%%%%%%%%%%%%%%%%%



%%%%%%%%%%%%%%%%%%%%%%%%%%%%%
%%%%%%%%%%%%%%%%%%%%%%%%%%%%%%
%%%%%%%%%%%%%%%    Proposicion    %%%%%%%%%%%%%%%%%%%%5

\newcounter{proposicion}[chapter] \setcounter{proposicion}{0}
\renewcommand{\theproposicion}{\arabic{chapter}.\arabic{section}.\arabic{proposicion}}
\newenvironment{proposicion}[2][]{%
\refstepcounter{proposicion}%
\mdfsetup{style=MiEstilo%
}
\ifstrempty{#1}
{
\begin{mdframed}[]\relax%
\strut \textbf{Proposición~\theproposicion}\label{#2}
}
{
\begin{mdframed}[]\relax%
\strut \textbf{Proposición~\theproposicion (~#1)}\label{#2}
}}{\end{mdframed}}
%%%%%%%%%%%%%%%%%%%%%%%%%%%%%

%%%%%%%%%%Ejemplo%%%%%%%%%%%%%%%%%%%%%%%%%%%%%%5555
\newcounter{ejemplo}[chapter] \setcounter{ejemplo}{0}
\renewcommand{\theejemplo}{\arabic{chapter}.\arabic{section}.\arabic{ejemplo}}
\newenvironment{ejemplo}[2][]{%
\refstepcounter{ejemplo}%
\relax\noindent\textbf{Ejemplo~\theejemplo}\label{#2}
}{}


%%%%%%%%%%Observacion%%%%%%%%%%%%%%%%%%%%%%%%%%%%%%5555
\newcounter{observacion}[chapter] \setcounter{observacion}{0}
\renewcommand{\theobservacion}{\arabic{chapter}.\arabic{section}.\arabic{observacion}}
\newenvironment{observacion}{%
\refstepcounter{observacion}%
\relax\noindent\textbf{Observacion~\theobservacion}
}{}

%%%%%%%%%%%%%%%Secciones%%%%%%%%%%%%%%%%%%%%%%%%%%%%%%%5


\makeatletter
\newcommand\makeSecHead[4][\fbox]{%
  \@namedef{#2}{\@ifnextchar*{\@nameuse{#2@i}}{\@nameuse{#2@ii}}}
%
    \expandafter\def\csname#2@i\endcsname*##1{\par\vspace{#4}\noindent
       #1{\parbox{\dimexpr\textwidth-2\fboxsep-2\fboxrule}{%
         \normalfont\normalsize#3\makebox[40pt][l]{}~##1}}\par\vspace{#4}}%
%
    \expandafter\def\csname#2@ii\endcsname{\@ifnextchar[{\@nameuse{#2@iii}}{\@nameuse{#2@iv}}}%
%
    \expandafter\def\csname#2@iii\endcsname[##1]##2{\par\vspace{#4}\noindent
      #1{\parbox{\dimexpr\textwidth-2\fboxsep-2\fboxrule}{%
        \refstepcounter{#2}\normalfont\normalsize#3\makebox[40pt][l]{\@nameuse{the#2}}~##2}}%
        \addcontentsline{toc}{#2}{\@nameuse{the#2}~##1}\par\vspace{#4}}%
%
   \expandafter\def\csname#2@iv\endcsname##1{\par\vspace{#4}\noindent
     #1{\parbox{\dimexpr\textwidth-2\fboxsep-2\fboxrule}{%
       \refstepcounter{#2}\normalfont\normalsize#3\makebox[40pt][l]{\@nameuse{the#2}}~##1}}%
       \addcontentsline{toc}{#2}{\@nameuse{the#2}~##1}\par\vspace{#4}}%
}
\makeatother    

\makeSecHead[\colorbox{gray!30}]{chapter}{\Huge\bfseries}{20pt}
\makeSecHead[\colorbox{gray!30}]{section}{\LARGE\bfseries}{15pt}
\makeSecHead[\colorbox{gray!30}]{subsection}{\Large\bfseries}{12pt}
\makeSecHead[\colorbox{gray!30}]{subsubsection}{\large\bfseries}{10pt}

%\renewcommand{\emph}[1]{\fontshape{it}\selectfont #1}


%%%%%%%%%%%%% Configuracion para notebook importadas

\input{python/jupyter_config.tex}

%%%%%%%%%%%%Configuración Encabezados Página
\fancyfoot{}
\fancyhead[RO,LE]{\thepage}
\fancyhead[LO]{\leftmark}
\fancyhead[RE]{\rightmark}


\titleformat{\section}
  {\normalfont\Large\bfseries}{\thesection}{1em}{}[{\titlerule[0.8pt]}]

  
%%%%%%%%%%%%Configuracion Apéndices %%%%%%%%%%%%%%%%%%%%%%%%55555  
\AtBeginEnvironment{subappendices}{%
\chapter*{Apéndices}
\addcontentsline{toc}{chapter}{Apéndices}
\counterwithin{figure}{section}
\counterwithin{table}{section}
}







% %%%%%%%%%%%%%Configuración de fuente para XeLaTeX


% %\setromanfont[Mapping=tex-text]{Oswald-Light}
\setmainfont{Roboto Condensed}
% %\setsansfont{Gentium Basic}
% %\setsansfont{FreeMono}
%\setromanfont{Oswald-Light}
%\setromanfont{Comfortaa}
 %
%
% \renewcommand{\familydefault}{\sfdefault}


%%%%%%%%%%%%%%%%%   Titulo  %%%%%%%%%%%%%%%%%%%%%%%%%%%%%%%%%%%%

\renewcommand{\emptyset}{\varnothing}

\title{Introducción al Análisis Matemático\\
\textcolor{red}{(BORRADOR)}
}

 

\author{}
\date{}
%%%%%%%%%%%%%%%%%%%%%%%%%%%%%%%%%%%%%%%%%%%%%%%%%%%%%%%%%%%%%%%%%%%%%%%%%%%%%%%%%%%%%%

%%%%%%%%%%%%%%   Indices %%%%%%%%%%%%%%%%%%%%%%%%%%%%%%%%%%%%%%%%%%%%%%%%%%%%%%
\makeindex[title=Indice Conceptos]
\makeindex[name=personas,title=Indice de Personas,columns=3]
\makeindex[name=simbolos,title=Indice Símbolos,columns=3]

\begin{document}



\fontsize{11pt}{11pt}\selectfont

\pagestyle{fancy}

%
 \maketitle
 \tableofcontents
%

%
\chapter*{Prólogo}

















%
%
%  \bibliographystyle{apalike-url}
%  \bibliography{diferenciales_ecuaciones,diferenciales_ecuaciones_sim}
 
 



%%%%=====Me tomé el atrevimiento de  RENOMBRAR los archivos de acuerdo al contenido. El número de la unidad no me alcanza para ubicar los temas.

\include{Uni0-Reales}
\include{Uni1-Conjuntos}
\include{Uni2-Topologia}
\include{Uni3-Sucesiones-funciones}====SE PODRÍA COLOCAR EL MATERIAL DE LA 1RA UNIDAD DE COMPLEMENTOS DE ANÁLISIS QUE ESTÁ BASADO EN TUS APUNTES, EL RUDIN CHICO Y OTROS LIBROS QUE USAMOS EN TOP II..
\include{Uni4-Riemann}
\chapter{Medida de Lebesgue en $\mathbb{R}$}

\section{Longitud de intervalos}





\section{Contexto}

Medida exterior


{Volúmen de rectángulos}
  Sea $R=[a_1,b_1]\times\ldots\times [a_n,b_n]\subset\rr^d$ un rectángulo cerrado
  \[|R|:=(b_1-a_1)\cdots(b_n-a_n).\]


{Medida exterior, definición}
  Sea $E\subset\rr^d$,
  \[\boxed{m_*(E):=\inf\left\{\sum_{j=1}^{\infty}|Q_j|\left| E\subset\bigcup_{j=1}^{\infty}Q_j, Q_j\text{ cubo cerrado }, j\in\mathbb{N}\right.\right\}}\]   

		


Medida exterior, propiedades

{Propiedades}
  \begin{description}
   \item[Monotonía:] $E_1\subset E_2\Rightarrow m_*(E_1)\leq m_*(E_2)$ 
   \item[$\sigma$-subaditividad:] $E_j\in\rr^d$, $j=1,2,\ldots$, $\Rightarrow \boxed{ m_*\left(\bigcup_{j=1}^{\infty}E_j\right)\leq\sum_{j=1}^{\infty}m_*(E_j)}$
   \item[$\sigma$-aditividad:] \emph{No se pudo demostrar la igualdad cuando los $E_j$ son mutuamente disjuntos}  
   \item[Regularidad:] $m_*(E)=\inf\{m_*(G)|G\text{ es abierto } G\supset E\}$.
  \end{description}




		



Conjuntos medibles

{Definición}
  $E\subset\rr^d$ se llama \emph{medible Lebesgue} si para todo $\epsilon>0$ existe $G\subset\rr^d$ abierto, $G\supset E$ tal que
  \[\boxed{m_*(G-E)<\epsilon}.\]
  Si $E$ es medible
  \[m(E):=m_*(E).\]



Conjuntos medibles, propiedades

{Propiedades}
    \begin{enumerate}
   \item Los conjuntos abiertos son medibles 
   \item Los conjuntos nulos ($m_*(Z)=0$) son medibles
   \item Las uniones e intersecciones numerables y las diferencias de 
conjuntos medibles resultan en conjuntos medibles.
   \item\textbf{$\sigma$-aditividad.} $E_j\in\rr^d$, $j=1,2,\ldots$, son 
medibles y $E_j\cap E_i=\emptyset$, $j\neq i$,  $\Rightarrow \boxed{ 
m\left(\bigcup_{j=1}^{\infty}E_j\right)=\sum_{j=1}^{\infty}m(E_j)}$  

  \end{enumerate}
  




% \section{Estructura conjuntos medibles}
% Estructura significados (RAE)}
% 
% 
%   \begin{enumerate}
%     \item Disposición o modo de estar relacionadas las distintas partes de un conjunto.
%   
%     \item Distribución y orden de las partes importantes de un edificio.
% 
%     \item Distribución y orden con que está compuesta una obra de ingenio, como un poema, una historia, etc.
% 
%     \item Armadura, generalmente de acero u hormigón armado, que, fija al suelo, sirve de sustentación a un edificio.
%   \end{enumerate}
% 
% 

Problema
  {Problema}
   Podemos describir un conjunto medible como una estructura integrada por partes? Idealmente estas partes deberían ser más familiares y fáciles de caracterizar.
  


\subsection{Aproximación}
\underline{Aproximación} por ``estructuras'' específicas
  {Teorema (Aproximación de conjuntos medibles)} Si $E$ es medible Lebesgue y $\epsilon$ es cualquier número real positivo:
  \begin{enumerate}
   \item\textbf{Abiertos por exceso.} Existe un abierto $G$ con $E\subset G$ y $m(G-E)<\epsilon$.
   \item\textbf{Cerrados por defecto.} Existe un cerrado $F$ con 
$F\subset E$ y $m(E-F)<\epsilon$.
   \item \textbf{Compacto por defecto.} Si $m(E)<\infty$, existe un compacto $K$ con $E\supset K$ y $m(E-K)<\epsilon$.
   \item\textbf{Elementales} Si $m(E)<\infty$, existe una unión finita de cubos cerrados   $K=\bigcup_{j=1}^NQ_j$ tal que $m(E\triangle K)<\epsilon$.
  \end{enumerate}

    
  
  



Demostración Teorema Aproximación
1) Definición de conjunto medible.

2) Aplicando 1), existe un abierto $G$ con  $G\supset E^c$ y 
\[\epsilon>m(G-E^c)=m(E\cap G)=m(E-G^c).\]
Debemos tomar $F=G^c$, que es cerrado y $F\subset E$.

3) Por 2)  Sea $F$ cerrado que aproxima a $E$ por defecto con error a lo sumo $\epsilon$:
\[m(E-F)<\epsilon.\]
Sea $K_n$, $n=1,\ldots$,su colección favorita de compactos con 
\[K_1\subset K_2\subset\cdots \quad\text{y}\quad \rr^d=\bigcup_{n=1}^{\infty}K_n.
\]
Mi favorito: $K_n:=\{x:|x|\leq n\}$. 


Demostración Teorema Aproximación
3) (continuación) Entonces $F_n:=F\cap K_n$ es compacto y
\[E-F_1\supset E-F_2\supset\cdots\quad\text{ y } \bigcap_{n=1}^{\infty}(E-F_n)=E-F.\]

Como $m(E-F_1)\leq m(E)<\infty$, el teorema de convergencia monótona de conjuntos 
\[\lim_{n\to\infty}m(E-F_n)=m(E-F)<\epsilon.\]
Luego existe un $n$ suficientemente grande para que 
\[m(E-F_n)<\epsilon.\]



Demostración Teorema Aproximación
4) Sean $Q_j$, $j=1,2,\ldots$, cubos con 
\[E\subset\bigcup_{j=1}^{\infty}Q_j,\quad \sum_{j=1}^{\infty}m(Q_j)<m(E)+\frac{\epsilon}{2}.\]
Como $m(E)<\infty$ la serie converge $\Rightarrow$ existe $N>0$ con 
\[\sum_{j=N+1}^{\infty}m(Q_j)<\frac{\epsilon}{2}.\]
Sea el cerrado
\[F=\bigcup_{j=1}^{N}Q_j\]



Demostración Teorema Aproximación

Entonces
\[
 \begin{split}
  m(E\triangle F)&=m(E-F)+m(F-E)\\
		 &\leq m\left(\bigcup_{j=N+1}^{\infty}Q_j\right)+m\left(\bigcup_{j=1}^{\infty}Q_j-E\right)\\
		 &\leq\sum_{j=N+1}^{\infty}m(Q_j)+\sum_{j=1}^{\infty}m(Q_j)-m(E)\\
		 &<\epsilon
 \end{split}
\]



\subsection{Conjuntos $G_{\delta}$ y $F_{\sigma}$}
Definición $G_{\delta}$ y $F_{\sigma}$
Hasta el momento no hemos expresado un conjunto medible como una 
estructura, sino que lo hemos aproximado, en un cierto sentido, por conjuntos 
con  estructuras determinadas. Para lograr el objetivo necesitamos introducir 
nuevos tipos de conjuntos.

{Definición  $G_{\delta}$ y $F_{\sigma}$}
 Un conjunto que es una intersección numerable de abiertos se denomina de clase $G_{\delta}$. El complemento de un conjunto de clase  $G_{\delta}$ se denomina de clase $F_{\sigma}$. 
 

\textbf{Observación:} $F$ es $F_{\sigma}$ si es unión  numerable de conjunto cerrados.






Ejemplos $G_{\delta}$ y $F_{\sigma}$
\begin{ejemplo}{}
  Obviamente todo abierto es $G_{\delta}$.
 \end{ejemplo}
 \begin{ejemplo}{}
   Si $-\infty<a<b<\infty$, $[a,b]=\bigcap_{n=1}^{\infty}(a-\frac{1}{n},b+\frac{1}{n})$. Así todo intervalo compacto es $G_{\delta}$.
   
  \end{ejemplo}
  
   \begin{ejemplo}{}
     $\mathbb{Q}$ es $F_{\sigma}$ pues es la unión numerable de conjuntos unitarios, que son cerrados. Por consiguiente, los irracionales $\rr-\mathbb{Q}$ es $G_{\delta}$. 
    \end{ejemplo}


      \begin{ejemplo}{}
     $\mathbb{Q}$ no es $G_{\delta}$. Este hecho no es sencillo de demostrar y requiere un teorema profundo
    \end{ejemplo} 
    

Ejemplos $G_{\delta}$ y $F_{\sigma}$
    



{Teorema de Baire}
 Si $(X,d)$ es un espacio métrico completo y $U_n\subset X$, $n=1,2,\ldots$, son abiertos y densos de $X$, entonces $\bigcap_{n=1}^{\infty}U_n$ es denso.  


Si $\mathbb{Q}$ fuese  $G_{\delta}$ y $U_n$ fuesen abiertos con $\mathbb{Q}=\bigcap_{n=1}^{\infty}U_n$, entonces cada $U_n$ es denso. Si ahora $\mathbb{Q}=\{q_1,q_2,\ldots\}$ y  ponemos $W_n=\rr-\{q_n\}$, entonces cada $W_n$ es abierto y denso. Si ponemos $\{V_n\}_{n=1}^{\infty}=\{U_n\}_{n=1}^{\infty}\cup \{W_n\}_{n=1}^{\infty}$, entonces 
 $\{V_n\}_{n=1}^{\infty}$ es una colección de abiertos densos de $\rr$ que contradice el teorema de Baire pues
 \[\bigcap_{n=1}^{\infty}V_n=\bigcap_{n=1}^{\infty}U_n\cap \bigcap_{n=1}^{\infty}W_n=\mathbb{Q}\cap(\rr-\mathbb{Q})=\emptyset.\]



Teorema estructura conjuntos medibles

{Teorema}
 Sea $E\subset\rr^d$. Son equivalentes
 \begin{enumerate}
  \item $E$ es medible,
  \item Existe $H\subset\rr^d$ de clase $G_{\delta}$ y $Z$ nulo tal que $E=H- Z$.
  \item Existe $F\subset\rr^d$ de clase $F_{\sigma}$ y $Z$ nulo tal que $E=F\cup Z$.
  
 \end{enumerate}





    


Demostración teorema estructura conjuntos medibles

Claramente items 2 y 3 implican el 1.


1$\Rightarrow$ 2. Sean $n\in\mathbb{N}$ y  $G_n\supset E$ abiertos tales que
\[m(G_n-E)<\frac{1}{n}.\]
y sean 
\[H=\bigcap_{n=1}^{\infty} G_n\quad\text{y}\quad Z=H-E.\]
$H$ es $G_{\delta}$, $E\subset H$  y
\[m(Z)=m(H-E)\leq m(G_n-E)<\frac{1}{n}.\]
Como $n$ es arbitrario $m(Z)=0$. 


Demostración teorema estructura conjuntos medibles
Restaría ver 1 $\Rightarrow$ 3. Si $E$ es medible, $E^c$ es medible y, por 1 $\Rightarrow$ 2, 
\[E^c=H-Z,\quad m(Z)=0.\]

Si $F=H^c$, tomando complementos tenemos

\[E=F\cup Z.\]
\qed




    


\subsection{Conjuntos de Borel}
$\sigma$-álgebras

{Definición}
 Una colección de subconjuntos de un conjunto dado $X$,  $\mathscr{A}\subset\mathcal{P}(X)$ se denomina $\sigma$-algebra si
 \begin{enumerate}
  \item $\emptyset\in\mathscr{A}$,
  \item $A\in \mathscr{A}\Rightarrow A^c\in\mathscr{A}$,
  \item $A_i\in \mathscr{A}$, $i=1,2,\ldots$, $\Rightarrow \bigcup_{i=1}^{\infty}A_i\in\mathscr{A}$
 \end{enumerate}



\textbf{Observaciones:} Si $\mathscr{A}$ es una $\sigma$-algebra 
\begin{enumerate}
 \item $X=\emptyset^c\in\mathscr{A}$,
 \item  $A_i\in \mathscr{A}$, $i=1,2,\ldots$, $\Rightarrow$
 $A_i^c\in \mathscr{A}$, $i=1,2,\ldots$, $\Rightarrow$ $\bigcup_{i=1}^{\infty}A_i^c\in\mathscr{A}$  $\Rightarrow$ $\bigcap_{i=1}^{\infty}A_i\in\mathscr{A}$ 
 \end{enumerate}



$\sigma$-álgebras, ejemplos
 \begin{ejemplo}{}
\begin{enumerate}
 \item  Triviales $\mathscr{A}=\mathcal{P}(X)$ y $\mathscr{A}=\{\emptyset,X\}$.
 \item  El conjunto $\mathscr{M}=\mathscr{M}(\rr^d)$ de todos los subconjuntos medibles Lebesgue de $\rr^d$
 \item El conjunto de los abiertos, cerrados, las clases $G_{\delta}$ y $F_{\sigma}$ no forman $\sigma$-algebras. Es decir, no tenemos clases caracterizadas por propiedades topológicas que sean ejemplos de $\sigma$-algebras    Cómo remediarlo? 
 
\end{enumerate}
 
 \end{ejemplo}



 $\sigma$-álgebra generada


{Ejercicio} Si  $\{\mathscr{A}_i\}_{i\in I}$ es una colección de $\sigma$-álgebras  (reparar en que $I$ es arbitrario) del conjunto $X$, entonces $\bigcap_{i\in I}\mathscr{A}_i$ es $\sigma$-álgebra.



{Definicion} Dado un subconjunto $\mathcal{C}$ de subconjuntos de $X$, $\mathcal{C}\subset\mathcal{P}(X)$, definimos la $\sigma$-álgebra generada por $\mathcal{C}$ como
\[\langle\mathcal{C}\rangle=\bigcap\{\mathscr{A}| \mathcal{C}\subset \mathscr{A}\text{ y } \mathscr{A} \text{ es $\sigma$-algebra}\}.\]


{Ejercicio} $\langle\mathcal{C}\rangle$ es la menor $\sigma$-algebra que contiene a $ \mathcal{C}$.




Definición
 {Definición} Sea $\mathcal{G}=\mathcal{G}(\rr^d)$ la colección de todos los conjuntos abiertos de $\rr^d$. Definimos la $\sigma$-algebra de Borel  $\mathscr{B}=\mathscr{B}(\rr^d)$ como  $\mathscr{B}=\langle \mathcal{G} \rangle$ 


{Ejercicio} Demostrar que la $\sigma$-algebra  $\mathscr{B}$ es también generada por:
\begin{enumerate}
 \item Los subconjuntos cerrados de $\rr^d$.
 \item Los subconjuntos compactos de $\rr^d$.
 \item Las bolas abiertas (o cerradas) de $\rr^d$.
 \item Los cubos de $\rr^d$.
\end{enumerate}

 

Observaciones
 \begin{enumerate}
 \item Claramente $G_{\delta},F_{\sigma}\subset\mathscr{B}$.
 \item La definición de conjuntos de Borel es notable, pues definimos los conjuntos de Borel, definiendo  la estructura que los contiene. Esto se manifiesta cuando se intenta demostrar alguna propiedad de los borelianos. 
\end{enumerate}
 \begin{ejemplo}{}
 Si $f:\rr^d\to\rr$ es continua y $B\in\mathscr{B}(\rr)$ entonces $f^{-1}(B)\in \mathscr{B}(\rr^d)$. 
\end{ejemplo}
 \textbf{Dem.} Definimos
\[
 \mathscr{A}=\{A\subset\rr|f^{-1}(A)\in  \mathscr{B}(\rr^d)\}.
\]

\begin{ejercicio}{} $\mathscr{A}$ es una $\sigma$-algebra que contiene a los abiertos.
 \end{ejercicio}



Entonces $\mathscr{B}(\rr)\subset \mathscr{A}$, que equivale a lo que queremos demostrar.




Borelianos y medibles

{Corolario}
 Sea $E\subset\rr^d$. Son equivalentes
 \begin{enumerate}
  \item $E$ es medible,
  \item Existe $B\in\mathscr{B}(\rr^d)$ y $Z$ nulo tal que $E=B\cup Z$.
  
 \end{enumerate}


 


\subsection{Preguntas}
Pregunta 1
\centerline{¿$\boxed{\mathscr{B}=\mathscr{M}}$?}

\textbf{Respuesta:} no.  

Se justifica más adelante, depende de la construcción de un $A\subset\rr$ con $A\notin \mathscr{M}$ y usa la función de Cantor. 



Pregunta 2
 Así como definimos los conjuntos $G_{\delta}$ y $F_{\sigma}$ podemos considerar 
\begin{enumerate}
 \item Uniones numerables de conjuntos de clase  $G_{\delta}$:   $G_{\delta\sigma}$,
 \item Intersecciones numerables de conjuntos de clase  $F_{\sigma}$:   $F_{\sigma\delta}$,
 \item $\vdots$
\end{enumerate}
 
¿Podríamos continuando este proceso construir $\mathscr{B}$?
 
 
 \textbf{Respuesta:} Si, con muchas sutilesas lógicas. Se necesita el concepto de ordinal e inducción transfinita.  Estamos en el terreno de la teoría descriptiva de conjuntos.






\chapter{Medida de Lebesgue -  versi\'on 2022}

COMPLETAR CON DATA DE LA  VERSION PDF DE LAS NOTAS DE CLASES QUE TIENEN
ALGUNO AGREGADOS EN COLORES!!!


\section{Preliminares}
Sea $\rr^d$ el espacio eucl\'ideo de dimensi\'on $d$. 

Si $x \in \rr^d$, entonces $x=(x_1,x_2,\ldots,x_d)$ siendo $x_i\in \rr$.

La \emph{norma} de $x$ se define como $|x|=\left(x_1^2+x_2^2+\ldots+x_d^2\right)^{1/2}$ y la \emph{distancia} de $x$ a $y$
se calcula mediante $d(x,y)=|x-y|$.

Si $E\subset \rr^d$, el \emph{complemento} de $E$ es 
$E^C=\left\{ x:x\notin E\right\}$.

Si $E,F\subset \rr^d,$ se tiene que 
$$E-F=E\cap F^C=\left\{x: x\in E \wedge x \notin F\right\}$$ y 
$$d(E,F)=\inf \left\{d(x,y):x\in E, y \in F \right\}.$$

Si $E \subset \rr^d$, entonces $\diametro(E)=\sup\left\{ d(x,y): x,y \in E \right\}$.

Ahora, la \emph{bola abierta} de centro $x$ y radio $r$ est\'a dada por \[B_r(x)=\left\{y \in \rr^d: d(x,y)<r  \right\}.\]

Si $E \subset \in \rr^d$ se dice \emph{abierto} si $\forall x \in E$, $\exists r>0:$\, $B(x,r)\subset E$. 

Y $F \subset \rr^d$ se denomina \emph{cerrado} si y s\'olo si $F^C$ es abierto.

\begin{itemize}
    \item Si $\{E_{\lambda}\}_{\lambda \in \Lambda}$ son abiertos $\Rightarrow$ $ \bigcup\limits_{\lambda \in \Lambda} E_{\lambda}$ 
    es abierto. 
    \item Si $\Lambda$ es finito $\Rightarrow$ $ \bigcap\limits_{\lambda \in \Lambda} E_{\lambda}$ es abierto.
    \item Si los conjuntos $E_{\lambda}$ son cerrados, se obtienen conjuntos cerrados \emph{intercambiando} uniones por intersecciones.
\end{itemize}

Si $E \subset \rr^d$ se dice \emph{acotado} si $E\subset B$ para alguna bola $B$.

Y, $E \subset \rr^d $ es \emph{compacto} si es cerrado y acotado.


\begin{teorema}{}[Cubrimiento Heine-Borel]
Si $E \subset \rr^d$ es compacto y $E\subset \bigcup\limits_{\alpha} \mathcal{O}_{\alpha}$
con $\mathcal{O}_{\alpha}$ abiertos $\forall \alpha$, entonces existen finitos $\alpha:\alpha_1,\alpha_2, \ldots,\alpha_N$, tal que 
$E \subset \bigcup\limits_{i=1}^N \mathcal{O}_{\alpha_i}$.
\end{teorema}

\begin{itemize}
    \item $x \in \rr^d$ es un \emph{punto l\'imite} \'o \emph{punto de clausura} de $E\subset \rr^d$
    si $\forall r>0$,\; $B(x,r)\cap E \neq \emptyset$.
    \item $x \in \rr^d$ es un \emph{punto aislado} de $E\subset \rr^d$ si $\exists r>0$ tal que $B(x,r)\cap E=\{x\}$.
    \item $x \in \rr^d$ es \emph{interior} a $E$ si $\exists r>0$ tal que $B(x,r) \subset E$.
\end{itemize}

Luego, se definen los siguientes conjuntos 
\begin{itemize}
    \item $E^{\circ}=\left\{x| x \;\mbox{ es interior a }\; E \right\}$.
    \item $\overline{E}=\left\{x| x \;\mbox{ es punto l\'imite de }\; E \right\}$.
    \item     $\partial E=\overline{E}-E^{\circ}$.
\end{itemize}

\begin{ejercicio}{}
\begin{enumerate}
    \item $\overline{E}$ es cerrado;
    \item $E$ es cerrado si y s\'olo su $E=\overline{E}$;
    \item $E$ es abierto si  s\'olo su $E=E^{\circ}$;
    \item $\partial E = \partial E^C$;
    \item $E^{\circ} = \overline{E^C}$.
\end{enumerate}
\end{ejercicio}

Por \'ultimo, un conjunto $E \subset \rr^d$ se llama \emph{perfecto} si no tiene puntos aislados.

\section{Rect\'angulos y cubos}
\begin{definicion}{}
$R \subset \rr^d$ es un \emph{rect\'angulo} si 
\[
R=[a_1,b_1]\times\ldots\times [a_n,b_n],
\]
donde $a_j\leq b_j$, para $j=1,2,\ldots,d$.
\end{definicion}

\begin{observacion}{}
Por definici\'on un rect\'angulo $R$ es cerrado y tiene lados paralelos a los ejes. Si 
\begin{itemize}
    \item $d=1$, $R$ es un intervalo cerrado;
    \item $d=2$, $R$ es un rect\'angulo cerrado con lados paralelos a los ejes;
    \item $d=3$. $R$ es un paralelep\'ipedo.
\end{itemize}
\end{observacion}

Si $R$ es rect\'angulo tal que la longitud de lado es $b_i-a_i$, $i=1,\ldots,d$, entonces el \emph{Volumen} de $R$ est\'a dado por
\[
|R|=(b_1-a_1)\ldots(b_d-a_d).
\]

\begin{itemize}
    \item Un \emph{rect\'angulo abierto} se define del modo natural.
    \item Un \emph{cubo} es un rect\'angulo con todos los lados de la misma longitud $l$. \\
    Si $Q$ es un cubo con lados de longitud $l$, entonces $|Q|=l^d$.
\end{itemize}

Una familia de rect\'angulos $\{R_{\lambda}\}_{\lambda \in \Lambda}$ se dice \emph{casi disjunta} si 
\[
R_{\lambda_1}^{\circ} \cap R_{\lambda_2}^{\circ}=\emptyset, \;\; \lambda_1,\lambda_2 \in \Lambda.
\]

Para $N>0$, consideramos el conjunto 
\[
\frac{1}{N}\zz=\left\{ \frac{m}{N}: m\in \zz   \right\}.
\]

\textbf{AGREGAR DIBUJITO DE LOS VALORES DEL CONJUNTO.}


Sea $I$ un intervalo de $\rr$ con longitud $l$. Queremos estimar 
\[
\# \left(I \cap \frac{1}{N}\zz \right).
\]

\begin{lema}{lema:cantidad-puntos-en-intervalo}
\[
Nl-1 \leq \# \left(I \cap \frac{1}{N}\zz \right) \leq Nl+1.\]
\end{lema}

\begin{demo}

Supongamos que $k=\# \left(I \cap \frac{1}{N}\zz \right)$.

Si $k>0$ y $a_1,\ldots,a_k \in I \cap \frac{1}{N}\zz$ tales que 
$a_1<a_2<\ldots<a_k$ y $a_1=\frac{J}{N},a_2=\frac{J+1}{N},\ldots,a_k=\frac{J+k-1}{N}$, con 
$I=[a,b]$. 

Por un lado, se tiene que
\[
a\leq \frac{J}{N}\leq \frac{J+k-1}{N}\leq b \Rightarrow
\frac{k-1}{N}\leq b-a=l.
\]
Por otro, 
\[
\frac{J-1}{N}< a\leq b<\frac{J+k}{N} \Rightarrow
l=b-a< \frac{k+1}{N}.
\]
\end{demo}

\begin{ejercicio}{}
Probar el Lema \ref{lema:cantidad-puntos-en-intervalo} para el caso $k=0$.
\end{ejercicio}

\begin{corolario}{}
Si $I$ es un intervalo de longitud $l$, entonces
\[
\lim\limits_{N \to \infty} \frac{1}{N}\# \left(I \cap \frac{1}{N}\zz \right)=l=|I|.
\]
\end{corolario}

Si $R$ es un rect\'angulo de $\rr^d$ con 
\[
R=\underbrace{[a_1,b_1]}_{I_1}\times \ldots \times \underbrace{[a_d,b_d]}_{I_d},
\]
entonces 
\[
\#\left(R \cap \frac{1}{N}\zz^d \right)=
\# \left(I_1 \cap \frac{1}{N}\zz \right)\ldots
\# \left(I_d \cap \frac{1}{N}\zz \right)
\]
y por tanto
\[
\lim\limits_{N \to \infty} \frac{1}{N^d}
\# \left(R \cap \frac{1}{N}\zz^d \right)
=|R|.
\]
Como 
\[
\# \left(R^{\circ} \cap \frac{1}{N}\zz^d \right) 
\leq
\# \left(R \cap \frac{1}{N}\zz^d \right)
\leq 
\# \left(R^{\circ} \cap \frac{1}{N}\zz^d \right) +2^d.
\]
Tambi\'en tenemos que 
\[
\lim\limits_{N \to \infty} \frac{1}{N^d}
\# \left(R^{\circ} \cap \frac{1}{N}\zz^d \right)
=|R|.
\]

\begin{corolario}{}
Si $R$ es un rect\'angulo y $R= \bigcup\limits_{j=1}^M R_j$ con 
$\{R_j\}_{j=1}^M$ una familia casi disjunta de rect\'angulos, entonces
\[
|R|=\sum\limits_{j=1}^M |R_j|.
\]
\end{corolario}

\begin{demo}
$\leq)$
\[
\begin{split}
|R|=&
\lim\limits_{N \to \infty} \frac{1}{N^d}
\# \left(R \cap \frac{1}{N}\zz^d \right)
\\
\leq & 
\lim\limits_{N \to \infty} \frac{1}{N^d}
\sum\limits_{j=1}^M 
\# \left(R_j \cap \frac{1}{N}\zz^d \right)
\\
=&
\sum\limits_{j=1}^M 
\lim\limits_{N \to \infty} \frac{1}{N^d}
\# \left(R_j \cap \frac{1}{N}\zz^d \right)
\\
=&
\sum\limits_{j=1}^M |R_j|.
\end{split}
\]

$\geq) $
\[\begin{split}
\sum\limits_{j=1}^M |R_j|=&
 \sum\limits_{j=1}^M
\lim\limits_{N \to \infty} 
\frac{1}{N^d} \# \left(R_j^{\circ} \cap \frac{1}{N}\zz^d \right)
\\
=&
\lim\limits_{N \to \infty} 
\frac{1}{N^d} 
\sum\limits_{j=1}^M
\# \left(R_j^{\circ} \cap \frac{1}{N}\zz^d \right)
\\
=&
\lim\limits_{N \to \infty} \frac{1}{N^d}
\# \left(\left(\bigcup\limits_{j=1}^M  R_j^{\circ}\right)
\cap \frac{1}{N} \zz^d\right)
\\
\leq& |R|.
\end{split}
\]
\end{demo}

\begin{lema}{lema:medida-rectangulo-incluido-en-union}
Si $R$ es rect\'angulo y $R\subset \bigcup\limits_{j=1}^M R_j$ donde
$R_j$ son rect\'angulos, entonces
\[
|R|\leq \sum\limits_{j=1}^M |R_j|. 
\]
\end{lema}

\begin{demo}
La prueba de este lema que como ejercicio para el lector. 
\end{demo}


\begin{ejercicio}{}
Demostrar el  Lema \ref{lema:medida-rectangulo-incluido-en-union}.
\end{ejercicio}


\begin{teorema}{teo:abierto-union-intervalos-abiertos-disj}
Todo conjunto abierto $\mathcal{O}$ de $\rr$ es uni\'on numerable \emph{\'unica} de intervalos abiertos disjuntos.
\end{teorema}

\begin{demo}
Sea $x \in \mathcal{O}$ y definimos 
\[
I_x=\bigcup \left\{  
I:\,I\mbox{ es intervalo abierto, }\, I\subset \mathcal{O}, x \in I
\right\}.
\]
\begin{enumerate}
    \item $I_x$ es abierto.
    \item $I_x$ es intervalo. En efecto, si $y,z\in I$ $\Rightarrow \exists\, I_1,I_2$ intervalos tales que $I_1,I_2\subset \mathcal{O}$, 
    $x \in I_1\cap I_2$, $y \in I_1$ y $z\in I_2$. 
    Entonces $I_1\cup I_2$ es intervalo e $I_1\cup I_2 \in \mathcal{O}$. Luego  $[y,z]\subset I_x$. 
    \item Si $I_x \cap I_y \neq \emptyset \Rightarrow I_x=I_y$.
    \'Esto se obtiene a partir de que $I_x \cup I_y$ es intervalo.
    \item\label{it:cantidad-numerable-intervalos} Hay a lo sumo una cantidad numerable de intervalos disjuntos.
    \item \label{it:desc-unica}La descomposici\'on en intervalos abiertos disjuntos es \'unica.
    \end{enumerate}
\end{demo}

\begin{ejercicio}{}
Demostrar  los apartados \ref{it:cantidad-numerable-intervalos} y \ref{it:desc-unica} de la prueba del Teorema \ref{teo:abierto-union-intervalos-abiertos-disj}.
\end{ejercicio}


La generalizaci\'on a $\rr^d$ con $d>1$ presenta algunas dificultades.

\begin{teorema}{}
Todo conjunto abierto de $\rr^d$ puede escribirse como uni\'on numerable de cubos casi disjuntos.
\end{teorema}

\begin{demo}
Sea $\mathcal{O}\in \rr^d$ abierto. Vamos a construir una familia $\mathcal{G}$ de cubos casi disjuntos con $\mathcal{O}=\bigcup\limits_{Q \in \mathcal{G}} Q$.

Procedemos inductivamente en etapas, 

\underline{Primera etapa.}
Consideremos la familia de cubos con v\'ertices en $\zz^d$ que cubre $\rr^d$.
\\
Todos aquellos cubos que quedan contenidos en $\mathcal{O}$ se agregan a $\mathcal{G}$.  Los  cubos  que son disjuntos con $\mathcal{O}$ se tiran y los que tienen parte en $\mathcal{O}$ y en $\mathcal{O}^C$ se dejan  como candidatos.

\underline{$k$-\'esima  etapa.}
Tomamos los candidatos de la etapa $k-1$, dividimos sus lados en $2$ y procedemos como en la primera etapa. 

\textbf{
AGREGAR DIBUJITOS!!!!}

La familia $\mathcal{G}$ es numerable y casi disjunta. 

Sea $x\in \mathcal{O}$. Luego,  $x$ pertenece a un cubo $Q$ de cada etapa $N$ cuyos lados miden $2^{-N}$. Si $N$ es suficientemente grande, se tiene  $Q\subset \mathcal{O}$. Entonces, o bien, $Q$ es agregado a $\mathcal{G}$ en la etapa $N$ o un padre de $Q$ fue agregado en una etapa anterior.
 \end{demo}
 
 \section{Expresiones s-\'adicas}
 
 El objetivo de esta secci\'on es representar n\'umeros reales mediante series num\'ericas.
 
 Sea $d=2,3,\ldots$. 
 
 Supongamos que $x \in [0,1)$. Dividamos el intervalo  $[0,1]$ en $d$ partes iguales, o sea, 
 \[
 0<\frac{1}{d}<\frac{2}{d}<\ldots<\frac{d-1}{d}<1.
 \]
 Ahora, existe un \'unico $j$ tal que 
 \[
 x \in \left[\frac{j}{d}, \frac{j-1}{d}\right).
 \]
 A ese valor de $j$ lo llamamos $a_1$. 
 Y, observamos que \[ \left|x-\frac{a_1}{d}\right|<\frac{1}{d}.\]
 
 A continuaci\'on,  dividimos $\left[\frac{a_1}{j}, \frac{a_1+1}{j} \right)$ en $d$ partes iguales y obtenemos
 \[
 \frac{a_1}{d}<\frac{a_1}{d}+\frac{1}{d^2}<\frac{a_1}{d}+\frac{2}{d^2}
 < \ldots<\frac{a_1}{d}+\frac{d-1}{d^2}<\frac{a_1+1}{d}.
 \]
 Nuevamente, hay un \'unico $j$ tal que 
 \[x \in \left[\frac{a_1}{d}+\frac{j}{d^2}, \frac{a_1}{d}+\frac{j+1}{d^2}\right).
 \]
 Ahora, llamamos $a_2$  a ese valor de $j$. Y notamos que 
 \[
 \left|x-\left(\frac{a_1}{d}+\frac{a_2}{d^2}\right)\right|<\frac{1}{d^2}.
 \]
 Continuando con este procedimiento, obtenemos una sucesi\'on $a_1,a_2,\ldots,a_n$ tal que 
 \[
 \left|  
 x-\left( \frac{a_1}{d}+\frac{a_2}{d^2}+\ldots+\frac{a_n}{d^n}\right)
 \right|<\frac{1}{d^n}.
 \]
 De este modo, habremos encontrado $a_j \in \{ 0,1,2,\ldots,d-1\}$ tales que 
 \begin{equation}\label{eq:expresion-d-adica}
 x=\sum\limits_{j=1}^{\infty} \frac{a_j}{d^j}.
 \end{equation}
 El desarrollo dado por  \eqref{eq:expresion-d-adica} se denomina \emph{expresi\'on  d-\'adica} de $x$. 
 
 Cuando $d=10$, \eqref{eq:expresion-d-adica} se llama \emph{expresi\'on decimal}.
 
Mientras que si $d=2$, \eqref{eq:expresion-d-adica} se denomina \emph{expresi\'on binaria}.
 
La expresi\'on d-\'adica \eqref{eq:expresion-d-adica} suele escribirse
\[
x=\left(0.a_1a_2\ldots\right)_d.
\]

Lamentablemente, la expresi\'on d-\'adica de $x$ no es \'unica. El problema se presenta con las sucesiones que toman el valor $d-1$ de un momento en adelante, pues
\[
\sum\limits_{j=k}^{\infty} \frac{d-1}{d^j}=
\frac{d-1}{d^k} \sum\limits_{j=0}^{\infty} \frac{1}{d^j}=
\frac{d-1}{d^k} \frac{1}{1-\frac{1}{d}}=\frac{1}{d^{k-1}}.
\]
Luego
\[
\left(0.a_1a_2\ldots a_{k-1} (d-1) (d-1)\dots\right)_d =
\left(0.a_1a_2\ldots (a_{k-1} +1) 0\dots\right)_d.
\]
 Si $a_{k-1}+1=d$, entonces 
 \[
 \left(0.a_1a_2\ldots (a_{k-1} +1)\dots\right)_d=
 \left(0.a_1a_2\ldots (a_{k-2} +1) 0\dots\right)_d.
\]
Si $a_{k-2}+1=d$, se procede de la misma manera. 


 \subsection{Conjunto de Cantor}
 Sea $\mathscr{C}_0=[0,1]$. 
 Entonces
 \[
 \begin{split}
 &\mathscr{C}_1=[0,1]-\left(\frac{1}{3},\frac{2}{3}\right)
 \\
  &\mathscr{C}_2=\mathscr{C}_1-\left(\frac{1}{9},\frac{2}{9}\right)
   -\left(\frac{7}{9},\frac{8}{9}\right)
   \\
   &\vdots
   \\
   &\mathscr{C}=\bigcap\limits_{k=1}^{\infty} \mathscr{C}_k.
 \end{split}
 \]
 
 \begin{proposicion}{prop:conjunto-Cantor-propiedades}
 $\mathscr{C}$ es compacto, totalmente disconexo y perfecto. 
 Adem\'as, $\# \mathscr{C}=\# \rr$.
 \end{proposicion}
 
 \begin{ejercicio}{}
 Demostrar la Proposici\'on \ref{prop:conjunto-Cantor-propiedades}.
 \end{ejercicio}
 
 \section{Medida exterior}
\chapter{Funciones medibles
}

\section{Introducci\'on}

\begin{definicion}{} Definimos la recta extendida $\overline{\rr}$ como el conjunto $\rr\cup\{\infty,-\infty\}$. 
Diremos que $H \subset \overline{\rr}$ es boreliano de la recta extendida si $H-\{ -\infty, \infty\} \in \mathscr{B}(\rr)$.
\end{definicion}
 

 Si $f:\mathbb{R}^d\to\overline{\mathbb{R}}$, y  $a\in\mathbb{R}$, definimos
 
$$
\{f>a\}=\left\{x \in \mathbb{R}^n: f(x)>a\right\}=f^{-1}((a, \infty]) .
$$

Análogamente se definen los conjuntos
$$
\{f \geq a\}, \quad\{f<a\}, \quad\{f \leq a\},
$$
que corresponden, respectivamente, a las desigualdades $f(x) \geq a, f(x)<a$ y $f(x) \leq a$. El símbolo $\{f=a\}$ indica el conjunto formado por todos los puntos donde $f$ toma el valor $a$.





\begin{definicion}{}Diremos que $f:\mathbb{R}^d\to\overline{\mathbb{R}}$  es una función medible si $\{f \geq a\}$ es un subconjunto medible del espacio $\mathbb{R}^d$ es decir, si para cada número real $a$, se verifica
 que   $f^{-1}([a,\infty]) \in \mathscr{M}$. 
\end{definicion}

\begin{teorema}{}
Si $H$ es boreliano de $\rr$ y $f$ es medible, entonces $f^{-1}(H)$ es medible.
\end{teorema}

\begin{demo}{}
Sea $\mathscr{M}^{'}=\{ H: f^{-1}(H)\;\mbox{ es medible}\}$.\\
$\mathscr{M}^{'}$ es $\sigma$-\'algebra y $\mathscr{I}\subset \mathscr{M}^{'}$.
Por lo tanto, $\mathscr{B}(\rr^n) \subset \mathscr{M}$.
\end{demo}



\begin{teorema}{}
Si $f:\rr^n \to \overline{\rr}$ es medible si y s\'olo si $f^{-1}(H) \in \mathscr{M}$ cada vez que $H$ es boreliano de $\overline{\rr}$.
\end{teorema}

\begin{demo}
$\Leftarrow)$ Inmediata a partir de la definici\'on de funci\'on medible.

$\Rightarrow)$ Se tiene que 
\[
\{f=+\infty\}=\bigcap\limits_{k=1}^{\infty} \{f \geq k\}
\;\;\mbox{ y }\;\;
\{f=-\infty\}=\bigcap\limits_{k=1}^{\infty} \{f \leq -k\}.
\]
Si $H$ es boreliano de $\overline{\rr}$, supongamos que $H=H^{'}\cup \{+\infty\}$ y $H^{'}$ es conjunto medible Borel de $\rr$.
Luego, $f^{-1}(H)=f^{-1}(H^{'})\cup \{f=+\infty\}$ es medible.
\end{demo}


\section{Funciones medibles sobre una $\sigma$-\'algebra}

\begin{definicion}{}
Si $\Sigma$ es una $\sigma$-\'algebra de $\rr^n$, diremos que $f:\rr^n\to \overline{\rr}$ es $\Sigma$-medible si 
\[\{f>a\} \in \Sigma\;\;\;\;\forall a \in \rr.\]
\end{definicion}

\begin{proposicion}{}
$f:\rr^n \to \overline{\rr}$ es $\Sigma$-medible si y s\'olo si
$f^{-1}(H)\in \Sigma$ cuando $H$ es boreliano de $\overline{\rr}$.
\end{proposicion}

Cuando $\Sigma=\mathscr{M}$, decimos que $f$ es \emph{medible}.

Cuando $\Sigma=\mathscr{B}$, llamamos a $f$ \emph{medible Borel} 
o funci\'on \emph{boreliana}.

\begin{ejercicio}{}
Si $f$ es semicontinua inferiormente, entonces $f$ es medible. 
\end{ejercicio}

Ahora, estudiaremos $g\circ f$ cuando 
$\rr^n \xrightarrow{f}\overline{\rr} \xrightarrow {g} \overline{\rr}$.

\begin{definicion}{}
Diremos que $g: \overline{\rr} \to \overline{\rr}$ es boreliana si $g^{-1}(M)$ es boreliano de $\overline{\rr}$ cuando $M$ lo es.
\end{definicion}

Luego, si $f:\rr^n \to \overline{\rr}$ es medible y $g:\overline{\rr} \to \overline{\rr}$ es boreliana, entonces $g\circ f$ es medible.

Tambi\'en, se tiene que si $f$ es medible entonces $|f|$, $|f|^2$, $\log|f|$ y $e^f$ son medibles. 

Si $f$ y $g$ son medibles, entonces $\{f<g\}$ es medible pues
\[
\{f<g\}=\bigcup\limits_{q \in \qq} \{ f<q\} \cap \{ g>q\}.
\]
\begin{teorema}{teo:propiedades-func-medibles}
Si $f$ y $g$ son medibles  con respecto a $\Sigma$ y finitas y si $c\in \rr$, entonces
$f+g$, $cf$ y $fg$ son medibles.
\end{teorema}

\begin{demo}
Supondremos que todas las funciones son finitas.

Veamos que 
\[
\{
f+g>a\}=\bigcup\limits_{r \in \qq} \{ f>r\} \cap \{g>a-r\}.
\]
Si $f(x)+g(x)>a,$ entonces existe $r \in \qq$ tal que 
\[
f(x)>r>a-g(x),\;\;\mbox{ es decir }\;\; 
f(x)>r\;\mbox{ y }\; g(x)>a-r.
\]
Rec\'iprocamente, si para alg\'un $r\in \qq$ se tiene que $f(x)>r$ y $g(x)>a-r$, luego $f(x)+g(x)>a$.

Si $c\in \rr$, entonces
\[
\{cf>a\}=
\left\{
\begin{array}{ll}
\{f>\frac{a}{c}\}&c>0
\medskip
\\
\{f<\frac{a}{c}\}&c<0
\medskip
\\
\emptyset\;\mbox{ \'o }\;\rr^n &c=0.
\end{array}
\right.
\]

Ahora, como 
\[
fg=\frac{1}{4}\left[(f+g)^2-(f-g)^2\right],
\]
luego $fg$ es medible con respecto a $\Sigma$. 

Para estudiar el cociente entre funciones medibles, denotamos por $\frac{1}{f}$
la funci\'on que toma el valor $\frac{1}{f(x)}$ si $f(x)\neq 0$ y el valor $0$ si $f(x)=0$.

Ahora, como 
\[
\left\{\frac{1}{f}>a\right\}
=
\left \{
\begin{array}{ll}
 \{f>0\} \cap \{f<\frac{1}{a}\}    &  a>0 
 \medskip
 \\
 \{ f>0\} \cup  (\{f<\frac{1}{a}\} \cap \{f<0\}) \cup \{f=0\}
     & a<0, 
\end{array}
\right.
\]
a partir de que $f$ es medible se tiene que  $1/f$ es medible. 
\end{demo}

\section{Sucesiones de funciones medibles}

\begin{proposicion}{prop:inf-y-sup-medibles}
Si $\{f_k\}$ es una sucesi\'on de funciones medibles con respecto a $\Sigma$, 
entonces
\[
g(x)=\inf\limits_{k} f_k(x) \;\;\mbox{ y }\;\; h(x)=\sup\limits_{k} f_k(x) 
\]
son $\Sigma$-medibles. 
\end{proposicion}


La demostraci\'on se deduce de las f\'ormulas
\[
\{h>a \}=\bigcup_{k=1}^{\infty} \{f_k>a\}
\;\;\mbox{ y }\;\;
\{g<a \}=\bigcup_{k=1}^{\infty} \{f_k<a\}.
\]

\begin{proposicion}{prop:liminf-y-limsup-medibles}
Si $\{f_k\}$ es una sucesi\'on de funciones $\Sigma$-medibles, entonces
\[
g(x)=\liminf\limits_{k \to \infty} f_k(x)
\;\;\mbox{ y }
\;\;
h(x)=\limsup\limits_{k \to \infty} f_k(x)
\]
son   medibles con respecto a $\Sigma$.
\end{proposicion}


La prueba resulta de aplicar la Proposici\'on \ref{prop:inf-y-sup-medibles}
a las relaciones 
\[
g(x)=\sup\limits_{j} \inf\limits_{k\geq j} f_k(x)
\;\;\mbox{ y }\;\;
h(x)=\inf\limits_{j}\sup\limits_{k \geq j} f_k(x).
\]

\begin{corolario}{}
Si $f_k(x) \to f(x)$ y $\{f_k\}$ es una sucesi\'on de funciones $\Sigma$-medibles, entonces $f$ es medible con respecto a $\Sigma$.
\end{corolario}

A continuaci\'on, completamos la demostraci\'on del Teorema \ref{teo:propiedades-func-medibles} que se reliz\'o suponiendo
que tanto $f$ como $g$ son finitas. 
Para evitar esa restricci\'on, ahora consideramos la sucesi\'on de funciones $\varphi_k:\overline{\rr} \to \overline{\rr}$ definidas por
\[
\varphi_k(t)=
\left\{
\begin{array}{rl}
   t  &  \mbox{si } |t|\leq k \\
   k  &  \mbox{si } t>k\\
   -k &  \mbox{si } t<-k.
\end{array}
\right.
\]
Cada $\varphi_k$ es una funci\'on boreliana pues la restricci\'on de $\varphi$ a $\rr$ es continua. Adem\'as, para cada $t\in \overline{\rr}$, 
se tiene que $\varphi_k \to t$ cuando $k \to \infty$. Entonces, las funciones
\[
f_k=\varphi_k \circ f \;\;\mbox{ y }
\;\;
g_k=\varphi_k \circ g,
\]
son medibles con respecto a $\Sigma$, finitas y convergen puntualmente a $f$ y $g$ respectivamente, cuando $k  \to \infty$. Luego, las funciones 
\[
f+g=\lim\limits_{k \to \infty} (f_k +g_k)
\;\;\mbox{ y }\;\;
fg=\lim\limits_{k \to \infty} f_kg_k,
\]
resultan medibles a partir de la aplicaci\'on de la Proposici\'on \ref{prop:liminf-y-limsup-medibles}.

\section{Funciones simples}
Definimos la funci\'on caracter\'istica  $\chi_E$ de un conjunto $E \subset \rr^n$ mediante
\[
\chi_E(x)=
\left\{
\begin{array}{ll}
   1  & \mbox{si } x \in E \\
   0  & \mbox{si } x \notin E.
\end{array}
\right.
\]

Se tiene la siguiente propiedad
\begin{itemize}
    \item $\chi_E$ es medible  si y s\'olo si $E$ es medible.
\end{itemize}


\begin{definicion}{defi:funcion-simple}
Una funci\'on medible y finita  $\varphi:\rr^n \to \rr$ 
se llama \emph{simple} si el conjunto de todos sus valores es finito, es decir, si $\varphi$ es medible y la imagen  $\varphi(\rr^n)$ es un subconjunto finito de $\rr$.
\end{definicion}

A partir de la Definici\'on \ref{defi:funcion-simple}, se tiene que si $\varphi,\psi$ son funciones simples y $c \in \rr$, entonces $\varphi + \psi$, $c\varphi$,  $\varphi \psi$ son simples.

Si $\varphi(\rr^n)=\{\alpha_1,\ldots,\alpha_N\}$, entonces $E_i=\varphi^{-1}(\{\alpha_i\})$ son medibles y 
\[
\varphi=\sum\limits_{i=1}^N \alpha_i \chi_{E_i}.
\]
Las funciones simples desempe\~nan un papel muy importante en la teor\'ia de integraci\'on en virtud del siguiente teorema.

\begin{teorema}{teo:simples-convergen-a-medible}
Si $f:\rr^n \to \overline{\rr}$ es una funci\'on medible no negativa, entonces existe una sucesi\'on $\{\varphi_k\}$ de funciones simples tal que
\[
\varphi_1\leq \varphi_2 \leq \varphi_3\leq \ldots \;\;\mbox{ y }\;\;
f(x)=\lim\limits_{k \to \infty} \varphi_k(x)
\]
en cada punto $x \in \rr^n$.
\end{teorema}

\begin{demo}
Para $k \in \nn$, dividimos $[0,k)$ en $k2^k$ intervalos disjuntos
\[
\left[\frac{i-1}{2^k},\frac{i}{2^k}\right), \;\;i=1,2,\ldots,k2^k.
\]
Definimos $g_k:\overline{\rr} \to \overline{\rr}$ por
\[
g_k(x)=
\left\{
\begin{array}{ll}
    \frac{i-1}{2^k} & \mbox{si }\; 0\leq t\leq k, \;\; (i-1)/2^k\leq t< i/2^k, 
    \medskip
    \\
    k     & \mbox{si }\;t\geq k
    \medskip
    \\
    0     & \mbox{si }\; t<0.
\end{array}
\right.
\]
Las funciones $g_k$ son borelianas, no negativas y verifican
\[
0\leq g_1\leq g_2\leq \ldots,\;\mbox{ y }\;\;
\lim\limits_{k \to \infty} g_k(t)=t, \;\;\mbox{ en } [0, +\infty].
\]
Las funciones $\varphi_k = g_k \circ f$ son simples y verifican  el teorema.
\end{demo}

\begin{observacion}{}
\begin{enumerate}
    \item Si $f$ es medible con respecto a una $\sigma$-\'algebra $\Sigma$, entonces las funciones $\varphi_k$ del Teorema \ref{teo:simples-convergen-a-medible} son tambi\'en medibles con respecto a $\Sigma$.
    \item Multiplicando a las $\varphi_k$ por $\chi_{B(0,k)}$ se obtiene una sucesi\'on de  funciones $\psi_k$ que verifican las hip\'otesis del Teorema \ref{teo:simples-convergen-a-medible} y que tienen soporte compacto.
    \item Si $f$ es acotada y positiva,  la convergencia es uniforme.
\end{enumerate}
\end{observacion}

\section{Partes positiva y negativa}

Si $f$ es medible, tambi\'en lo son 
\[
f^+=\sup\{0,f \}\;\;\mbox{ y }\;\;f^{-}=\sup\{0,-f\}
\]
llamadas \emph{parte positiva} y \emph{parte negativa} de $f$.

Se verifica que 
\[
f=f^{+}-f^{-}\;\;\mbox{ y }\;\; |f|=f^{+}+f^{-}.
\]

\begin{teorema}{}
Si $f$ es medible y $f=f_1-f_2$, con $f_i\geq 0$ para $i=1,2$, entonces
$f^{+}\leq f_1$ y $f^{-}\leq f_2$.
\end{teorema}

\begin{demo}
Se tiene que $f\leq f_1$ de donde $f^+=\sup\{ f,0\}\leq f_1$. 

Adem\'as, $-f \leq f_2$. Luego, $f^{-}\leq f_2$
\end{demo}

Si $f$ es medible,  aplicando el Teorema \ref{teo:simples-convergen-a-medible} a $f^+$ y $f^{-}$, 
existen funciones simples $\varphi_k,\psi_k$ tales que $\varphi_k \to f^+$
y $\psi_k \to f^{-}$. Luego, 
$\varphi_k - \psi_k \to f$, y adem\'as, $|\varphi_k-\psi_k|\leq \varphi_k+\psi_k \leq f^{+}+f^{-}=|f|$.

\section{Propiedades verdaderas en casi todo punto}

Si $P$ es una propiedad sobre puntos de $\rr^n$ ($P(x)$) diremos que $P$ es verdadera en casi todo punto si $P(x)$ es verdadera excepto, posiblemente, en un conjunto de medida cero. 

As\'i, por ejemplo:
\begin{enumerate}
    \item Casi todo n\'umero es irracional.
    \item  Si $f$ y $g$ son funciones definidas sobre todo $\rr^n$, diremos que $f=g$ en casi todo punto si $f(x)=g(x)$ para todo  
    $x \notin E$  siendo la  $m(E)=0$.
\end{enumerate}

\begin{teorema}{}
Si $h=0$ en c.t.p., entonces $h$ es medible.
\end{teorema}

\begin{demo}
Sea $Z=\{h \neq 0\}$. 

Si $a \geq 0$, entonces $\{h>a\}\subset Z$ y por tanto $m(\{h>a\})=0.$ Luego, $\{h>a\} \in \mathscr{M}$.

Si $a < 0$, se tiene que  $\{h\leq a\}\subset Z$. Ahora,  como $\{h>a\}=\rr^n-\{h\leq a\}$, entonces  $\{h>a\} \in \mathscr{M}$ por ser complemento de un conjunto medible.
\end{demo}



\begin{corolario}{}
Si $f$ es medible y $f=g$ en c.t.p., entonces $g$ es medible. 
\end{corolario}


Ser\'a frecuente decir que $f_k(x) \to f(x)$ en c.t.p.

\begin{teorema}{}
Si $f_k \to f$ en c.t.p. y las funciones $f_k$ son medibles, entonces 
$f$ es medible.
\end{teorema}

\begin{demo}
La funci\'on $g=\liminf\limits_{k \to \infty} f_k$ es medible y $g=f$ en c.t.p.
\end{demo}

Si $f$ y $g$ son medibles, definimos $f \sim g$ si $f=g$ en c.t.p.

Entonces, $\sim$ es una relaci\'on de equivalencia. Adem\'as, si $f_1\sim f_2$ y $g_1 \sim g_2$, entonces $f_1+g_1\sim f_2+g_2$ y $f_1g_1\sim f_2g_2$.

Si $f=g$ en c.t.p., entonces $f$ es esencialmente igual a $g$.

\section{Convergencia en medida}

Si para cada $x\in \rr^n$ tenemos una propiedad, enunciado o afirmaci\'on $P(x)$ que puede ser tildada de  verdadera o falsa, escribiremos $E(P)$ para denotar $E\cap \{x:\,P(x)\}$.

\begin{definicion}{}
Sean $f_k$ y $f$  medibles sobre $E$.\\
Se dice que $f_k$ converge en medida a $f$ si $\forall \delta>0$ se tiene que 
\[
m(E(|f_k-f|\geq \delta))\xrightarrow[k \to \infty]{} 0.
\]
Notaremos: $f_k \xrightarrow[]{m} f$.
\end{definicion}

\begin{teorema}{}
Si $f_k \xrightarrow[]{m}f$ y $f_k \xrightarrow[]{m} g$, entonces
$f=g$ en c.t.p.
\end{teorema}

\begin{demo}
A partir de que 
\[
\{|f-g|\geq \delta\} \subset 
\left\{|f-f_k|\geq\frac{\delta}{2}\right\} \cup \left\{|f_k-g|\geq\frac{\delta}{2}\right\}, 
\]
se tiene que 
\[
\begin{split}
&m\left(E\left(|f-g|\geq \delta\right)\right)\leq 
\\
&m\left(E\left(|f-f_k|\geq \frac{\delta}{2}\right)\right)+
m\left(E\left(|f_k-g|\geq \frac{\delta}{2}\right)\right) \xrightarrow[k \to \infty]{} 0.
\end{split}
\]
Luego, 
\[
m(\{f\neq g\})=
m\left( \bigcup\limits_{k=1}^{\infty} \left\{|f-g|\geq \frac{1}{k}  \right\} \right)=0.
\]
As\'i, $f=g$ en c.t.p.
\end{demo}


\begin{teorema}{teo:conv-puntual-implica-medida}
Si $m(E)<\infty$ y $f_k \to f$ en c.t.p. de $E$, entonces 
$f_k \xrightarrow[]{m}f$.
\end{teorema}


\begin{demo}
Sea $Z$ el conjunto de puntos donde $f_k$ no tiende a $f$. Entonces $m(Z)=0$. 

Dado $\delta>0$, definimos 
\[
B_j=\bigcup\limits_{k=j}^{\infty} E(|f_k-f|\geq \delta).
\]
 y se tiene que $ \bigcap\limits_{j=1}^{\infty} B_j\subset Z.$
 Luego, $m(B_j)\to 0$ cuando $j \to \infty$.
 
 Como si $k \geq j$, se tiene que $E(|f_k-f|\geq \delta )\subset B_j$.
Luego $m(E(|f_k-f|\geq \delta))\to 0$ cuando $k \to \infty$
y por lo tanto $f_k \xrightarrow[]{m}f.$
\end{demo}

\begin{observacion}{}
\begin{enumerate}
    \item Si $m(E)=+\infty$, el Teorema \ref{teo:conv-puntual-implica-medida} no es cierto. 
    Por ejemplo,  $f_k=\chi_{B(0,k)} \to 1$ en c.t.p., mientras que 
    $m(E(|f_k-1|=1 ))=m(\rr^n-B(0,k))= +\infty$\; $\forall k\in \nn$.
    \item La rec\'iproca del Teorema \ref{teo:conv-puntual-implica-medida} no es cierta. 
    Basta tomar $f_{k_n}=\chi_{\left[\frac{k}{2^n}, \frac{k+1}{2^n}\right)}$ con 
    $k=0,1, \ldots, 2^{n-1}$ y $n=1,2,\ldots$
\end{enumerate}
\end{observacion}{}

\begin{definicion}{}
Diremos que $f_k$ es fundamental en medida sobre $E$ si $\forall \delta>0$ se tiene que 
\[
m\left(E\left(\left|f_k-f_j\right|\geq \delta\right)\right)\xrightarrow[k,j \to \infty]{} 0.
\]
\end{definicion}

\begin{observacion}{}
Si $f_k \xrightarrow[]{m}f$ y $f$ es finita, entonces $f_k$ es fundamental en medida. 
\end{observacion}



\begin{teorema}{}
Si $f_k$ es fundamental en medida sobre $E$, entonces existe una subsucesi\'on $k_j$ y una funci\'on $f$ medible sobre $E$ tal que $f_{k_j}\to f$ en c.t.p. de $E$. Adem\'as, $f$ es finita y $f_{k} \xrightarrow[]{m}f$.
\end{teorema}

\begin{demo}
Para cada $i>0$, existe $k_i\in \nn$ tal que 
\[
m\left(E\left(|f_k-f_j|\geq \frac{1}{2^i}\right)\right)\leq
\frac{1}{2^i},
\]
para $k,j\geq k_i$.
Podemos suponer que $k_1<k_2<\ldots$. Sea 
\[
E_i=E\left(   |f_{k_i} -f_{k_{i+1}}|\geq \frac{1}{2^i}\right)
\]
Tenemos que $m(E_i)<\frac{1}{2^i}$. 

Sea 
\[ Z=\bigcap\limits_{j=1}^{\infty}\bigcup\limits_{i=j}^{\infty} E_i
=\limsup\limits_{i \to \infty} E_i.
\]
Ahora, $m(Z)=0$. 

Si $x \in E-Z$, existe un $j$ tal que $x \notin E_i$ para $i\geq j$, es decir, 
\[
x\in E\left( \left|f_{k_i}-f_{k_{i+1}}\right|<\frac{1}{2^i}\right).
\]
Luego, la serie
\begin{equation}\label{eq:serie-telesc-conv-med-a-puntual}
f_{k_1}(x)+\left(f_{k_2}(x)-f_{k_1}(x)\right)+\ldots
\end{equation}
converge absolutamente en $E-Z$.

Sea $f(x)$ la suma de \eqref{eq:serie-telesc-conv-med-a-puntual} en $E-Z$ y sea $f(x)=0$ en $Z$. A partir de la definici\'on de $f$ es claro que $f$ es finita. Adem\'as, pasando a sumas parciales, tenemos
\[
f(x)=\lim\limits_{i \to \infty} f_{k_i}(x)\;\mbox{ en c.t.p. de } \,E.
\]

A continuaci\'on, veamos que $f_{k_i} \xrightarrow[]{m} f$.

Sea $\delta>0$ y elijamos $j$ tal que $\frac{1}{2^{j-1}}<\delta$. 
Si  $x\notin Z$, entonces 
\[
f(x)=
f_{k_j}(x)+\left(f_{k_{j+1}}(x)-f_{k_j}(x)\right)+\ldots.
\]
As\'i
\[
E\left(\left|f(x)-f_{k_j}(x)  \right|\geq \delta\right)\subset 
Z\cup \left(\bigcup\limits_{i\geq j} E_i\right)
\]
de donde
\[
m\left(
E\left(\left|f(x)-f_{k_j}(x)  \right|\geq \delta\right)
\right)
\leq \sum\limits_{i\geq j} m(E_i)=\frac{1}{2^{j-1}}.
\]
De este modo, obtenemos $f_{k_j} \xrightarrow[]{m} f$.

Por \'ultimo, a partir de 
\[
E\left(|f_k-f|\geq\delta\right)\subset 
E\left(|f_k -f_{k_j}|\geq \frac{\delta}{2}\right) \cup
E\left(|f_{k_j} -f|\geq \frac{\delta}{2}\right),
\]
tomando $k$ y $k_j$ grandes, deducimos
\[
m\left(E\left(\left|f_k-f\right|\geq \delta\right)\right)< \epsilon
\]
para valores de $k$ grandes. En consecuencia, $f_k \xrightarrow[]{m}f$.
\end{demo}

\section{Funci\'on singular de Cantor}

El conjunto de Cantor se define como 
\[
P=\bigcap\limits_{n=1}^{\infty} F_n,
\]
donde $F_n$ es la uni\'on de $2^n$ intervalos cerrados y disjuntos contenidos en el $[0,1]$.

El conjunto $[0,1]-F_n$ es la uni\'on de $2^n-1$ intervalos abiertos disjuntos. Si los numeramos de izquierda a derecha, formamos los intervalos abiertos $J_{n,i}$, para $i=1,2,\ldots, 2^n-1$ y se tiene la relaci\'on
\[ 
J_{n,i}=J_{n+1,2i}.
\]

Sea $\varphi_n$ la funci\'on que toma los siguientes valores
$\varphi_n(0)=0$, $\varphi_n(1)=1$, $\varphi_n(x)=\frac{i}{2^n}$ en $J_{n,i}$, es lineal entre los $F_n$ y es continua.

Tenemos que $\varphi_{n+1}=\varphi_n$ en  $J_{n,i}$   y adem\'as
\[
\left|\varphi_{n+1} -\varphi_n\right|<\frac{1}{2^{n+1}},
\]
en cada punto de $[0,1]$.
Luego, la serie
\[
\varphi_1+(\varphi_2-\varphi_1)+(\varphi_3-\varphi_2)+\ldots
\]
converge uniformemente a una funci\'on continua $\varphi$ que se llama 
\emph{funci\'on singular de Cantor}. 

Es claro que $\varphi$ es mon\'otona creciente y su restricci\'on a cualquiera de los intervalos $J_{n,i}$ es constante.


  \chapter{Integral de Lebesgue}

\section{Definici\'on y propiedades inmediatas}

\begin{definicion}{defi:integral-de-Lebesgue}
Sean $E\subset \rr^n$,  $f:\rr^n\to \overline{\rr}$  y $f\geq 0$ sobre $E$ medible. 
\\
La integral de Lebesgue de $f$ se define  mediante
\[
\int_E f(x)\.dx=\sup\left\{ \sum\limits_{i=1}^{N} m(E_i) \right\},
\]
donde el supremo se toma sobre toda descomposici\'on del conjunto $E$ en uni\'on de conjuntos medibles $E_i$ y mutuamente disjuntos,  siendo 
\[\alpha_i =\inf\limits_{E_i}f.\]
\end{definicion}

Se usa la convenci\'on $0{.}(+\infty)=+\infty{.}0=0.$

De la Definici\'on \ref{defi:integral-de-Lebesgue} se deduce que si $f\equiv c\in \rr$ sobre $E$, 
entonces
\[ \int f\,dx=c m(E).\]

\begin{teorema}{}
Si $E=A\cup B$ y $f\geq 0$ en $E$, entonces
\[
\int_E f=\int_A f +\int_B f.
\]
\end{teorema}

\begin{demo}
Sean $E=\bigcup\limits_{i=1}^N E_i$ y $\alpha_i=\inf\limits_{E_i} f$.
  $A_i=A\cap E_i$ y $B_i=B\cap E_i$ y llamamos $\beta_i=\inf\limits_{A_i} f$ y $\gamma_i=\inf\limits_{B_i} f$. 
Entonces $\alpha_i \leq \beta_i$, $\alpha_i\leq \gamma_i$ y 
\[
\begin{split}
\sum\limits_{i=1}^N \alpha_i m(E_i)=
&\sum\limits_{i=1}^N \alpha_i (m(A_i)+m(B_i))
\\
\leq & 
\sum\limits_{i=1}^N \beta_i m(A_i)+\sum\limits_{i=1}^N \gamma_i m(B_i)
\\
\leq & \int_A f\,dx +\int_B f\,dx. 
\end{split}
\]
Luego 
\[\int_E f \leq \int_A f +\int_B f.
\]
Sean $A=\bigcup\limits_{i=1}^N A_i$, $B=\bigcup\limits_{i=1}^M B_i$, $\beta_i$ y $\gamma_i$ como antes. 
Entonces 
\[
\sum \limits_{i=1}^N \beta_i m(A_i) +\sum\limits_{i=1}^M \gamma_i m(B_i) \leq \int_E f.
\]
Luego
\[
\int_A f + \int_B f \leq \int_E f.
\]
\end{demo}

\begin{itemize}
    \item Si $0\leq f\leq g$, entonces 
    \[\int_E f \leq \int_E g,\] 
    pues $\inf\limits_{E_i} f \leq \inf\limits_{E_i} g.$
    \item Si $f\geq 0$ sobre $E$ y $A\subset E$, entonces \[\int_A f \leq \int_E f,\] pues 
    $\int_E f=\int_A f +\int_{E-A} f$.
    \item Si $m(E)=0$ y $f\geq 0$ sobre $E$, entonces
    \[\int_E f =0.\]
    \item Si $f=g$ en c.t.p. de $E$, entonces
    \[\int_E f =\int_E g.\]
    \begin{demo}
    Sea $A=\{f \neq g\}$ entonces $m(A)=0$ y 
    \[\int_E f =\int_{E-A} f =\int_{E-A} g=\int_E g.\]
    \end{demo}
    \item Si $f\geq 0$ en $E$, entonces
    \[\int_E  f = \int_{\rr^n}  f \chi_E\,dx.\]
    \item 
    Cuando $E= \rr^n$, escribimos 
    \[\int_{\rr^n} f = \int f.\]
    \item Si $E=(a,b)\subset \rr$, ponemos
     \[\int_{E} f = \int_a^b f.\]
    \end{itemize}
    
    \section{Integral de funciones simples}
    
    \begin{teorema}{}
    Sean $E=E_1 \dot{\cup} E_2\dot{\cup} \ldots \dot{\cup} E_N$ y 
    $\beta_i \in \rr,\; i=1,2,\ldots,N$, $\beta_i\geq 0$. Luego, 
    \[
    \int_E \sum\limits_{i=1}^N \beta_i \chi_{E_i}=
    \sum\limits_{i=1}^N \beta_i m(E_i)
    \]
    \end{teorema}
    
    \begin{demo}
    \[
    \int_E \sum\limits_{i=1}^N \beta_i \chi_{E_i}=
    \sum\limits_{j=1}^N \int_{E_j} \sum\limits_{i=1}^N \beta_i \chi_{E_i} =\sum\limits_{j=1}^N \beta_j m(E_j).
    \]
    \end{demo}
    
    \begin{teorema}{}
    Si $\varphi$  y $\psi$ son funciones simples no negativas y $c\in \rr^+$, entonces
    \[\begin{split}
            \int_E (\varphi+\psi) &= \int_E \varphi +\int_E \psi \;\;\mbox{  y   }\;\;\\
        \int_E c\varphi &=c\int_E \varphi.
        \end{split}
    \]
    \end{teorema}
    
    \begin{demo}
    Sean $\varphi=\sum\limits_{i=1}^N \alpha_i \chi_{E_i}$ y $\psi=\sum\limits_{j=1}^M \beta_j \chi_{F_j}$. 
    Tenemos 
    \[
    \varphi+\psi=
    \sum\limits_{i=1}^N\sum\limits_{j=1}^M (\alpha_i+\beta_j) \chi_{E_i \cap F_j},
    \]
    y entonces
    \[
    \begin{split}
    \int_E \varphi+\psi=& 
    \sum\limits_{i=1}^N\sum\limits_{j=1}^M (\alpha_i+\beta_j) m(E_i \cap F_j)
    \\
    =&\sum\limits_{i=1}^N \alpha_i \sum\limits_{j=1}^M m(E_i \cap F_j)+
    \sum\limits_{j=1}^M \beta_j \sum\limits_{i=1}^N m(E_i \cap F_j)
    \\
    =&\int_E \varphi +\int_E \psi.
    \end{split}
        \]
    Como $c\varphi=\sum\limits_{i=1}^N c \alpha_i \chi_{E_i}$, luego
    \[
    \int_E c\varphi = \sum\limits_{i=1}^N c\alpha_i m(E_i)=\alpha \int_E \varphi.
    \]
    \end{demo}
    
    \begin{teorema}{}
    Si $f\geq 0$, entonces
    \[\int_E f = \sup\limits_{0\leq \varphi\leq f} \int_E \varphi.\]
    \end{teorema}
    
    \begin{demo}
    Sean $E=E_1\cup E_2\cup\ldots \cup E_N$ y $\alpha_i=\inf\limits_{E_i} f$. 
    Supongamos que $\alpha_i<\infty$. Luego, si $\varphi=\sum\limits_{i=1}^N \alpha_i \chi_{E_i}$ tenemos 
    \[
    \int_E \varphi =\sum\limits_{i=1}^N \alpha_i m(E_i)\;\;
    \mbox{ y }\;\; 0\leq \varphi \leq f.
    \]
    En general, para $k\in \nn$ definamos 
    \[\alpha_{ik}=\min\{k,\alpha_i\}\]
    y 
    \[ \varphi_k=\sum\limits_{i=1}^n \alpha_{ik} \chi_{E_i}.\]
    Ahora, $0\leq \varphi_k \leq f$ y $\alpha_{i1}\leq \alpha_{i2}\leq \ldots \nearrow \alpha_i$. 
    LUego 
    \[
    \begin{split}
    \sum\limits_{i=1}^N \alpha_i m(E_i)=&
    \lim\limits_{k \to \infty} \sum\limits_{i=1}^N \alpha_{ik} m(E_i)
    \\
    =&\lim\limits_{k \to \infty} \int_E \varphi_k 
     \leq  \sup\limits_{0\leq \varphi \leq f} \int_E \varphi.
    \end{split}
    \]
    Luego
    \[
    \int_E f \leq \sup\limits_{0\leq \varphi \leq f} \int_E \varphi.
    \]    
    La otra desigualdad es inmediata.
    \end{demo}
    
    
    \begin{lema}{}
    Sean $f_1\leq f_2\leq \ldots$ funciones no negativas y $\varphi$ funci\'on simple no negativa tal que 
    \[
    \varphi \leq \lim\limits_{k \to \infty} f_k.
    \]
    \end{lema}
    
\begin{demo}
Sean $0\leq \alpha_1< \alpha_2<\ldots < \alpha_N$ los distintos valores ordenados que toma $\varphi$ y sea 
\[  
A_i=\{\varphi=\alpha_i\}.
\]
Supongamos probado el lema cuando $\alpha_1>0$. Entonces, sale para $\alpha_1=0$, pues aplicando el caso probado en $E-A_1$ tenemos 
\[
\int_E \varphi =\int_{E-A_1}\varphi 
\leq \lim\limits_{k\to \infty} \int_{E-A_1} f_k
\leq \lim\limits_{k \to \infty} \int_E f_k.
\]
Supongamos ahora que $\alpha_1>0$.\\ Sea $0<\epsilon$ con $0<\epsilon<\alpha_1$.
Luego $\varphi-\epsilon$ es una funci\'on simple que toma los valores $\alpha_i-\epsilon$ sobre $A_i$. 
Poniendo 
\[
E_k=\{\alpha \in E: f_k(x)> \varphi (x)-\epsilon \}.
\]
Por hip\'otesis
\[
E_k \subset E_{k+1}  \;\mbox { y }\; \bigcup\limits_{k=1}^{\infty}E_k
=E.\]
Luego $m(E_k) \to m(E)$ cuando $k \to \infty$.

Se presentan dos casos
\begin{enumerate}
    \item Si $m(E)<\infty$, tenemos $m(E-E_k) \xrightarrow[]{k \to \infty}0$
     y 
    \[
    \begin{split}
    \int_E f_k \geq \int_{E_k} f_k &\geq \int_{E_k} \varphi(x) -\epsilon
    \\
    &=\int_E \varphi(x)-\epsilon -\int_{E-E_k}\varphi(x)-\epsilon
    \\
    &\geq \int_E \varphi -\epsilon m(E)-(\alpha_N-\epsilon)m(E-E_k).
    \end{split}
         \]
    Luego 
    \[
    \lim\limits_{k \to \infty} \int_E f_k \geq \int_E \varphi-\epsilon m(E).
    \]
    Como $\epsilon$ es arbitrario, se obtiene la desigualdad del lema.
    \item Si $m(E)=\infty$, consideramos $m(E_k)\nearrow +\infty$
    y obtenemos
    \[
    \int_E f_k \geq \int_{E_k} f_k 
    \geq \int_{E_k} \varphi-\epsilon \geq (\alpha_1-\epsilon)m(E_k)\nearrow +\infty, 
    \]
    lo cual lleva a la desigualdad del lema.
\end{enumerate}
\end{demo}

\section{Paso al l\'imite bajo el signo de integral}

\begin{teorema}{teo:Beppo-Levi}[Beppo-Levi]
Sea $0\leq f_1\leq f_2\leq\ldots \nearrow f$. 
Entonces
\[ \int_E f =
\lim\limits_{k \to \infty} \int_E f_k.\]
\end{teorema}
    
    \begin{demo}
    Sea $\varphi$ funci\'on simple tal que $0\leq \varphi \leq f$. Luego
    \[ \varphi \leq \lim\limits_{k \to \infty} f_k,\]
    y por tanto
    \[
\int_E \varphi \leq \lim\limits_{k \to \infty} \int_E f_k.
    \]
    As\'i, llegamos a 
    \[
    \int_E f=
    \sup\limits_{0\leq \varphi\leq f} \int_E \varphi
    \leq \lim\limits_{k \to \infty} \int_E f_k.
    \]
La otra desigualdad es inmediata. 
    \end{demo}
    
    
    \begin{teorema}{}
    Si $f,g\geq 0$ y $c \in \rr$, entonces
    \[ 
    \int_E (f+g)= \int_E f+\int_E g \;\;\;\mbox{ y }\;\;\;
    \int_E cf=c\int_E f.
    \]
    \end{teorema}
    
     \begin{demo}
     Sean $\varphi_k$ y $\psi_k$ funciones simples tales que $\varphi_k \nearrow f$ y $\psi_k \nearrow g$.
     Luego $\varphi_k +\psi_k \nearrow f+g$ y entonces
     \[
     \begin{split}
         \int_E f+g =&\lim\limits_{k \to \infty} \int_E \varphi_k +\psi_k
         \\
         =&\lim\limits_{k \to \infty}\int_E \varphi_k
         +
         \lim\limits_{k \to \infty}\int_E \psi_k
         \\
         =&\int_E f +\int_E g.
     \end{split}
     \]
          \end{demo}
          
          \begin{corolario}{cor:Beppo-Levi}
          Si $f_k\geq 0$, $k=1,2,\ldots,$ entonces
          \[
          \int_E \sum\limits_{k=1}^{\infty} f_k
          =
          \sum\limits_{k=1}^{\infty} \int_E f_k.
          \]
          \end{corolario}
          
          \begin{demo}
          Sean $S_N=\sum\limits_{k=1}^N f_k$ y 
          $s=\sum\limits_{k=1}^{\infty} f_k=\lim\limits_{N \to \infty} S_N$. \\Luego
          $0\leq S_1 \leq S_2 \leq \ldots \nearrow s$. 
          De este modo, 
          \[
          \begin{split}
          \int_E s=&\lim\limits_{N \to \infty} \int_E S_N
          \\
          =&\lim\limits_{N \to  \infty} \sum\limits_{k=1}^N \int_E f_k
          = \sum\limits_{k=1}^{\infty} \int_E f_k.
          \end{split}
          \]
                    \end{demo}
                    
                    
                    \begin{lema}{lema:Fatou}[Lema de Fatou]
                    Si $f_k\geq 0$, entonces
                    \[
                    \int_E \liminf\limits_{k \to \infty} f_k
                    \leq 
                    \liminf\limits_{k\to \infty} \int_E f_k.
                    \]
                    \end{lema}
                    
                    \begin{demo}
                    Sean 
                    \[
                    g(x)=\liminf\limits_{k \to \infty} f_k\;\;
                    \mbox{  y }\;\;
                      g_k(x)=\inf\limits_{j\geq k} f_j.
                    \]
                    As\'i, 
                    $g_1\leq g_2\leq \ldots \nearrow g $ y 
                    \[ g(x)=\sup\limits_{k} g_k(x)=\lim\limits_{k \to \infty} g_k(x).  
                    \]
                    Por el Teorema \ref{teo:Beppo-Levi} y como $g_k \leq f_k$, se tiene 
                    \[
                    %\begin{split}
                    \int_E g=\lim\limits_{k \to \infty} \int_E g_k
                    =\liminf\limits_{k \to \infty} \int_E g_k
                   % \\
                    \leq \liminf\limits_{k \to \infty} \int_E f_k.
                    %\end{split}
                    \]
                    \end{demo}
                    
                    
                    \begin{corolario}{cor:Fatou}
                    Si $f_k \to f$ en $E$, entonces
                    \[
                    \int_E f \leq \liminf\limits_{k \to \infty}\int_E f_k.
                    \]
                    \end{corolario}
                    
                    \section{Integrales de funciones de distinto signo}
                    
                    \begin{definicion}{}
                    Si $f=f^+-f^-$, diremos que $f$ es integrable sobre $E$ si y s\'olo si
                    \[
                    \int_E f^+ \;\;\mbox{ y }\;\;
                    \int_E f^-
                    \]
                    son finitas. 
                    
                    En este caso, escribimos
                    \[
                    \int_E f =\int_E f^+ - \int_E f^-.
                    \]
                    \end{definicion}
                    
                    \begin{teorema}{}
                    $f$ es integrable sobre $E$ si y s\'olo si $|f|$ lo es.  Y, en este caso, vale
                    \[
                    \left|\int_E f\right|\leq \int_E |f|.
                    \]
                    \end{teorema}
                    
                    \begin{demo}
                    $\Rightarrow)$
                    Como $|f|=f^++f^-$, si $f$ es integrable entonces $|f|$ tambi\'en los es. 
                    
                    $\Leftarrow)$
                    Si $|f|$ es integrable, como $f^+\leq |f|$ y $f^-\leq |f|$, entonces $f$ tambi\'en resulta integrable.
                    \end{demo}
                    
                    \begin{teorema}{teo:diferencia-de-integrables}
                    Si $f_1,f_2\geq 0$ son integrables sobre $E$ y $f=f_1-f_2$, entonces $f$ es integrable y 
                    \[\int_E f_1-f_2 = \int_E f_1 - \int_E f_2.\]
                    \end{teorema}
                    
                    \begin{demo}
                    A partir de  que $f^+\leq f_1$ y $f^-f_2$ se deduce que $f$ es integrable.  
                    \\
                    Como $f=f^+-f^-=f_1-f_2$ entonces 
                    $f^+ + f_2=f_1 + f^-$ y 
                    \[
                    \int_E f^+ + \int_E f_2=
                    \int_E f_1 + \int_E f^-.
                    \]
                    \end{demo}
                    
                    \begin{teorema}{}
                    Si $f\geq 0$ es integrable y $|g|\leq f$, entonces $g$ es integrable. 
                    \end{teorema}
                    
                    \begin{demo}
                     La prueba sale a partir de que 
                    $g^+\leq f$ y $g^- \leq f$.
                    \end{demo}
                   
                   \begin{teorema}{}
                   Si $f$ y $g$ son integrables sobre $E$ y $c\in \rr$, entonces $f+g$ y $cf$ son integrables sobre $E$. Adem\'as, 
                   \[
                   \int_E f+g =\int_E f +\int_E g
                   \;\;\mbox{ y }\;\;
                   \int_E cf =c\int_E f.
                   \]
                 \end{teorema}
                    
                    \begin{demo}
                     Como $f+g=f^+ + g^+ -(f^-+g^-)$, aplicando el Teorema \ref{teo:diferencia-de-integrables} se obtiene que $f+g$ es integrable y         \[
                     \begin{split}
                     \int_E f+g =&\int_E f^{+} +g^{+}  - \int_E f^{-} + g^{-}
                     \\
                     =&\int_E f^{+} +\int_E g^{+}  - \int_E f^{-} -\int_E g^{-}
                     \\
                     =&\int_E f + \int_E g.
                     \end{split}
                     \]
                     
                     Si $c\geq 0$, entonces $cf=cf^{+}-cf^{-}$ y el resultado se obtiene por el Teorema \ref{teo:diferencia-de-integrables}.
                     
                     Si $c<0$, el resultado se obtiene a partir de que $cf=(-c)f^{-}-(-c)f^{+}$.
                    \end{demo}
                    
                    \begin{corolario}{}
                    Si $f$ y $g$ son integrables tales que $f\leq g$, entonces 
                    \[\int_E f \leq \int_E g.\]
                    \end{corolario}
                    
                    
                    As\'i, 
                    \[
                    L(E)=\{
                    f: f\;\mbox{ es integrable sobre }\;E
                    \}
                    \]
                    es un espacio vectorial y la aplicaci\'on
                    \[
                    f \longmapsto \int_E f
                    \]
                    es una aplicaci\'on lineal.
                    
                    \section{Convergencia Mayorada}
                    
                    
                    Si $f_k$ son integrables sobre $E$ y $f_k \to f$ en c.t.p. de $E$ siendo $f$  integrable en $E$, \textbf{en general no es cierto} que 
                    \[
                    \int_E f_k \to \int_E f.
                    \]
                    
                    \begin{ejemplo}{}
                    Si $f_k=k \chi_{(0,\frac{1}{k})}$ entonces $f_k \to 0$ puntualmente en $(0,1)$. Sin embargo,
                     \[
                    \int_{(0,1)} f_k = 1,\;\; \forall k\in \nn.
                    \]
                     y por lo tanto 
                     \[
                    \int_{(0,1)} f_k  \nrightarrow \int_{(0,1)} f.
                    \]
                    \end{ejemplo}
                    
                    \begin{teorema}{}[Convergencia Mayorada de Lebesgue]
                    Sea $\Phi$ func\'ion integrable sore $E$. 
                    Si $f_k$ son funciones integrables tales que 
                    \[|f_k|\leq \Phi\;\mbox{ en }\;E,\]    
                    entonces
                    $g=\liminf f_k$, $h=\limsup
                    f_k$ son integrables sobre $E$ y 
                    \[
                    \int_E \liminf f_k \leq 
                    \liminf \int_E f_k \leq\limsup \int_E f_k
                    \leq \int_E h.
                    \]
                    \end{teorema}
                    
                    \begin{demo}
                    Por hip\'otesis se tiene que $-\Phi \leq f_k \leq \Phi$, a partir de lo cual se deduce que 
                    $-\Phi \leq g\leq \Phi$ y  $-\Phi \leq h\leq \Phi$ y por lo tanto $g$ y $h$ resultan integrables sobre $E$.
                    
                    Por otra parte,  $f_k+\Phi \geq 0$ y $\Phi-f_k\geq 0$ y 
                    \[\begin{split}
                    \liminf\limits_{k \to \infty} (f_k+\Phi)=g+\Phi
                    \\
                    \liminf\limits_{k \to \infty} (\Phi-f_k)=\Phi-h.
                    \end{split}
                    \]
                    Ahora, por el Lema de Fatou (Lema \ref{lema:Fatou}) tenemos
                    \[
                    \int_E g + \int_E \Phi =
                    \int_E g+\Phi \leq \liminf\limits_{k \to \infty} \int_E f_k +\Phi=
                    \liminf\limits_{k \to \infty} \int_E f_k +\int_E \Phi.
                    \]
                    Luego 
                    \[
                    \int_E g \leq \liminf \int_E f_k.
                    \]
                    La otra desigualdad se obtiene de manera an\'aloga.
                    \end{demo}
                    
                    \begin{corolario}{cor:conv-mayorada}
                    Si $f_k \to f$ en cada punto de $E$ y $|f_k|\leq \Phi \in L(E)$, entonces $f \in L(E)$ y 
                    \[
                    \lim\limits_{k \to \infty}\int_E f_k = \int_E f.
                    \]
                    \end{corolario}
                    
                    
                    \section{La integral y los conjuntos de medida nula}
                    
                    Si $f,g\geq 0$ y $f=g$ en c.t.p de $E$, entonces
                    \[
                    \int_E f =\int_E g.
                    \]
                    
                    En particular, sobre un conjunto de medida nula cualquier $f$ medible e integrable  con integral que vale 0.
                    
                    Usaremos la \textbf{desigualdad de Chebyshev} que establece que si $f$ es medible sobre $E$ se cumple 
                    \[
                    m\left(\{ x\in E: |f(x)|>\lambda\}\right) \leq \frac{1}{\lambda} \int_E |f|.
                    \]
                    La prueba de la desigualdad de Chebyshev es sencilla, a saber, 
                    \[
                    \int_E |f| \geq \int_{\{|f|\geq \lambda\}} |f|
                    \geq \lambda m\left(\{|f|\geq \lambda\}\right).
                    \]
                    
                    \begin{teorema}{}
                    Si $f\geq 0$ sobre $E$ y $\int_E f=0$, entonces $f=0$ en c.t.p. de $E$.
                    \end{teorema}
                    
                    \begin{demo}
                    Sean $Z_k=\{f >\frac{1}{k}\}$ y $Z=\{f>0\}$, entonces 
                    \[
                    Z=\bigcup\limits_{k=1}^{\infty} Z_k.
                    \]
                    Ahora, 
                    \[
                    m(Z_k)=m\left(\left\{f>\frac{1}{k}\right\}\right)\leq k \int_E f=0,
                    \]
                    y en consecuencia  $m(Z)=0$.
                    \end{demo}
                    
                    \section{ Invariancia bajo traslaciones }
                    
                    \begin{teorema}{}
                Sea $f\geq 0$, entonces $\forall h \in \rr^n$ se tiene
                \begin{enumerate}
                    \item \label{it:invariancia-f-x+h} $\int f(x+h) =\int f(x)$,
                    \item \label{it:invariancia-f-y-E-x+h} $\int_E f(x+h) =\int_{E+h} f(x)$.
                \end{enumerate}
                    \end{teorema}
                    
                    \begin{demo}
                    \begin{enumerate}
                        \item 
                    \begin{itemize}
                        \item Si $f=\chi_E$, entonces 
                        $f(x+h)=\chi_E (x+h) =\chi_{E-h}(x).$
                        As\'i, 
                        \[
                        \int f(x+h)= \int \chi_{E-h} (x) =m(E-h)
                        \]
                        y 
                        \[
                        \int f(x) = \int \chi_E(x)=m(E),
                        \]
                        y son iguales
                        \item Si $f$ es simple, entonces 
                        \[  f=\sum\limits_{i=1}^N \alpha_i f_i, \;\; \alpha_i\geq 0,\;f_i=\chi_{E_i}.
                        \]
                        Luego, 
                        \[
                        \int f(x+h) =\sum\limits_{i=1}^N \alpha_i \int f_i(x+h)=
                        \sum\limits_{i=1}^N \alpha_i \int_{E_i} f_i= \int f.
                        \]
                        \item Sea $f$ arbitraria no negativa y sean $\varphi_k$ funciones simples no negativas tales que $\varphi_k \nearrow f$. Luego, 
                        \[
                        \varphi_k(x+h)\to f(x+h), \;\;\forall x \in \rr^n.
                        \]
                        Por el Teorema de Beppo-Levi (Teorema \ref{teo:Beppo-Levi}), se tiene
                        \[
                        \int f(x+h)\,dx=\lim\limits_{k \to \infty} \int \varphi_k(x+h)\,dx=
                        \lim\limits_{k\to \infty} \int \varphi_k(x) =\int f(x)\,dx.
                        \]
                    \end{itemize}
                    As\'i queda demostrado el item \ref{it:invariancia-f-x+h}.
                    \item es consecuencia del item \ref{it:invariancia-f-x+h}. En efecto, 
                    \[\begin{split} 
                    \int_E f(x+h) \,dx=& \int \chi_{E+h}(x+h) f(x+h) \,dx
                    \\
                    =&
                    \int \chi_{E+h} (x) f(x)\,dx 
                    \\
                    =&
                    \int_{E+h} f(x)\,dx.
                    \end{split}\]
                    \end{enumerate}
                    Por \'ultimo, los item \ref{it:invariancia-f-x+h} y   \ref{it:invariancia-f-y-E-x+h} son ciertos para $f\in L(E)$,  a partir de la usual descomposici\'on de $f$ dada por $f=f^{+}-f^{-}$.
                    \end{demo}
                    
                    \section{La integral como funci\'on de conjunto}
                    
                    \textbf{FALTA DEFINIR $\mathcal{M}$ o RECORDARLO!!!} 
                    
                    Sea $f\in L(\rr)$ y definimos $\Phi: \mathcal{M} \to \rr$
                    por
                    \[
                    \Phi(E)=\int_E f.
                    \]
                    
                    $\Phi$ se llama integral indefinida de $f$.
                    
                    \begin{teorema}{}
                    Si $E_j \in \mathcal{M}$ son mutuamente disjuntos y sea
                    $E=\bigcup\limits_{j=1}^{\infty}E_j$, entonces
                    \[\Phi(E)=\sum\limits_{j=1}^{\infty} \Phi(E_j).\]
                    \end{teorema}
                    
                    \begin{demo}
                    Si $f\geq 0$, tenemos 
                    $\sum\limits_{j=1}^{\infty} \chi_{E_j}=\chi_E$ y luego
                    \[
                    \Phi(E)=\int_E f = \int f \chi_E =\int \sum\limits_j f \chi_{E_j}=\sum\limits_{j} \int f \chi_{E_j}=\sum\limits_j \Phi(E_j).
                    \]
                    Si $f \in L(E)$, trabajamos con  $f=f^{+}-f^{-}$ donde 
                    $f^+$ y $f^-$ son funciones medibles no negativas.
                    \end{demo}
                    
                    \begin{definicion}{}
                    Si $X$ es un conjunto y $\Sigma$ es una sigma-\'algebra de subconjuntos de $X$. 
                    Una funci\'on $\mu:\Sigma \to \rr^+$ se llama medida si
                    \begin{enumerate}
                        \item $\mu(\emptyset)=0,$
                        \item $\mu(\bigcup\limits_j E_j) = \sum\limits_j \mu(E_j)$, donde $E_j \in \Sigma$ y son mutuamente disjuntos.
                    \end{enumerate}
                    \end{definicion}
                    
                    
                    Para una medida valen los teoremas de convergencia mon\'otona de conjuntos, se puede definir una integral y \emph{casi todo} lo visto para la medida de Lebesgue es v\'alido. Una propiedad que no siempre es cierta es la invariancia por traslaciones.
                    
                    Ahora bien,  $\Phi$ es una medida. 
                    La pregunta que surge es ?`ser\'a toda medida sobre
                    $\mathcal{M}$ de la forma de $\Phi$ para alguna $f$?\\
                    La respuesta a esta pregunta ser\'a dada por el \textbf{Teorema de Radom-Nikodim}.
                    
                    
                    La siguiente propiedad se llama \emph{continuidad absoluta}.
                    
                    \begin{teorema}{}
                    Sea $f \in L(\rr^n)$, entonces $\forall \epsilon>0$ \;$\exists \delta >0$ tal que 
                    \[
                    m(E)<\delta \Longrightarrow |\Phi(E)|<\epsilon.
                    \]
                    \end{teorema}
                    
                    \begin{demo}
                    Se puede suponer que $f\geq 0$. \\
                    Sea $f_k=\min\{f,k\}$, entonces $f_k \nearrow f$.
                    Por el Teorema de Beppo-Levi (Teorema \ref{teo:Beppo-Levi}) se tiene que 
                    $\int_E f_k \nearrow \int_E f$ y por tanto
                    $
                    \int_E f-f_k \to 0.
                    $
                    As\'i, existe $k \in \nn$ tal que 
                    \[
                    \int_E f-f_k < \frac{\epsilon}{2}.
                    \]
                    Sea $\delta <\frac{\epsilon}{2k}$. Luego, si $m(E)<\delta$ entonces 
                    \[
                    \int_E f =\int_E f-f_k +\int_E f_k <\frac{\epsilon}{2}+km(E)<\epsilon. 
                    \]
                    \end{demo}
                    
                    \section{Comparaci\'on con la integral de Riemann}
                    
                    Sea $f$ acotada en $[a,b]\subset \rr$. 
                    Si $a=x_0<x_1<x_2<\ldots<x_N=b$, llamamos suma inferior de Riemann $s$ y suma superior de Riemann $S$ a 
                    \[
                    s=\sum\limits_{i=1}^N m_i (x_i-x_{i-1})  \;\;\mbox{ y }\;\;
                     S=\sum\limits_{i=1}^N M_i (x_i-x_{i-1}),
                    \]
                    respectivamente, donde 
                    \[
                    m_i=\inf\limits_{[x_{i-1},x_i]} f\;\;\mbox{ y }\;\;
                    M_i=\sup\limits_{[x_{i-1},x_i]} f.
                    \]
                    
                    Una funci\'on $f$ se llama integrable seg\'un Riemann si y s\'olo si
                    $\forall \epsilon>0$ existe una partici\'on para la cual 
                    \[
                    S-s<\epsilon.
                    \]
                    
                    
                    La integral de Riemann se define 
                    \[ \R\, \int_a^b f =\sup s = \inf S. \]
                    
                    \begin{teorema}{}
                    Si $f$ es integrable Riemann sobre $[a,b]$, entonces
                    $f$ es medible e integrable  Lebesgue. Adem\'as, 
                    \[
                    \R  \,\int_a^b f = \int_a^b f.
                    \]
                    \end{teorema}
                    
                    \begin{demo}
        Si $a=x_0<x_1<x_2<\ldots<x_N=b$, definimos las funciones escalonadas 
        \[
        \varphi(x)=\sum\limits_{i=1}^N m_i \chi_{J_i} \;\;\mbox{ y }\;\;
        \psi(x)=\sum\limits_{i=1}^N M_i \chi_{J_i} 
        \]
        donde 
        $J_i=[x_{i-1},x_i]$. Entonces $\varphi \leq f \leq psi$ en c.t.p. de $[a,b]$.
        
        Si $f$ es integrable Riemann, existen dos sucesiones de funciones escaleras $\varphi_k$ y $\psi_k$ tales que 
        $\varphi_k \leq f \leq \psi_k$ y 
        \[
        \int_a^b \psi_k -\varphi_k <\frac{1}{k}.
        \]
        Adem\'as, si $g=\sup\varphi_k$ y $h=\inf \psi_k$, entonces
        $g$ y $h$ son borelianas y $g\leq f\leq h$.
        Adem\'as
        \[
        \int_a^b \varphi_k \leq \int_a^b f \leq \int_a^b \psi_k
        \]
        y 
        \[
        \R\,\int_a^b \varphi_k \leq\; \R\, \int_a^b f \leq\; \R\, \int_a^b \psi_k,
        \]
        de donde
        \[
        \left|
        \int_a^b f -\; \R\,\int_a^b f
        \right|\leq 
        \int_a^b \psi_k - \varphi_k <\frac{1}{k}.
        \]
                            \end{demo}
                            
                            \section{Integraci\'on parcial: Teorema de Fubini}
                            
        Si $u \in \rr^{n+m}$, pondremos $u=(x,y)$
        con $x \in \rr^n$ e $y \in \rr^m$. 
        
        Si $E\subset \rr^{n+m}$ e $y\in \rr^n$, entonces
        \[
        E_x=\{y \in \rr^m: (x,y)\in E\},
        \]
        se llama la \emph{secci\'on} de $E$ en $x$.
                An\'alogamente, se define $E_y$.
        
        Se puede demostrar que 
        \[
        \left(\bigcup\limits_{k=1}^{\infty} E_k \right)_x
        = \bigcup\limits_{k=1}^{\infty} \left(E_k\right)_x\]
        y
        \[
        \left(\bigcap\limits_{k=1}^{\infty} E_k \right)_x
        = \bigcap\limits_{k=1}^{\infty} \left(E_k\right)_x,\]
        para cualquier sucesi\'on de conjuntos $E_k$ contenidos en $\rr^{n+m}$. 
        
        Si $E_1\subset E_2$ entonces 
        \[
        \left(E_1\right)_x  \subset   \left(E_2\right)_x   
       \; \mbox{ y }\;
        \left(E_1-E_2\right)_x=\left(E_1\right)_x -\left(E_2\right)_x.
        \]
        
        \begin{teorema}{}[Principio de Cavalieri]
        Sea $E$ medible en $\rr^{n+m}$, entonces
        \begin{enumerate}
            \item $E_x$ es medible de $\rr^{n+m}$ en c.t.p. $x \in \rr^n$;
            \item $m\left(E_x\right)$ es medible como funci\'on de $x$;
            \item $m(E)=\int m\left(E_x\right)\, dx$.
        \end{enumerate}
        \end{teorema}
        
        \begin{demo}
        \emph{Primer Paso)} Si $E$ es un intervalo de $\rr^{n+m}$, supongamos que $E=I\times J$ con $I$ intervalo de $\rr^n$, $J$ intervalo de $\rr^m$, entonces $E_x=J$ $\forall x \in I$ y $E_x=\emptyset$ si $x \notin I$.
        As\'i, $E_x$ es conjunto medible $\forall x \in I$, $m(E_x)=\chi_I m(J)$ es funci\'on medible  en $x$ y 
        \[
        \int m(E_x)\,dx=m(J)m(I)=m(E).
        \]
        \emph{Segundo Paso)} Si $E$ es abierto, entonces $E_x=\bigcup\limits_{k=1}^{\infty} (I_k)_x$ con $I_k$ intervalos mutuamente disjuntos y donde $E_x=\bigcup\limits_{k=1}^{\infty} \left(I_k\right)_x$. 
        As\'i $E_x$ es conjunto medible. Adem\'as, 
        \[
        m(E_x)=\sum\limits_{k=1}^{\infty} m\left((I_k)_x\right)
        \]
        es una funci\'on medible de $x$ y 
        \[
        \int m(E_x)=\sum\limits_{k=1}^{\infty} \int m\left((I_k)_x\right)\,dx=
        \sum\limits_{k=1}^{\infty} m(I_k)=m(E).
        \]
        \emph{Tercer Paso)} Si $E$ es un conjunto acotado y de tipo $G_{\delta}$, entonces existe una bola $B$ y una sucesi\'on de conjuntos abiertos $G_k$ tales que 
        \[
        E\subset B\;\mbox{  y   }\;E=\bigcap\limits_{k=1}^{\infty}G_k.
        \]
        Tomando $G_k^{'}=B\cap G_1\cap \ldots \cap G_k$, podemos suponer
        \[B\supset G_1 \supset \ldots \supset E.\]
        Ahora, se tiene que 
        \[
        E_x=\bigcap\limits_{k=1}^{\infty} \left(G_k\right)_x
        \]
        es conjunto medible y 
        \[
        m(E_x)=\lim\limits_{k \to \infty} m\left((G_k)_x\right)
        \]
        es funci\'on medible de $x$. Adem\'as, 
        \[
        m\left((G_k)_x\right)\leq m(B_x)\in L(\rr^n).
        \]
        A continuaci\'on, por aplicaci\'on de Convergencia Mayorada (Corolario \ref{cor:conv-mayorada}), se obtiene 
        \[
        \int m(E_x)\,dx =\lim\limits_{k \to \infty} \int m\left((G_k)_x\right)\,dx =\lim\limits_{k \to \infty} m(G_k)=m(E).
        \]
        \emph{Cuarto Paso)}
        Supongamos que  $E$ es un conjunto de tipo $G_{\delta}$. 
        Sean $B_k=B(0,k)$ y $E_k=E\cap B_k \in G_{\delta}$, entonces $E=\bigcup\limits_{k=1}^{\infty} E_k$ y $E_1\subset E_2\subset \ldots$ Luego
        \[
        E_x=\bigcup\limits_{k=1}^{\infty} \left(E_k\right)_x
        \;\mbox{ y }\;
        (E_1)_x \subset (E_2)_x \subset \ldots
        \]
        De este modo, resulta que $E_x$ es conjunto medible y 
        \[m(E_x) = \lim\limits_{k \to \infty} m((E_k)_x).\]
                Como $m((E_k))_x$ es una sucesi\'on mon\'otona creciente de funciones medibles, por el Teorema de Beppo-Levi (Teorema \ref{teo:Beppo-Levi}) llegamos a 
     \[\int m(E_x)\,dx=\lim\limits_{k \to \infty} \int m\left(\left(E_k\right)_x\right)\,dx=
     \lim\limits_{k \to \infty} m(E_k)=m(E).
                    \]
    \emph{Quinto Paso)} Sea $E$ un conjunto de medida nula. Luego, existe $H\in G_{\delta}$ tal que $E\subset H$ y $m(H)=0$.
    A partir de 
    \[
    \int m(H_x)\,dx=m(H)=0,
    \]
    se tiene que $0\leq m(E_x)\leq m(H_x)=0$ en c.t.p. $x$. Luego, $m(E_x)=0$
    en c.t.p. $x$ y por lo tanto es funci\'on medible en $x$.
    Adem\'as, 
    \[
    \int m(E_x)\,dx \leq \int m(H_x)\,dx=0=m(E).\]

 \emph{Sexto Paso)} Sea $E$ medible. Entonces existen $H\in G_{\delta}$ y $Z$ de medida nula tal que $E=H-Z$. 
 Luego
 \[E_x=H_x -Z_x\]
 y $m(Z_x)=0$ en c.t.p. $x$, de donde $E_x$ es un conjunto medible en c.t.p. $X$ y 
 \[
 m(E_x)=m(H_x)-m(Z_x)=m(H_x)\;\mbox{  en c.t.p. } x.
 \]
 Es as\'i que, $m(E_x)$ es funci\'on medible siempre que $E_x$ sea medible y 
 definimos $m(E_x)=0$ para el caso en que $E_x$ no sea medible. Por \'ultimo,  \[
 \int m(E_x)\,dx=\int m(H_x)\,dx=m(H)=m(E).
 \]
\end{demo}

\begin{ejemplo}{}
\begin{enumerate}
    \item 
Sea $H$ un hiperplano de ecuaci\'on $a_1x_1+\ldots+a_nx_n=a$. 
\\Veamos por inducci\'on que $m(H)=0$.

El caso $n=1$ es trivial. 
Supongamos que $(a_2,a_3,\ldots,a_n)\neq 0$, luego 
\[H_{x_1}=\{(x_2,\ldots,x_n)| a_2x_2+\ldots+a_nx_n=a-a_1x_1\}\]
y 
\[
m(H)=\int m(H_{x_1})\,dx_1=0.
\]
\item El simple $S$ de altura $a$ es $x_1+x_2+\ldots+x_n\leq a$ con $x_i\geq 0$. 

Veamos por inducci\'on que $m(S)=\frac{a^n}{n!}$. 

El caso $n=1$ es trivial. Para  $n>1$, se tiene que $S_{x_1}=\emptyset$ si $x_1\notin [0,a]$ y si $x_1 \in [0,a]$ entonces 
$S_{x_1}=\{(x_2,\ldots,x_n)|x_2+\dots+x_n\leq a-x_1 \}$ es el simple de altura $a-x_1$. Luego, 
\[
m(S)=\int_0^a m(S_{x_1})\,dx_1= \int_0^a \frac{(a-x_1)^{n-1}}{(n-1)!}\,dx_1=\frac{a^n}{n!}.
\]     
\end{enumerate}   
\end{ejemplo}
   
   
   \begin{teorema}{teo:Fubini-Tonelli}[Fubini-Tonelli]
   Si $f(u)=f(x,y)$ es medible no negativa sobre $\rr^{n+m}$, entonces
   \begin{enumerate}
       \item $f(x,y)$ es medible en $y$ para c.t.p. $x$;
       \item $g(x)=\int_{\rr^n} f(x,y)\,dy$ es medible sobre $\rr^n$;
       \item 
       \[
       \int g(x)\,dx=\int dx \int f(x,y)\,dy=\int f(u)\,du.
       \]
   \end{enumerate}
      \end{teorema}     
      
      \begin{demo}
      \emph{Paso 1)}
      Si $f=\chi_E$, vale  \[ \chi_E(x,y)=\chi_{E_x} (y).\]
      $E_x$ es medible en c.t.p. $x$. As\'i, $f$ es medible en $y$ para c.t.p. $x$. Adem\'as
      \[
       \int\,dx \int \chi_{E_x}(y)\,dy =
       \int m(E_x)\,dx =m(E)=\int f(u)\,du.
      \]
      
       \emph{Paso 2)}
       Si $f$ es simple, entonces 
       \[f(x,y)=\sum\limits_{k=1}^N \alpha_k f_k(x,y),\]
       donde $\alpha_k\geq 0$ y $f_k$ son funciones caracter\'isticas 
       $\chi_{E_k}$.
       
       Por el \emph{Paso 1)}, $f_k(x,y)$ es medible en $y$ y $x$ est\'a en un conjunto de la forma $\rr^n-Z_k$, donde $m(Z_k)=0$. 
       Ahora, llamando $Z=\bigcup\limits_{k=1}^{N} Z_k$, $f$ es medible en $y$ siempre que $x \in \rr^n-Z$. O sea, $f$ resulta medible en $y$ para c.t.p. $x \in \rr^n$.

       Si $x \in \rr^n-Z$, la funci\'on 
       \[
       g(x)=\int f(x,y)\,dy=\sum\limits_{k=1}^N \alpha_k \int f_k(x,y)\,dy
       \]
        es medible.  Adem\'as, por el \emph{Paso 1)}, 
        \[
        \int g(x)\,dx=
        \sum\limits_{k=1}^N \alpha_k \int\, dx \int f_k(x,y)\,dy
        =
        \sum\limits_{k=1}^N \alpha_k \int f_k(u)\,du=\int f(u)\,du.
        \]
        
        \emph{Paso 3)}
        Si $f$ es medible no negativa, existe una sucesi\'on de funciones simples $f_k:\rr^{n+m}\to \rr$ tales que $0\leq f_1\leq f_2\leq \ldots$ y $f_k(u)\nearrow f(u)$ $\forall u\in \rr^{n+m}$.
        Dado que $f_k$ es simple, existe $Z_k$ tal que $f_k(x,y)$ es medible en $y$  si $x \notin Z_k$. 
        \\
        Sea $Z=\bigcup\limits_{k=1}^N Z_k$, entonces $m(Z)=0$ y cada $f_k(x,y)$ es medible en $y$ si $x \notin Z$. As\'i, $f$ es medible  en $y$ $\forall x \notin Z$. 
        Aplicando el Teorema de Beppo-Levi (Teorema \ref{teo:Beppo-Levi}), si $x \notin Z$  se tiene que
        \[
        g(x)=\int f(x,y)\,dy=\lim\limits_{k\to \infty} \int f_k(x,y)\,dy,
        \]      
        y $g$ es medible tomando la precauci\'on de definir $g=0$ en $Z$.
        \\
        Por \'ultimo, aplicando nuevamente  el Teorema de Beppo-Levi (Teorema \ref{teo:Beppo-Levi}), obtenemos
        \[
        \int g(x)\,dx=\lim\limits_{k \to \infty} \int\,dx \int f_k(x,y)\,dy
        =\lim\limits_{k \to \infty} \int f_k(u)\,du
        =\int f(u)\,du.
        \]
        \end{demo}
        
        \begin{teorema}{}[Fubini]
        Si $f(u)=f(x,y) \in L(\rr^{n+m})$, entonces
        \begin{enumerate}
            \item para casi todo $x$, $f(x,y)$ es integrable en $y$;
            \item la funci\'on $g(x)=\int f(x,y)\,dy$ es integrable en $x$;
            \item \[
            \int g(x)\,dx=\int \,dx \int f(x,y)\,dy=\int f(u)\,du
            \]
        \end{enumerate}
                \end{teorema}
               
                
                \begin{demo}
                 Por aplicaci\'on del Teorema de Fubini-Tonelli (Teorema \ref{teo:Fubini-Tonelli}) se tiene que 
                 \[
                 \int\,dx \int f^{+}(x,y)\,dy=\int f^{+}(u)\,du<\infty
                 \]
                 y 
                 \[
                  \int\,dx \int f^{-}(x,y)\,dy=\int f^{-}(u)\,du<\infty.
                 \]
                 Las funciones no negativas
\[
g_1(x)=\int f^{+}(x,y)\,dy\; \mbox{ y }\; 
g_2(x)=\int f^{-}(x,y)\,dy
\]
son integrables y por ende finitas en casi todo punto. 
Luego, 
\[g(x)=\int f(x,y)\,dy=g_1(x)-g_2(x)
\]
es integrable en $y$ para casi todo $x$. Adem\'as, $g \in L(\rr^n)$ y 
\[
\begin{split}
\int g(x)\,dx&=\int g_1(x)\,dx - \int g_2(x)\,dx
\\&=\int f^{+}(u)\,du - \int f^{-}(u)\,du=\int f(u)\,du.
\end{split}
\]
                \end{demo}
                
                \begin{corolario}{}
Si $f$ satisface que 
\[
\int\,dx \int |f(x,y)|\,dy<\infty\; \mbox{ \'o }\;
\int\,dy \int |f(x,y)|\,dx<\infty,
\]
vale el Teorema de Fubini.
\end{corolario}
        
        
Si $f$ es integrable sobre un conjunto $E\subset \rr^{n+m}$, la integral de $f$ sobre $E$ se calcula por medio de la f\'ormula
\[
\int_E f(u)\,du=\int\,dx \int_{E_x} f(x,y)\,dy.
\]

En efecto, $f\chi_E$ es integrable sobre todo el espacio $\rr^{n+m}$ y por consiguiente
\[
\begin{split}
\int_E f(u)\,du &=\int\int_E f(x,y)\,dx\,dy
\\
&=\int\int \chi_E(x,y)f(x,y)\,dx\,dy
\\
&=\int\,dx\int \chi_{E_x(y)}f(x,y)\,dy
\\
&=\int\,dx\int_{E_x} f(x,y)\,dy.
\end{split}
\]

 \include{Uni9-Espacios-Lp}
  \include{Uni6-Medida-abstracta}
%  
% \chapter*{Apéndice}

\section{Topología}

\begin{teorema}[Principio de encaje de intervalos\index{Intervalos encajados}]{}  Sea $I_n=[a_n,b_n]\subset\mathbb{R}$ una sucessión de intervalos con las siguientes propiedades
\begin{enumerate}
 \item $\forall n\in\mathbb{N}: I_n\subset I_{n+1},$
 \item $\lim\limits_{n\to\infty}(b_n-a_n)=0.$
\end{enumerate}
Entonces $\bigcap_{n=1}^{\infty}I_n$ consiste de uno, y solo un, punto $x\in\mathbb{R}$.
 
\end{teorema}

\begin{proof}
 
\end{proof}





\begin{teorema}[Heine-Borel]{}\index[personas]{Heine}\index[personas]{Borel} Toda sucesión acotada de $\mathbb{R}$ 
tiene una subsucesión convergente.
 
\end{teorema}

\begin{proof} Uso encajes de intervalos.
 
\end{proof}


\begin{definicion}[Continuidad uniforme \index{Continuidad uniforme}]{} Sea $f:A\subset\mathbb{R}\to\mathbb{R}$ una función. Diremos que $f$ es uniformemente continua si 
\[
 \forall \epsilon>0\exists \delta>0 \forall x,y\in A:|x-y|<\delta\Rightarrow |f(x)-f(y)|<\epsilon.
 \]

\end{definicion}

\begin{ejemplo} Varios ilustrando diferencia con continuidad
 
\end{ejemplo}


\begin{teorema}{} Sea $f:[a,b]\to\mathbb{R}$ continua. Entonces $f$ es uniformemente continua. 
 \end{teorema}
\begin{proof} Uso Heine-Borel
 \end{proof}


% \include{Apendice0}
 
%\chapter*{Apéndice}

\section{Topología}

\begin{teorema}[Principio de encaje de intervalos\index{Intervalos encajados}]{}  Sea $I_n=[a_n,b_n]\subset\mathbb{R}$ una sucessión de intervalos con las siguientes propiedades
\begin{enumerate}
 \item $\forall n\in\mathbb{N}: I_n\subset I_{n+1},$
 \item $\lim\limits_{n\to\infty}(b_n-a_n)=0.$
\end{enumerate}
Entonces $\bigcap_{n=1}^{\infty}I_n$ consiste de uno, y solo un, punto $x\in\mathbb{R}$.
 
\end{teorema}

\begin{proof}
 
\end{proof}





\begin{teorema}[Heine-Borel]{}\index[personas]{Heine}\index[personas]{Borel} Toda sucesión acotada de $\mathbb{R}$ 
tiene una subsucesión convergente.
 
\end{teorema}

\begin{proof} Uso encajes de intervalos.
 
\end{proof}


\begin{definicion}[Continuidad uniforme \index{Continuidad uniforme}]{} Sea $f:A\subset\mathbb{R}\to\mathbb{R}$ una función. Diremos que $f$ es uniformemente continua si 
\[
 \forall \epsilon>0\exists \delta>0 \forall x,y\in A:|x-y|<\delta\Rightarrow |f(x)-f(y)|<\epsilon.
 \]

\end{definicion}

\begin{ejemplo} Varios ilustrando diferencia con continuidad
 
\end{ejemplo}


\begin{teorema}{} Sea $f:[a,b]\to\mathbb{R}$ continua. Entonces $f$ es uniformemente continua. 
 \end{teorema}
\begin{proof} Uso Heine-Borel
 \end{proof}


%\include{Apendice0}


 %\nocite{*}
 \addcontentsline{toc}{chapter}{Bibliografía}
 
\bibliographystyle{apalike}
\bibliography{biblio}
 
 \printindex
\printindex[personas]
\printindex[simbolos]



\end{document}
