\documentclass[oneside]{book}

%%%%%%%%%%%%%%%%%%%%%%%%%%%%%%Paquetes%%%%%%%%%%%%%%%%%%%%%%%%%%%%%%%%%%%%%%%%%%%%%%%5
%%%%%%%%%%%%%%%%%%%%%%%%%%%%%%%%%%%%%%%%%%%%%%%%%%%%%%%%%%%%%%%%%%%%%%%%%%%%%%%%%%%%%
%\usepackage{empheq}
\usepackage[spanish]{babel}
\usepackage{amssymb,amsmath,amsthm}
\usepackage{enumerate}
\usepackage{verbatim}
\usepackage{array}
\usepackage{ wasysym }%simbolos especiales
\usepackage{hyperref}
\usepackage{color}
\usepackage[spanish]{varioref} % ESTILO PARA REFERENCIAS
\usepackage{fontspec} %para xelatex
\usepackage[a4paper,driver=xetex,top=2.5cm, bottom=2.7cm,%
layouthoffset=10mm, left=1.5cm, right=6.2cm,marginparwidth=3.5cm]{geometry}
\usepackage{fancyhdr} %Encabezados mejorados
\usepackage{marginnote} % Notas al margen mejoradas
\usepackage{titlesec} %para encabezados
\usepackage{pst-all} %pstricks graficos
\let\clipbox\relax
\usepackage{pgf,tikz,pgfplots} %tikz graficos
\pgfplotsset{compat=1.15}
\usetikzlibrary{arrows}
\usepackage{imakeidx} %supongo que para indices multiples
\usepackage{pythontex}%Ejecutar python dentro latex
\usepackage{diagrams}%diagramas comutativos
\usepackage{hyperref}%hipertexto
\defaultfontfeatures{Ligatures=TeX}
\usepackage[framemethod=TikZ]{mdframed}
\usepackage[breakable,many]{tcolorbox}
\tcbset{nobeforeafter} % prevents tcolorboxes being placing in paragraphs
\usepackage{float}
\floatplacement{figure}{H} % forces figures to be placed at the correct location
\usepackage{graphicx}
    % We will generate all images so they have a width \maxwidth. This means
    % that they will get their normal width if they fit onto the page, but
    % are scaled down if they would overflow the margins.
\makeatletter
\def\maxwidth{\ifdim\Gin@nat@width>\linewidth\linewidth
\else\Gin@nat@width\fi}
\makeatother
\let\Oldincludegraphics\includegraphics
% Set max figure width to be 80% of text width, for now hardcoded.
\renewcommand{\includegraphics}[1]{\Oldincludegraphics[width=.8\maxwidth]{#1}}
    % Ensure that by default, figures have no caption (until we provide a
    % proper Figure object with a Caption API and a way to capture that
    % in the conversion process - todo).
%     \usepackage{caption}
%     \DeclareCaptionLabelFormat{nolabel}{}
%     \captionsetup{labelformat=nolabel}
\usepackage{adjustbox} % Used to constrain images to a maximum size 
\usepackage{textcomp} % defines textquotesingle
    % Hack from http://tex.stackexchange.com/a/47451/13684:
\AtBeginDocument{%
\def\PYZsq{\textquotesingle}% Upright quotes in Pygmentized code
    }
\usepackage{upquote} % Upright quotes for verbatim code
\usepackage{eurosym} % defines \euro
\usepackage[mathletters]{ucs} % Extended unicode (utf-8) support
\usepackage{fancyvrb} % verbatim replacement that allows latex
\usepackage{grffile} % extends the file name processing of package graphics 
                         % to support a larger range 
    % The hyperref package gives us a pdf with properly built
    % internal navigation ('pdf bookmarks' for the table of contents,
    % internal cross-reference links, web links for URLs, etc.)
    
%\usepackage{longtable} % longtable support required by pandoc >1.10
\usepackage{booktabs}  % table support for pandoc > 1.12.2
\usepackage[inline]{enumitem} % IRkernel/repr support (it uses the enumerate* environment)
\usepackage[normalem]{ulem} % ulem is needed to support strikethroughs (\sout)
                                % normalem makes italics be italics, not underlines
\usepackage{mathrsfs}



%%%%%%%%%% Nuevos Entornos-Comandos%%%%%%%%%%%%%%%%%%%%%%%%%%%%%%%%%%%5
%\input{entornos_comandos.tex}

% %%%%%%%%%%%%%%%%%%%%%%%%%%Nuevos comandos entornos%%%%%%%%%%%%%%%%%%%%%%%%%%%%%%%%
% %%%%%%%%%%%%%%%%%%%%%%%%%%%%%%%%%%%%%%%%%%%%%%%%%%%%%%%%%%%%%%%%%%%%%%%%
\newenvironment{demo}{\noindent\emph{Dem.}}{\hfill\qed \newline\vspace{5pt}}

\newenvironment{observa}{\noindent\textbf{Observación:}}{}
\renewcommand{\C}{\overline{C}}
\newcommand{\com}{\mathbb{C}}
\newcommand{\rr}{\mathbb{R}}
\newcommand{\nn}{\mathbb{N}}
\newcommand{\qq}{\mathbb{R}}
\newcommand{\R}{\text{(R)}}
\renewcommand{\epsilon}{\varepsilon}
\renewcommand{\lim}{\mathop{\rm lím}}
\renewcommand{\inf}{\mathop{\rm ínf}}
\renewcommand{\liminf}{\mathop{\rm líminf}}
\renewcommand{\limsup}{\mathop{\rm límsup}}
\renewcommand{\min}{\mathop{\rm mín}}
\renewcommand{\max}{\mathop{\rm máx}}
\renewcommand{\b}[1]{\boldsymbol{#1}}
\renewenvironment{frame}[1]{}{}

%%%%%%%%%%%%%%%% Funcion característica %%%%%%%%%%%5555555

\DeclareRobustCommand{\rchi}{{\mathpalette\irchi\relax}}
\newcommand{\irchi}[2]{\raisebox{\depth}{$#1\chi$}} % inner command, used by \rchi


% 
% 
% 


%\renewcommand{\lim}{displaystyle\lim}
\DeclareMathOperator{\atan2}{atan2}
\DeclareMathOperator{\sen}{sen}
\DeclareMathOperator{\sgn}{sgn}
\DeclareMathOperator{\diametro}{diam}


\pgfdeclareverticalshading{exersicebackground}{100bp}
  {color(0bp)=(black!40);color(50bp)=(black!0)}

\mdfdefinestyle{MiEstilo}{innertopmargin=10pt,linecolor=white!100,%
linewidth=2pt,topline=true,tikzsetting={shading=exersicebackground}}  


%%%%%%%%%%%%  Teorema %%%%%%%%%%%%%%%%%%%%%%%%%%55

\newcounter{teorema}[chapter] \setcounter{teorema}{0}
\renewcommand{\theteorema}{\arabic{chapter}.\arabic{section}.\arabic{teorema}}
\newenvironment{teorema}[2][]{%
\refstepcounter{teorema}%
\mdfsetup{style=MiEstilo%
}
\ifstrempty{#1}
{
\begin{mdframed}[]\relax%
\strut \textbf{Teorema~\theteorema}\label{#2}
}
{
\begin{mdframed}[]\relax%
\strut \textbf{Teorema~\theteorema~(#1)}\label{#2}
}
}{\end{mdframed}}
%%%%%%%%%%%%%%%%%%%%%%%%%%%%%
%Lemma


\newcounter{lema}[chapter] \setcounter{lema}{0}
\renewcommand{\thelema}{\arabic{chapter}.\arabic{section}.\arabic{lema}}
\newenvironment{lema}[2][]{%
\refstepcounter{lema}%
\mdfsetup{style=MiEstilo%
}
\ifstrempty{#1}
{
\begin{mdframed}[]\relax%
\strut \textbf{Lema~\thelema}\label{#2}
}
{
\begin{mdframed}[]\relax%
\strut \textbf{Lema~\thelema~(#1)}\label{#2}
}
}{\end{mdframed}}
%%%%%%%%%%%%%%%%%%%%%%%%%%%%%


%% Definicion
\newcounter{definicion}[chapter] \setcounter{definicion}{0}
\renewcommand{\thedefinicion}{\arabic{chapter}.\arabic{section}.\arabic{definicion}}
\newenvironment{definicion}[2][]{%
\refstepcounter{definicion}%
\mdfsetup{style=MiEstilo%
}
\ifstrempty{#1}
{
\begin{mdframed}[]\relax%
\strut \textbf{Definición~\thedefinicion}\label{#2}
}
{
\begin{mdframed}[]\relax%
\strut \textbf{Definición~\thedefinicion~(#1)}\label{#2}
}}{\end{mdframed}}
%%%%%%%%%%%%%%%%%%%%%%%%%%%%%
%%%%%%%%%%%%%%%%%%%%%%%%%%%%%%

%%%%%%%%%%%%%%%%% Demostracion

\newenvironment{prf}{\noindent\emph{Dem.}}{$\square$ \newline\vspace{5pt}}


%Corolario
\newcounter{corolario}[chapter] \setcounter{corolario}{0}
\renewcommand{\thecorolario}{\arabic{chapter}.\arabic{section}.\arabic{corolario}}
\newenvironment{corolario}[2][]{%
\refstepcounter{corolario}%
\mdfsetup{style=MiEstilo%
}
\ifstrempty{#1}
{
\begin{mdframed}[]\relax%
\strut \textbf{Corolario~\thecorolario}\label{#2}
}
{
\begin{mdframed}[]\relax%
\strut \textbf{Corolario~\thecorolario~(#1)}\label{#2}
}}{\end{mdframed}}
%%%%%%%%%%%%%%%%%%%%%%%%%%%%%




%%%%%%%%%%%%%%%%%%%%%%%%%%%%%%
%% Ejercicio
\newcounter{ejercicio}[chapter] \setcounter{ejercicio}{0}
\renewcommand{\theejercicio}{\arabic{chapter}.\arabic{section}.\arabic{ejercicio}}
\newenvironment{ejercicio}[2][]{%
\refstepcounter{ejercicio}%
\mdfsetup{style=MiEstilo%
}
\ifstrempty{#1}
{
\begin{mdframed}[]\relax%
\strut \textbf{Ejercicio~\theejercicio}\label{#2}
}
{
\begin{mdframed}[]\relax%
\strut \textbf{Ejercicio~\theejercicio~(#1)}\label{#2}
}}{\end{mdframed}}
%%%%%%%%%%%%%%%%%%%%%%%%%%%%%



%%%%%%%%%%%%%%%%%%%%%%%%%%%%%
%%%%%%%%%%%%%%%%%%%%%%%%%%%%%%
%%%%%%%%%%%%%%%    Proposicion    %%%%%%%%%%%%%%%%%%%%5

\newcounter{proposicion}[chapter] \setcounter{proposicion}{0}
\renewcommand{\theproposicion}{\arabic{chapter}.\arabic{section}.\arabic{proposicion}}
\newenvironment{proposicion}[2][]{%
\refstepcounter{proposicion}%
\mdfsetup{style=MiEstilo%
}
\ifstrempty{#1}
{
\begin{mdframed}[]\relax%
\strut \textbf{Proposición~\theproposicion}\label{#2}
}
{
\begin{mdframed}[]\relax%
\strut \textbf{Proposición~\theproposicion (~#1)}\label{#2}
}}{\end{mdframed}}
%%%%%%%%%%%%%%%%%%%%%%%%%%%%%

%%%%%%%%%%Ejemplo%%%%%%%%%%%%%%%%%%%%%%%%%%%%%%5555
\newcounter{ejemplo}[chapter] \setcounter{ejemplo}{0}
\renewcommand{\theejemplo}{\arabic{chapter}.\arabic{section}.\arabic{ejemplo}}
\newenvironment{ejemplo}[2][]{%
\refstepcounter{ejemplo}%
\relax\noindent\textbf{Ejemplo~\theejemplo}\label{#2}
}{}


%%%%%%%%%%Observacion%%%%%%%%%%%%%%%%%%%%%%%%%%%%%%5555
\newcounter{observacion}[chapter] \setcounter{observacion}{0}
\renewcommand{\theobservacion}{\arabic{chapter}.\arabic{section}.\arabic{observacion}}
\newenvironment{observacion}{%
\refstepcounter{observacion}%
\relax\noindent\textbf{Observacion~\theobservacion}
}{}

%%%%%%%%%%%%%%%Secciones%%%%%%%%%%%%%%%%%%%%%%%%%%%%%%%5


\makeatletter
\newcommand\makeSecHead[4][\fbox]{%
  \@namedef{#2}{\@ifnextchar*{\@nameuse{#2@i}}{\@nameuse{#2@ii}}}
%
    \expandafter\def\csname#2@i\endcsname*##1{\par\vspace{#4}\noindent
       #1{\parbox{\dimexpr\textwidth-2\fboxsep-2\fboxrule}{%
         \normalfont\normalsize#3\makebox[40pt][l]{}~##1}}\par\vspace{#4}}%
%
    \expandafter\def\csname#2@ii\endcsname{\@ifnextchar[{\@nameuse{#2@iii}}{\@nameuse{#2@iv}}}%
%
    \expandafter\def\csname#2@iii\endcsname[##1]##2{\par\vspace{#4}\noindent
      #1{\parbox{\dimexpr\textwidth-2\fboxsep-2\fboxrule}{%
        \refstepcounter{#2}\normalfont\normalsize#3\makebox[40pt][l]{\@nameuse{the#2}}~##2}}%
        \addcontentsline{toc}{#2}{\@nameuse{the#2}~##1}\par\vspace{#4}}%
%
   \expandafter\def\csname#2@iv\endcsname##1{\par\vspace{#4}\noindent
     #1{\parbox{\dimexpr\textwidth-2\fboxsep-2\fboxrule}{%
       \refstepcounter{#2}\normalfont\normalsize#3\makebox[40pt][l]{\@nameuse{the#2}}~##1}}%
       \addcontentsline{toc}{#2}{\@nameuse{the#2}~##1}\par\vspace{#4}}%
}
\makeatother    

\makeSecHead[\colorbox{gray!30}]{chapter}{\Huge\bfseries}{20pt}
\makeSecHead[\colorbox{gray!30}]{section}{\LARGE\bfseries}{15pt}
\makeSecHead[\colorbox{gray!30}]{subsection}{\Large\bfseries}{12pt}
\makeSecHead[\colorbox{gray!30}]{subsubsection}{\large\bfseries}{10pt}

%\renewcommand{\emph}[1]{\fontshape{it}\selectfont #1}


%%%%%%%%%%%%% Configuracion para notebook importadas

%\input{python/jupyter_config.tex}

%%%%%%%%%%%%Configuración Encabezados Página
\fancyfoot{}
\fancyhead[RO,LE]{\thepage}
\fancyhead[LO]{\leftmark}
\fancyhead[RE]{\rightmark}


\titleformat{\section}
  {\normalfont\Large\bfseries}{\thesection}{1em}{}[{\titlerule[0.8pt]}]

  
%%%%%%%%%%%%Configuracion Apéndices %%%%%%%%%%%%%%%%%%%%%%%%55555  
\AtBeginEnvironment{subappendices}{%
\chapter*{Apéndices}
\addcontentsline{toc}{chapter}{Apéndices}
\counterwithin{figure}{section}
\counterwithin{table}{section}
}







% %%%%%%%%%%%%%Configuración de fuente para XeLaTeX


% %\setromanfont[Mapping=tex-text]{Oswald-Light}
\setmainfont{Roboto Condensed}
% %\setsansfont{Gentium Basic}
% %\setsansfont{FreeMono}
%\setromanfont{Oswald-Light}
%\setromanfont{Comfortaa}
 %
%
% \renewcommand{\familydefault}{\sfdefault}


%%%%%%%%%%%%%%%%%   Titulo  %%%%%%%%%%%%%%%%%%%%%%%%%%%%%%%%%%%%

\title{Introducción al Análisis Matemático}
 

\author{}
\date{}
%%%%%%%%%%%%%%%%%%%%%%%%%%%%%%%%%%%%%%%%%%%%%%%%%%%%%%%%%%%%%%%%%%%%%%%%%%%%%%%%%%%%%%

%%%%%%%%%%%%%%   Indices %%%%%%%%%%%%%%%%%%%%%%%%%%%%%%%%%%%%%%%%%%%%%%%%%%%%%%
\makeindex[title=Indice Conceptos]
\makeindex[name=personas,title=Indice de Personas,columns=3]
\makeindex[name=simbolos,title=Indice Símbolos,columns=3]

\begin{document}



\fontsize{11pt}{11pt}\selectfont

\pagestyle{fancy}

%
 \maketitle
 \tableofcontents
%

%
\chapter*{Prólogo}

















%
%
%  \bibliographystyle{apalike-url}
%  \bibliography{diferenciales_ecuaciones,diferenciales_ecuaciones_sim}
 
 



%
\chapter{Breve introducci\'on a Python y SymPy}




\section{Descripción}
\href{https://www.python.org/}{Python} es un lenguaje de programación interpretado, 
abierto, facil de aprender, potente y portátil. Es utilizado en proyectos de todo tipo, 
no sólo aplicaciones científicas.
\marginnote{
%\begin{tabular}{b{.7in} b{.7in}}
 \includegraphics[scale=.25]{imagenes/python-logo.png}
% &\begin{pspicture}(.7in,.7in)
%        \psbarcode{https://www.python.org/}{}{qrcode}
%    \end{pspicture}\\
%    &
% {\tiny https://www.python.org/}
% \end{tabular}
}



\href{http://www.scipy.org/}{SciPy}, 
Python científico, es un conjunto de módulos de python para distintos tipos de cálculos. 
Está integrado por los módulos, SymPy (para cálculos simbólicos), 
numpy (cálculos numéricos), matplotlib (gráficos) entre otros.  
En este curso sólo usaremos SymPy.
\marginnote{
%\begin{tabular}{b{.7in} b{.7in}}
 \includegraphics[scale=.5]{imagenes/scipy_logo.png}
%&\begin{pspicture}(.7in,.7in)
%       \psbarcode{http://www.scipy.org/}{}{qrcode}
%   \end{pspicture}\\
%   &
%{\tiny http://www.scipy.org/}
%\end{tabular}
}


\href{http://www.sympy.org/}{SymPy}
es una biblioteca de Python para matemática simbólica. Su objetivo es convertirse en 
un sistema de álgebra computacional (SAC) completo, manteniendo el código lo más simple 
posible para que sea comprensible y fácilmente extensible. SymPy está escrito enteramente 
en Python y no requiere de ninguna biblioteca externa.
\marginnote{
 % \begin{tabular}{b{.7in} b{.7in}}
    \includegraphics[scale=.2]{imagenes/sympy_logo.png}
% 	&
% 	  \begin{pspicture}(.7in,.7in)
% 	    \psbarcode{http://www.sympy.org/}{}{qrcode}
% 	  \end{pspicture}
% \\
% 	&
% 	  {\tiny http://www.sympy.org/}
%   \end{tabular}
}
% 
% 



\href{http://matplotlib.org/}{Matplotlib} es una biblioteca de trazado de gráficos de Python que produce figuras de calidad de publicación en una variedad de formatos impresos y entornos interactivos a través de plataformas. Matplotlib se puede utilizar en scripts Python, en el shell Python e IPython, el portátil jupyter, servidores de aplicaciones web.
\marginnote{
  \includegraphics[scale=.1]{imagenes/matplotlib.jpg}
}






% 
% \section{Local y online} 
% 
% Se pueden usar todos los recursos anteriores de dos formas
% \begin{enumerate}
% \item Instalando el software necesario en una computadora. Nos referiremos a este modo como de acceso local.
% 
% \item A traves de trasacciones en línea que permiten usar una computadora remota que ejecuta las instrucciones y programas que se tipean en una página web con la que se interactúa usualmente por medio de un navegador.  Hay varios sitios que ofrecen este servicio. 
%Sugerimos la \href{https://cloud.sagemath.com/}{SageMathCloud}. El usuario debe registrase.
% \end{enumerate}
% 


\section{Instalación}


Son muchas las componentes requeridas para poder ejecutar los programas con los que trabajaremos 
en esta asignatura. Hay que instalar un interprete de python, los módulos que utilizaremos 
(sympy, matplotlib), es útil utilizar entornos integrados de desarrollo (IDE), que facilitan al usuario
editores de código fuente (especializados con la sintáxis de python), consolas de comandos
mejoradas (ipython, qt, etc). Otro recurso que se dispone son las notebooks, de las cuales 
hablaremos más adelante. Sería engorroso instalar todas estas componentes, que muchas veces 
tienen orígenes en desarrolladores diferentes, de manera independiente. Para nuestra fortuna
existen, las así llamadas, \emph{distribuciones}. Estas en algunos casos son archivos ejecutables que instalan todas 
las componentes necesarias, o al menos muchas  de ellas, de un determinada aplicación.
Recomendamos las siguientes distribuciones.  

\subsection{\href{https://www.continuum.io/downloads}{Anaconda}} 
La versión de código abierto de Anaconda es una distribución de alto rendimiento de Python y R 
e incluye más de 100 de los paquetes científicos más populares asociados a estos lenguajes.
%\reversemarginpar\marginpar{\includegraphics[scale=.12]{imagenes/library.png} } 
Además, se puede acceder a más de 720 paquetes que pueden ser fácilmente instalados con Conda, 
 un programa incluído en Anaconda para la gestión de paquetes.
 Anaconda tiene licencia BSD que da permiso para utilizar Anaconda comercialmente 
 y para su redistribución. Al día que se escriben estas líneas, anaconda parece la opción más 
 sencilla y completa para instalar todos los recursos necesarios para desarrollar los contenidos de 
 estas notas. Existen versiones para linux, OS X y Windows. 
\marginnote{
    \includegraphics[scale=.3]{imagenes/anaconda.png}
}



\subsection{Windows} Hay distribuciones específicas para distintos sistemas operativos. La distribución  \href{https://code.google.com/p/pythonxy/}{python(x,y)}  instala el interprete de python y todos los módulos de scipy. Además el entorno de desarrollo integrado (IDE) spyder.
\marginnote{\includegraphics[scale=.07]{imagenes/windows-logo.png}}


\subsection{linux} Aquí todo es más sencillo, el interprete de python suele venir con 
la distribución del SO y se pueden instalar los módulos, SymPy, NumPy, etc, 
recurriendo al administrador de paquetes o tipeando la sentencia adecuada en la línea 
de comandos.  
\marginnote{\includegraphics[scale=.4]{imagenes/linux.jpeg}}

\subsection{Android} \href{http://qpython.com/}{Qpython} es una aplicación que permite ejecutar código python y una versión básica de sympy desde tablets y smartphones. Se descarga desde la plataforma \href{https://play.google.com/store/apps/details?id=com.hipipal.qpyplus}{google play}.
\marginnote{\includegraphics[scale=.1]{imagenes/android.jpg}}. 

\subsection{\href{https://es.wikipedia.org/wiki/Computación_en_la_nube}{Computación en la nube}}

En los últimos tiempos se ha popularizado el uso de la computación en la nube. Esto se trata de servidores que algunas empresas o asociaciones sin fines de lucro facilitan en la web para ejecutar programas en diversos lenguajes. Citamos como ejemplo \href{http://www.cocalc.com}{cocalc}. Una vez registrado en el sitio se pueden subir o crear notebooks de jupyter. Soporta varios lenguajes, incluído Python y sus librerías.  Como uno utiliza los recursos instalados en el servidor, no se necesita tener instalado ningún interprete de los lenguajes. Sólo se necesita un navegador web actualizado. De este modo pueden ejecutarse programas desde un smartphone o tablet. 



\section{Forma de trabajo: por medio de scripts e interactiva}


Se puede trabajar de tres formas

\begin{enumerate}
\item Interactivamente, ingresando sentencias, de a una por vez, en la línea de comandos y obteniendo respuestas. Se requiere una consola.

\item Haciendo un script (programa) donde se guardan todas las sentencias que se desea ejecutar. Posteriormente este script se puede ejecutar, ya sea desde la línea de comandos o desde un IDE (spyder) oprimiendo un botón de ejecución.

\item En una notebook. Se hacen celdas que contienen porciones de código que pueden ejecutarse.

\end{enumerate}





\section{Características sobresalientes del lenguaje}

Seguiremos en esta exposición a \cite{wiki_python} de manera cercana. \link
Las principales características del lenguaje son:

\normalmarginpar
\begin{itemize}
\item Interpretado. Es necesario un conjunto de programas, 
el interprete, que entienda el código python y ejecute las acciones contenidas en él.
\item Implementa  tipos dinámicos.
\item  Multiparadigma, ya que soporta orientación a objetos, programación imperativa y, en menor medida, programación funcional.
\item Multiplataforma.

\item Es comprendido  con facilidad. Usa  palabras donde otros lenguajes utilizarían símbolos. Por ejemplo, los operadores lógicos \verb~!, || y \&\&~ en Python se escriben not, or y and, respectivamente.


\item  El contenido de los bloques de código (bucles, funciones, clases, etc.) es delimitado mediante espacios o tabuladores.

\item Empieza a contar desde cero (elementos en listas, vectores, etc).



\end{itemize}




\section{Elementos del Lenguaje}

\subsection{Comentarios}

Hay dos formas de producir comentarios, texto que el interprete  no ejecuta y que sirve para entender un programa.

Para comentarios largos se utilizan las tildes: \linebreak\verb~''' comentario '''~ .


 La segunda notación utiliza el símbolo \verb~#~, no necesita símbolo de finalización 
 pues se extiende hasta el final de la línea.

 \begin{pyverbatim}
'''
Comentario  largo en un script de Python
'''
print "Hola mundo" # Comentario corto
\end{pyverbatim}




El intérprete no tiene en cuenta los comentarios, lo cual es útil si deseamos poner información adicional en nuestro código como, por ejemplo, una explicación sobre el comportamiento de una sección del programa.








\subsection{Variables}
Las variables se definen de forma dinámica, lo que significa que no se tiene que especificar cuál es su tipo de antemano y que una variable puede tomar distintos valores en distintos momentos de un programa, incluso puede tomar
 un tipo diferente al que tenía previamente. \emph{Se usa el símbolo = para asignar valores a variables.}
\advertencia Es importante distinguir este = (de asignación) con el igual que es utilizado para definir igualdades en sympy, para ecuaciones por ejemplo.



\begin{pyverbatim}
x = 1
x = "texto" 
\end{pyverbatim}
Esto es posible porque los tipos son asignados
dinamicamente

\subsection{Tipo de datos}

Python implementa diferentes tipos de datos. Para la noción de \emph{tipos de datos} en 
general 
ver \cite{wiki:tipo_dato}\link. A continuación describimos sumariamente algunos de los tipos 
más comunes presentes en Python. Cuando se utilizan módulos específicos (p. ej. sympy) la 
diversidad  de tipos de datos se expande, con la incorporación de tipos con significación
matemática, p.ej. matrices,  expresiones algebraicas, etc. 


\includegraphics[scale=.4]{imagenes/tipo_datos.jpg}

Se clasifican en:
\begin{description}
 \item[Mutable] si su contenido puede cambiarse.
 \item[Inmutable] si su contenido no puede cambiarse.
\end{description}

Se usa el comando \verb~\type~ para averiguar que tipo de dato contiene una variable

 \begin{pyconsole}
x=1
type(x)
x='Ecuaciones'
type(x)
\end{pyconsole}





\subsection{Listas y tuplas}


\begin{itemize}

\item Es una estructura de dato, que contiene, como su nombre lo indica, listas de otros datos en cierto orden. Listas y tuplas son muy similares.

\item Para declarar una lista se usan los corchetes [], en cambio, para declarar una tupla se usan los paréntesis (). En ambos casos los elementos se separan por comas, y en el caso de las tuplas es necesario que tengan como mínimo una coma.

\item    Tanto las listas como las tuplas pueden contener elementos de diferentes tipos. No obstante las listas suelen usarse para elementos del mismo tipo en cantidad variable mientras que las tuplas se reservan para elementos distintos en cantidad fija.
    
\item Para acceder a los elementos de una lista o tupla se utiliza un índice entero (empezando por "0", no por "1"). Se pueden utilizar índices negativos para acceder elementos a partir del final.


\item Las listas se caracterizan por ser mutables, mientras que las tuplas son inmutables.

\end{itemize}






\begin{pyconsole}
lista = ["abc", 42, 3.1415]
lista[0] # Acceder a un elemento por su indice
lista[-1] # Acceder a un elemento usando un indice negativo
lista.append(True) # Agregar un elemento al final de la lista
lista
del lista[3] # Borra un elemento de la lista usando un indice
lista[0] = "xyz" # Re-asignar el valor del primer elemento
lista[0:2]#elementos del indice "0" al "1" 
lista_anidada = [lista, [True, 42L]] #Es posible anidar listas
lista_anidada
lista_anidada[1][0] #accede lista dentro de otra lista
\end{pyconsole}

\begin{pyconsole}
tupla = ("abc", 42, 3.1415)
tupla[0] # Acceder a un elemento por su indice
del tupla[0] # No es posible borrar ni agregar
tupla[0] = "xyz" # Tampoco es posible re-asignar
tupla[0:2] # elementos del indice "0" al "2" sin incluir
tupla_anidada = (tupla, (True, 3.1415)) # es posible anidar
1, 2, 3, "abc" # Esto tambien es una tupla
(1) #  no es una tupla, ya que no posee al menos una coma
(1,) # si es una tupla
(1, 2) # Con mas de un elemento no es necesaria la coma final
(1, 2,) # Aunque agregarla no modifica el resultado
\end{pyconsole}

\subsection{Diccionarios}
\begin{itemize}
\item    Para declarar un diccionario se usan las llaves \verb~{}~. Contienen elementos separados por comas, donde cada elemento está formado por un par clave:valor (el símbolo : separa la clave de su valor correspondiente).
 \item   Los diccionarios son mutables, es decir, se puede cambiar el contenido de un valor en tiempo de ejecución.
\item    En cambio, las claves de un diccionario deben ser inmutables. Esto quiere decir, por ejemplo, que no podremos usar ni listas ni diccionarios como claves.
\item    El valor asociado a una clave puede ser de cualquier tipo de dato, incluso un diccionario.

\end{itemize}






\begin{pyconsole}
dicci = {"cadena": "abc", "numero": 42, "lista": [True, 42L]}
dicci["cadena"] # Usando una clave, se accede a su valor
dicci["lista"][0]
dicci["cadena"] = "xyz" # Re-asignar el valor de una clave
dicci["cadena"]
dicci["decimal"] = 3.1415927 # nuevo elemento clave:valor
dicci["decimal"]
dicci_mixto = {"tupla": (True, 3.1415), "diccionario": dicci}
dicci_mixto["diccionario"]["lista"][1]
dicci = {("abc",): 42} # tupla puede ser clave 
dicci = {["abc"]: 42} # una clave no puede ser lista
\end{pyconsole}



\subsection{Listas por comprensión}
Una lista por comprensión es una expresión compacta para definir listas. Al igual que el operador lambda, aparece en lenguajes funcionales. Ejemplos:

\pyv{range(n)} devuelve una lista, empezando en 0 y terminando en $n-1$.

\begin{pyconsole}
range(5) #  
[i*i for i in range(5)]
lista = [(i, i + 2) for i in range(5)]
lista
\end{pyconsole}



\subsection{Funciones}
\begin{itemize}

  \item  Las funciones se definen con la palabra clave \verb~def~, seguida del nombre de
  la función y sus parámetros. 
  Otra forma de escribir funciones, aunque menos utilizada, es con la palabra clave \verb~lambda~ (que aparece en lenguajes funcionales como Lisp). Generalemente esta forma es apropiada para funciones que es posible definir en una sola línea.
  
  
  \item  El valor devuelto en las funciones con \verb~def~ será el dado con la instrucción \verb~return~.
  \end{itemize}


\begin{pyconsole}
def suma(x, y = 2): #el argumento y tiene un valor por defecto
    return x + y # Retornar la suma

suma(4) # La variable "y" no se modifica, siendo su valor: 2
suma(4, 10) # La variable "y" si se modifica
\end{pyconsole}


\begin{pyconsole}
suma = lambda x, y = 2: x + y
suma(4) # La variable "y" no se modifica
suma(4, 10) # La variable "y" si se modifica
\end{pyconsole}

\subsection{Condicionales}
 Una sentencia condicional (\pyv{if} \verb~condicion~) ejecuta su bloque de código interno 
 sólo si \verb~condicion~ tiene el valor booleano \pyv{True}.  Condiciones adicionales, si las hay, se introducen usando \pyv{elif} seguida de la condición y su bloque de código. Todas las condiciones se evalúan secuencialmente hasta encontrar la primera que sea verdadera, y su bloque de código asociado es el único que se ejecuta. Opcionalmente, puede haber un bloque final (la palabra clave \pyv{else} seguida de un bloque de código) que se ejecuta sólo cuando todas las condiciones fueron falsas.



\begin{pyconsole}
verdadero = True
if verdadero: # No es necesario poner "verdadero == True"
    print "Verdadero"

else:
    print "Falso"

lenguaje = "Python"
if lenguaje == "C": 
    print "Lenguaje de programacion: C"
elif lenguaje == "Python": # tantos "elif" como se quiera
    print "Lenguaje de programacion: Python"
else: 
    print "Lenguaje de programacion: indefinido"

if verdadero and lenguaje == "Python": 
    print "Verdadero y Lenguaje de programacion: Python"

\end{pyconsole}






\subsection{Bucles}
El bucle \pyv{for} es similar a  otros lenguajes. Recorre un objeto \emph{iterable},
esto es  una lista o una tupla, y por cada elemento del iterable 
ejecuta el bloque de código interno. 
Se define con la palabra clave \pyv{for} seguida de un nombre de variable, 
seguido de \pyv{in} seguido del iterable, y finalmente el bloque de código interno. 
En cada iteración, el elemento siguiente del iterable se asigna al nombre de variable 
especificado:

\begin{pyconsole}
lista = ["a", "b", "c"]
for i in lista: # Iteramos sobre una lista, que es iterable
    print i

cadena = "abc"
for i in cadena: # Iteramos sobre una cadena, que es iterable
    print i # una coma al final evita un salto de linea

\end{pyconsole}



El bucle \pyv{while} evalúa una condición y, si es verdadera, ejecuta el bloque
de código interno. Continúa evaluando y ejecutando mientras la condición sea verdadera.
Se define con la palabra clave \pyv{while} seguida de la condición, y a continuación 
el bloque de código interno:
\begin{pyconsole}
numero = 0
while numero < 3:
    print numero
    numero += 1  

\end{pyconsole}


% 
%  \bibliographystyle{plain}
%  \bibliography{biblio}
% 
% 
% \end{document}

\bibliographystyle{plain}%{apalike-url}
\bibliography{biblio,diferenciales_ecuaciones,diferenciales_ecuaciones_sim}




\chapter{Sucessiones, series de funciones y sus amigos}
\chapter{Integral de Riemann}

\section{Introducción}

\begin{quotation}
<< Bernard Riemann recibió su doctorado en 1851, su \emph{Habilitación} en 1854. La habilitación confiere el reconocimiento de la capacidad de crear sustanciales contribuciones en la investigación más allá de la tesis doctoral, y es un prerequisito necesario para ocupar un cargo de profesor en una universidad Alemana. Riemann eligió como tema  de habilitación el problema de las series de Fourier. Su tesis fue titulada \emph{\"Uber die Darstellbarkeit einer Function durch eine trigonometrische Reine} (Sobre la representación de una función por series trigonométricas) y respondía la pregunta:  Cuándo una función definida en el intervalo $(-\pi,\pi)$ puede ser respresentada por la serie trigonométrica $a_0/2+\sum_{n=1}^{\infty}[a_n\cos(nx)+b_n\sen(nx)]$? 
\marginpar{\includegraphics[scale=.4]{imagenes/Riemann.jpeg}\\
Bernhard Riemann 1826-1866
} 
En este trabajo  es donde hallamos   la Integral de Riemann, introducida en una sección corta antes del nucleo principal de la tesis, como parte del trabajo preparatorio que él necesitó desarrollar antes de abordar el problema de representabilidad por series trigonométricas. >> 
\end{quotation}
\begin{flushright}
 David M. Bressoud\\
 A Radical Approach to Lebesgue's Theory of Integration.\lectura
\end{flushright}


En este capítulo vamos a desarrollar el concepto de la integral de Riemman. Vamos a exponer la definición de la integral debida a Riemann y la ideada por J. G. Darboux.
Mostraemos la equivalencia de las dos definiciones y discutiremos las propiedades de la intergal, sus alcances y límites. Preparamos así el camino para la introducción de la integral de Lebesgue. 
\marginpar{\includegraphics[scale=.6]{imagenes/Darboux.jpg}\\
Jean G. Darboux  1842-1917
} 

Debemos advertir \advertencia  al alumno que en este curso dejaremos un poco de lado las cuestiones procedimentales de cómo calcular integrales, aspecto que seguramente abordó en cursos anteriores y del cual nos vamos a valer. Tampoco debe esperar que las actividades prácticas se centren en esa dirección.   Nuestro principal objetivo aquí es discutir la materia conceptual ligada a la integral y cómo es previsible las actividades prácticas estarán orientadas con ese propósito.


El concepto de integral encuentra su motivación en diversos problemas. Aparece cuando se busca el centro de masas de un determinado cuerpo, cuando se quieren hallar longitudes de arco, volúmenes, cuando se quiere reconstruir el movimiento de cuerpo conocida su velocidad, etc. La integral es utilizada en incontables otros conceptos matemáticos, como ser el mencionado már arriba relativo a las series de Fourier. 

Quizás el 
problema más simple donde aparece la integral es el que utilizaremos como motivación para introducirla y es el concepto de área.  Vamos a tratar de reconstruir este concepto desde su base, esto es analizando la noción de área de figuras tan simples como rectángulos, triángulos, etc. 



\section{Área de figuras elementales planas} 

  
El cálculo de áreas es necesario en multitud de actividades humanas, por ejemplo con el comercio. La cantidad de muchos productos y servicios se estima en medidas de área, por ejemplo: las telas,  el trabajo de un colocador de pisos,  el precio de la construcción,  el valor de las extensiones de tierra, etc.  
 
 


Por figuras elementales planas nos referimos a rectángulos, triángulos, trapecios, etc. Sin duda el alumno  debe estar  muy familiarizado con las áreas de estas figuras, el área de un rectángulo viene dada por la conocida fórmula $b\times h$, donde $b$ es la base del rectángulo y $h$ su altura.  Ahora bien, ¿Cómo 
se llega a esta fórmula? Porque esta fórmula es apropiada para calcular el precio de un terreno por ejemplo. En esta sección vamos a justificar esta fórmula a partir de algunos hechos elementales.



Vamos a considerar un plano $\mathcal{P}$. En este plano $\mathcal{P}$ supondremos fijada una unidad de longitud.  Pretendemos asignar un área a las figuras, es decir a los subconjuntos, de $\mathcal{P}$. De ahora en más, cómo es usual en esta materia  nos referiremos a \emph{medida}\index{medida} en lugar de área. La medida es un concepto más general  que el concepto de área. No obstante en el contexto en que estamos actualmente son sinónimos.  

Queremos construir pues una función $m$ tal que $m(A)$ reppresente la medida  de  $A\subset\mathcal{P}$. Ahora bien ¿qué podemos usar de guía con ese objetivo? Si, como dijimos,  desconocemos todas las fórmulas previamente aprendidas, sobre que partimos para construir la medida o área. La respuesta es que tomaremos como principio rector  ciertas propiedades que son deseables  que una medida satisfaga. Ellas son las  siguientes. 




\begin{description}
 \item[Positividad.] debería ser una magnitud no negativa.  
 \item[Invariancia por movimientos rígidos.] Si una región es transformada en otra por medio de un movimiento rígido, ambas regiones deberían tener la misma área. Otra manera de expresar esta propiedad es diciendo que dos figuras \emph{congruentes}\index{congruencia} tienen la misma área. 
 \item[Aditividad.] Si una región es la unión de cierta cantidad de regiones más chicas mutuamente disjuntas  
\end{description}

\begin{figure}[h]
\begin{center}
 
\definecolor{xdxdff}{rgb}{0.49,0.49,1}
\definecolor{zzttqq}{rgb}{0.6,0.2,0}
\definecolor{ududff}{rgb}{0.30,0.30,1}
\begin{tikzpicture}[line cap=round,line join=round,x=.9cm,y=.9cm]
\clip(-2.261345665671131,-2.6483801457242535) rectangle (15.749723988011487,4.927708528477943);
\fill[line width=2pt,color=zzttqq,fill=zzttqq,fill opacity=0.10000000149011612] (-1.62,-0.89) -- (-1.62,3.13) -- (1.28,3.15) -- (1.3,-0.89) -- cycle;
\fill[line width=2pt,color=zzttqq,fill=zzttqq,fill opacity=0.10000000149011612] (0.09169246661626662,0.6507662540293884) -- (1.2899992647959808,1.130148511211861) -- (1.28,3.15) -- (-0.3199239037382289,3.1389660420431844) -- cycle;
\fill[line width=2pt,color=zzttqq,fill=zzttqq,fill opacity=0.10000000149011612] (-1.62,1.45) -- (-1.62,3.13) -- (-0.3199239037382289,3.1389660420431844) -- (0.09169246661626662,0.6507662540293884) -- (-0.9454716981132074,0.2358490566037732) -- cycle;
\fill[line width=2pt,color=zzttqq,fill=zzttqq,fill opacity=0.10000000149011612] (-1.62,-0.89) -- (-0.32,-0.89) -- (-1.62,1.45) -- cycle;
\fill[line width=2pt,color=zzttqq,fill=zzttqq,fill opacity=0.10000000149011612] (-0.32,-0.89) -- (-0.9454716981132074,0.2358490566037732) -- (0.09169246661626662,0.6507662540293884) -- (1.2899992647959808,1.130148511211861) -- (1.3,-0.89) -- cycle;
\fill[line width=2pt,color=zzttqq,fill=zzttqq,fill opacity=0.10000000149011612] (8.76,1.77) -- (8.134528301886792,2.8958490566037733) -- (9.171692466616268,3.3107662540293887) -- (10.36999926479598,3.790148511211861) -- (10.38,1.77) -- cycle;
\fill[line width=2pt,color=zzttqq,fill=zzttqq,fill opacity=0.10000000149011612] (3.8264466094067267,-1.199878066911803) -- (2.6385072170133266,-0.011938674518402692) -- (3.551459891609421,0.9136938983359697) -- (5.601939561385962,-0.5546723179904587) -- (5.16194451123264,-1.5814488960049211) -- cycle;
\fill[line width=2pt,color=zzttqq,fill=zzttqq,fill opacity=0.10000000149011612] (5.6888024763961145,1.8814084921516754) -- (6.19715889449669,3.0677137999207305) -- (4.761837661840736,4.4888939366884495) -- (3.638322806619573,3.3497747084781038) -- cycle;
\fill[line width=2pt,color=zzttqq,fill=zzttqq,fill opacity=0.10000000149011612] (7.14,0.65) -- (5.84,0.65) -- (7.14,-1.69) -- cycle;
\draw [line width=2pt,color=zzttqq] (-1.62,-0.89)-- (-1.62,3.13);
\draw [line width=2pt,color=zzttqq] (-1.62,3.13)-- (1.28,3.15);
\draw [line width=2pt,color=zzttqq] (1.28,3.15)-- (1.3,-0.89);
\draw [line width=2pt,color=zzttqq] (1.3,-0.89)-- (-1.62,-0.89);
\draw [line width=2pt] (-1.62,1.45)-- (-0.32,-0.89);
\draw [line width=2pt] (-0.9454716981132074,0.2358490566037732)-- (1.2899992647959808,1.130148511211861);
\draw [line width=2pt] (-0.3199239037382289,3.1389660420431844)-- (0.09169246661626662,0.6507662540293884);
\draw [line width=2pt,color=zzttqq] (0.09169246661626662,0.6507662540293884)-- (1.2899992647959808,1.130148511211861);
\draw [line width=2pt,color=zzttqq] (1.2899992647959808,1.130148511211861)-- (1.28,3.15);
\draw [line width=2pt,color=zzttqq] (1.28,3.15)-- (-0.3199239037382289,3.1389660420431844);
\draw [line width=2pt,color=zzttqq] (-0.3199239037382289,3.1389660420431844)-- (0.09169246661626662,0.6507662540293884);
\draw [line width=2pt,color=zzttqq] (-1.62,1.45)-- (-1.62,3.13);
\draw [line width=2pt,color=zzttqq] (-1.62,3.13)-- (-0.3199239037382289,3.1389660420431844);
\draw [line width=2pt,color=zzttqq] (-0.3199239037382289,3.1389660420431844)-- (0.09169246661626662,0.6507662540293884);
\draw [line width=2pt,color=zzttqq] (0.09169246661626662,0.6507662540293884)-- (-0.9454716981132074,0.2358490566037732);
\draw [line width=2pt,color=zzttqq] (-0.9454716981132074,0.2358490566037732)-- (-1.62,1.45);
\draw [line width=2pt,color=zzttqq] (-1.62,-0.89)-- (-0.32,-0.89);
\draw [line width=2pt,color=zzttqq] (-0.32,-0.89)-- (-1.62,1.45);
\draw [line width=2pt,color=zzttqq] (-1.62,1.45)-- (-1.62,-0.89);
\draw [line width=2pt,color=zzttqq] (-0.32,-0.89)-- (-0.9454716981132074,0.2358490566037732);
\draw [line width=2pt,color=zzttqq] (-0.9454716981132074,0.2358490566037732)-- (0.09169246661626662,0.6507662540293884);
\draw [line width=2pt,color=zzttqq] (0.09169246661626662,0.6507662540293884)-- (1.2899992647959808,1.130148511211861);
\draw [line width=2pt,color=zzttqq] (1.2899992647959808,1.130148511211861)-- (1.3,-0.89);
\draw [line width=2pt,color=zzttqq] (1.3,-0.89)-- (-0.32,-0.89);
\draw [line width=2pt,color=zzttqq] (8.76,1.77)-- (8.134528301886792,2.8958490566037733);
\draw [line width=2pt,color=zzttqq] (8.134528301886792,2.8958490566037733)-- (9.171692466616268,3.3107662540293887);
\draw [line width=2pt,color=zzttqq] (9.171692466616268,3.3107662540293887)-- (10.36999926479598,3.790148511211861);
\draw [line width=2pt,color=zzttqq] (10.36999926479598,3.790148511211861)-- (10.38,1.77);
\draw [line width=2pt,color=zzttqq] (10.38,1.77)-- (8.76,1.77);
\draw [line width=2pt,color=zzttqq] (3.8264466094067267,-1.199878066911803)-- (2.6385072170133266,-0.011938674518402692);
\draw [line width=2pt,color=zzttqq] (2.6385072170133266,-0.011938674518402692)-- (3.551459891609421,0.9136938983359697);
\draw [line width=2pt,color=zzttqq] (3.551459891609421,0.9136938983359697)-- (5.601939561385962,-0.5546723179904587);
\draw [line width=2pt,color=zzttqq] (5.601939561385962,-0.5546723179904587)-- (5.16194451123264,-1.5814488960049211);
\draw [line width=2pt,color=zzttqq] (5.16194451123264,-1.5814488960049211)-- (3.8264466094067267,-1.199878066911803);
\draw [line width=2pt,color=zzttqq] (5.6888024763961145,1.8814084921516754)-- (6.19715889449669,3.0677137999207305);
\draw [line width=2pt,color=zzttqq] (6.19715889449669,3.0677137999207305)-- (4.761837661840736,4.4888939366884495);
\draw [line width=2pt,color=zzttqq] (4.761837661840736,4.4888939366884495)-- (3.638322806619573,3.3497747084781038);
\draw [line width=2pt,color=zzttqq] (3.638322806619573,3.3497747084781038)-- (5.6888024763961145,1.8814084921516754);
\draw [line width=2pt,color=zzttqq] (7.14,0.65)-- (5.84,0.65);
\draw [line width=2pt,color=zzttqq] (5.84,0.65)-- (7.14,-1.69);
\draw [line width=2pt,color=zzttqq] (7.14,-1.69)-- (7.14,0.65);
\draw (-1.1356538123159674,2.2366014415507594) node[anchor=north west] {$A_1$};
\draw (3.9651373981996176,0.03798454046645946) node[anchor=north west] {$A_1$};
\draw (0.23628313396063827,2.5883801457242477) node[anchor=north west] {$A_2$};
\draw (4.703872676963944,3.83719454554013) node[anchor=north west] {$A_2$};
\draw (0.09557165229124281,0.5304747263093427) node[anchor=north west] {$A_3$};
\draw (9.101106479132552,3.1160482019844795) node[anchor=north west] {$A_3$};
\draw (-1.5753771925328282,0.31940750380524985) node[anchor=north west] {$A_4$};
\draw (6.445177262622713,0.5480636615180171) node[anchor=north west] {$A_4$};
\begin{scriptsize}
\draw [fill=ududff] (-1.62,-0.89) circle (2.5pt);
\draw [fill=ududff] (-1.62,3.13) circle (2.5pt);
\draw [fill=ududff] (1.28,3.15) circle (2.5pt);
\draw [fill=ududff] (1.3,-0.89) circle (2.5pt);
\draw [fill=xdxdff] (-1.62,1.45) circle (2.5pt);
\draw [fill=xdxdff] (-0.32,-0.89) circle (2.5pt);
\draw [fill=xdxdff] (-0.9454716981132074,0.2358490566037732) circle (2.5pt);
\draw [fill=xdxdff] (1.2899992647959808,1.130148511211861) circle (2.5pt);
\draw [fill=xdxdff] (-0.3199239037382289,3.1389660420431844) circle (2.5pt);
\draw [fill=xdxdff] (0.09169246661626662,0.6507662540293884) circle (2.5pt);
\draw [fill=xdxdff] (8.76,1.77) circle (2.5pt);
\draw [fill=xdxdff] (8.134528301886792,2.8958490566037733) circle (2.5pt);
\draw [fill=xdxdff] (10.36999926479598,3.790148511211861) circle (2.5pt);
\draw [fill=ududff] (10.38,1.77) circle (2.5pt);
\draw [fill=xdxdff] (3.8264466094067267,-1.199878066911803) circle (2.5pt);
\draw [fill=ududff] (2.6385072170133266,-0.011938674518402692) circle (2.5pt);
\draw [fill=xdxdff] (3.551459891609421,0.9136938983359697) circle (2.5pt);
\draw [fill=xdxdff] (5.601939561385962,-0.5546723179904587) circle (2.5pt);
\draw [fill=xdxdff] (5.16194451123264,-1.5814488960049211) circle (2.5pt);
\draw [fill=xdxdff] (5.6888024763961145,1.8814084921516754) circle (2.5pt);
\draw [fill=xdxdff] (6.19715889449669,3.0677137999207305) circle (2.5pt);
\draw [fill=ududff] (4.761837661840736,4.4888939366884495) circle (2.5pt);
\draw [fill=xdxdff] (3.638322806619573,3.3497747084781038) circle (2.5pt);
\draw [fill=ududff] (7.14,0.65) circle (2.5pt);
\draw [fill=xdxdff] (5.84,0.65) circle (2.5pt);
\draw [fill=xdxdff] (7.14,-1.69) circle (2.5pt);
\end{scriptsize}
\end{tikzpicture}


 \caption{El área del rectángulo es la suma de sus partes}\label{fig:rect_descop} 
\end{center}
\end{figure}

Utilizando la segunda y tercer propiedad se pueden relacionar el área del rectángulo de la figura \ref{fig:rect_descop} con las cuatro regiones en la que es dividido.

Como veremos a lo largo de la materia la propiedad de aditividad debe ser estudiada con cuidado, esto ocurre por las intrincadas maneras en que una región puede ser unión de otras regiones. A lo largo de esta materia elaboraremos una  teoría que nos dará una descripción  precisa de a que conjuntos podemos asignarle una medida de modo que las propiedades previas sean ciertas. 

Por el momento veamos como las propiedades anteriores determinan practicamente de manera unívoca la medida de regiones elementales planas.  


Hablando de propiedades de la medida, supongamos que $A$ y $B$ son dos regiones con $A\subset B$. Entonces como $B=A\cup (B-A)$ y por la propiedad de aditividad y positividad

\[
 m(B)=m(A)+m(B-A)\geq m(A).
\]

Descubrimos así que nuestra medida deberá tener adicionalmente la siguiente propiedad:
\begin{description}
 \item[Monotonía.] Si $A\subset B$ entonces $m(A)\leq m(B)$. 
\end{description}

Es claro que si logramos construir una medida que satisfaga las propiedades anteriores cualquier multiplo por un número real positivo  de ella seguirá cumpliendo las propiedades. Esto es una manera de expresar el hecho que podemos usar diferentes unidades de medición. Esta cuestión se sortea proponiendo la unidad de medida. Esta unidad es completamente arbitraria, ud. podría elegir su figura plana preferida como unidad de área. \marginpar{ Podríamos por ejemplo elegir el círculo de radio uno como unidad de área. Así ya no tendríamos el problema de ese número raro $\pi$ que aparece en la fórmula del área del círculo. ¡El área de cualquier círculo sería igual a su radio al cuadrado! Claro que aparecería $\pi$  en la fórmula del área del cuadrado de lado 1. Nos tapamos los pies y se destapa el cuerpo.}  Cómo es habitual, elijamos el cuadrado cuyos lados miden la unidad de longitud previamente fijada. 


Supongamos ahora que tenemos un rectángulo de un lado igual a la unidad y el otro de lado un racional $n/m$, $n,m\in\mathbb{N}$. Veamos que la aditividad, la invariancia por movimientos rígidos y el hecho que decidimos que el cuadrado de lados igual a la unidad determinan el área de este rectángulo. Primero observar que si dividimos el lado de cuadrado unidad en $m$ segmentos iguales de longitud. \marginpar{
 
\definecolor{xdxdff}{rgb}{0.49,0.49,1}
\definecolor{zzttqq}{rgb}{0.6,0.2,0}
\definecolor{ududff}{rgb}{0.30,0.30,1}
\begin{tikzpicture}[x=2.1cm,y=2.1cm]
\clip(-0.07,0) rectangle (1.5,1.5);
\draw [line width=2pt,color=zzttqq] (0,0) -- (1,0) -- (1,1) -- (0,1) -- cycle;
\draw [line width=1pt,color=zzttqq, dashed] (0,.2)--(1,.2);
\draw [line width=1pt,color=zzttqq, dashed] (0,.4)--(1,.4);
\draw [line width=1pt,color=zzttqq, dashed] (0,.6)--(1,.6);
\draw [line width=1pt,color=zzttqq, dashed] (0,.8)--(1,.8);
\draw (0,1) node[anchor=south] {$Q$};
\draw (0.5,.1) node[anchor=center] {$R_1$};
\draw (0.5,.3) node[anchor=center] {$R_2$};
\draw (0.5,.5) node[anchor=center] {$R_3$};
\draw (0.5,.7) node[anchor=center] {$R_4$};
\draw (0.5,.9) node[anchor=center] {$R_5$};



\end{tikzpicture}
\\
 }
Queda dividido el cuadrado en $m$ rectángulos $R_1,\ldots,R_m$ (ver figura en el margen), todos ellos  congruentes entre si, de modo que todos tienen la misma medida, digamos $m(R_1)$. La unión de ellos es el cuadrado que por convención dijimos que tiene medida 1. De modo que por la aditividad debe ocurrir que $m(R_1)=\cdots =m(R_m))=1/m$. Recordemos nuestra pretención de inferir la medida de un rectángulo $R$ de lado 1 y otro $n/m$. Este rectángulo esta compuesto de $n$ rectángulos congruentes a los $R_i$, $i=1,\ldots,m$, nuevamente por la aditividad inferimos que $m(R)=n/m$. 

Sea ahora una rectángulo $R$ con un lado unidad y el otro un real cualquiera $l>0$. Existen sendas sucesiones $0<q_k,p_k\in\mathbb{Q}$, $k\in\mathbb{N}$, tales que $q_1\leq q_2\leq\cdots \leq l \leq \cdots\leq p_2\leq p_1$ y $\lim_{k\to\infty}q_k =\lim_{k\to\infty} p_k=l$. Consideremos una dos sucesiones de rectángulos $R_k$ y $S_k$ que comparten el lado de $R$ igual a la unidad, mientras que el otro lado de $R_k$ y $S_k$ es igual a $q_k$ y $p_k$ respectivamente. Luego por la monotonía
\[
 q_k=m(R_k)\leq m(R) \leq m(S_k)\leq p_k.
\]
Tomando límite cuando $k\to\infty$ inferimos que $m(R)=l$. 



\begin{figure}[h]
 \begin{center}
 \input{imagenes/ParalTria.tikz} 
 \end{center}
 \caption{Áreas de otras figuras elementales.}\label{fig:paral-trig}
\end{figure}


A partir de las propiedades fundamentales que postulamos para la medida o área inferimos la famosa fórmula del área de un rectángulo en el caso que uno de los lados sea igual a la unidad. Para un  rectángulo arbitrario. En la figura \ref{fig:paral-trig} se muestra como relacionar el área de un paralelepípedo con la de un rectángulo y la de un triángulo con la de un paralelepípedo para inferir las conocidas fórmulas para estas figuras.



\section{Integral de Riemann}

En esta sección abordaremos el problema del área de regiones planas. Vamos a contextualizarnos dentro del marco conceptual que nos brinda la geometría analítica. Mediante coordenadas cartesianas ortogonales los puntos del plano se identifican con pares ordenados $(x,y)\in\mathbb{R}^2$ y el plano con el conjunto $\rr^2$.  Nuestro propósito es entonces definir la medida de subconjuntos de $\mathbb{R}^2$. La geometría analítica abre así nuevas posibilidades para abordar el problema del área. 

Nuestra primera aproximación será la que propuso Bernhard Riemann en 1854, pero seguiremos  el enfoque de Jean Darboux. En esta parte de nuestra aproximación consideraremos subconjuntos de $\mathbb{R}^2$ de un tipo especial, concretamente a conjuntos que quedan encerrados entre la gráfica de una función y del eje coordenadas $x$.   
 


%\chapter{Sucesiones, series de funciones y sus amigos}


\section{Sucesiones de funciones}
Sea $K$ un espacio métrico, usualmente $K\subset \mathbb{R}^n$ para algún $n$. 

Una colección $f_n:K\to \mathbb{R}$ para $n=1,2,3,\ldots$ se llama sucesión de funciones.

Dada una sucesión de funciones $f_n$, $n=1,2,3,\ldots$  y $x\in K$, $f_n(x)$
es una sucesión de números reales y como tal puede o no converger a cierto límite.

La mayor diferencia entre una sucesión de números reales y una sucesión de funciones es
el hecho que en una sucesión de funciones los términos de la sucesión cambian cuando la variable
$x$ cambia. 
Por lo tanto el límite también puede cambiar, en caso de existir, y por consiguiente el límite
también es una función de $x$.
De manera que es necesario tener presente que cuando una sucesión de funciones es evaluada en
un valor de  $x$ particular resulta una sucesión de números reales.

Supongamos que para todo $x \in K$ la sucesión de números reales $f_n(x)$ converge,  
es decir que existe el $\lim\limits_{n \to \infty} f_n(x)$ y lo denotaremos $f(x)$. 
En este caso diremos que $f_n$ converge puntualmente a $f$.
 
\begin{ejemplo}{ej:sucesion-conv-puntual}
La sucesión $f_n(x)=\frac{1}{1+nx^2}$ converge puntualmente 
\[f(x)=\left\{\begin{array}{ll}
0&x\neq 0
\\
1&x=0
\end{array}
\right.\]
\textbf{Justificación:}
\\
Claramente si $x=0$ tenemos que $f_n(0)=1$ para todo $n \in \mathbb{N}$ y entonces $\lim\limits_{n \to \infty}f_n(0)=1$.

Si $x\neq 0$ entonces $nx^2\to \infty$ cuando $n\to \infty$  y por lo tanto $\lim\limits_{n \to \infty} \frac{1}{1+nx^2}=0$.

Como vemos, la determinación de la convergencia puntual suele reducirse al cálculo de un límite. 
Para este propósito es lícito usar todas las técnicas estudiadas en cursos anteriores como puede ser la Regla de L'H\^opital.
\end{ejemplo}

\begin{ejemplo}{ej:sucesion-conv-unif-1}
La sucesión $f_n(x)=\frac{n^2x-n^2}{1+n^2x}$ converge a $f(x)=\frac{x-1}{x}$ si $x \neq 0$.

Si $x=0$ no converge.

Es necesario ser cuidadoso con la justificación. Por ejemplo, la Regla de L'H\^opital  sólo puede usarse en casos de indeterminaciones.

Si $x=0$ no hay indeterminación pues $f_n(0)=-n^2 $ y 
$\lim\limits_{n \to \infty}f_n(0)=\lim\limits_{n \to \infty}(-n^2)=-\infty$. 

Es lícito decir $\lim\limits_{n \to \infty}f_n(0)=-\infty$ en lugar de que $f_n$ no converge en $x=0$.

Cuando $x=1$ tampoco hay indeterminación pues $f_n(1)=\frac{0}{1+n^2}$ y por tanto $\lim\limits_{n \to \infty} f_n(1)=0$. 
 
Si $x\neq 0$ y $x \neq 1$ podemos usar la Regla de L'H\^opital dado que se tiene la indeterminación $\frac{\infty}{\infty}$. 
En efecto, 
$\lim\limits_{n \to \infty}f_n(x)=
\lim\limits_{n \to \infty}\frac{n^2x-n^2}{1+n^2x}=
\lim\limits_{n \to \infty}\frac{2nx-2n}{2nx}=
\lim\limits_{n \to \infty}\frac{2x-2}{2x}=\frac{1-x}{x}.
$
Observemos que si $f(x)=\frac{1-x}{x}$ entonces $f(1)=0$ y por tanto $\lim\limits_{n\to \infty}f_n(x)=f(x)$ $\forall x\neq 0$.

Suele ser útil graficar algunas funciones de la sucesión y la función límite, ya sea empleando los procedimientos aprendidos 
en materias anteriores o usando sympy.
\end{ejemplo}

Para el Ejemplo \ref{ej:sucesion-conv-puntual} APARECE MAL LA REFERENCIA DEL EJEMPLO!!!!!!!!!!

 \begin{tcolorbox}[breakable, size=fbox, boxrule=1pt, pad at break*=1mm,colback=cellbackground, colframe=cellborder]
\prompt{In}{incolor}{1}{\hspace{4pt}}
\begin{Verbatim}[commandchars=\\\{\}]
\PY{k+kn}{from} \PY{n+nn}{sympy} \PY{k+kn}{import} \PY{o}{*}
\PY{n}{init\PYZus{}printing}\PY{p}{(}\PY{p}{)}
\end{Verbatim}
\end{tcolorbox}

    %Ejemplo \(f_n(x)=\frac{1}{1+nx^2}\)

    \begin{tcolorbox}[breakable, size=fbox, boxrule=1pt, pad at break*=1mm,colback=cellbackground, colframe=cellborder]
\prompt{In}{incolor}{2}{\hspace{4pt}}
\begin{Verbatim}[commandchars=\\\{\}]
\PY{n}{x}\PY{p}{,}\PY{n}{n}\PY{o}{=}\PY{n}{symbols}\PY{p}{(}\PY{l+s+s1}{\PYZsq{}}\PY{l+s+s1}{x,n}\PY{l+s+s1}{\PYZsq{}}\PY{p}{)}
\PY{n}{fn}\PY{o}{=}\PY{l+m+mi}{1}\PY{o}{/}\PY{p}{(}\PY{l+m+mi}{1}\PY{o}{+}\PY{n}{n}\PY{o}{*}\PY{n}{x}\PY{o}{*}\PY{o}{*}\PY{l+m+mi}{2}\PY{p}{)}
\PY{n}{p}\PY{o}{=}\PY{n}{plot}\PY{p}{(}\PY{n}{fn}\PY{o}{.}\PY{n}{subs}\PY{p}{(}\PY{n}{n}\PY{p}{,}\PY{l+m+mi}{1}\PY{p}{)}\PY{p}{,} \PY{p}{(}\PY{n}{x}\PY{p}{,}\PY{o}{\PYZhy{}}\PY{l+m+mi}{5}\PY{p}{,}\PY{l+m+mi}{5}\PY{p}{)}\PY{p}{,}\PY{n}{show}\PY{o}{=}\PY{n}{false}\PY{p}{)}
\PY{k}{for} \PY{n}{k} \PY{o+ow}{in} \PY{n+nb}{range}\PY{p}{(}\PY{l+m+mi}{2}\PY{p}{,}\PY{l+m+mi}{100}\PY{p}{)}\PY{p}{:}
    \PY{n}{p}\PY{o}{.}\PY{n}{append}\PY{p}{(}\PY{n}{plot}\PY{p}{(}\PY{n}{fn}\PY{o}{.}\PY{n}{subs}\PY{p}{(}\PY{n}{n}\PY{p}{,}\PY{n}{k}\PY{p}{)}\PY{p}{,} \PY{p}{(}\PY{n}{x}\PY{p}{,}\PY{o}{\PYZhy{}}\PY{l+m+mi}{5}\PY{p}{,}\PY{l+m+mi}{5}\PY{p}{)}\PY{p}{,}\PY{n}{show}\PY{o}{=}\PY{n}{false}\PY{p}{)}\PY{p}{[}\PY{l+m+mi}{0}\PY{p}{]}\PY{p}{)}
\PY{n}{p}\PY{o}{.}\PY{n}{show}\PY{p}{(}\PY{p}{)}
\end{Verbatim}
\end{tcolorbox}

    \begin{center}
    \adjustimage{max size={0.9\linewidth}{0.9\paperheight}}{python/uni3/output_3_0.png}
    \end{center}
    { \hspace*{\fill} \\}

Para el Ejemplo \ref{ej:sucesion-conv-unif-1}


    \begin{tcolorbox}[breakable, size=fbox, boxrule=1pt, pad at break*=1mm,colback=cellbackground, colframe=cellborder]
\prompt{In}{incolor}{3}{\hspace{4pt}}
\begin{Verbatim}[commandchars=\\\{\}]
\PY{n}{x}\PY{p}{,}\PY{n}{n}\PY{o}{=}\PY{n}{symbols}\PY{p}{(}\PY{l+s+s1}{\PYZsq{}}\PY{l+s+s1}{x,n}\PY{l+s+s1}{\PYZsq{}}\PY{p}{)}
\PY{n}{fn}\PY{o}{=}\PY{p}{(}\PY{n}{n}\PY{o}{*}\PY{o}{*}\PY{l+m+mi}{2}\PY{o}{*}\PY{n}{x}\PY{o}{\PYZhy{}}\PY{n}{n}\PY{o}{*}\PY{o}{*}\PY{l+m+mi}{2}\PY{p}{)}\PY{o}{/}\PY{p}{(}\PY{l+m+mi}{1}\PY{o}{+}\PY{n}{n}\PY{o}{*}\PY{n}{x}\PY{o}{*}\PY{o}{*}\PY{l+m+mi}{2}\PY{p}{)}
\PY{n}{p}\PY{o}{=}\PY{n}{plot}\PY{p}{(}\PY{n}{fn}\PY{o}{.}\PY{n}{subs}\PY{p}{(}\PY{n}{n}\PY{p}{,}\PY{l+m+mi}{1}\PY{p}{)}\PY{p}{,} \PY{p}{(}\PY{n}{x}\PY{p}{,}\PY{o}{\PYZhy{}}\PY{l+m+mi}{5}\PY{p}{,}\PY{l+m+mi}{5}\PY{p}{)}\PY{p}{,}\PY{n}{show}\PY{o}{=}\PY{n}{false}\PY{p}{,}\PY{n}{ylim}\PY{o}{=}\PY{p}{(}\PY{o}{\PYZhy{}}\PY{l+m+mi}{20}\PY{p}{,}\PY{l+m+mi}{10}\PY{p}{)}\PY{p}{)}
\PY{k}{for} \PY{n}{k} \PY{o+ow}{in} \PY{n+nb}{range}\PY{p}{(}\PY{l+m+mi}{2}\PY{p}{,}\PY{l+m+mi}{10}\PY{p}{)}\PY{p}{:}
    \PY{n}{p}\PY{o}{.}\PY{n}{append}\PY{p}{(}\PY{n}{plot}\PY{p}{(}\PY{n}{fn}\PY{o}{.}\PY{n}{subs}\PY{p}{(}\PY{n}{n}\PY{p}{,}\PY{n}{k}\PY{p}{)}\PY{p}{,} \PY{p}{(}\PY{n}{x}\PY{p}{,}\PY{o}{\PYZhy{}}\PY{l+m+mi}{5}\PY{p}{,}\PY{l+m+mi}{5}\PY{p}{)}\PY{p}{,}\PY{n}{show}\PY{o}{=}\PY{n}{false}\PY{p}{)}\PY{p}{[}\PY{l+m+mi}{0}\PY{p}{]}\PY{p}{)}
\PY{n}{p}\PY{o}{.}\PY{n}{show}\PY{p}{(}\PY{p}{)}
\end{Verbatim}
\end{tcolorbox}

    \begin{center}
    \adjustimage{max size={0.9\linewidth}{0.9\paperheight}}{python/uni3/output_5_0.png}
    \end{center}
    { \hspace*{\fill} \\}		
En los ejemplos anteriores se forma una ``montaña'' alrededor de un punto \emph{fijo} ($x=0$). 
Pero, puede ocurrir otro comportamiento que observaremos los siguientes ejemplos.

\begin{ejemplo}{}
Si  $f_n(x)=\left\{
\begin{array}{ll}
1&n\leq x\leq n+1
\\
0&\mbox{en otro caso}
\end{array}
\right.$
entonces $\lim\limits_{n \to \infty} f_n(x)=0$. 
\end{ejemplo} 

GRAFICAR CON SYMPY!!!

A partir del gráfico vemos que los términos de la sucesión $f_n(x)$ son ``montañas móviles'' de altura 1.


\begin{ejemplo}{ej:montaña-movil-cte}
Si  $f_n(x)=\frac{nx}{1+n^2x^2}$ en $[0,\infty)$
entonces 
$\lim\limits_{n \to \infty} \frac{nx}{1+n^2x^2}=
\lim\limits_{n\to \infty} \frac{x}{x}\frac{x}{\frac{1}{n}+nx^2}=
\lim\limits_{n \to \infty}\frac{x}{\frac{1}{n}+nx^2}=0.$ 
\end{ejemplo}


    \begin{tcolorbox}[breakable, size=fbox, boxrule=1pt, pad at break*=1mm,colback=cellbackground, colframe=cellborder]
\prompt{In}{incolor}{4}{\hspace{4pt}}
\begin{Verbatim}[commandchars=\\\{\}]
\PY{n}{x}\PY{p}{,}\PY{n}{n}\PY{o}{=}\PY{n}{symbols}\PY{p}{(}\PY{l+s+s1}{\PYZsq{}}\PY{l+s+s1}{x,n}\PY{l+s+s1}{\PYZsq{}}\PY{p}{)}
\PY{n}{fn}\PY{o}{=}\PY{n}{n}\PY{o}{*}\PY{n}{x}\PY{o}{/}\PY{p}{(}\PY{l+m+mi}{1}\PY{o}{+}\PY{n}{n}\PY{o}{*}\PY{o}{*}\PY{l+m+mi}{2}\PY{o}{*}\PY{n}{x}\PY{o}{*}\PY{o}{*}\PY{l+m+mi}{2}\PY{p}{)}
\PY{n}{p}\PY{o}{=}\PY{n}{plot}\PY{p}{(}\PY{n}{fn}\PY{o}{.}\PY{n}{subs}\PY{p}{(}\PY{n}{n}\PY{p}{,}\PY{l+m+mi}{1}\PY{p}{)}\PY{p}{,} \PY{p}{(}\PY{n}{x}\PY{p}{,}\PY{l+m+mi}{0}\PY{p}{,}\PY{l+m+mi}{5}\PY{p}{)}\PY{p}{,}\PY{n}{show}\PY{o}{=}\PY{n}{false}\PY{p}{)}
\PY{k}{for} \PY{n}{k} \PY{o+ow}{in} \PY{n+nb}{range}\PY{p}{(}\PY{l+m+mi}{2}\PY{p}{,}\PY{l+m+mi}{10}\PY{p}{)}\PY{p}{:}
    \PY{n}{p}\PY{o}{.}\PY{n}{append}\PY{p}{(}\PY{n}{plot}\PY{p}{(}\PY{n}{fn}\PY{o}{.}\PY{n}{subs}\PY{p}{(}\PY{n}{n}\PY{p}{,}\PY{n}{k}\PY{p}{)}\PY{p}{,} \PY{p}{(}\PY{n}{x}\PY{p}{,}\PY{l+m+mi}{0}\PY{p}{,}\PY{l+m+mi}{5}\PY{p}{)}\PY{p}{,}\PY{n}{show}\PY{o}{=}\PY{n}{false}\PY{p}{)}\PY{p}{[}\PY{l+m+mi}{0}\PY{p}{]}\PY{p}{)}
\PY{n}{p}\PY{o}{.}\PY{n}{show}\PY{p}{(}\PY{p}{)}
\end{Verbatim}
\end{tcolorbox}

    \begin{center}
    \adjustimage{max size={0.9\linewidth}{0.9\paperheight}}{python/uni3/output_7_0.png}
    \end{center}
    { \hspace*{\fill} \\}
En este caso también se observa una montaña móvil. 

HACER EL ANÁLISIS CON LA DERIVADA!!!

\begin{ejemplo}{}
Si $f_n(x)=\sqrt{x^2+\frac{1}{n^2}}$, $x \in \mathbb{R}$ luego 
$f_n^{'}(x)=\frac{x}{\sqrt{x^2+\frac{1}{n^2}}}$. 
Entonces $f_n^{'}(x)>0$ en $(0,+\infty)$ y $f_n^{'}(x)<0$ en $(0,+\infty)$, de donde  $(0,+\infty)$ es intervalo de
crecimiento para cada $f_n(x)$ y $(-\infty,0)$ es intervalo de decrecimiento para cada $f_n(x)$.
Luego cada $f_n(x)$ tiene un mínimo en $x=0$ y el valor mínimo es $f_n(0)=\frac{1}{n}.$

Por otra parte, $\lim\limits_{n \to \infty} f_n(x)=\lim\limits_{n \to \infty} \sqrt{x^2+\frac{1}{n^2}}=\sqrt{x^2}=|x|$.
\end{ejemplo}

 \begin{tcolorbox}[breakable, size=fbox, boxrule=1pt, pad at break*=1mm,colback=cellbackground, colframe=cellborder]
\prompt{In}{incolor}{5}{\hspace{4pt}}
\begin{Verbatim}[commandchars=\\\{\}]
\PY{k}{def} \PY{n+nf}{grafica}\PY{p}{(}\PY{n}{f}\PY{p}{,}\PY{n}{x1}\PY{p}{,}\PY{n}{x2}\PY{p}{,}\PY{n}{m}\PY{p}{)}\PY{p}{:}
    \PY{n}{p}\PY{o}{=}\PY{n}{plot}\PY{p}{(}\PY{n}{f}\PY{o}{.}\PY{n}{subs}\PY{p}{(}\PY{n}{n}\PY{p}{,}\PY{l+m+mi}{1}\PY{p}{)}\PY{p}{,} \PY{p}{(}\PY{n}{x}\PY{p}{,}\PY{n}{x1}\PY{p}{,}\PY{n}{x2}\PY{p}{)}\PY{p}{,}\PY{n}{show}\PY{o}{=}\PY{n}{false}\PY{p}{)}
    \PY{k}{for} \PY{n}{k} \PY{o+ow}{in} \PY{n+nb}{range}\PY{p}{(}\PY{l+m+mi}{2}\PY{p}{,}\PY{n}{m}\PY{p}{)}\PY{p}{:}
        \PY{n}{p}\PY{o}{.}\PY{n}{append}\PY{p}{(}\PY{n}{plot}\PY{p}{(}\PY{n}{f}\PY{o}{.}\PY{n}{subs}\PY{p}{(}\PY{n}{n}\PY{p}{,}\PY{n}{k}\PY{p}{)}\PY{p}{,} \PY{p}{(}\PY{n}{x}\PY{p}{,}\PY{n}{x1}\PY{p}{,}\PY{n}{x2}\PY{p}{)}\PY{p}{,}\PY{n}{show}\PY{o}{=}\PY{n}{false}\PY{p}{)}\PY{p}{[}\PY{l+m+mi}{0}\PY{p}{]}\PY{p}{)}
    \PY{n}{p}\PY{o}{.}\PY{n}{show}\PY{p}{(}\PY{p}{)}
\PY{n}{f}\PY{o}{=}\PY{n}{sqrt}\PY{p}{(}\PY{n}{x}\PY{o}{*}\PY{o}{*}\PY{l+m+mi}{2}\PY{o}{+}\PY{l+m+mf}{1.0}\PY{o}{/}\PY{n}{n}\PY{o}{*}\PY{o}{*}\PY{l+m+mi}{2}\PY{p}{)}
\PY{n}{grafica}\PY{p}{(}\PY{n}{f}\PY{p}{,}\PY{o}{\PYZhy{}}\PY{l+m+mi}{5}\PY{p}{,}\PY{l+m+mi}{5}\PY{p}{,}\PY{l+m+mi}{10}\PY{p}{)}
\end{Verbatim}
\end{tcolorbox}

    \begin{center}
    \adjustimage{max size={0.9\linewidth}{0.9\paperheight}}{python/uni3/output_9_0.png}
    \end{center}
    { \hspace*{\fill} \\}

En el Análisis Matemático, además de límites tenemos conceptos como continuidad, derivadas, integrales, etc.
Es común operar expresiones conjugando varios de ellos y queremos contar con relaciones entre ellos que permitan 
transformar las expresiones. 

Por ejemplo, ?`es importante el orden en que se realizan  las operaciones?
?`Es lo mismo tomar límite y luego derivar que hacerlo en el orden inverso? 
Si se tienen dos límites, ?`se pueden permutar?

\begin{ejemplo}{ej:sucesion-conv-2 puntos}
Si $f_n(x)=\sen(nx)$ para $x\in [0,\pi]$ entonces $f_n(0)=f_n(\pi)=0$ y la sucesión converge en estos valores.

Veamos que la sucesión de funciones dada no converge en ningún otro valor. 

Supongamos que $x\neq0$, $x\neq \pi$ y $\lim\limits_{n \to \infty}\sen(nx)=\alpha$. 

Si se tuviese $\alpha\neq 0$ entonces
\[
1=\lim\limits_{n \to \infty}\frac{\sen(2nx)}{\sen(x)}=2\lim\limits_{n \to \infty}\cos(nx)
\]
de donde $\lim\limits_{n \to \infty} \cos(nx)=\frac{1}{2}$.
Sin embargo, 
\[
\lim\limits_{n \to \infty}\cos(2nx)=
\lim\limits_{n \to \infty}2\cos^2(nx)-1=-\frac{1}{2}
\]
lo que nos lleva a una contradicción. 

Por lo tanto, $\lim\limits_{n \to \infty}\sen(nx)=0$.
Entonces 
$\lim\limits_{n \to \infty}|\cos(2nx)|=1
$
y 
\[
0=\lim\limits_{n \to \infty} |\sen[(n+1)x]|=
\lim\limits_{n \to \infty} |\sen(nx)\cos x+\cos(nx)\sen x|=|\sen x|
\]
y necesariamente  $x=0$ \'o $x=\pi$.

 \begin{tcolorbox}[breakable, size=fbox, boxrule=1pt, pad at break*=1mm,colback=cellbackground, colframe=cellborder]
\prompt{In}{incolor}{6}{\hspace{4pt}}
\begin{Verbatim}[commandchars=\\\{\}]
\PY{n}{f}\PY{o}{=}\PY{n}{sin}\PY{p}{(}\PY{n}{n}\PY{o}{*}\PY{n}{x}\PY{p}{)}
\PY{n}{grafica}\PY{p}{(}\PY{n}{f}\PY{p}{,}\PY{l+m+mi}{0}\PY{p}{,}\PY{n}{pi}\PY{p}{,}\PY{l+m+mi}{10}\PY{p}{)}
\end{Verbatim}
\end{tcolorbox}

    \begin{center}
    \adjustimage{max size={0.9\linewidth}{0.9\paperheight}}{python/uni3/output_11_0.png}
    \end{center}
    { \hspace*{\fill} \\}
\end{ejemplo}


\begin{ejemplo}{}
En el Ejemplo \ref{ej:sucesion-conv-puntual} 0.1 OJO CON LA REFERENCIA!!!!
vimos que  la sucesión 
$f_n(x)=\frac{1}{1+nx^2}$ converge puntualmente 
\[f(x)=\left\{\begin{array}{ll}
0&x\neq 0
\\
1&x=0
\end{array}
\right.\]

Si calculamos 
\[
\lim\limits_{n\to \infty}\lim\limits_{x \to 0}f_n(x)=\lim\limits_{n \to \infty}1=1
\]
y a continuación permutamos los límites obtenemos
\[
\lim\limits_{x\to 0}\lim\limits_{n \to \infty}f (x)=\lim\limits_{x \to 0}=0
\]
Por lo tanto, la permutación de los límites produce resultados \textbf{distintos}.

También vemos que la función límite es discontinua a pesar de que cada $f_n(x)$ es continua para cada $n$.
\end{ejemplo}

\begin{ejemplo}{}
Con las funciones del Ejemplo \ref{ej:montaña-movil-cte} 3 tenemos
\[
\int_{-\infty}^{+\infty} \lim\limits_{n \to  \infty} f_n(x)\,dx=\int_{-\infty}^{\infty} 0\,dx=0
\]
mientras que 
\[
\lim\limits_{n \to \infty} \int_{-\infty}^{\infty} f_n(x)\,dx=\lim\limits_{n \to \infty} \int_{n}^{n+1}dx=1
\]
En este caso, la permutación entre la operación de integración y la de límite también produce resultados \textbf{distintos}.
\end{ejemplo}

\begin{ejemplo}{}
Cada $f_n(x)=\sqrt{x^2+\frac{1}{n^2}}$ del Ejemplo \ref{ej:sucesion-conv-2 puntos} 4 ó 5 OJO CON LA REFERENCIA!!!!
es derivable y las derivadas son
$f_n^{'}(x)=\frac{x}{\sqrt{x^2+\frac{1}{n^2}}}$. 
Si computamos
\[
\lim\limits_{n \to \infty} f_n^{'}(x)=\frac{x}{|x|}=
\left\{
\begin{array}{ll}
1&x>0
\\
-1&x<0
\end{array}
\right.\]
cuando $x\neq 0$.
Entonces la funci\'on límite $f(x)=\frac{x}{|x|}$ no es derivable en 0. Luego  
\[
0=\lim\limits_{n \to \infty}f_n^{'}(0)\neq \frac{d}{dx}\left(\lim\limits_{n \to \infty}f_n(x)\right)\left. \right|_{x=0}
\]
pues ni siquiera tiene sentido el miembro de la derecha.
\end{ejemplo}

Es así que tenemos  
\begin{mdframed}[style=MiEstilo]\relax%
Encontrar condiciones que permitan permutar las operaciones anteriores.
\end{mdframed}



Antes de atacar este problema vamos a presentar varios ejemplo \textit{famosos} de sucesiones.


OJO!!!! LO QUE SIGUE EN EL APUNTE NO TIENE EJEMPLOS FAMOSOS, VIENEN LAS SERIES!!!!!

\section{Series de funciones}
Dada una sucesión de funciones $f_n(x)$, $n=1,2,\ldots$ podemos formar otra sucesión tomando las sumas acumuladas o
sumas parciales
\[
\begin{split}
s_1(x)&=f_1(x),
\\
s_2(x)&=f_1(x)+f_2(x),
\\
&\vdots
\\
s_n(x)&=f_1(x)+f_2(x)+\ldots+f_n(x).
\end{split}
\]

Si la nueva sucesión $\{s_n(x)\}$ converge a $f$ se dice que la serie $\sum\limits_{n=1}^{\infty} f_n(x)$
converge a $f$ ó que 
\[
\sum\limits_{n=1}^{\infty} f_n(x)=f(x).
\]

En pocas ocasiones se puede determinar que una serie converge hallando una expresión simple para $s_n(x)$ y
calculando su límite.

\begin{ejemplo}{}
Si $f_n(x)=c$ con $c \in \mathbb{R}$ un número independiente de $n$, entonces
\[
\begin{split}
s_1(x)&=f_1(x)=c,
\\
s_2(x)&=f_1(x)+f_2(x)=2c,
\\
&\vdots
\\
s_n(x)&=f_1(x)+f_2(x)+\ldots+f_n(x)=nc.
\end{split}
\]
Entonces
\[
\lim\limits_{n \to \infty} s_n(x)=
\left\{
\begin{array}{lll}
0&si&c=0,
\\
\infty&si&c\neq 0.
\end{array}
\right.
\]
y por lo tanto la series converge sólo cuando $c=0$.
\end{ejemplo}

\begin{ejemplo}{}
Si $f_n(z)=z^n$ para $n=0,1,2,\ldots$, luego
 
$s_n(z)=f_0(z)+\ldots+f_n(z)=1+z+\ldots+z^n$ y $zs_n(z)=z+z^2+\ldots+z^n+z^{n+1}$ 

entonces
$zs_n(z)-s_n(z)=z^{n+1}-1$ y por tanto $s_n(z)=\frac{z^{n+1}-1}{z-1}$.

De este modo, logramos expresar $s_n(z)$ en una fórmula relativamente sencilla. Ahora, 
\[
\lim\limits_{n \to \infty} s_n(z)=
\lim\limits_{n \to \infty} \frac{z^{n+1}-1}{z-1}=
\left\{\begin{array}{ll}
\frac{1}{1-z}&|z|<1,
\\
\mbox{no converge}&|z|\geq 1.
\end{array}
\right.
\]
\end{ejemplo}

Es interesante ver qué ocurre en $|z|=1$. 

Si $|z|=1$ entonces $z=e^{i\theta}=\cos \theta +i \sen \theta$ y 
\[
\begin{split}
s_n(z)&=
\frac{z^{n+1}-1}{z-1}=
\frac{z^{n+1}}{z-1}-\frac{1}{z-1}
\\
&=\frac{z^{n+1}(\overline z-1)}{|z-1|^2}-\frac{1}{z-1}
\\
&=\frac{ [\cos(n+1)\theta+i\sen(n+1)\theta](\overline z-1)}{|z-1|^2}-\frac{1}{z-1}
\\
&=\cos(n+1)\theta \frac{\overline z-1}{|z-1|^2}+i \sen(n+1)\theta \frac{(\overline z-1)}{|z-1|^2}-\frac{1}{z-1}
\end{split}
\]
son funciones oscilantes como en el Ejemplo \ref{ej:sucesion-conv-2 puntos} 5 y 1/2. VER GRÁFICOS!!!

En los ejemplos anteriores pudimos justificar la convergencia calculando explícitamente el límite. 
Ésto es posible las menos de las veces. En materias anteriores se estudiaron criterios para la 
convergencia de series  numéricas. Estos criterios establecen condiciones, algunas necesarias, otras 
suficientes y algunas necesarias y suficientes para que la serie converja. Recordaremos algunos de ellos.

\begin{teorema}[Criterio del Resto]{}
Si  $\sum\limits_{n=1}^{\infty} a_n$ converge entonces $\lim\limits_{n \to \infty} a_n=0$.
\end{teorema}

Como es una condición  necesaria sólo sirve para decir cuándo una serie no converge.

\begin{ejemplo}{}
Si $f_n(x)= \sen(nx)$ para $x \in [0,\pi]$. 

Como ya vimos, $\sen(nx)$ no converge excepto para $x=0$ \'o $x=\pi$. 

Luego, $\sum\limits_{n=1}^{\infty} \sen(nx)$ no converge.
\end{ejemplo}

Como veremos más adelante, la serie $\sum\limits_{n=1}^{\infty}\frac{1}{n}$ no converge y sin embargo
$\lim\limits_{n \to \infty}\frac{1}{n}=0$. 

Entonces el \textit{Criterio del Resto} no sirve para determinar
la convergencia de una serie.

\begin{teorema}[Convergencia Absoluta]{}
Si$\sum\limits_{n=1}^{\infty} |a_n|$ converge entonces $\sum\limits_{n=1}^{\infty} a_n$ converge.
\end{teorema}

\begin{ejemplo}{}
$\sum \limits_{n=1}^{\infty} (-1)^n z^n$ converge cuando $|z|<1$.
\end{ejemplo}

\begin{teorema}
[Criterio de Comparación]Si $0\leq a_n\leq b_n$ y $\sum\limits_{n=1}^{\infty} b_n$ converge entonces $\sum\limits_{n=1}^{\infty}a_n$ converge.
Dicho de otro modo, si $\sum\limits_{n=1}^{\infty} a_n$ diverge (su suma es $+\infty$) entonces 
$\sum\limits_{n=1}^{\infty}$ diverge.
\end{teorema}

\begin{ejemplo}{}
\begin{enumerate}
\item $\sum\limits_{n=1}^{\infty} \frac{1}{2^n} \sen(nx)$ converge pues $|\frac{1}{2^n} \sen(nx)|\leq \frac{1}{2^n}$.
\item $\sum\limits_{n=1}^{\infty} \frac{1}{n^2} \sin(nx)$ converge pues para $n>1$ tenemos
\[
\left|\frac{1}{n^2}\sin(nx)\right|\leq \frac{1}{n^2}\leq \frac{1}{(n-1)n}=\frac{1}{n-1}-\frac{1}{n}
\]
y
\[
\sum\limits_{n=2}^{m}\frac{1}{n-1}-\frac{1}{n}=1-\frac{1}{m}\to 1\mbox{  cuando  } m\to \infty.
\]
Luego $\sum\limits_{n=1}^{\infty} \frac{1}{n^2}$ converge y 
$\sum\limits_{n=1}^{\infty} \frac{1}{n^2} \sen(nx)$ también.
\end{enumerate}
\end{ejemplo}

\begin{teorema}[Criterio del Cociente]{}
Si $0<a_n$, $\lim\limits_{n \to \infty} \frac{a_{n+1}}{a_n}$ existe y es igual a $r$ entonces: 
\begin{enumerate}
\item si  $r<1$ entonces $\sum\limits_{n=1}^{\infty} a_n$ converge;
\item si $r>1$ entonces $\sum\limits_{n=1}^{\infty} a_n$ diverge;
\item\label{it:r=1} si $r=1$ no se sabe nada sobre la convergencia o divergencia de $\sum\limits_{n=1}^{\infty} a_n$.
\end{enumerate}
\end{teorema}

La situación planteada por el item \ref{it:r=1} ocurre en una cantidad exasperante de casos.

\begin{ejemplo}{}
Consideramos la serie $\sum\limits_{n=1}^{\infty} z^n$ con $z \in \mathbb{C}$ y calculamos
\[
\lim\limits_{n \to \infty} \frac{|n||z|^{n+1}}{|n+1||z|^n}=|z|\lim\limits_{n \to \infty} \frac{n}{n+1}=|z|.
\]
La serie converge cuando $|z|<1$, diverge cuando $|z|>1$ y nada se sabe cuando $|z|=1$.
\end{ejemplo}

\begin{teorema}[Criterio del Cociente]{}
Si $a_n\geq a_{n+1}\geq 0$ para todo $n$ y $\lim\limits_{n \to \infty} a_n=0$, entonces
$\sum\limits_{n=1}^{\infty}(-1)^n a_n$ converge.
\end{teorema}

\begin{ejemplo}{}
La series $f(z)=\sum\limits_{n=1}^{\infty}\frac{z^n}{n}$ converge en $z=-1$ pues 
$f(-1)=\sum\limits_{n=1}^{\infty} \frac{(-1)^n}{n}$, $\frac{1}{n}>\frac{1}{n+1}$ para todo $n\geq 1$ y $\lim\limits_{n \to \infty} \frac{1}{n}=0$.
\end{ejemplo}


\section{Series de Potencias}

Las series de la forma 
\[
\sum\limits_{n=0}^{\infty} a_n (z-z_0)^n,\quad a_n\in \mathbb{C},\quad z\in \mathbb{C},\quad z_0 \in \mathbb{C},
\]
se llaman series de potencias.

Veremos para qué valores de $z$ esta serie converge.
Por simplicidad asumiremos que $z_0=0$.

\begin{lema}{}
Si $\sum\limits_{n=0}^{\infty} a_n z^n$ converge en $z_1\in \mathbb{C}$ entonces 
la serie converge uniformemente para todo $z$ tal que $|z|<|z_1|$.
\end{lema}

\begin{proof}
Como $\sum\limits_{n=0}^{\infty} a_nz_1^n$ converge entonces $\lim\limits_{n\to \infty} a_nz_1^n=0$.
En particular, existe $M>0$ tal que $|a_nz_1^n|\geq M$.
Sea $|z|<|z_1|$ luego
\[|a_n z^n|<|a_n z_1^n|\left|\frac{z}{z_1}\right|^n\leq M \left|\frac{z}{z_1}\right|^n. \]
Como $|\frac{z}{z_1}|<1$ entonces $\sum\limits_{n=0}^{\infty} (\frac{z}{z_1})^n$ converge y por el 
Criterio de Comparación obtenemos que $\sum\limits_{n=0}^{\infty} |a_nz^n|$ converge siempre que 
$|z|<|z_1|$.
\end{proof}

\begin{corolario}{}
Dada una serie de funciones $\sum\limits_{n=0}^{\infty} a_n (z-z_0)^n$ existe $R\geq 0$ tal que la serie
converge en $|z-z_0|<R$ y no converge en $|z-z_0|>R$. 
\end{corolario}

El criterio del cociente suele ser útil para determinar el valor de $R$ que se denomina \emph{radio de convergencia}.

\begin{ejemplo}{}
La serie $\sum\limits_{n=1}^{\infty} \frac{z^n}{n}$ converge si 
\[
1>\lim\limits_{n \to \infty} \frac{\frac{|z|^{n+1}}{n+1}}{\frac{|z|^n}{n}}=
|z|\lim\limits_{n \to \infty} \frac{n }{n+1}=|z|,
\]
y no converge si $|z|>1$. Luego, el radio de convergencia es $1$.

?`Qué pasa en el borde $|z|=1$?

Si  $|z|=1$ $\Leftrightarrow$ $z=\cos \theta +i\sen \theta$ y $z^n=\cos(n \theta)+i \sen(n \theta)$. 
Luego
\[
\sum\limits_{n=1}^{\infty} \frac{z^n}{n}=
\sum\limits_{n=1}^{\infty} \frac{\cos(n \theta)}{n}+i \sum \limits_{n=1}^{\infty} \frac{\sen (n \theta)}{n}.
\]
Esto nos lleva a considerar otras series.
\end{ejemplo}


\section{Series de Fourier}
Una serie de Fourier es una expresión de la forma
\[
f(x)=\frac{a_0}{2}+\sum\limits_{n=1}^{\infty} a_n \cos(n x)+b_n \sen(nx).
\]

Las series de Fourier tiene la capacidad de aproximar funciones $2\pi$-periódicas. 
En primer lugar,   es necesario saber elegir los coeficientes y para ello se usa la propiedad que se presenta a continuación.

\begin{lema}{}
\[
\begin{split}
&\int_{-\pi}^{\pi} \cos(nx) \cos(mx)\,dx=
	\left\{
	\begin{array}{ll}
0&n\neq m
\\
\pi&n=m\neq0
\\
2\pi&n=m=0,
	\end{array}
	\right.
\\
&
\int_{-\pi}^{\pi} \sen(nx) \sen(mx)\,dx=
\left\{
\begin{array}{ll}
0&n\neq m
\\
\pi&n=m\neq0
\\
2\pi&n=m=0,
\end{array}
\right.
\\
&\int_{-\pi}^{\pi} \cos(nx) \sen(mx)\,dx=0.
\end{split}
\]
\end{lema}


%\[\sum\limits_{k=0}^n {n \choose k} x^k (1-x)^{n-k}\]


Luego, si nos permitimos permutar integrales son sumas de series, tenemos
\[
\begin{split}
&\int_{-\pi}^{\pi} f(x)\sen(kx)\,dx
\\
&=\frac{a_0}{2}\int_{-\pi}^{\pi} \sen(kx)\,dx
\\
&+ 
\sum\limits_{n=1}^{\infty} 
a_n \int_{-\pi}^{\pi} \cos(nx)\sen(kx)\,dx
\\
&+
b_n \int_{-\pi}^{\pi}  \sen(nx) \sen(kx)\,dx
\\
&=\pi b_k 
\end{split}\]

\begin{definicion}{}
Dada una función $2\pi$-periódica $f(x)$ definimos los coeficientes de Fourier por
\begin{equation}
a_n=\frac{1}{\pi} \int_{-\pi}{\pi} f(x)\cos(nx)\,dx,\quad n\geq 0
\end{equation}
y 
\begin{equation}
b_n=\frac{1}{\pi} \int_{-\pi}{\pi} f(x)\sen(nx)\,dx,\quad n\geq 1.
\end{equation}
\end{definicion}

En la notebook de Sympy indagamos las facultades aproximativas de la serie de Fourier.
    \begin{tcolorbox}[breakable, size=fbox, boxrule=1pt, pad at break*=1mm,colback=cellbackground, colframe=cellborder]
\prompt{In}{incolor}{10}{\hspace{4pt}}
\begin{Verbatim}[commandchars=\\\{\}]
\PY{n}{x}\PY{o}{=}\PY{n}{symbols}\PY{p}{(}\PY{l+s+s1}{\PYZsq{}}\PY{l+s+s1}{x}\PY{l+s+s1}{\PYZsq{}}\PY{p}{,}\PY{n}{real}\PY{o}{=}\PY{n+nb+bp}{True}\PY{p}{)}
\PY{n}{n}\PY{p}{,}\PY{n}{m}\PY{o}{=}\PY{n}{symbols}\PY{p}{(}\PY{l+s+s1}{\PYZsq{}}\PY{l+s+s1}{n,m}\PY{l+s+s1}{\PYZsq{}}\PY{p}{,}\PY{n}{integer}\PY{o}{=}\PY{n+nb+bp}{True}\PY{p}{,}\PY{n}{positive}\PY{o}{=}\PY{n+nb+bp}{True}\PY{p}{)}
\PY{n}{Integral}\PY{p}{(}\PY{n}{cos}\PY{p}{(}\PY{n}{n}\PY{o}{*}\PY{n}{x}\PY{p}{)}\PY{o}{*}\PY{n}{cos}\PY{p}{(}\PY{n}{m}\PY{o}{*}\PY{n}{x}\PY{p}{)}\PY{p}{,}\PY{p}{(}\PY{n}{x}\PY{p}{,}\PY{o}{\PYZhy{}}\PY{n}{pi}\PY{p}{,}\PY{n}{pi}\PY{p}{)}\PY{p}{)}\PY{o}{.}\PY{n}{doit}\PY{p}{(}\PY{p}{)}
\end{Verbatim}
\end{tcolorbox}
 
            
\prompt{Out}{outcolor}{10}{}
    
    $$\begin{cases} 0 & \text{for}\: m \neq n \\\pi & \text{otherwise} \end{cases}$$

    

    \begin{tcolorbox}[breakable, size=fbox, boxrule=1pt, pad at break*=1mm,colback=cellbackground, colframe=cellborder]
\prompt{In}{incolor}{11}{\hspace{4pt}}
\begin{Verbatim}[commandchars=\\\{\}]
\PY{n}{Integral}\PY{p}{(}\PY{n}{sin}\PY{p}{(}\PY{n}{n}\PY{o}{*}\PY{n}{x}\PY{p}{)}\PY{o}{*}\PY{n}{sin}\PY{p}{(}\PY{n}{m}\PY{o}{*}\PY{n}{x}\PY{p}{)}\PY{p}{,}\PY{p}{(}\PY{n}{x}\PY{p}{,}\PY{o}{\PYZhy{}}\PY{n}{pi}\PY{p}{,}\PY{n}{pi}\PY{p}{)}\PY{p}{)}\PY{o}{.}\PY{n}{doit}\PY{p}{(}\PY{p}{)}
\end{Verbatim}
\end{tcolorbox}
 
            
\prompt{Out}{outcolor}{11}{}
    
    $$\begin{cases} 0 & \text{for}\: m \neq n \\\pi & \text{otherwise} \end{cases}$$

    

    \begin{tcolorbox}[breakable, size=fbox, boxrule=1pt, pad at break*=1mm,colback=cellbackground, colframe=cellborder]
\prompt{In}{incolor}{12}{\hspace{4pt}}
\begin{Verbatim}[commandchars=\\\{\}]
\PY{n}{Integral}\PY{p}{(}\PY{n}{cos}\PY{p}{(}\PY{n}{n}\PY{o}{*}\PY{n}{x}\PY{p}{)}\PY{o}{*}\PY{n}{sin}\PY{p}{(}\PY{n}{m}\PY{o}{*}\PY{n}{x}\PY{p}{)}\PY{p}{,}\PY{p}{(}\PY{n}{x}\PY{p}{,}\PY{o}{\PYZhy{}}\PY{n}{pi}\PY{p}{,}\PY{n}{pi}\PY{p}{)}\PY{p}{)}\PY{o}{.}\PY{n}{doit}\PY{p}{(}\PY{p}{)}
\end{Verbatim}
\end{tcolorbox}
 
            
\prompt{Out}{outcolor}{12}{}
    
    $$0$$

    

    \begin{tcolorbox}[breakable, size=fbox, boxrule=1pt, pad at break*=1mm,colback=cellbackground, colframe=cellborder]
\prompt{In}{incolor}{13}{\hspace{4pt}}
\begin{Verbatim}[commandchars=\\\{\}]
\PY{n}{S}\PY{o}{=}\PY{n+nb}{sum}\PY{p}{(}\PY{p}{[}\PY{n}{sin}\PY{p}{(}\PY{n}{n}\PY{o}{*}\PY{n}{theta}\PY{p}{)}\PY{o}{/}\PY{n}{n} \PY{k}{for} \PY{n}{n} \PY{o+ow}{in} \PY{n+nb}{range}\PY{p}{(}\PY{l+m+mi}{1}\PY{p}{,}\PY{l+m+mi}{50}\PY{p}{)}\PY{p}{]}\PY{p}{)}
\PY{n}{plot}\PY{p}{(}\PY{n}{S}\PY{p}{,}\PY{p}{(}\PY{n}{theta}\PY{p}{,}\PY{o}{\PYZhy{}}\PY{l+m+mi}{4}\PY{o}{*}\PY{n}{pi}\PY{p}{,}\PY{l+m+mi}{4}\PY{o}{*}\PY{n}{pi}\PY{p}{)}\PY{p}{)}
\end{Verbatim}
\end{tcolorbox}

    \begin{center}
    \adjustimage{max size={0.9\linewidth}{0.9\paperheight}}{python/uni3/output_20_0.png}
    \end{center}
    { \hspace*{\fill} \\}
    
            \begin{tcolorbox}[breakable, boxrule=.5pt, size=fbox, pad at break*=1mm, opacityfill=0]
\prompt{Out}{outcolor}{13}{\hspace{3.5pt}}
\begin{Verbatim}[commandchars=\\\{\}]
<sympy.plotting.plot.Plot at 0x7f413f16afd0>
\end{Verbatim}
\end{tcolorbox}
        
    Ejemplo
\[f(x)=\left\{\begin{array}{cc} \frac{\pi-x}{2}, & \text{ si } x\in [0,\pi]\\
    -\frac{\pi+x}{2}, & \text{ si } x\in [-\pi,0) \end{array}    \right.\]

    \begin{tcolorbox}[breakable, size=fbox, boxrule=1pt, pad at break*=1mm,colback=cellbackground, colframe=cellborder]
\prompt{In}{incolor}{14}{\hspace{4pt}}
\begin{Verbatim}[commandchars=\\\{\}]
\PY{n}{f}\PY{o}{=}\PY{n}{Piecewise}\PY{p}{(}\PY{p}{(}\PY{p}{(}\PY{n}{pi}\PY{o}{\PYZhy{}}\PY{n}{x}\PY{p}{)}\PY{o}{/}\PY{l+m+mi}{2}\PY{p}{,} \PY{n}{x}\PY{o}{\PYZgt{}}\PY{o}{=}\PY{l+m+mi}{0}  \PY{p}{)}\PY{p}{,}\PY{p}{(}\PY{o}{\PYZhy{}}\PY{p}{(}\PY{n}{pi}\PY{o}{+}\PY{n}{x}\PY{p}{)}\PY{o}{/}\PY{l+m+mi}{2}\PY{p}{,}\PY{n}{x}\PY{o}{\PYZlt{}}\PY{l+m+mi}{0} \PY{p}{)}\PY{p}{)}
\PY{n}{plot}\PY{p}{(}\PY{n}{f}\PY{p}{,}\PY{p}{(}\PY{n}{x}\PY{p}{,}\PY{o}{\PYZhy{}}\PY{l+m+mi}{4}\PY{p}{,}\PY{l+m+mi}{4}\PY{p}{)}\PY{p}{)}
\end{Verbatim}
\end{tcolorbox}

    \begin{center}
    \adjustimage{max size={0.9\linewidth}{0.9\paperheight}}{python/uni3/output_22_0.png}
    \end{center}
    { \hspace*{\fill} \\}
    
            \begin{tcolorbox}[breakable, boxrule=.5pt, size=fbox, pad at break*=1mm, opacityfill=0]
\prompt{Out}{outcolor}{14}{\hspace{3.5pt}}
\begin{Verbatim}[commandchars=\\\{\}]
<sympy.plotting.plot.Plot at 0x7f413f019b50>
\end{Verbatim}
\end{tcolorbox}
        
    \begin{tcolorbox}[breakable, size=fbox, boxrule=1pt, pad at break*=1mm,colback=cellbackground, colframe=cellborder]
\prompt{In}{incolor}{15}{\hspace{4pt}}
\begin{Verbatim}[commandchars=\\\{\}]
\PY{k}{def} \PY{n+nf}{a}\PY{p}{(}\PY{n}{g}\PY{p}{,}\PY{n}{k}\PY{p}{)}\PY{p}{:}
    \PY{k}{return} \PY{l+m+mi}{1}\PY{o}{/}\PY{n}{pi}\PY{o}{*}\PY{n}{Integral}\PY{p}{(}\PY{n}{g}\PY{o}{*}\PY{n}{cos}\PY{p}{(}\PY{n}{k}\PY{o}{*}\PY{n}{x}\PY{p}{)}\PY{p}{,}\PY{p}{(}\PY{n}{x}\PY{p}{,}\PY{o}{\PYZhy{}}\PY{n}{pi}\PY{p}{,}\PY{n}{pi}\PY{p}{)}\PY{p}{)}\PY{o}{.}\PY{n}{doit}\PY{p}{(}\PY{p}{)}
\PY{k}{def} \PY{n+nf}{b}\PY{p}{(}\PY{n}{g}\PY{p}{,}\PY{n}{k}\PY{p}{)}\PY{p}{:}
    \PY{k}{return} \PY{l+m+mi}{1}\PY{o}{/}\PY{n}{pi}\PY{o}{*}\PY{n}{Integral}\PY{p}{(}\PY{n}{g}\PY{o}{*}\PY{n}{sin}\PY{p}{(}\PY{n}{k}\PY{o}{*}\PY{n}{x}\PY{p}{)}\PY{p}{,}\PY{p}{(}\PY{n}{x}\PY{p}{,}\PY{o}{\PYZhy{}}\PY{n}{pi}\PY{p}{,}\PY{n}{pi}\PY{p}{)}\PY{p}{)}\PY{o}{.}\PY{n}{doit}\PY{p}{(}\PY{p}{)}
\end{Verbatim}
\end{tcolorbox}

    \begin{tcolorbox}[breakable, size=fbox, boxrule=1pt, pad at break*=1mm,colback=cellbackground, colframe=cellborder]
\prompt{In}{incolor}{16}{\hspace{4pt}}
\begin{Verbatim}[commandchars=\\\{\}]
\PY{p}{[}\PY{n}{a}\PY{p}{(}\PY{n}{f}\PY{p}{,}\PY{n}{k}\PY{p}{)} \PY{k}{for} \PY{n}{k} \PY{o+ow}{in} \PY{n+nb}{range}\PY{p}{(}\PY{l+m+mi}{1}\PY{p}{,}\PY{l+m+mi}{10}\PY{p}{)}\PY{p}{]}
\end{Verbatim}
\end{tcolorbox}
 
            
\prompt{Out}{outcolor}{16}{}
    
    $$\left [ 0, \quad 0, \quad 0, \quad 0, \quad 0, \quad 0, \quad 0, \quad 0, \quad 0\right ]$$

    

    \begin{tcolorbox}[breakable, size=fbox, boxrule=1pt, pad at break*=1mm,colback=cellbackground, colframe=cellborder]
\prompt{In}{incolor}{17}{\hspace{4pt}}
\begin{Verbatim}[commandchars=\\\{\}]
\PY{p}{[}\PY{n}{b}\PY{p}{(}\PY{n}{f}\PY{p}{,}\PY{n}{k}\PY{p}{)} \PY{k}{for} \PY{n}{k} \PY{o+ow}{in} \PY{n+nb}{range}\PY{p}{(}\PY{l+m+mi}{1}\PY{p}{,}\PY{l+m+mi}{10}\PY{p}{)}\PY{p}{]}
\end{Verbatim}
\end{tcolorbox}
 
            
\prompt{Out}{outcolor}{17}{}
    
    $$\left [ 1, \quad \frac{1}{2}, \quad \frac{1}{3}, \quad \frac{1}{4}, \quad \frac{1}{5}, \quad \frac{1}{6}, \quad \frac{1}{7}, \quad \frac{1}{8}, \quad \frac{1}{9}\right ]$$

    

    \begin{tcolorbox}[breakable, size=fbox, boxrule=1pt, pad at break*=1mm,colback=cellbackground, colframe=cellborder]
\prompt{In}{incolor}{18}{\hspace{4pt}}
\begin{Verbatim}[commandchars=\\\{\}]
\PY{n}{S}\PY{o}{=}\PY{n+nb}{sum}\PY{p}{(}\PY{p}{[}\PY{n}{b}\PY{p}{(}\PY{n}{f}\PY{p}{,}\PY{n}{k}\PY{p}{)}\PY{o}{*}\PY{n}{sin}\PY{p}{(}\PY{n}{k}\PY{o}{*}\PY{n}{x}\PY{p}{)} \PY{k}{for} \PY{n}{k} \PY{o+ow}{in} \PY{n+nb}{range}\PY{p}{(}\PY{l+m+mi}{1}\PY{p}{,}\PY{l+m+mi}{20}\PY{p}{)}\PY{p}{]}\PY{p}{)}
\PY{n}{S}
\end{Verbatim}
\end{tcolorbox}
 
            
\prompt{Out}{outcolor}{18}{}
    
    %$$
		\[\begin{split}
		&\sin{\left (x \right )} + \frac{\sin{\left (2 x \right )}}{2} + \frac{\sin{\left (3 x \right )}}{3} + \frac{\sin{\left (4 x \right )}}{4}
		\\&+\frac{\sin{\left (5 x \right )}}{5} + \frac{\sin{\left (6 x \right )}}{6} + \frac{\sin{\left (7 x \right )}}{7} +\frac{\sin{\left (8 x \right )}}{8} \\
		&+ \frac{\sin{\left (9 x \right )}}{9} + \frac{\sin{\left (10 x \right )}}{10} + \frac{\sin{\left (11 x \right )}}{11} 
		+ \frac{\sin{\left (12 x \right )}}{12} \\
		&+ \frac{\sin{\left (13 x \right )}}{13} + \frac{\sin{\left (14 x \right )}}{14} + \frac{\sin{\left (15 x \right )}}{15} + \frac{\sin{\left (16 x \right )}}{16} \\
		&+ \frac{\sin{\left (17 x \right )}}{17} + \frac{\sin{\left (18 x \right )}}{18} + \frac{\sin{\left (19 x \right )}}{19}
		\end{split}\]
		%$$

    

    \begin{tcolorbox}[breakable, size=fbox, boxrule=1pt, pad at break*=1mm,colback=cellbackground, colframe=cellborder]
\prompt{In}{incolor}{19}{\hspace{4pt}}
\begin{Verbatim}[commandchars=\\\{\}]
\PY{n}{S}\PY{o}{=}\PY{n+nb}{sum}\PY{p}{(}\PY{p}{[}\PY{n}{b}\PY{p}{(}\PY{n}{f}\PY{p}{,}\PY{n}{k}\PY{p}{)}\PY{o}{*}\PY{n}{sin}\PY{p}{(}\PY{n}{k}\PY{o}{*}\PY{n}{x}\PY{p}{)} \PY{k}{for} \PY{n}{k} \PY{o+ow}{in} \PY{n+nb}{range}\PY{p}{(}\PY{l+m+mi}{1}\PY{p}{,}\PY{l+m+mi}{20}\PY{p}{)}\PY{p}{]}\PY{p}{)}
\PY{n}{plot}\PY{p}{(}\PY{n}{f}\PY{p}{,}\PY{n}{S}\PY{p}{,} \PY{p}{(}\PY{n}{x}\PY{p}{,}\PY{o}{\PYZhy{}}\PY{n}{pi}\PY{p}{,}\PY{n}{pi}\PY{p}{)}\PY{p}{)}
\end{Verbatim}
\end{tcolorbox}

    \begin{center}
    \adjustimage{max size={0.9\linewidth}{0.9\paperheight}}{python/uni3/output_27_0.png}
    \end{center}
    { \hspace*{\fill} \\}
    
            \begin{tcolorbox}[breakable, boxrule=.5pt, size=fbox, pad at break*=1mm, opacityfill=0]
\prompt{Out}{outcolor}{19}{\hspace{3.5pt}}
\begin{Verbatim}[commandchars=\\\{\}]
<sympy.plotting.plot.Plot at 0x7f413f1ae210>
\end{Verbatim}
\end{tcolorbox}


Deberíamos justificar el uso de propiedades como 
\[
\int_a^b \sum\limits_{n=1}^{\infty} f_n(x)\,dx=\sum\limits_{n=1}^{\infty} \int_a^b f_n(x)\,dx,
\]
que se justificar\'ia  si 
\[
\begin{split}
&\int_a^b \sum\limits_{n=1}^{\infty} f_n(x)= \int_a^b \lim\limits_{N \to \infty} \sum\limits_{n=1}^{N}f_n(x)
\\
\underset{?}{=}
\lim\limits_{N \to \infty} &\int_a^b \sum\limits_{n=1}^{N} f_n(x)\,dx=
\lim\limits_{N \to \infty} \sum\limits_{n=1}^N \int_a^b f_n(x)\,dx
\end{split}
\]
Pero, ya hemos visto ejemplos de que éste no es siempre el caso. 
Vamos a identificar otro modo de convergencia que hace esta regla posible. 

Si $f_n(x)$ converge puntualmente a $f$ entonces
\[
\forall x \forall \epsilon>0 \exists N=N(\epsilon, x)>0:
n\geq N\Rightarrow|f_n(x)-f(x)|<\epsilon.
\]

\begin{definicion}{}
$f_n$ converge uniformemente a $f$ si y sólo si
\[
\forall \epsilon>0 \exists N=N(\epsilon)>0 \forall x:
n\geq N\Rightarrow|f_n(x)-f(x)|<\epsilon.
\]
\end{definicion}
IDEA GRÁFICA 

\begin{ejemplo}{}
\begin{enumerate}
\item
$f_n(x)=\frac{1}{n}xe^{-n^2x^2}$ converge uniformemente a cero. 

Se tiene que 
$f^{'}_n(x)=\frac{1}{n}e^{-n^2x^2}-n2x^2 e^{-n^2x^2}$ y 
\[f^{'}_n(x)=0 \Longleftrightarrow 0=e^{-n^2x^2}\left(\frac{1}{n}-2nx^2\right) \Leftrightarrow x=\pm\frac{1}{\sqrt{2}n}.\]
Luego $f^{'}_n(x)>0$ en $|x|<\frac{1}{\sqrt{2}n}$ y $f^{'}_n(x)<0$ en $|x|>\frac{1}{\sqrt{2}n}$.
Y, $f\left(\pm\frac{1}{\sqrt{2}n}\right)=\frac{1}{n}(\pm\frac{1}{\sqrt{2}n} )e^{-\frac{1}{2}}=
\pm \frac{1}{\sqrt{2}}e^{-\frac{1}{2}} \frac{1}{n^2}.$

Luego, dado $\epsilon>0$ existe $N\geq \left(\frac{e^{\frac{1}{2}}}{\sqrt{2}\epsilon}\right)^{\frac{1}{2}}$.

    \begin{tcolorbox}[breakable, size=fbox, boxrule=1pt, pad at break*=1mm,colback=cellbackground, colframe=cellborder]
\prompt{In}{incolor}{20}{\hspace{4pt}}
\begin{Verbatim}[commandchars=\\\{\}]
\PY{n}{x}\PY{p}{,}\PY{n}{n}\PY{o}{=}\PY{n}{symbols}\PY{p}{(}\PY{l+s+s1}{\PYZsq{}}\PY{l+s+s1}{x,n}\PY{l+s+s1}{\PYZsq{}}\PY{p}{)}
\PY{n}{fn}\PY{o}{=}\PY{l+m+mi}{1}\PY{o}{/}\PY{n}{n}\PY{o}{*}\PY{n}{exp}\PY{p}{(}\PY{o}{\PYZhy{}}\PY{n}{n}\PY{o}{*}\PY{o}{*}\PY{l+m+mi}{2}\PY{o}{*}\PY{n}{x}\PY{o}{*}\PY{o}{*}\PY{l+m+mi}{2}\PY{p}{)}\PY{o}{*}\PY{n}{x}
\PY{n}{p}\PY{o}{=}\PY{n}{plot}\PY{p}{(}\PY{n}{fn}\PY{o}{.}\PY{n}{subs}\PY{p}{(}\PY{n}{n}\PY{p}{,}\PY{l+m+mi}{1}\PY{p}{)}\PY{p}{,} \PY{p}{(}\PY{n}{x}\PY{p}{,}\PY{o}{\PYZhy{}}\PY{l+m+mi}{5}\PY{p}{,}\PY{l+m+mi}{5}\PY{p}{)}\PY{p}{,}\PY{n}{show}\PY{o}{=}\PY{n}{false}\PY{p}{)}
\PY{k}{for} \PY{n}{k} \PY{o+ow}{in} \PY{n+nb}{range}\PY{p}{(}\PY{l+m+mi}{2}\PY{p}{,}\PY{l+m+mi}{10}\PY{p}{)}\PY{p}{:}
    \PY{n}{p}\PY{o}{.}\PY{n}{append}\PY{p}{(}\PY{n}{plot}\PY{p}{(}\PY{n}{fn}\PY{o}{.}\PY{n}{subs}\PY{p}{(}\PY{n}{n}\PY{p}{,}\PY{n}{k}\PY{p}{)}\PY{p}{,} \PY{p}{(}\PY{n}{x}\PY{p}{,}\PY{o}{\PYZhy{}}\PY{l+m+mi}{5}\PY{p}{,}\PY{l+m+mi}{5}\PY{p}{)}\PY{p}{,}\PY{n}{show}\PY{o}{=}\PY{n}{false}\PY{p}{)}\PY{p}{[}\PY{l+m+mi}{0}\PY{p}{]}\PY{p}{)}
\PY{n}{p}\PY{o}{.}\PY{n}{show}\PY{p}{(}\PY{p}{)}
\end{Verbatim}
\end{tcolorbox}

    \begin{center}
    \adjustimage{max size={0.9\linewidth}{0.9\paperheight}}{python/uni3/output_29_0.png}
    \end{center}
    { \hspace*{\fill} \\}
\item $f_n(x)=nxe^{-n^2x^2}$ converge puntualmente pero no uniformemente a cero.

Si $x\neq 0$, $\lim\limits_{n\to \infty} \frac{nx}{e^{n^2x^2}}=\lim\limits_{n \to \infty} \frac{x}{2ne^{n^2x^2}}=0$.

Por otra parte, 
\[
0=f^{'}_n(x)=-n^2e^{-n^2 x^2}2x^2+e^{-n^2x^2}\Leftrightarrow 1-2n^2x^2=0 \Leftrightarrow x=\pm \frac{1}{\sqrt{2}n}.
\]
Y, $f\left(\pm\frac{1}{\sqrt{2}n}\right)=\pm e^{\frac{1}{2}}\frac{1}{\sqrt{2}}$.

    \begin{tcolorbox}[breakable, size=fbox, boxrule=1pt, pad at break*=1mm,colback=cellbackground, colframe=cellborder]
\prompt{In}{incolor}{21}{\hspace{4pt}}
\begin{Verbatim}[commandchars=\\\{\}]
\PY{n}{fn}\PY{o}{=}\PY{n}{n}\PY{o}{*}\PY{n}{exp}\PY{p}{(}\PY{o}{\PYZhy{}}\PY{n}{n}\PY{o}{*}\PY{o}{*}\PY{l+m+mi}{2}\PY{o}{*}\PY{n}{x}\PY{o}{*}\PY{o}{*}\PY{l+m+mi}{2}\PY{p}{)}\PY{o}{*}\PY{n}{x}
\PY{n}{p}\PY{o}{=}\PY{n}{plot}\PY{p}{(}\PY{n}{fn}\PY{o}{.}\PY{n}{subs}\PY{p}{(}\PY{n}{n}\PY{p}{,}\PY{l+m+mi}{1}\PY{p}{)}\PY{p}{,} \PY{p}{(}\PY{n}{x}\PY{p}{,}\PY{o}{\PYZhy{}}\PY{l+m+mi}{5}\PY{p}{,}\PY{l+m+mi}{5}\PY{p}{)}\PY{p}{,}\PY{n}{show}\PY{o}{=}\PY{n}{false}\PY{p}{)}
\PY{k}{for} \PY{n}{k} \PY{o+ow}{in} \PY{n+nb}{range}\PY{p}{(}\PY{l+m+mi}{2}\PY{p}{,}\PY{l+m+mi}{10}\PY{p}{)}\PY{p}{:}
    \PY{n}{p}\PY{o}{.}\PY{n}{append}\PY{p}{(}\PY{n}{plot}\PY{p}{(}\PY{n}{fn}\PY{o}{.}\PY{n}{subs}\PY{p}{(}\PY{n}{n}\PY{p}{,}\PY{n}{k}\PY{p}{)}\PY{p}{,} \PY{p}{(}\PY{n}{x}\PY{p}{,}\PY{o}{\PYZhy{}}\PY{l+m+mi}{5}\PY{p}{,}\PY{l+m+mi}{5}\PY{p}{)}\PY{p}{,}\PY{n}{show}\PY{o}{=}\PY{n}{false}\PY{p}{)}\PY{p}{[}\PY{l+m+mi}{0}\PY{p}{]}\PY{p}{)}
\PY{n}{p}\PY{o}{.}\PY{n}{show}\PY{p}{(}\PY{p}{)}
\end{Verbatim}
\end{tcolorbox}

    \begin{center}
    \adjustimage{max size={0.9\linewidth}{0.9\paperheight}}{python/uni3/output_31_0.png}
    \end{center}
    { \hspace*{\fill} \\}
\end{enumerate}
\end{ejemplo}


\begin{teorema}{}
Si $f_n: A\to B$ (aquí $A,B \subset \mathbb{R}^n$ ó $A,B \subset \mathbb{C}$) 
son continuas y convergen uniformemente a $f$, entonces $f$ es continua y 
\[
\lim\limits_{x \to a}\lim\limits_{n \to \infty} f_n(x)=
\lim\limits_{n \to \infty} f_n(a)=
\lim\limits_{n \to \infty} \lim \limits_{x \to a} f_n(x).
\]
\end{teorema}

\begin{proof}
Sea $f(x)=\lim\limits_{n \to \infty} f_n(x)$ y $a \in A$. 

Sea $\epsilon>0$, entonces $\forall x \exists N: n\geq N\Rightarrow|f_n(x)-f(x)|<\frac{\epsilon}{3} $. 

Fijamos un $n$ cualquiera que satisfaga la desigualdad anterior.

Como $f_n$ es continua en $a$ $\exists \delta=\delta(\epsilon,a)$ tal que 
$|x-a|<\delta \Rightarrow |f_n(x)-f_n(a)|<\frac{\epsilon}{3}$. 

Entonces 
\[
|f(x)-f(a)|\leq |f(x)-f_n(x)|+|f_n(x)-f_n(a)|+|f_n(a)-f(a)|<\epsilon.
\]
\end{proof}

\begin{ejemplo}{}
$x^n$ converge puntualmente en $[0,1]$ pero no uniformemente.
\end{ejemplo}

\begin{teorema}{}
Si $f_n$ converge uniformemente a $f$ en $[a,b]$ entonces
\[
\lim\limits_{n \to \infty} \int_a^b f_n(x)\,dx=\int_a^b f(x)\,dx.
\]
\end{teorema}

\begin{proof}
La prueba se deja como ejercicio.
\end{proof}

\begin{teorema}[M-test Weiertrass]{}
Si $f_n:A\subset \mathbb{R}^n \to \mathbb{R}$ y $|f_n(x)|\leq M_n$ independientemente de $x$
y $\sum\limits_{n=1}^{\infty} M_n<\infty$ entonces la serie $\sum\limits_{n=1}^{\infty}f_n(x)$
converge uniformemente en $A$.
\end{teorema}

\begin{proof}
La serie converge puntualmente por aplicación del \textit{Criterio de Comparación}.

Sea $f(x)=\sum\limits_{n=1}^{\infty} f_n(x)$ y sea $\epsilon>0$.
Entonces $\exists N>0 $ tal que $\sum\limits_{n=N+1}^{\infty} M_n<\epsilon$. 

Luego, 
\[
\begin{split}
&\left|f(x)-\sum\limits_{n=1}^{N} f_n(x)\right|=\left|\sum\limits_{n=N+1}^{\infty} f_n(x)\right|
\\
=&\left|\lim\limits_{M\to \infty} \sum\limits_{n=N+1}^{M} f_n(x)\right|
\leq \lim\limits_{M\to \infty} \sum\limits_{n=N+1}^{M} M_n
<\epsilon.
\end{split}\]
\end{proof}

\begin{corolario}{}
Si la serie de potencias $\sum\limits_{n=0}^{\infty} a_n (z-z_0)^n$ tiene radio de convergencia $R>0$, 
entonces converge uniformemente en $|z-z_0|\leq r$ $\forall r<R$.
\end{corolario}

\begin{proof}
Supongamos $z_0=0$.

Sea $0<r<R$ entonces la serie converge abasolutamente en $|z|=r$ y por tanto 
\[
\sum\limits_{n=0}^{\infty}|a_n|r^n
<\infty.\]
Luego si $|z|<r$ entonces $|a_nz^n|\leq |a_n|r^n$ y se verifican las hipótesis del M-Test de Weierstrass.
\end{proof}

\begin{ejemplo}{}
Analizar con Sympy la convergencia de $\sum\limits_{n=0}^{\infty} x^n$.
\end{ejemplo}

\begin{corolario}{}
Si $\sum\limits_{n=0}^{\infty}|a_n|+|b_n|<\infty$ entonces la serie de Fourier 
\[
\frac{a_0}{2}+\sum\limits_{n=1}^{\infty} a_n \cos(nx)+b_n \sen(nx),
\]
converge uniformemente a una función $f$ en $[-\pi,\pi]$ y 
\[
\begin{split}
a_n=\frac{1}{\pi}\int_{-\pi}^{\pi} f(x)\cos(nx)\,dx,\\
b_n=\frac{1}{\pi}\int_{-\pi}^{\pi} f(x)\sen(nx)\,dx.
\end{split}
\]
\end{corolario}

\textbf{Problema:} Hallar $\sum\limits_{n=1}^{\infty} \frac{1}{n^2}$.


\section{Productos infinitos}

Si un polinomio
\[
p(x)=a_nx^n+a_{n-1}x^{n-1}+\ldots+a_1x+1,
\]
tiene raíces reales $x_1,x_2,\ldots,x_n$ entonces
\[
x_1^{-1}+x_2^{-1}+\ldots+x_n^{-1}=-a_1.
\]

\begin{proof}
Tenemos que $p(x)=a_n(x-x_1)(x-x_2)\ldots(x-x_n)$ entonces $1=p(0)=(-1)^n x_1\ldots x_n$
y \[p'(0)=a_1=a_n (-1)^{n-1}(x_1x_3\ldots x_n+x_2x_3\ldots x_n+\ldots +x_1x_2\ldots x_{n-1}).\]

Luego
\[
\begin{split}
a_1&=\frac{p'(0)}{p(0)}
\\
&=\frac{a_n (-1)^{n-1}(x_1x_3\ldots x_n+x_2x_3\ldots x_n+\ldots +x_1x_2\ldots x_{n-1})}{(-1)^n x_1\ldots x_n}
\\&=
-\left(\frac{1}{x_1}+\frac{1}{x_2}+\ldots+\frac{1}{x_n}\right).
\end{split}
\]
Además 
\[
p(x)=(-1)^n x_1x_2\ldots x_n \left(1-\frac{x}{x_1}\right) \left(1-\frac{x}{x_2}\right)\ldots \left(1-\frac{x}{x_n}\right).
\]
\end{proof}

\begin{teorema}[Euler(1748)]{}
\[\sen x=
x \prod\limits_{k=1}^{\infty}
\left(1-\frac{x^2}{k^2 \pi^2}\right).
\]
\end{teorema}

\begin{proof}
\begin{equation}\label{eq:previa-euler-a-wallis}
\begin{split}
\frac{\sen x}{x}=&\frac{x-\frac{x^3}{3!}+\frac{x^5}{5!}+\ldots+(-1)^{n+1}\frac{x^{2n-1}}{(2n-1)!}+\ldots}{x}
\\
=&\left(1-\frac{x}{\pi}\right)\left(1+\frac{x}{\pi}\right)\left(1-\frac{x}{2\pi}\right)\left(1+\frac{x}{2\pi}\right)\ldots
\\
=& \left(1-\frac{x^2}{\pi^2}\right) \left(1-\frac{x^2}{4\pi^2}\right)\ldots \left(1-\frac{x^2}{k^2 \pi^2}\right)\ldots
\end{split}
\end{equation}
\end{proof}
Tomando $x^2=y$  en \eqref{eq:previa-euler-a-wallis} llegamos a 
\[1-\frac{y}{3!}+\frac{y^2}{5!}+\frac{y^4}{7!}+\ldots=\left(1-\frac{y}{\pi^2}\right)\ldots\left(1-\frac{y}{k^2\pi^2}\right).
\]
Luego
\[
\frac{1}{\pi^2}+\frac{1}{4\pi^2}+\ldots=\frac{1}{6}
\]
y 
\[
1+\frac{1}{4}+\ldots=\frac{\pi^2}{6}.
\]
Evaluando en $x=\frac{\pi}{2}$ obtenemos
\[
\begin{split}
\frac{2}{\pi}&=
\left(1-\frac{(\frac{\pi}{2})^2}{\pi^2}\right) \left(1-\frac{(\frac{\pi}{2})^2}{4\pi^2}\right)\ldots
\left(1-\frac{(\frac{\pi}{2})^2}{k^2\pi^2}\right)\ldots
\\
&=\prod\limits_{k=1}^{\infty} \left(1-\frac{1}{4k^2}\right)=\prod\limits_{k=1}^{\infty}\frac{4k^2-1}{4k^2}
\\
&=\prod\limits_{k=1}^{\infty} \frac{2k-1}{2k}\frac{2k+1}{2k}=\frac{1\cdot3}{2\cdot 2}\frac{3 \cdot 5}{4 \cdot 4}\ldots
\end{split}
\]
y por lo tanto obtenemos la fórmula de Wallis
\[
\frac{\pi}{2}=\frac{2\cdot 2}{1 \cdot 3}\frac{4 \cdot 4}{3\cdot 5}\ldots
\]

\section{Aproximación de funciones}

\begin{teorema}[Weierstrass]{}
Si $f$ es continua en $[0,1]$ entonces $f$ es límite uniforme de polinomios.
\end{teorema}

\begin{definicion}{}
Si $f$ es una función se define su polinomio de Berstein de grado n por
\[
B_n(f)=\sum\limits_{k=0}^{n} {n \choose k} f\left(\frac{k}{n}\right) x^k (1-x)^{n-k}. 
\]
\end{definicion}

Si $f \equiv 1$ entonces 
\[
B_n(f)=\sum\limits_{k=0}^{n} {n \choose k} x^k (1-x)^{n-k} =[x+(1-x)]^n=1.
\]


Si $f\equiv x$ entonces 
\[\begin{split}
B_n(f)&=\sum\limits_{k=0}^{n} {n \choose k} \frac{k}{n} x^k (1-x)^{n-k} 
\\&=
x \sum\limits_{k=1}^{n} \frac{(n-1)!}{(k-1)![n-1-(k-1)]!}  x^{k-1} (1-x)^{[n-1-(k-1)]}
\\&=
x B_{n-1}(1)=x.
\end{split}
\]


Si $f\equiv x^2$ entonces 
\[
\begin{split}
B_n(f)
&=\sum\limits_{k=0}^{n} {n \choose k} \frac{k^2}{n^2} x^k (1-x)^{n-k} 
\\
&=
x \sum\limits_{k=1}^{n}  \frac{(n-1)!} {(k-1)![n-1-(k-1)]!} \frac{k}{n}  x^{k-1} (1-x)^{n-k}
\\
&=
x \sum\limits_{k=1}^{n}  \frac{k-1}{n} {{n-1} \choose{k-1}}  x^{k-1} (1-x)^{n-k}
+
\frac{x}{n} \sum\limits_{k=1}^{n} {{n-1 }\choose{k-1}}   x^{k-1} (1-x)^{n-k}
\\
&=
x \frac{n-1}{n} B_{n-1}(x) +  \frac{x}{n} B_{n-1}(1)=\frac{n-1}{n}x^2+\frac{x}{n}.
\end{split}
\]
Luego, tenemos que $B_n(x^2) \to x^2$ uniformemente.

Dado que $f$ es continua, para cada $\epsilon>0$  existe $\delta>0$ tal  que si $|x-y|<\delta$
entonces $|f(x)-f(y)|<\epsilon$.

Sean $I_{x}=\{ k| \left|\frac{k}{n}-x\right|<\delta \}$ y  $J_x=\{ 1,\dots,n\}-I_x$, luego
\[
\begin{split}
&|f(x)-B_n(f)(x)|
\\
=&\left|\sum\limits_{k=0}^n f(x){n \choose k} x^k (1-x)^{n-k}-
\sum\limits_{k=0}^n f\left(\frac{k}{n}\right){n \choose k} x^k (1-x)^{n-k}\right|
\\
\leq&
\sum\limits_{k=0}^n \left|f(x)-f\left(\frac{k}{n}\right)\right|{n \choose k} x^k (1-x)^{n-k}
\\
\leq 
&\sum\limits_{I_x}+\sum\limits_{J_x}<\epsilon+2M \sum\limits_{k=0}^n {n \choose k} x^k (1-x)^{n-k}
\end{split}
\]
siendo $M=\sup|f|$.

Ahora, 
\[
\begin{split}
\sum\limits_{k=0}^n {n \choose k} x^k (1-x)^{n-k}
\leq 
&\frac{1}{\delta^2} \sum\limits_{k=0}^n \left(x-\frac{k}{n}\right)^2 {n \choose k} x^k (1-x)^{n-k}
\\
=&
\frac{1}{\delta^2} \left(x^2-2x^2+\frac{n-1}{n}x^2+\frac{x}{n}\right).
\end{split}
\]

Luego
\[
\left|\sum\limits_{J_x} {n \choose k} x^k (1-x)^{n-k}
\right|
\leq \frac{1}{\delta^2}\left\{\frac{|x|^2+|x|}{n}\right\}\leq \frac{1}{n\delta^2},
\]
y podemos elegir $n$ tal que $\frac{1}{n \delta^2}<\epsilon$.

\newpage
 Ejemplos unidad 2

    \begin{tcolorbox}[breakable, size=fbox, boxrule=1pt, pad at break*=1mm,colback=cellbackground, colframe=cellborder]
\prompt{In}{incolor}{1}{\hspace{4pt}}
\begin{Verbatim}[commandchars=\\\{\}]
\PY{k+kn}{from} \PY{n+nn}{sympy} \PY{k+kn}{import} \PY{o}{*}
\PY{n}{init\PYZus{}printing}\PY{p}{(}\PY{p}{)}
\end{Verbatim}
\end{tcolorbox}

    Ejemplo \(f_n(x)=\frac{1}{1+nx^2}\) En este ejemplo 

    \begin{tcolorbox}[breakable, size=fbox, boxrule=1pt, pad at break*=1mm,colback=cellbackground, colframe=cellborder]
\prompt{In}{incolor}{1}{\hspace{4pt}}
\begin{Verbatim}[commandchars=\\\{\}]
\PY{n}{x}\PY{p}{,}\PY{n}{n}\PY{o}{=}\PY{n}{symbols}\PY{p}{(}\PY{l+s+s1}{\PYZsq{}}\PY{l+s+s1}{x,n}\PY{l+s+s1}{\PYZsq{}}\PY{p}{)}
\PY{n}{fn}\PY{o}{=}\PY{l+m+mi}{1}\PY{o}{/}\PY{p}{(}\PY{l+m+mi}{1}\PY{o}{+}\PY{n}{n}\PY{o}{*}\PY{n}{x}\PY{o}{*}\PY{o}{*}\PY{l+m+mi}{2}\PY{p}{)}
\PY{n}{p}\PY{o}{=}\PY{n}{plot}\PY{p}{(}\PY{n}{fn}\PY{o}{.}\PY{n}{subs}\PY{p}{(}\PY{n}{n}\PY{p}{,}\PY{l+m+mi}{1}\PY{p}{)}\PY{p}{,} \PY{p}{(}\PY{n}{x}\PY{p}{,}\PY{o}{\PYZhy{}}\PY{l+m+mi}{5}\PY{p}{,}\PY{l+m+mi}{5}\PY{p}{)}\PY{p}{,}\PY{n}{show}\PY{o}{=}\PY{n}{false}\PY{p}{)}
\PY{k}{for} \PY{n}{k} \PY{o+ow}{in} \PY{n+nb}{range}\PY{p}{(}\PY{l+m+mi}{2}\PY{p}{,}\PY{l+m+mi}{100}\PY{p}{)}\PY{p}{:}
    \PY{n}{p}\PY{o}{.}\PY{n}{append}\PY{p}{(}\PY{n}{plot}\PY{p}{(}\PY{n}{fn}\PY{o}{.}\PY{n}{subs}\PY{p}{(}\PY{n}{n}\PY{p}{,}\PY{n}{k}\PY{p}{)}\PY{p}{,} \PY{p}{(}\PY{n}{x}\PY{p}{,}\PY{o}{\PYZhy{}}\PY{l+m+mi}{5}\PY{p}{,}\PY{l+m+mi}{5}\PY{p}{)}\PY{p}{,}\PY{n}{show}\PY{o}{=}\PY{n}{false}\PY{p}{)}\PY{p}{[}\PY{l+m+mi}{0}\PY{p}{]}\PY{p}{)}
\PY{n}{p}\PY{o}{.}\PY{n}{show}\PY{p}{(}\PY{p}{)}
\end{Verbatim}
\end{tcolorbox}

    \begin{center}
    \adjustimage{max size={0.9\linewidth}{0.9\paperheight}}{python/uni3/output_3_0.png}
    \end{center}
    { \hspace*{\fill} \\}
    
    Ejemplo \(f_n(x)=\frac{n^2x-n^2}{1+nx^2}\)

    \begin{tcolorbox}[breakable, size=fbox, boxrule=1pt, pad at break*=1mm,colback=cellbackground, colframe=cellborder]
\prompt{In}{incolor}{3}{\hspace{4pt}}
\begin{Verbatim}[commandchars=\\\{\}]
\PY{n}{x}\PY{p}{,}\PY{n}{n}\PY{o}{=}\PY{n}{symbols}\PY{p}{(}\PY{l+s+s1}{\PYZsq{}}\PY{l+s+s1}{x,n}\PY{l+s+s1}{\PYZsq{}}\PY{p}{)}
\PY{n}{fn}\PY{o}{=}\PY{p}{(}\PY{n}{n}\PY{o}{*}\PY{o}{*}\PY{l+m+mi}{2}\PY{o}{*}\PY{n}{x}\PY{o}{\PYZhy{}}\PY{n}{n}\PY{o}{*}\PY{o}{*}\PY{l+m+mi}{2}\PY{p}{)}\PY{o}{/}\PY{p}{(}\PY{l+m+mi}{1}\PY{o}{+}\PY{n}{n}\PY{o}{*}\PY{n}{x}\PY{o}{*}\PY{o}{*}\PY{l+m+mi}{2}\PY{p}{)}
\PY{n}{p}\PY{o}{=}\PY{n}{plot}\PY{p}{(}\PY{n}{fn}\PY{o}{.}\PY{n}{subs}\PY{p}{(}\PY{n}{n}\PY{p}{,}\PY{l+m+mi}{1}\PY{p}{)}\PY{p}{,} \PY{p}{(}\PY{n}{x}\PY{p}{,}\PY{o}{\PYZhy{}}\PY{l+m+mi}{5}\PY{p}{,}\PY{l+m+mi}{5}\PY{p}{)}\PY{p}{,}\PY{n}{show}\PY{o}{=}\PY{n}{false}\PY{p}{,}\PY{n}{ylim}\PY{o}{=}\PY{p}{(}\PY{o}{\PYZhy{}}\PY{l+m+mi}{20}\PY{p}{,}\PY{l+m+mi}{10}\PY{p}{)}\PY{p}{)}
\PY{k}{for} \PY{n}{k} \PY{o+ow}{in} \PY{n+nb}{range}\PY{p}{(}\PY{l+m+mi}{2}\PY{p}{,}\PY{l+m+mi}{10}\PY{p}{)}\PY{p}{:}
    \PY{n}{p}\PY{o}{.}\PY{n}{append}\PY{p}{(}\PY{n}{plot}\PY{p}{(}\PY{n}{fn}\PY{o}{.}\PY{n}{subs}\PY{p}{(}\PY{n}{n}\PY{p}{,}\PY{n}{k}\PY{p}{)}\PY{p}{,} \PY{p}{(}\PY{n}{x}\PY{p}{,}\PY{o}{\PYZhy{}}\PY{l+m+mi}{5}\PY{p}{,}\PY{l+m+mi}{5}\PY{p}{)}\PY{p}{,}\PY{n}{show}\PY{o}{=}\PY{n}{false}\PY{p}{)}\PY{p}{[}\PY{l+m+mi}{0}\PY{p}{]}\PY{p}{)}
\PY{n}{p}\PY{o}{.}\PY{n}{show}\PY{p}{(}\PY{p}{)}
\end{Verbatim}
\end{tcolorbox}

    \begin{center}
    \adjustimage{max size={0.9\linewidth}{0.9\paperheight}}{python/uni3/output_5_0.png}
    \end{center}
    { \hspace*{\fill} \\}
    
    Ejemplo \(f_n(x)=\frac{nx}{1+n^2x^2}\)

    \begin{tcolorbox}[breakable, size=fbox, boxrule=1pt, pad at break*=1mm,colback=cellbackground, colframe=cellborder]
\prompt{In}{incolor}{4}{\hspace{4pt}}
\begin{Verbatim}[commandchars=\\\{\}]
\PY{n}{x}\PY{p}{,}\PY{n}{n}\PY{o}{=}\PY{n}{symbols}\PY{p}{(}\PY{l+s+s1}{\PYZsq{}}\PY{l+s+s1}{x,n}\PY{l+s+s1}{\PYZsq{}}\PY{p}{)}
\PY{n}{fn}\PY{o}{=}\PY{n}{n}\PY{o}{*}\PY{n}{x}\PY{o}{/}\PY{p}{(}\PY{l+m+mi}{1}\PY{o}{+}\PY{n}{n}\PY{o}{*}\PY{o}{*}\PY{l+m+mi}{2}\PY{o}{*}\PY{n}{x}\PY{o}{*}\PY{o}{*}\PY{l+m+mi}{2}\PY{p}{)}
\PY{n}{p}\PY{o}{=}\PY{n}{plot}\PY{p}{(}\PY{n}{fn}\PY{o}{.}\PY{n}{subs}\PY{p}{(}\PY{n}{n}\PY{p}{,}\PY{l+m+mi}{1}\PY{p}{)}\PY{p}{,} \PY{p}{(}\PY{n}{x}\PY{p}{,}\PY{l+m+mi}{0}\PY{p}{,}\PY{l+m+mi}{5}\PY{p}{)}\PY{p}{,}\PY{n}{show}\PY{o}{=}\PY{n}{false}\PY{p}{)}
\PY{k}{for} \PY{n}{k} \PY{o+ow}{in} \PY{n+nb}{range}\PY{p}{(}\PY{l+m+mi}{2}\PY{p}{,}\PY{l+m+mi}{10}\PY{p}{)}\PY{p}{:}
    \PY{n}{p}\PY{o}{.}\PY{n}{append}\PY{p}{(}\PY{n}{plot}\PY{p}{(}\PY{n}{fn}\PY{o}{.}\PY{n}{subs}\PY{p}{(}\PY{n}{n}\PY{p}{,}\PY{n}{k}\PY{p}{)}\PY{p}{,} \PY{p}{(}\PY{n}{x}\PY{p}{,}\PY{l+m+mi}{0}\PY{p}{,}\PY{l+m+mi}{5}\PY{p}{)}\PY{p}{,}\PY{n}{show}\PY{o}{=}\PY{n}{false}\PY{p}{)}\PY{p}{[}\PY{l+m+mi}{0}\PY{p}{]}\PY{p}{)}
\PY{n}{p}\PY{o}{.}\PY{n}{show}\PY{p}{(}\PY{p}{)}
\end{Verbatim}
\end{tcolorbox}

    \begin{center}
    \adjustimage{max size={0.9\linewidth}{0.9\paperheight}}{python/uni3/output_7_0.png}
    \end{center}
    { \hspace*{\fill} \\}
    
    Ejemplo \(f_n(x)=\sqrt{x^2+\frac{1}{n^2}}\)

    \begin{tcolorbox}[breakable, size=fbox, boxrule=1pt, pad at break*=1mm,colback=cellbackground, colframe=cellborder]
\prompt{In}{incolor}{5}{\hspace{4pt}}
\begin{Verbatim}[commandchars=\\\{\}]
\PY{k}{def} \PY{n+nf}{grafica}\PY{p}{(}\PY{n}{f}\PY{p}{,}\PY{n}{x1}\PY{p}{,}\PY{n}{x2}\PY{p}{,}\PY{n}{m}\PY{p}{)}\PY{p}{:}
    \PY{n}{p}\PY{o}{=}\PY{n}{plot}\PY{p}{(}\PY{n}{f}\PY{o}{.}\PY{n}{subs}\PY{p}{(}\PY{n}{n}\PY{p}{,}\PY{l+m+mi}{1}\PY{p}{)}\PY{p}{,} \PY{p}{(}\PY{n}{x}\PY{p}{,}\PY{n}{x1}\PY{p}{,}\PY{n}{x2}\PY{p}{)}\PY{p}{,}\PY{n}{show}\PY{o}{=}\PY{n}{false}\PY{p}{)}
    \PY{k}{for} \PY{n}{k} \PY{o+ow}{in} \PY{n+nb}{range}\PY{p}{(}\PY{l+m+mi}{2}\PY{p}{,}\PY{n}{m}\PY{p}{)}\PY{p}{:}
        \PY{n}{p}\PY{o}{.}\PY{n}{append}\PY{p}{(}\PY{n}{plot}\PY{p}{(}\PY{n}{f}\PY{o}{.}\PY{n}{subs}\PY{p}{(}\PY{n}{n}\PY{p}{,}\PY{n}{k}\PY{p}{)}\PY{p}{,} \PY{p}{(}\PY{n}{x}\PY{p}{,}\PY{n}{x1}\PY{p}{,}\PY{n}{x2}\PY{p}{)}\PY{p}{,}\PY{n}{show}\PY{o}{=}\PY{n}{false}\PY{p}{)}\PY{p}{[}\PY{l+m+mi}{0}\PY{p}{]}\PY{p}{)}
    \PY{n}{p}\PY{o}{.}\PY{n}{show}\PY{p}{(}\PY{p}{)}
\PY{n}{f}\PY{o}{=}\PY{n}{sqrt}\PY{p}{(}\PY{n}{x}\PY{o}{*}\PY{o}{*}\PY{l+m+mi}{2}\PY{o}{+}\PY{l+m+mf}{1.0}\PY{o}{/}\PY{n}{n}\PY{o}{*}\PY{o}{*}\PY{l+m+mi}{2}\PY{p}{)}
\PY{n}{grafica}\PY{p}{(}\PY{n}{f}\PY{p}{,}\PY{o}{\PYZhy{}}\PY{l+m+mi}{5}\PY{p}{,}\PY{l+m+mi}{5}\PY{p}{,}\PY{l+m+mi}{10}\PY{p}{)}
\end{Verbatim}
\end{tcolorbox}

    \begin{center}
    \adjustimage{max size={0.9\linewidth}{0.9\paperheight}}{python/uni3/output_9_0.png}
    \end{center}
    { \hspace*{\fill} \\}
    
    Ejemplo \(f_n(x)=\sin(nx)\)

    \begin{tcolorbox}[breakable, size=fbox, boxrule=1pt, pad at break*=1mm,colback=cellbackground, colframe=cellborder]
\prompt{In}{incolor}{6}{\hspace{4pt}}
\begin{Verbatim}[commandchars=\\\{\}]
\PY{n}{f}\PY{o}{=}\PY{n}{sin}\PY{p}{(}\PY{n}{n}\PY{o}{*}\PY{n}{x}\PY{p}{)}
\PY{n}{grafica}\PY{p}{(}\PY{n}{f}\PY{p}{,}\PY{l+m+mi}{0}\PY{p}{,}\PY{n}{pi}\PY{p}{,}\PY{l+m+mi}{10}\PY{p}{)}
\end{Verbatim}
\end{tcolorbox}

    \begin{center}
    \adjustimage{max size={0.9\linewidth}{0.9\paperheight}}{python/uni3/output_11_0.png}
    \end{center}
    { \hspace*{\fill} \\}
    
    Series de Potencias

Ejemplo \(S(z)=\sum\limits_{n=1}^{\infty}z^n\)

Serie geométrica converge \(|z|<1\). Ponemos \(z=r e^{i\theta}\). Luego
\(z^j=r^je^{j\theta i}\).

    \begin{tcolorbox}[breakable, size=fbox, boxrule=1pt, pad at break*=1mm,colback=cellbackground, colframe=cellborder]
\prompt{In}{incolor}{7}{\hspace{4pt}}
\begin{Verbatim}[commandchars=\\\{\}]
\PY{n}{n}\PY{p}{,}\PY{n}{theta}\PY{p}{,}\PY{n}{r}\PY{o}{=}\PY{n}{symbols}\PY{p}{(}\PY{l+s+s1}{\PYZsq{}}\PY{l+s+s1}{n,theta,r}\PY{l+s+s1}{\PYZsq{}}\PY{p}{,}\PY{n}{real}\PY{o}{=}\PY{n+nb+bp}{True}\PY{p}{)}

\PY{n}{S}\PY{o}{=}\PY{n+nb}{sum}\PY{p}{(}\PY{p}{[}\PY{n}{r}\PY{o}{*}\PY{o}{*}\PY{n}{j}\PY{o}{*}\PY{n}{exp}\PY{p}{(}\PY{n}{j}\PY{o}{*}\PY{n}{theta}\PY{o}{*}\PY{n}{I}\PY{p}{)} \PY{k}{for} \PY{n}{j} \PY{o+ow}{in} \PY{n+nb}{range}\PY{p}{(}\PY{l+m+mi}{1}\PY{p}{,}\PY{l+m+mi}{5}\PY{p}{)}\PY{p}{]}\PY{p}{)}
\PY{n}{SS}\PY{o}{=}\PY{n}{re}\PY{p}{(}\PY{n}{S}\PY{p}{)}
\end{Verbatim}
\end{tcolorbox}

    \begin{tcolorbox}[breakable, size=fbox, boxrule=1pt, pad at break*=1mm,colback=cellbackground, colframe=cellborder]
\prompt{In}{incolor}{8}{\hspace{4pt}}
\begin{Verbatim}[commandchars=\\\{\}]
\PY{k+kn}{from} \PY{n+nn}{sympy.plotting} \PY{k+kn}{import} \PY{n}{plot3d\PYZus{}parametric\PYZus{}surface}
\end{Verbatim}
\end{tcolorbox}

    \begin{tcolorbox}[breakable, size=fbox, boxrule=1pt, pad at break*=1mm,colback=cellbackground, colframe=cellborder]
\prompt{In}{incolor}{9}{\hspace{4pt}}
\begin{Verbatim}[commandchars=\\\{\}]
\PY{n}{plot3d\PYZus{}parametric\PYZus{}surface}\PY{p}{(}\PY{n}{r}\PY{o}{*}\PY{n}{cos}\PY{p}{(}\PY{n}{theta}\PY{p}{)}\PY{p}{,}\PY{n}{r}\PY{o}{*}\PY{n}{sin}\PY{p}{(}\PY{n}{theta}\PY{p}{)}\PY{p}{,}\PY{n}{SS}\PY{p}{,}\PY{p}{(}\PY{n}{theta}\PY{p}{,}\PY{o}{\PYZhy{}}\PY{n}{pi}\PY{p}{,}\PY{n}{pi}\PY{p}{)}\PY{p}{,}\PY{p}{(}\PY{n}{r}\PY{p}{,}\PY{l+m+mi}{0}\PY{p}{,}\PY{l+m+mi}{1}\PY{p}{)}\PY{p}{)}
\end{Verbatim}
\end{tcolorbox}

    \begin{center}
    \adjustimage{max size={0.9\linewidth}{0.9\paperheight}}{python/uni3/output_15_0.png}
    \end{center}
    { \hspace*{\fill} \\}
    
            \begin{tcolorbox}[breakable, boxrule=.5pt, size=fbox, pad at break*=1mm, opacityfill=0]
\prompt{Out}{outcolor}{9}{\hspace{3.5pt}}
\begin{Verbatim}[commandchars=\\\{\}]
<sympy.plotting.plot.Plot at 0x7f413ec776d0>
\end{Verbatim}
\end{tcolorbox}
        
    Series de Fourier 

    \begin{tcolorbox}[breakable, size=fbox, boxrule=1pt, pad at break*=1mm,colback=cellbackground, colframe=cellborder]
\prompt{In}{incolor}{10}{\hspace{4pt}}
\begin{Verbatim}[commandchars=\\\{\}]
\PY{n}{x}\PY{o}{=}\PY{n}{symbols}\PY{p}{(}\PY{l+s+s1}{\PYZsq{}}\PY{l+s+s1}{x}\PY{l+s+s1}{\PYZsq{}}\PY{p}{,}\PY{n}{real}\PY{o}{=}\PY{n+nb+bp}{True}\PY{p}{)}
\PY{n}{n}\PY{p}{,}\PY{n}{m}\PY{o}{=}\PY{n}{symbols}\PY{p}{(}\PY{l+s+s1}{\PYZsq{}}\PY{l+s+s1}{n,m}\PY{l+s+s1}{\PYZsq{}}\PY{p}{,}\PY{n}{integer}\PY{o}{=}\PY{n+nb+bp}{True}\PY{p}{,}\PY{n}{positive}\PY{o}{=}\PY{n+nb+bp}{True}\PY{p}{)}
\PY{n}{Integral}\PY{p}{(}\PY{n}{cos}\PY{p}{(}\PY{n}{n}\PY{o}{*}\PY{n}{x}\PY{p}{)}\PY{o}{*}\PY{n}{cos}\PY{p}{(}\PY{n}{m}\PY{o}{*}\PY{n}{x}\PY{p}{)}\PY{p}{,}\PY{p}{(}\PY{n}{x}\PY{p}{,}\PY{o}{\PYZhy{}}\PY{n}{pi}\PY{p}{,}\PY{n}{pi}\PY{p}{)}\PY{p}{)}\PY{o}{.}\PY{n}{doit}\PY{p}{(}\PY{p}{)}
\end{Verbatim}
\end{tcolorbox}
 
            
\prompt{Out}{outcolor}{10}{}
    
    $$\begin{cases} 0 & \text{for}\: m \neq n \\\pi & \text{otherwise} \end{cases}$$

    

    \begin{tcolorbox}[breakable, size=fbox, boxrule=1pt, pad at break*=1mm,colback=cellbackground, colframe=cellborder]
\prompt{In}{incolor}{11}{\hspace{4pt}}
\begin{Verbatim}[commandchars=\\\{\}]
\PY{n}{Integral}\PY{p}{(}\PY{n}{sin}\PY{p}{(}\PY{n}{n}\PY{o}{*}\PY{n}{x}\PY{p}{)}\PY{o}{*}\PY{n}{sin}\PY{p}{(}\PY{n}{m}\PY{o}{*}\PY{n}{x}\PY{p}{)}\PY{p}{,}\PY{p}{(}\PY{n}{x}\PY{p}{,}\PY{o}{\PYZhy{}}\PY{n}{pi}\PY{p}{,}\PY{n}{pi}\PY{p}{)}\PY{p}{)}\PY{o}{.}\PY{n}{doit}\PY{p}{(}\PY{p}{)}
\end{Verbatim}
\end{tcolorbox}
 
            
\prompt{Out}{outcolor}{11}{}
    
    $$\begin{cases} 0 & \text{for}\: m \neq n \\\pi & \text{otherwise} \end{cases}$$

    

    \begin{tcolorbox}[breakable, size=fbox, boxrule=1pt, pad at break*=1mm,colback=cellbackground, colframe=cellborder]
\prompt{In}{incolor}{12}{\hspace{4pt}}
\begin{Verbatim}[commandchars=\\\{\}]
\PY{n}{Integral}\PY{p}{(}\PY{n}{cos}\PY{p}{(}\PY{n}{n}\PY{o}{*}\PY{n}{x}\PY{p}{)}\PY{o}{*}\PY{n}{sin}\PY{p}{(}\PY{n}{m}\PY{o}{*}\PY{n}{x}\PY{p}{)}\PY{p}{,}\PY{p}{(}\PY{n}{x}\PY{p}{,}\PY{o}{\PYZhy{}}\PY{n}{pi}\PY{p}{,}\PY{n}{pi}\PY{p}{)}\PY{p}{)}\PY{o}{.}\PY{n}{doit}\PY{p}{(}\PY{p}{)}
\end{Verbatim}
\end{tcolorbox}
 
            
\prompt{Out}{outcolor}{12}{}
    
    $$0$$

    

    \begin{tcolorbox}[breakable, size=fbox, boxrule=1pt, pad at break*=1mm,colback=cellbackground, colframe=cellborder]
\prompt{In}{incolor}{13}{\hspace{4pt}}
\begin{Verbatim}[commandchars=\\\{\}]
\PY{n}{S}\PY{o}{=}\PY{n+nb}{sum}\PY{p}{(}\PY{p}{[}\PY{n}{sin}\PY{p}{(}\PY{n}{n}\PY{o}{*}\PY{n}{theta}\PY{p}{)}\PY{o}{/}\PY{n}{n} \PY{k}{for} \PY{n}{n} \PY{o+ow}{in} \PY{n+nb}{range}\PY{p}{(}\PY{l+m+mi}{1}\PY{p}{,}\PY{l+m+mi}{50}\PY{p}{)}\PY{p}{]}\PY{p}{)}
\PY{n}{plot}\PY{p}{(}\PY{n}{S}\PY{p}{,}\PY{p}{(}\PY{n}{theta}\PY{p}{,}\PY{o}{\PYZhy{}}\PY{l+m+mi}{4}\PY{o}{*}\PY{n}{pi}\PY{p}{,}\PY{l+m+mi}{4}\PY{o}{*}\PY{n}{pi}\PY{p}{)}\PY{p}{)}
\end{Verbatim}
\end{tcolorbox}

    \begin{center}
    \adjustimage{max size={0.9\linewidth}{0.9\paperheight}}{python/uni3/output_20_0.png}
    \end{center}
    { \hspace*{\fill} \\}
    
            \begin{tcolorbox}[breakable, boxrule=.5pt, size=fbox, pad at break*=1mm, opacityfill=0]
\prompt{Out}{outcolor}{13}{\hspace{3.5pt}}
\begin{Verbatim}[commandchars=\\\{\}]
<sympy.plotting.plot.Plot at 0x7f413f16afd0>
\end{Verbatim}
\end{tcolorbox}
        
    Ejemplo
\[f(x)=\left\{\begin{array}{cc} \frac{\pi-x}{2}, & \text{ si } x\in [0,\pi]\\
    -\frac{\pi+x}{2}, & \text{ si } x\in [-\pi,0) \end{array}    \right.\]

    \begin{tcolorbox}[breakable, size=fbox, boxrule=1pt, pad at break*=1mm,colback=cellbackground, colframe=cellborder]
\prompt{In}{incolor}{14}{\hspace{4pt}}
\begin{Verbatim}[commandchars=\\\{\}]
\PY{n}{f}\PY{o}{=}\PY{n}{Piecewise}\PY{p}{(}\PY{p}{(}\PY{p}{(}\PY{n}{pi}\PY{o}{\PYZhy{}}\PY{n}{x}\PY{p}{)}\PY{o}{/}\PY{l+m+mi}{2}\PY{p}{,} \PY{n}{x}\PY{o}{\PYZgt{}}\PY{o}{=}\PY{l+m+mi}{0}  \PY{p}{)}\PY{p}{,}\PY{p}{(}\PY{o}{\PYZhy{}}\PY{p}{(}\PY{n}{pi}\PY{o}{+}\PY{n}{x}\PY{p}{)}\PY{o}{/}\PY{l+m+mi}{2}\PY{p}{,}\PY{n}{x}\PY{o}{\PYZlt{}}\PY{l+m+mi}{0} \PY{p}{)}\PY{p}{)}
\PY{n}{plot}\PY{p}{(}\PY{n}{f}\PY{p}{,}\PY{p}{(}\PY{n}{x}\PY{p}{,}\PY{o}{\PYZhy{}}\PY{l+m+mi}{4}\PY{p}{,}\PY{l+m+mi}{4}\PY{p}{)}\PY{p}{)}
\end{Verbatim}
\end{tcolorbox}

    \begin{center}
    \adjustimage{max size={0.9\linewidth}{0.9\paperheight}}{python/uni3/output_22_0.png}
    \end{center}
    { \hspace*{\fill} \\}
    
            \begin{tcolorbox}[breakable, boxrule=.5pt, size=fbox, pad at break*=1mm, opacityfill=0]
\prompt{Out}{outcolor}{14}{\hspace{3.5pt}}
\begin{Verbatim}[commandchars=\\\{\}]
<sympy.plotting.plot.Plot at 0x7f413f019b50>
\end{Verbatim}
\end{tcolorbox}
        
    \begin{tcolorbox}[breakable, size=fbox, boxrule=1pt, pad at break*=1mm,colback=cellbackground, colframe=cellborder]
\prompt{In}{incolor}{15}{\hspace{4pt}}
\begin{Verbatim}[commandchars=\\\{\}]
\PY{k}{def} \PY{n+nf}{a}\PY{p}{(}\PY{n}{g}\PY{p}{,}\PY{n}{k}\PY{p}{)}\PY{p}{:}
    \PY{k}{return} \PY{l+m+mi}{1}\PY{o}{/}\PY{n}{pi}\PY{o}{*}\PY{n}{Integral}\PY{p}{(}\PY{n}{g}\PY{o}{*}\PY{n}{cos}\PY{p}{(}\PY{n}{k}\PY{o}{*}\PY{n}{x}\PY{p}{)}\PY{p}{,}\PY{p}{(}\PY{n}{x}\PY{p}{,}\PY{o}{\PYZhy{}}\PY{n}{pi}\PY{p}{,}\PY{n}{pi}\PY{p}{)}\PY{p}{)}\PY{o}{.}\PY{n}{doit}\PY{p}{(}\PY{p}{)}
\PY{k}{def} \PY{n+nf}{b}\PY{p}{(}\PY{n}{g}\PY{p}{,}\PY{n}{k}\PY{p}{)}\PY{p}{:}
    \PY{k}{return} \PY{l+m+mi}{1}\PY{o}{/}\PY{n}{pi}\PY{o}{*}\PY{n}{Integral}\PY{p}{(}\PY{n}{g}\PY{o}{*}\PY{n}{sin}\PY{p}{(}\PY{n}{k}\PY{o}{*}\PY{n}{x}\PY{p}{)}\PY{p}{,}\PY{p}{(}\PY{n}{x}\PY{p}{,}\PY{o}{\PYZhy{}}\PY{n}{pi}\PY{p}{,}\PY{n}{pi}\PY{p}{)}\PY{p}{)}\PY{o}{.}\PY{n}{doit}\PY{p}{(}\PY{p}{)}
\end{Verbatim}
\end{tcolorbox}

    \begin{tcolorbox}[breakable, size=fbox, boxrule=1pt, pad at break*=1mm,colback=cellbackground, colframe=cellborder]
\prompt{In}{incolor}{16}{\hspace{4pt}}
\begin{Verbatim}[commandchars=\\\{\}]
\PY{p}{[}\PY{n}{a}\PY{p}{(}\PY{n}{f}\PY{p}{,}\PY{n}{k}\PY{p}{)} \PY{k}{for} \PY{n}{k} \PY{o+ow}{in} \PY{n+nb}{range}\PY{p}{(}\PY{l+m+mi}{1}\PY{p}{,}\PY{l+m+mi}{10}\PY{p}{)}\PY{p}{]}
\end{Verbatim}
\end{tcolorbox}
 
            
\prompt{Out}{outcolor}{16}{}
    
    $$\left [ 0, \quad 0, \quad 0, \quad 0, \quad 0, \quad 0, \quad 0, \quad 0, \quad 0\right ]$$

    

    \begin{tcolorbox}[breakable, size=fbox, boxrule=1pt, pad at break*=1mm,colback=cellbackground, colframe=cellborder]
\prompt{In}{incolor}{17}{\hspace{4pt}}
\begin{Verbatim}[commandchars=\\\{\}]
\PY{p}{[}\PY{n}{b}\PY{p}{(}\PY{n}{f}\PY{p}{,}\PY{n}{k}\PY{p}{)} \PY{k}{for} \PY{n}{k} \PY{o+ow}{in} \PY{n+nb}{range}\PY{p}{(}\PY{l+m+mi}{1}\PY{p}{,}\PY{l+m+mi}{10}\PY{p}{)}\PY{p}{]}
\end{Verbatim}
\end{tcolorbox}
 
            
\prompt{Out}{outcolor}{17}{}
    
    $$\left [ 1, \quad \frac{1}{2}, \quad \frac{1}{3}, \quad \frac{1}{4}, \quad \frac{1}{5}, \quad \frac{1}{6}, \quad \frac{1}{7}, \quad \frac{1}{8}, \quad \frac{1}{9}\right ]$$

    

    \begin{tcolorbox}[breakable, size=fbox, boxrule=1pt, pad at break*=1mm,colback=cellbackground, colframe=cellborder]
\prompt{In}{incolor}{18}{\hspace{4pt}}
\begin{Verbatim}[commandchars=\\\{\}]
\PY{n}{S}\PY{o}{=}\PY{n+nb}{sum}\PY{p}{(}\PY{p}{[}\PY{n}{b}\PY{p}{(}\PY{n}{f}\PY{p}{,}\PY{n}{k}\PY{p}{)}\PY{o}{*}\PY{n}{sin}\PY{p}{(}\PY{n}{k}\PY{o}{*}\PY{n}{x}\PY{p}{)} \PY{k}{for} \PY{n}{k} \PY{o+ow}{in} \PY{n+nb}{range}\PY{p}{(}\PY{l+m+mi}{1}\PY{p}{,}\PY{l+m+mi}{20}\PY{p}{)}\PY{p}{]}\PY{p}{)}
\PY{n}{S}
\end{Verbatim}
\end{tcolorbox}
 
            
\prompt{Out}{outcolor}{18}{}
    
    $$\sin{\left (x \right )} + \frac{\sin{\left (2 x \right )}}{2} + \frac{\sin{\left (3 x \right )}}{3} + \frac{\sin{\left (4 x \right )}}{4} + \frac{\sin{\left (5 x \right )}}{5} + \frac{\sin{\left (6 x \right )}}{6} + \frac{\sin{\left (7 x \right )}}{7} + \frac{\sin{\left (8 x \right )}}{8} + \frac{\sin{\left (9 x \right )}}{9} + \frac{\sin{\left (10 x \right )}}{10} + \frac{\sin{\left (11 x \right )}}{11} + \frac{\sin{\left (12 x \right )}}{12} + \frac{\sin{\left (13 x \right )}}{13} + \frac{\sin{\left (14 x \right )}}{14} + \frac{\sin{\left (15 x \right )}}{15} + \frac{\sin{\left (16 x \right )}}{16} + \frac{\sin{\left (17 x \right )}}{17} + \frac{\sin{\left (18 x \right )}}{18} + \frac{\sin{\left (19 x \right )}}{19}$$

    

    \begin{tcolorbox}[breakable, size=fbox, boxrule=1pt, pad at break*=1mm,colback=cellbackground, colframe=cellborder]
\prompt{In}{incolor}{19}{\hspace{4pt}}
\begin{Verbatim}[commandchars=\\\{\}]
\PY{n}{S}\PY{o}{=}\PY{n+nb}{sum}\PY{p}{(}\PY{p}{[}\PY{n}{b}\PY{p}{(}\PY{n}{f}\PY{p}{,}\PY{n}{k}\PY{p}{)}\PY{o}{*}\PY{n}{sin}\PY{p}{(}\PY{n}{k}\PY{o}{*}\PY{n}{x}\PY{p}{)} \PY{k}{for} \PY{n}{k} \PY{o+ow}{in} \PY{n+nb}{range}\PY{p}{(}\PY{l+m+mi}{1}\PY{p}{,}\PY{l+m+mi}{20}\PY{p}{)}\PY{p}{]}\PY{p}{)}
\PY{n}{plot}\PY{p}{(}\PY{n}{f}\PY{p}{,}\PY{n}{S}\PY{p}{,} \PY{p}{(}\PY{n}{x}\PY{p}{,}\PY{o}{\PYZhy{}}\PY{n}{pi}\PY{p}{,}\PY{n}{pi}\PY{p}{)}\PY{p}{)}
\end{Verbatim}
\end{tcolorbox}

    \begin{center}
    \adjustimage{max size={0.9\linewidth}{0.9\paperheight}}{python/uni3/output_27_0.png}
    \end{center}
    { \hspace*{\fill} \\}
    
            \begin{tcolorbox}[breakable, boxrule=.5pt, size=fbox, pad at break*=1mm, opacityfill=0]
\prompt{Out}{outcolor}{19}{\hspace{3.5pt}}
\begin{Verbatim}[commandchars=\\\{\}]
<sympy.plotting.plot.Plot at 0x7f413f1ae210>
\end{Verbatim}
\end{tcolorbox}
        
    Convergencia uniforme

Ejemplo $ f\_n(x)=\frac{1}{n}e^{-n2x^2}x $

    \begin{tcolorbox}[breakable, size=fbox, boxrule=1pt, pad at break*=1mm,colback=cellbackground, colframe=cellborder]
\prompt{In}{incolor}{20}{\hspace{4pt}}
\begin{Verbatim}[commandchars=\\\{\}]
\PY{n}{x}\PY{p}{,}\PY{n}{n}\PY{o}{=}\PY{n}{symbols}\PY{p}{(}\PY{l+s+s1}{\PYZsq{}}\PY{l+s+s1}{x,n}\PY{l+s+s1}{\PYZsq{}}\PY{p}{)}
\PY{n}{fn}\PY{o}{=}\PY{l+m+mi}{1}\PY{o}{/}\PY{n}{n}\PY{o}{*}\PY{n}{exp}\PY{p}{(}\PY{o}{\PYZhy{}}\PY{n}{n}\PY{o}{*}\PY{o}{*}\PY{l+m+mi}{2}\PY{o}{*}\PY{n}{x}\PY{o}{*}\PY{o}{*}\PY{l+m+mi}{2}\PY{p}{)}\PY{o}{*}\PY{n}{x}
\PY{n}{p}\PY{o}{=}\PY{n}{plot}\PY{p}{(}\PY{n}{fn}\PY{o}{.}\PY{n}{subs}\PY{p}{(}\PY{n}{n}\PY{p}{,}\PY{l+m+mi}{1}\PY{p}{)}\PY{p}{,} \PY{p}{(}\PY{n}{x}\PY{p}{,}\PY{o}{\PYZhy{}}\PY{l+m+mi}{5}\PY{p}{,}\PY{l+m+mi}{5}\PY{p}{)}\PY{p}{,}\PY{n}{show}\PY{o}{=}\PY{n}{false}\PY{p}{)}
\PY{k}{for} \PY{n}{k} \PY{o+ow}{in} \PY{n+nb}{range}\PY{p}{(}\PY{l+m+mi}{2}\PY{p}{,}\PY{l+m+mi}{10}\PY{p}{)}\PY{p}{:}
    \PY{n}{p}\PY{o}{.}\PY{n}{append}\PY{p}{(}\PY{n}{plot}\PY{p}{(}\PY{n}{fn}\PY{o}{.}\PY{n}{subs}\PY{p}{(}\PY{n}{n}\PY{p}{,}\PY{n}{k}\PY{p}{)}\PY{p}{,} \PY{p}{(}\PY{n}{x}\PY{p}{,}\PY{o}{\PYZhy{}}\PY{l+m+mi}{5}\PY{p}{,}\PY{l+m+mi}{5}\PY{p}{)}\PY{p}{,}\PY{n}{show}\PY{o}{=}\PY{n}{false}\PY{p}{)}\PY{p}{[}\PY{l+m+mi}{0}\PY{p}{]}\PY{p}{)}
\PY{n}{p}\PY{o}{.}\PY{n}{show}\PY{p}{(}\PY{p}{)}
\end{Verbatim}
\end{tcolorbox}

    \begin{center}
    \adjustimage{max size={0.9\linewidth}{0.9\paperheight}}{python/uni3/output_29_0.png}
    \end{center}
    { \hspace*{\fill} \\}
    
    Ejemplo \$ f\_n(x)=ne\textsuperscript{\{-n}2x\^{}2\}x \$

    \begin{tcolorbox}[breakable, size=fbox, boxrule=1pt, pad at break*=1mm,colback=cellbackground, colframe=cellborder]
\prompt{In}{incolor}{21}{\hspace{4pt}}
\begin{Verbatim}[commandchars=\\\{\}]
\PY{n}{fn}\PY{o}{=}\PY{n}{n}\PY{o}{*}\PY{n}{exp}\PY{p}{(}\PY{o}{\PYZhy{}}\PY{n}{n}\PY{o}{*}\PY{o}{*}\PY{l+m+mi}{2}\PY{o}{*}\PY{n}{x}\PY{o}{*}\PY{o}{*}\PY{l+m+mi}{2}\PY{p}{)}\PY{o}{*}\PY{n}{x}
\PY{n}{p}\PY{o}{=}\PY{n}{plot}\PY{p}{(}\PY{n}{fn}\PY{o}{.}\PY{n}{subs}\PY{p}{(}\PY{n}{n}\PY{p}{,}\PY{l+m+mi}{1}\PY{p}{)}\PY{p}{,} \PY{p}{(}\PY{n}{x}\PY{p}{,}\PY{o}{\PYZhy{}}\PY{l+m+mi}{5}\PY{p}{,}\PY{l+m+mi}{5}\PY{p}{)}\PY{p}{,}\PY{n}{show}\PY{o}{=}\PY{n}{false}\PY{p}{)}
\PY{k}{for} \PY{n}{k} \PY{o+ow}{in} \PY{n+nb}{range}\PY{p}{(}\PY{l+m+mi}{2}\PY{p}{,}\PY{l+m+mi}{10}\PY{p}{)}\PY{p}{:}
    \PY{n}{p}\PY{o}{.}\PY{n}{append}\PY{p}{(}\PY{n}{plot}\PY{p}{(}\PY{n}{fn}\PY{o}{.}\PY{n}{subs}\PY{p}{(}\PY{n}{n}\PY{p}{,}\PY{n}{k}\PY{p}{)}\PY{p}{,} \PY{p}{(}\PY{n}{x}\PY{p}{,}\PY{o}{\PYZhy{}}\PY{l+m+mi}{5}\PY{p}{,}\PY{l+m+mi}{5}\PY{p}{)}\PY{p}{,}\PY{n}{show}\PY{o}{=}\PY{n}{false}\PY{p}{)}\PY{p}{[}\PY{l+m+mi}{0}\PY{p}{]}\PY{p}{)}
\PY{n}{p}\PY{o}{.}\PY{n}{show}\PY{p}{(}\PY{p}{)}
\end{Verbatim}
\end{tcolorbox}

    \begin{center}
    \adjustimage{max size={0.9\linewidth}{0.9\paperheight}}{python/uni3/output_31_0.png}
    \end{center}
    { \hspace*{\fill} \\}
    

    % Add a bibliography block to the postdoc
    
\chapter{Integral de Riemann}

\section{Introducción}

\begin{quotation}
\marginnote{\adjustimage{max size={0.9\linewidth}{0.9\paperheight}}{imagenes/Riemann.jpeg}\\
Bernhard Riemann 1826-1866
} 
<< Bernard Riemann recibió su doctorado en 1851, su \emph{Habilitación} en 1854. La habilitación confiere el reconocimiento de la capacidad de crear sustanciales contribuciones en la investigación más allá de la tesis doctoral, y es un prerequisito necesario para ocupar un cargo de profesor en una universidad Alemana. Riemann eligió como tema  de habilitación el problema de las series de Fourier. Su tesis fue titulada \emph{\"Uber die Darstellbarkeit einer Function durch eine trigonometrische Reine} (Sobre la representación de una función por series trigonométricas) y respondía la pregunta:  Cuándo una función definida en el intervalo $(-\pi,\pi)$ puede ser respresentada por la serie trigonométrica $a_0/2+\sum_{n=1}^{\infty}[a_n\cos(nx)+b_n\sen(nx)]$? 
En este trabajo  es donde hallamos   la Integral de Riemann, introducida en una sección corta antes del nucleo principal de la tesis, como parte del trabajo preparatorio que él necesitó desarrollar antes de abordar el problema de representabilidad por series trigonométricas. >> 
\end{quotation}
\begin{flushright}
 David M. Bressoud\\
 A Radical Approach to Lebesgue's Theory of Integration.
\end{flushright}


En este capítulo vamos a desarrollar el concepto de la integral de Riemman. Vamos a exponer la definición de la integral debida a Riemann y la ideada por J. G. Darboux.
Mostraemos la equivalencia de las dos definiciones y discutiremos las propiedades de la intergal, sus alcances y límites. Preparamos así el camino para la introducción de la integral de Lebesgue. 
\marginnote{\adjustimage{max size={0.9\linewidth}{0.9\paperheight}}{imagenes/Darboux.jpg}\\
Jean G. Darboux  1842-1917
} 

Debemos advertir  al alumno que en este curso dejaremos un poco de lado las cuestiones procedimentales de cómo calcular integrales, aspecto que seguramente abordó en cursos anteriores y del cual nos vamos a valer. Tampoco debe esperar que las actividades prácticas se centren en esa dirección.   Nuestro principal objetivo aquí es discutir la materia conceptual ligada a la integral y cómo es previsible las actividades prácticas estarán orientadas con ese propósito.


El concepto de integral encuentra su motivación en diversos problemas. Aparece cuando se busca el centro de masas de un determinado cuerpo, cuando se quieren hallar longitudes de arco, volúmenes, cuando se quiere reconstruir el movimiento de cuerpo conocida su velocidad, etc. La integral es utilizada en incontables otros conceptos matemáticos, como ser el mencionado már arriba relativo a las series de Fourier. 

Quizás el 
problema más simple donde aparece la integral es el que utilizaremos como motivación para introducirla y es el concepto de área.  Vamos a tratar de reconstruir este concepto desde su base, esto es analizando la noción de área de figuras tan simples como rectángulos, triángulos, etc. 



\section{Área de figuras elementales planas}\label{sec:area_elem}

  
El cálculo de áreas es necesario en multitud de actividades humanas, por ejemplo con el comercio. La cantidad de muchos productos y servicios se estima en medidas de área, por ejemplo: las telas,  el trabajo de un colocador de pisos,  el precio de la construcción,  el valor de las extensiones de tierra, etc.  
 
 


Por figuras elementales planas nos referimos a rectángulos, triángulos, trapecios, etc. Sin duda el alumno  debe estar  muy familiarizado con las áreas de estas figuras, el área de un rectángulo viene dada por la conocida fórmula $b\times h$, donde $b$ es la base del rectángulo y $h$ su altura.  Ahora bien, ¿Cómo 
se llega a esta fórmula? Porque esta fórmula es apropiada para calcular el precio de un terreno por ejemplo. En esta sección vamos a justificar esta fórmula a partir de algunos hechos elementales.



Vamos a considerar un plano $\mathcal{P}$. En este plano $\mathcal{P}$ supondremos fijada una unidad de longitud.  Pretendemos asignar un área a las figuras, es decir a los subconjuntos, de $\mathcal{P}$. De ahora en más, cómo es usual en esta materia  nos referiremos a \emph{medida}\index{medida} en lugar de área. La medida es un concepto más general  que el concepto de área. No obstante en el contexto en que estamos actualmente son sinónimos.  

Queremos construir pues una función $m$ tal que $m(A)$ reppresente la medida  de  $A\subset\mathcal{P}$. Ahora bien ¿qué podemos usar de guía con ese objetivo? Si, como dijimos,  desconocemos todas las fórmulas previamente aprendidas, sobre que partimos para construir la medida o área. La respuesta es que tomaremos como principio rector  ciertas propiedades que son deseables  que una medida satisfaga. Ellas son las  siguientes. 




\begin{description}
 \item[Positividad.] debería ser una magnitud no negativa.  
 \item[Invariancia por movimientos rígidos.] Si una región es transformada en otra por medio de un movimiento rígido, ambas regiones deberían tener la misma área. Otra manera de expresar esta propiedad es diciendo que dos figuras \emph{congruentes}\index{congruencia} tienen la misma área. 
 \item[Aditividad.] Si una región es la unión de cierta cantidad de regiones más chicas mutuamente disjuntas  
\end{description}

\begin{figure}[h]
\begin{center}
 
\definecolor{xdxdff}{rgb}{0.49,0.49,1}
\definecolor{zzttqq}{rgb}{0.6,0.2,0}
\definecolor{ududff}{rgb}{0.30,0.30,1}
\begin{tikzpicture}[line cap=round,line join=round,x=.9cm,y=.9cm]
\clip(-2.261345665671131,-2.6483801457242535) rectangle (15.749723988011487,4.927708528477943);
\fill[line width=2pt,color=zzttqq,fill=zzttqq,fill opacity=0.10000000149011612] (-1.62,-0.89) -- (-1.62,3.13) -- (1.28,3.15) -- (1.3,-0.89) -- cycle;
\fill[line width=2pt,color=zzttqq,fill=zzttqq,fill opacity=0.10000000149011612] (0.09169246661626662,0.6507662540293884) -- (1.2899992647959808,1.130148511211861) -- (1.28,3.15) -- (-0.3199239037382289,3.1389660420431844) -- cycle;
\fill[line width=2pt,color=zzttqq,fill=zzttqq,fill opacity=0.10000000149011612] (-1.62,1.45) -- (-1.62,3.13) -- (-0.3199239037382289,3.1389660420431844) -- (0.09169246661626662,0.6507662540293884) -- (-0.9454716981132074,0.2358490566037732) -- cycle;
\fill[line width=2pt,color=zzttqq,fill=zzttqq,fill opacity=0.10000000149011612] (-1.62,-0.89) -- (-0.32,-0.89) -- (-1.62,1.45) -- cycle;
\fill[line width=2pt,color=zzttqq,fill=zzttqq,fill opacity=0.10000000149011612] (-0.32,-0.89) -- (-0.9454716981132074,0.2358490566037732) -- (0.09169246661626662,0.6507662540293884) -- (1.2899992647959808,1.130148511211861) -- (1.3,-0.89) -- cycle;
\fill[line width=2pt,color=zzttqq,fill=zzttqq,fill opacity=0.10000000149011612] (8.76,1.77) -- (8.134528301886792,2.8958490566037733) -- (9.171692466616268,3.3107662540293887) -- (10.36999926479598,3.790148511211861) -- (10.38,1.77) -- cycle;
\fill[line width=2pt,color=zzttqq,fill=zzttqq,fill opacity=0.10000000149011612] (3.8264466094067267,-1.199878066911803) -- (2.6385072170133266,-0.011938674518402692) -- (3.551459891609421,0.9136938983359697) -- (5.601939561385962,-0.5546723179904587) -- (5.16194451123264,-1.5814488960049211) -- cycle;
\fill[line width=2pt,color=zzttqq,fill=zzttqq,fill opacity=0.10000000149011612] (5.6888024763961145,1.8814084921516754) -- (6.19715889449669,3.0677137999207305) -- (4.761837661840736,4.4888939366884495) -- (3.638322806619573,3.3497747084781038) -- cycle;
\fill[line width=2pt,color=zzttqq,fill=zzttqq,fill opacity=0.10000000149011612] (7.14,0.65) -- (5.84,0.65) -- (7.14,-1.69) -- cycle;
\draw [line width=2pt,color=zzttqq] (-1.62,-0.89)-- (-1.62,3.13);
\draw [line width=2pt,color=zzttqq] (-1.62,3.13)-- (1.28,3.15);
\draw [line width=2pt,color=zzttqq] (1.28,3.15)-- (1.3,-0.89);
\draw [line width=2pt,color=zzttqq] (1.3,-0.89)-- (-1.62,-0.89);
\draw [line width=2pt] (-1.62,1.45)-- (-0.32,-0.89);
\draw [line width=2pt] (-0.9454716981132074,0.2358490566037732)-- (1.2899992647959808,1.130148511211861);
\draw [line width=2pt] (-0.3199239037382289,3.1389660420431844)-- (0.09169246661626662,0.6507662540293884);
\draw [line width=2pt,color=zzttqq] (0.09169246661626662,0.6507662540293884)-- (1.2899992647959808,1.130148511211861);
\draw [line width=2pt,color=zzttqq] (1.2899992647959808,1.130148511211861)-- (1.28,3.15);
\draw [line width=2pt,color=zzttqq] (1.28,3.15)-- (-0.3199239037382289,3.1389660420431844);
\draw [line width=2pt,color=zzttqq] (-0.3199239037382289,3.1389660420431844)-- (0.09169246661626662,0.6507662540293884);
\draw [line width=2pt,color=zzttqq] (-1.62,1.45)-- (-1.62,3.13);
\draw [line width=2pt,color=zzttqq] (-1.62,3.13)-- (-0.3199239037382289,3.1389660420431844);
\draw [line width=2pt,color=zzttqq] (-0.3199239037382289,3.1389660420431844)-- (0.09169246661626662,0.6507662540293884);
\draw [line width=2pt,color=zzttqq] (0.09169246661626662,0.6507662540293884)-- (-0.9454716981132074,0.2358490566037732);
\draw [line width=2pt,color=zzttqq] (-0.9454716981132074,0.2358490566037732)-- (-1.62,1.45);
\draw [line width=2pt,color=zzttqq] (-1.62,-0.89)-- (-0.32,-0.89);
\draw [line width=2pt,color=zzttqq] (-0.32,-0.89)-- (-1.62,1.45);
\draw [line width=2pt,color=zzttqq] (-1.62,1.45)-- (-1.62,-0.89);
\draw [line width=2pt,color=zzttqq] (-0.32,-0.89)-- (-0.9454716981132074,0.2358490566037732);
\draw [line width=2pt,color=zzttqq] (-0.9454716981132074,0.2358490566037732)-- (0.09169246661626662,0.6507662540293884);
\draw [line width=2pt,color=zzttqq] (0.09169246661626662,0.6507662540293884)-- (1.2899992647959808,1.130148511211861);
\draw [line width=2pt,color=zzttqq] (1.2899992647959808,1.130148511211861)-- (1.3,-0.89);
\draw [line width=2pt,color=zzttqq] (1.3,-0.89)-- (-0.32,-0.89);
\draw [line width=2pt,color=zzttqq] (8.76,1.77)-- (8.134528301886792,2.8958490566037733);
\draw [line width=2pt,color=zzttqq] (8.134528301886792,2.8958490566037733)-- (9.171692466616268,3.3107662540293887);
\draw [line width=2pt,color=zzttqq] (9.171692466616268,3.3107662540293887)-- (10.36999926479598,3.790148511211861);
\draw [line width=2pt,color=zzttqq] (10.36999926479598,3.790148511211861)-- (10.38,1.77);
\draw [line width=2pt,color=zzttqq] (10.38,1.77)-- (8.76,1.77);
\draw [line width=2pt,color=zzttqq] (3.8264466094067267,-1.199878066911803)-- (2.6385072170133266,-0.011938674518402692);
\draw [line width=2pt,color=zzttqq] (2.6385072170133266,-0.011938674518402692)-- (3.551459891609421,0.9136938983359697);
\draw [line width=2pt,color=zzttqq] (3.551459891609421,0.9136938983359697)-- (5.601939561385962,-0.5546723179904587);
\draw [line width=2pt,color=zzttqq] (5.601939561385962,-0.5546723179904587)-- (5.16194451123264,-1.5814488960049211);
\draw [line width=2pt,color=zzttqq] (5.16194451123264,-1.5814488960049211)-- (3.8264466094067267,-1.199878066911803);
\draw [line width=2pt,color=zzttqq] (5.6888024763961145,1.8814084921516754)-- (6.19715889449669,3.0677137999207305);
\draw [line width=2pt,color=zzttqq] (6.19715889449669,3.0677137999207305)-- (4.761837661840736,4.4888939366884495);
\draw [line width=2pt,color=zzttqq] (4.761837661840736,4.4888939366884495)-- (3.638322806619573,3.3497747084781038);
\draw [line width=2pt,color=zzttqq] (3.638322806619573,3.3497747084781038)-- (5.6888024763961145,1.8814084921516754);
\draw [line width=2pt,color=zzttqq] (7.14,0.65)-- (5.84,0.65);
\draw [line width=2pt,color=zzttqq] (5.84,0.65)-- (7.14,-1.69);
\draw [line width=2pt,color=zzttqq] (7.14,-1.69)-- (7.14,0.65);
\draw (-1.1356538123159674,2.2366014415507594) node[anchor=north west] {$A_1$};
\draw (3.9651373981996176,0.03798454046645946) node[anchor=north west] {$A_1$};
\draw (0.23628313396063827,2.5883801457242477) node[anchor=north west] {$A_2$};
\draw (4.703872676963944,3.83719454554013) node[anchor=north west] {$A_2$};
\draw (0.09557165229124281,0.5304747263093427) node[anchor=north west] {$A_3$};
\draw (9.101106479132552,3.1160482019844795) node[anchor=north west] {$A_3$};
\draw (-1.5753771925328282,0.31940750380524985) node[anchor=north west] {$A_4$};
\draw (6.445177262622713,0.5480636615180171) node[anchor=north west] {$A_4$};
\begin{scriptsize}
\draw [fill=ududff] (-1.62,-0.89) circle (2.5pt);
\draw [fill=ududff] (-1.62,3.13) circle (2.5pt);
\draw [fill=ududff] (1.28,3.15) circle (2.5pt);
\draw [fill=ududff] (1.3,-0.89) circle (2.5pt);
\draw [fill=xdxdff] (-1.62,1.45) circle (2.5pt);
\draw [fill=xdxdff] (-0.32,-0.89) circle (2.5pt);
\draw [fill=xdxdff] (-0.9454716981132074,0.2358490566037732) circle (2.5pt);
\draw [fill=xdxdff] (1.2899992647959808,1.130148511211861) circle (2.5pt);
\draw [fill=xdxdff] (-0.3199239037382289,3.1389660420431844) circle (2.5pt);
\draw [fill=xdxdff] (0.09169246661626662,0.6507662540293884) circle (2.5pt);
\draw [fill=xdxdff] (8.76,1.77) circle (2.5pt);
\draw [fill=xdxdff] (8.134528301886792,2.8958490566037733) circle (2.5pt);
\draw [fill=xdxdff] (10.36999926479598,3.790148511211861) circle (2.5pt);
\draw [fill=ududff] (10.38,1.77) circle (2.5pt);
\draw [fill=xdxdff] (3.8264466094067267,-1.199878066911803) circle (2.5pt);
\draw [fill=ududff] (2.6385072170133266,-0.011938674518402692) circle (2.5pt);
\draw [fill=xdxdff] (3.551459891609421,0.9136938983359697) circle (2.5pt);
\draw [fill=xdxdff] (5.601939561385962,-0.5546723179904587) circle (2.5pt);
\draw [fill=xdxdff] (5.16194451123264,-1.5814488960049211) circle (2.5pt);
\draw [fill=xdxdff] (5.6888024763961145,1.8814084921516754) circle (2.5pt);
\draw [fill=xdxdff] (6.19715889449669,3.0677137999207305) circle (2.5pt);
\draw [fill=ududff] (4.761837661840736,4.4888939366884495) circle (2.5pt);
\draw [fill=xdxdff] (3.638322806619573,3.3497747084781038) circle (2.5pt);
\draw [fill=ududff] (7.14,0.65) circle (2.5pt);
\draw [fill=xdxdff] (5.84,0.65) circle (2.5pt);
\draw [fill=xdxdff] (7.14,-1.69) circle (2.5pt);
\end{scriptsize}
\end{tikzpicture}


 \caption{El área del rectángulo es la suma de sus partes}\label{fig:rect_descop} 
\end{center}
\end{figure}

Utilizando la segunda y tercer propiedad se pueden relacionar el área del rectángulo de la figura \ref{fig:rect_descop} con las cuatro regiones en la que es dividido.

Como veremos a lo largo de la materia la propiedad de aditividad debe ser estudiada con cuidado, esto ocurre por las intrincadas maneras en que una región puede ser unión de otras regiones. A lo largo de esta materia elaboraremos una  teoría que nos dará una descripción  precisa de a que conjuntos podemos asignarle una medida de modo que las propiedades previas sean ciertas. 

Por el momento veamos como las propiedades anteriores determinan practicamente de manera unívoca la medida de regiones elementales planas.  


Hablando de propiedades de la medida, supongamos que $A$ y $B$ son dos regiones con $A\subset B$. Entonces como $B=A\cup (B-A)$ y por la propiedad de aditividad y positividad

\[
 m(B)=m(A)+m(B-A)\geq m(A).
\]

Descubrimos así que nuestra medida deberá tener adicionalmente la siguiente propiedad:
\begin{description}
 \item[Monotonía.] Si $A\subset B$ entonces $m(A)\leq m(B)$. 
\end{description}
\marginpar{ Podríamos por ejemplo elegir el círculo de radio uno como unidad de área. Así ya no tendríamos el problema de ese número raro $\pi$ que aparece en la fórmula del área del círculo. ¡El área de cualquier círculo sería igual a su radio al cuadrado! Claro que aparecería $\pi$  en la fórmula del área del cuadrado de lado 1. Nos tapamos los pies y se destapa el cuerpo.}
Es claro que si logramos construir una medida que satisfaga las propiedades anteriores cualquier multiplo por un número real positivo  de ella seguirá cumpliendo las propiedades. Esto es una manera de expresar el hecho que podemos usar diferentes unidades de medición. Esta cuestión se sortea proponiendo la unidad de medida. Esta unidad es completamente arbitraria, ud. podría elegir su figura plana preferida como unidad de área.   Cómo es habitual, elijamos el cuadrado cuyos lados miden la unidad de longitud previamente fijada. 


Supongamos ahora que tenemos un rectángulo de un lado igual a la unidad y el otro de lado un racional $n/m$, $n,m\in\mathbb{N}$. Veamos que la aditividad, la invariancia por movimientos rígidos y el hecho que decidimos que el cuadrado de lados igual a la unidad determinan el área de este rectángulo. Primero observar que si dividimos el lado de cuadrado unidad en $m$ segmentos iguales de longitud. \marginnote{
 
\definecolor{xdxdff}{rgb}{0.49,0.49,1}
\definecolor{zzttqq}{rgb}{0.6,0.2,0}
\definecolor{ududff}{rgb}{0.30,0.30,1}
\begin{tikzpicture}[x=2.1cm,y=2.1cm]
\clip(-0.07,0) rectangle (1.5,1.5);
\draw [line width=2pt,color=zzttqq] (0,0) -- (1,0) -- (1,1) -- (0,1) -- cycle;
\draw [line width=1pt,color=zzttqq, dashed] (0,.2)--(1,.2);
\draw [line width=1pt,color=zzttqq, dashed] (0,.4)--(1,.4);
\draw [line width=1pt,color=zzttqq, dashed] (0,.6)--(1,.6);
\draw [line width=1pt,color=zzttqq, dashed] (0,.8)--(1,.8);
\draw (0,1) node[anchor=south] {$Q$};
\draw (0.5,.1) node[anchor=center] {$R_1$};
\draw (0.5,.3) node[anchor=center] {$R_2$};
\draw (0.5,.5) node[anchor=center] {$R_3$};
\draw (0.5,.7) node[anchor=center] {$R_4$};
\draw (0.5,.9) node[anchor=center] {$R_5$};



\end{tikzpicture}
\\
 Descomposición rectángulo $R$
 }
Queda dividido el cuadrado en $m$ rectángulos $R_1,\ldots,R_m$ (ver figura en el margen), todos ellos  congruentes entre si, de modo que todos tienen la misma medida, digamos $m(R_1)$. La unión de ellos es el cuadrado que por convención dijimos que tiene medida 1. De modo que por la aditividad debe ocurrir que $m(R_1)=\cdots =m(R_m))=1/m$. Recordemos nuestra pretención de inferir la medida de un rectángulo $R$ de lado 1 y otro $n/m$. Este rectángulo esta compuesto de $n$ rectángulos congruentes a los $R_i$, $i=1,\ldots,m$, nuevamente por la aditividad inferimos que $m(R)=n/m$. 

Sea ahora una rectángulo $R$ con un lado unidad y el otro un real cualquiera $l>0$. Existen sendas sucesiones $0<q_k,p_k\in\mathbb{Q}$, $k\in\mathbb{N}$, tales que $q_1\leq q_2\leq\cdots \leq l \leq \cdots\leq p_2\leq p_1$ y $\lim_{k\to\infty}q_k =\lim_{k\to\infty} p_k=l$. Consideremos una dos sucesiones de rectángulos $R_k$ y $S_k$ que comparten el lado de $R$ igual a la unidad, mientras que el otro lado de $R_k$ y $S_k$ es igual a $q_k$ y $p_k$ respectivamente. Luego por la monotonía
\[
 q_k=m(R_k)\leq m(R) \leq m(S_k)\leq p_k.
\]
Tomando límite cuando $k\to\infty$ inferimos que $m(R)=l$. 



\begin{figure}[h]
 \begin{center}
 \input{imagenes/ParalTria.tikz} 
 \end{center}
 \caption{Áreas de otras figuras elementales.}\label{fig:paral-trig}
\end{figure}


A partir de las propiedades fundamentales que postulamos para la medida o área inferimos la famosa fórmula del área de un rectángulo en el caso que uno de los lados sea igual a la unidad. Para un  rectángulo arbitrario. En la figura \ref{fig:paral-trig} se muestra como relacionar el área de un paralelepípedo con la de un rectángulo y la de un triángulo con la de un paralelepípedo para inferir las conocidas fórmulas para estas figuras.



\section{Integral de Riemann}

En esta sección abordaremos el problema del área de regiones planas. Vamos a contextualizarnos dentro del marco conceptual que nos brinda la geometría analítica. Mediante coordenadas cartesianas ortogonales los puntos del plano se identifican con pares ordenados $(x,y)\in\mathbb{R}^2$ y el plano con el conjunto $\rr^2$.  Nuestro propósito es entonces definir la medida de subconjuntos de $\mathbb{R}^2$. La geometría analítica abre así nuevas posibilidades para abordar el problema del área. 

Nuestra primera aproximación será la que propuso Bernhard Riemann en 1854, pero seguiremos  el enfoque de Jean Darboux. En esta parte de nuestra exposición consideraremos subconjuntos de $\mathbb{R}^2$ de un tipo especial, concretamente a conjuntos que quedan encerrados entre la gráfica de una función y del eje coordenadas $x$. Esto nos lleva alconcepto de integral. 


\begin{definicion}[Partición]{} Sea $[a,b]$ un intervalo. Una {\em partición}\index{Partición} $P$ es un conjunto ordenado y finito de puntos, donde el primer elemento es $a$ y el último $b$. Es decir $P=\{x_0,x_1,\ldots,x_n\}$, donde $a=x_0<x_1<\cdots<x_n=b$. 
 
\end{definicion}




\begin{definicion}[Sumas de Darboux]{} Sea $f:[a,b]\to\mathbb{R}$ una función acotada y $P=\{x_0,x_1,\ldots,x_n\}$ una partición de $[a,b]$. Consideremos las siguientes magnitudes
\[
 \begin{split}
    m_i&:=\inf\{f(x)| x\in [x_{i-1},x_i]\}\\
    M_i&:=\sup\{f(x)| x\in [x_{i-1},x_i]\}\\
 \end{split}
\]

Definimos la \emph{Suma superior de Darboux}\index{Suma superior} como
\[
 \overline{S}(P,f)=\sum_{i=1}^nM_i(x_i-x_{i-1}),
\]
y la \emph{Suma inferior de Darboux}\index{Suma inferiorr} como
\[
 \underline{S}(P,f)=\sum_{i=1}^nm_i(x_i-x_{i-1}),
\] 
\end{definicion}
\marginpar{
  \begin{center}
    \begin{tikzpicture}[scale=0.45]
\begin{axis}[
    xtick={0,...,5},ytick={5,10,15,20,25},
    y=0.3cm, xmax=5.4,ymax=26.9,ymin=0,xmin=0,
    enlargelimits=true,
    axis lines=middle,
    clip=false,yticklabels=\empty,xticklabels=\empty,
    ]
\addplot+[color=red,fill=red!10!white,const plot, mark=none]
    coordinates {(0,2) (1,5) (2,10) (3,17) (4,26) (5,26)}\closedcycle;
\addplot+[color=green,fill=green!10!white,const plot, mark=none]
    coordinates {(0,1) (1,2) (2,5) (3,10) (4.0,17) (5,17)}\closedcycle;
\addplot[smooth, thick,domain=0:5]{1+x^2};
\addplot[const plot,domain=0:5,color=red] coordinates {(1,0) (1,2)};
\addplot[const plot,domain=0:5,color=red] coordinates {(2,0) (2,5)};
\addplot[const plot,domain=0:5,color=red] coordinates {(3,0) (3,10)};
\addplot[const plot,domain=0:5,color=red] coordinates {(4,0) (4,17)};
\addplot[const plot,domain=0:5,color=red] coordinates {(5,0) (5,26)};
\addplot[color=black] coordinates {(0,-0.8)} node {$a$};
\addplot[color=black] coordinates {(5,-0.8)} node {$b$};

\end{axis}
\end{tikzpicture}
    Sumas de Darboux. 
  \end{center}
}

\begin{lema}[Monotonía sumas de Darboux]{}  Sea $f:[a,b]\to\mathbb{R}$ una función acotada y $P=\{x_0,x_1,\ldots,x_n\}$ una partición de $[a,b]$. Supongamos que $P'$ es otra partición que tiene un pnto más que $P$. Entoces
 \[
  \underline{S}(P',f)\geq \underline{S}(P,f)\quad\text{y}\quad \overline{S}(P',f)\leq \overline{S}(P,f)
 \]
\end{lema}

\begin{ejercicio}{} Sea $f:[a,b]\to\mathbb{R}$ una función acotada y $P,P'$ particiones de $[a,b]$ con $P\subset P'$. Demostrar que 
  \[
  \underline{S}(P,f)\leq \underline{S}(P',f)\quad\text{y}\quad \overline{S}(P',f)\leq \overline{S}(P,f).
 \]
 Inferir que para cualesquiera $P,P'$ (sin importar que una este o no contenida dentro de la otra)
   \[
\underline{S}(P,f)\leq \overline{S}(P,f).
 \]
 
\end{ejercicio}

\begin{definicion}[Funciones integrables]{} Sea $f:[a,b]\to\mathbb{R}$ una función acotada. Diremos que $f$ es {\em integrable Riemann} \index{Integrable Riemann} si 
\begin{equation}\label{eq:integrable}
 \sup\left\{\underline{S}(P,f)| P \text{partición de }[a,b]\right\}=\inf\left\{\overline{S}(P,f)| P \text{partición de }[a,b]\right\}
\end{equation}
En caso que $f$ sea integrable llamamos {\em integral} \index{Integral} entre $a$ y $b$ de $f$ al valor de los dos miembros de \eqref{eq:integrable} y este número se denota
\[
\int_a^bf(x)dx. 
\]
 
\end{definicion}
 
\begin{teorema}[Primer criterio de integrabilidad]{}  Sea $f:[a,b]\to\mathbb{R}$ una función acotada. Entonces  $f$ sea integrable si para todo $\epsilon>0$ existe una partición $P$ tal que 
\begin{equation}\label{eq:Crit1Int}
 \overline{S}(P;f)-\underline{S}(P;f)<\epsilon.
\end{equation}
 
\end{teorema}

\begin{demo} La demo
 
\end{demo}


\begin{ejemplo}{} Sea $0\leq a<b$ veamos que 
\[
 \int_a^b x dx=\frac{b^2}{2}-\frac{a^2}{2}.
\]
\end{ejemplo}


\begin{ejercicio}{} Sea $0\leq a<b$ veamos que 
\[
 \int_a^b x^2 dx=\frac{b^3}{3}-\frac{a^3}{3}.
\]
{\em Ayuda:} Usar particiones uniformes y la fórmula $\sum_{i=1}^nn^2= n(n+1)(2n+1)/6$.
\end{ejercicio}

\begin{ejemplo}{} Sea $0\leq a<b$ y $n\in\mathbb{N}$,  veamos que 
\[
 \int_a^b x^n dx=\frac{b^{n+1}}{n+1}-\frac{a^{n+1}}{n+1}.
\]
Usamos particiones no uniformes
\end{ejemplo}

\begin{ejercicio}{} Sea $0\leq a<b$ y $n$ un entero negativo, veamos que
\[
 \int_a^b x^n dx=\begin{cases}
                  \frac{b^{n+1}}{n+1}-\frac{a^{n+1}}{n+1} & \text{ si } n\neq -1\\
                  \ln(b)-\ln(a) & \text{ si } n=-1\\
                 \end{cases}
\]
\end{ejercicio}


\begin{ejemplo}{} Sea $0\leq a<b\leq \pi/2$, veamos que
\[
 \int_a^b \sen x dx=-(\cos(b)-\cos(a)).
\]
\end{ejemplo}



\begin{ejercicio}{} Sea $0\leq a<b\leq \pi$, veamos que
\[
 \int_a^b \sen x dx=-(\cos(b)-\cos(a)).
\]
\end{ejercicio}


\begin{ejemplo}{} Usamos SymPy y sumas de Darboux aproximar el valor de $\pi$. Utilizamos el hecho que $\pi/4$ es el área de un cuarto de círculo de radio $1$. Entonces 
$$\pi=4\int_0^1\sqrt{1-x^2}dx.$$

\begin{sympyblock}
from sympy import *
N=1000.0
lim=int(N+1)
x=symbols('x')
f=sqrt(1-x**2)
Sinf=sum([ f.subs(x,i/N)*1/N for i in range(1,lim)])
Ssup=sum([f.subs(x,(i-1)/N)*1/N for i in range(1,lim)])
\end{sympyblock}
Encontramos la estimación
\[\sympy{4*Sinf}\leq \pi \leq \sympy{4*Ssup}\]
\end{ejemplo}

\begin{ejercicio}{} Usando SymPy estimar las siguientes integrales 
$$\int_1^2\frac{1}{x}dx,$$
 comparar con $\ln(2)$,

$$\int_{-1}^{1}x^2dx$$
¿A qué parece aproximarse las sumas inferiores y superiores?

$$\int_0^{\frac{1}{2}}\frac{1}{\sqrt{1-x^2}}dx,$$
¿Por qué el resultado puede usarse para aproximar $\pi$ ?

 
\end{ejercicio}




\begin{teorema}[Propiedades elementales de la integral]{} Sean $f,g:[a,b]\to\mathbb{R}$ integrables, $\alpha,\beta\in\mathbb{R}$ y $c\in (a,b)$. Entonces
\begin{description}
 \item[Linealidad] $\alpha f+\beta g$ es integrable y 
 \[
  \int_a^b\alpha f(x)+\beta g(x)dx=\alpha \int_a^bf(x)dx+\beta\int_a^b g(x)dx.
 \]
 \item[Monotonía] Si $f(x)\leq g(x)$  para $x\in [a,b]$ entonces 
 \[
  \int_a^b f(x)dx\leq \int_a^b g(x)dx.
 \]
 \item[Aditividad del Intervalo]  \[
  \int_a^b\alpha f(x)dx= \int_a^cf(x)dx+\int_c^b f(x)dx.
 \]
\end{description}


\end{teorema}

\begin{observa} Las propiedadades anteriores son compatibles con las propiedades que habíamos propuesto para el concepto de área en la sección  \ref{sec:area_elem}.
\end{observa}

\section{Integrabilidad y continuidad}


 
\begin{teorema}[Segundo criterio de integrabilidad]{}  Sea $f:[a,b]\to\mathbb{R}$ una función acotada. Entonces  $f$ sea integrable si para todo $\epsilon>0$ existe un $\delta>0$ tal que para cualquier partición $P$ que satisface
\[\max_i\{x_i-x_{i-1}\}<\delta,\]
se tiene que
\begin{equation}\label{eq:Crit1Int}
 \overline{S}(P;f)-\underline{S}(P;f)<\epsilon.
\end{equation}
 
\end{teorema}
\begin{demo} Agarrate catalina
 
\end{demo}

\begin{teorema}[Continuidad implica integrabilidad]{}  Si $f:[a,b]\to\mathbb{R}$ es una función continua entonces es integrable.
\end{teorema}
\begin{demo} hacer
 \end{demo}

 
¿Qué ocurre con las funciones discontinuas? 

\begin{ejemplo}[Función de Heavside]{} Es la función
\[
 H(x)=\begin{cases}0 & \text{ si } x<0\\1 & \text{ si } x\geq 0\end{cases}.
\]
Es discontinua en $[-1,1]$ pero integrable.
 
\end{ejemplo}



\marginnote{\adjustimage{max size={0.9\linewidth}{0.9\paperheight}}{imagenes/dirichlet.png}\\
Función de Dirichlet}
\begin{ejemplo}[Función de Dirichlet]{} Es la función $f:[0,1]\to\mathbb{R}$ definida por 
\[
 f(x)=\begin{cases} 1 & \text{ si } x\in\mathbb{Q}\\0 & \text{ si }   x\notin\mathbb{Q}\\
\end{cases}
\]
Veamos que $f$ es discontinua en todo punto y no integrable.
\end{ejemplo}





\begin{ejemplo}[Función de Thomae]{} Es la función $f:[0,1]\to\mathbb{R}$ definida por 
\[
 f(x)=\begin{cases} \frac{1}{q} & \text{ si } x=\frac{p}{q},p,q\in\mathbb{Z}, \text{m.c.d}(p,q)=1
 \\0 & \text{ si }   x\notin\mathbb{Q}\\
\end{cases}
\]
\end{ejemplo}
\marginnote{\adjustimage{max size={0.9\linewidth}{0.9\paperheight}}{imagenes/thomae.png}\\
Función de Thomae}

Para graficarla

\begin{sympyverbatim}
from matplotlib import pyplot as plt
total=500
q=[]
f=[]
for i in range(1,total):
    for j in range(1,i):
        if gcd(j,i)==1:
            q.append(Rational(j,i))
            f.append(1.0/i)
plt.plot(q,f,'.',markersize=12)
\end{sympyverbatim}

Veamos que es discontinua en todo punto racional y es integrable integrable.

\marginnote{\adjustimage{max size={0.9\linewidth}{0.9\paperheight}}{imagenes/escalera.png}\\
Función creciente y discontinua en $\mathbb{Q}\cap [0,1]$}
\begin{ejemplo}[Escalera discontinua]{} Sea $\mathbb{Q}\cap [0,1]=\{q_1,q_2,\ldots\}$ una numeración de los racionales del $[0,1]$. Definamos $f:[0,1]\to\mathbb{R}$ como
 \[
  f(x)=\sum_{n=1}^{\infty}H(x-q_n),
 \]
donde $H$ es la función de \emph{Heavside}\index{Heavside}. 
\begin{sympyverbatim}
q=[]
f=[]
total=20
for i in range(1,total):
    for j in range(1,i):
        if gcd(j,i)==1:
            q.append(float(Rational(j,i)))
x=symbols('x')
Heavside=Piecewise((0,x<0),(1,x>=0))
f=sum([Heavside.subs(x,x-q[n])/2**n for n in range(len(q))])
plot(f,(x,0,1))
\end{sympyverbatim}

Veamos que $f$ es monotona no decreciente y discontinua en todo punto de $[0,1]\cap \mathbb{Q}$. Además $f$ es integrable.


\end{ejemplo}
\begin{definicion}[Oscilación sobre un intervalo]{} Sea $f:[a,b]\to\mathbb{R}$ acotada y $I=[\alpha,\beta]\subset [a,b]$. Definimos la \emph{oscilación}\index{oscilación} de $f$ en $I$ por 
\[
 w(f,I)=\sup\{f(x)| x\in I\}-\inf\{f(x)| x\in I\}.
\]
\end{definicion}

\begin{ejemplo}{} 
\begin{enumerate}
 \item Para la función de Dirichlet $w(f,I)=1$ para todo $I$ con interior no vacío.
 \item Para la función de Heavside e $I=[\alpha,\beta]$
 \[
  w(f,I)=\begin{cases}
          1 & \text{ si } 0\in (\alpha,\beta]\\
          0 & \text{ si } 0\notin (\alpha,\beta]\\
         \end{cases}
 \]
  \item Si $I^o\neq\emptyset$, $f$ la función de Thomae e $I\subset [0,1]$ entonces 
     $ w(f,I)=1/q^*$, donde $q^*$ es el mínimo valor de $q$ para el que existe $p\leq q$ tal que $p/q\in I$.
  \item Para la función escalera discontinua e $I\subset [0,1]$
  \[
   w(f,I)=\sum_{q_n\in I}\frac{1}{2^n}.
  \]
\end{enumerate}

 
\end{ejemplo}



\begin{definicion}{} Sea $f:[a,b]\to\mathbb{R}$ acotada, $\sigma>0$ y $P=\{x_0,x_1,\ldots,x_n\}$ una partición. Definimos
\[
 I_{\sigma}:=\{i\in\{1,\ldots,n\}|w(f,[x_{i-1},x_i])>\sigma\}.
\]
y
\[
 R(P,f,\sigma)=\sum_{i\in I_{\sigma}}(x_i-x_{i-1}).
\]
\end{definicion}





\begin{proposicion} Si $f$ es continua en $[a,b]$ para todo $\sigma>0$ existe $\delta>0$ tal que 
\[
\max_i(x_i-x_{i-1})<\delta\Rightarrow I_{\sigma}=\emptyset\Rightarrow R(P,f,\sigma)=0. 
\]

 
\end{proposicion}

\begin{ejemplo}{} Para la función de Dirichlet y para todo $0<\sigma<1$ y para toda partición de $[0,1]$ tenemos $I_{\sigma}=\{x_1,\ldots,x_n\}$ y $R(P,f,\sigma)=[0,1]$
 
\end{ejemplo}

\begin{ejemplo}{} Para la función de Heavside,   para todo $0<\sigma<1$ y para toda partición de $[0,1]$ tenemos $I_{\sigma}=i$, donde $i$ es el índice para el que $i\in (x_{i-1},x_i]$ y $R(P,F,\sigma)=x_i-x_{i-1}$.
 
\end{ejemplo}

\begin{teorema}[Criterio de integrabilidad de Riemann]{} Sea $f$ acotada en $[a,b]$ entonces $f$ es integrable si y sólo si  para todo $\epsilon>0$ y $\sigma>0$ existe $\delta>0$ talque $R(P,f,\sigma)<\epsilon$.
\end{teorema}

\begin{ejemplo}{} Discutir los ejemplos Dirichlet, Heavside, Continuas, escalera discontinua
 
\end{ejemplo}

\begin{ejemplo}[Función de Riemann]{} Definimos
\[
 ((x))=x-[x+0.5]
\]

\marginnote{\adjustimage{max size={0.9\linewidth}{0.9\paperheight}}{imagenes/serrucho.pdf}\\
Función serrucho}
\begin{sympyverbatim}
x=symbols('x')
g=x-floor(x+.5)
plot(g,(x,-5,5))
\end{sympyverbatim}
Definimos la función de Riemann Porque
\[
 f(x)=\sum_{n=1}^{\infty}\frac{((x))}{n^2}.
\]

\begin{sympyverbatim}
f=sum([g.subs(x,n*x)/n**2 for n in range(1,20)])
plot(f,(x,0,1))
\end{sympyverbatim}
\marginnote{\adjustimage{max size={0.9\linewidth}{0.9\paperheight}}{imagenes/funcRiemann.png}\\
Función de Riemann}
 
Demostramos que la función de Riemann es discontinua en los racionales $p/q$ donde $\text{m.c.d}(p,q)=1$ y $q$ par. Es integrable en $[0,1]$.  
 
\end{ejemplo}


\begin{definicion}[Oscilación de una función en un punto]{}\index{oscilación} 
Sea $f:[a,b]\to\mathbb{R}$ y $x\in [a,b]$ 
acotada definimos la {\em oscilación de $f$ en $x$} como 
\[
 w(f;x)=\inf\limits_{x\in I^o}w(f,I),
\]
donde el ínfimo se toma sobre todos los intervalos que contienen a $x$ en su interior.
\end{definicion}

\begin{ejercicio}{} $f$ es continua en $x$ si y solo si $w(f;x)=0$.
 \end{ejercicio}

 \begin{definicion}[Contenido exterior]{}\index{Contenido exterior} Sea $S\subset\mathbb{R}$. Un \emph{cubrimiento finito}\index{cubrimiento finito} de $S$ es una colección de intervalos $\left\{ [x_{i-1},x_i]\right\}_{i=1,\ldots,n}$ tal que $S\subset \cup_{i=1}^n[x_{i-1},x_i]$.
 
 El \emph{contenido exterior} de $S$ se define por 
 \[
  c_e(S)=\inf \sum_{i=1}^n (x_i-x_{i-1}),
 \]
donde el ínfimo es tomado sobre todos los cubrimientos finitos de $S$.
  
 \end{definicion}
 
 \begin{teorema}[Criterio de integrabilidad de Hankel]{}  Sea $f$ acotada en $[a,b]$ entonces $f$ es integrable si y sólo si para todo $\sigma>0$ el conjunto $S_{\sigma}:=\{x\in [a,b]| w(f,x)>\sigma\}$ tiene contenido exterior igual a $0$ ($c_e(S_{\sigma})=0)$.
   \end{teorema}


\section{Teorema Fundamental de Cálculo}


\section{Función de Volterra}

\begin{pyblock}
import numpy as np
import scipy.optimize
from matplotlib import pyplot as plt
\end{pyblock}

Consideramos  la función $f(x)=x^2\sen(1/x)$. 

\begin{pyblock}
def G(x):
    return x**2*np.sin(1/x)
x=np.arange(0,.15,0.0000001)
y=G(x)
plt.plot(x,y)
\end{pyblock}


 \begin{figure}[h]
 \begin{center}
\adjustimage{max size={0.9\linewidth}{0.9\paperheight}}{imagenes/VolterraPrecursora.png}
 \caption{Función precursora de Volterra}
\end{center}
 \end{figure}





\begin{pyblock}
def F(x):
    return 2*x*np.sin(1/x) - np.cos(1/x)
x = scipy.optimize.broyden2(F, .13, f_tol=1e-14)
x,1-x, G(x)
\end{pyblock}

Se alcanza un máximo en $x=\py{np.float32(x)}$ y toma el valor $G(x)=\py{np.float32(G(x))}$. Hay que utilizar el punto simétrico a $x$, es decir $1-x=\py{np.float32(1-x)}$.


Definimos la función ``madre''.



\begin{pyblock}
def f0(x):
    x1=x[x<=0]
    x2=x[(x<=0.13163877)*(x>0)]
    x3=x[(x>0.13163877)*(x<0.868361226)]
    x4=x[(x>=0.868361226)*(x<1)]
    x5=x[x>=1]
    y1=np.zeros(np.shape(x1))
    y2=x2**2*np.sin(1/x2)
    y3=0.01675771541054875*np.ones(np.shape(x3))
    y4=(1-x4)**2*np.sin(1/(1-x4))
    y5=np.zeros(np.shape(x5))
    return np.concatenate((y1,y2,y3,y4,y5), axis=None)
\end{pyblock}

Definimos la función de Volterra
\begin{pyverbatim}
def volterra(x,n,a=0,b=1):
    if n == 0:
        return 0
    
    a1,b1 = 2.*a/3. + b/3., a/3. + 2.*b/3.
    pto_med = .5*(a+b)
    return volterra(x,n-1,a,a1) + (b1-a1)*f0((x-a1)/(b1-a1))\
    + volterra(x,n-1,b1,b)
\end{pyverbatim}

Graficamos

\begin{pyverbatim}
x=np.arange(0,1,0.0000001)
y=volterra(x,12)
plt.plot(x,y)
\end{pyverbatim}


 \begin{figure}[h]
 \begin{center}
\adjustimage{max size={0.9\linewidth}{0.9\paperheight}}{imagenes/volterra.png}
 \caption{Función de Volterra}
\end{center}
 \end{figure}


\section{Integral de Riemann y pasos al límite}


\chapter{Medida de Lebesgue en $\mathbb{R}$}

\section{Longitud de intervalos}

En el present


\definecolor{cffff00}{RGB}{255,255,0}


\begin{tikzpicture}[y=0.80pt, x=0.80pt, yscale=-1.000000, xscale=1.000000, inner sep=0pt, outer sep=0pt]
  \shade[left color=cffff00,right color=white,rounded corners=0.0000cm] (163.5714,519.5051) rectangle
    (405.0000,820.9336);


    
\end{tikzpicture}


swdasd


\chapter{Medidas abstractas}

\begin{quote}
 <<Me gustaría enfatizar nuevamente, antes de comenzar esta presentación, que la nueva definición será aplicable no solo a un espacio con $n$ dimensiones sino a un conjunto abstracto. Es decir, ni siquiera es necesario, por ejemplo, suponer que sabemos cuál es el límite de elementos en este conjunto>> 
\end{quote}
\begin{flushright}
 M. Frechet \index[personas]{Frechet}\\
Sur l’intégrale d’une fonctionnelle étendueà un ensemble abstrait\\
Bulletin de la S. M. F., tome  43 (1915), p. 248-265.
\end{flushright}

En todos los capítulos anteriores hemos tratado de fundar los conceptos que fuimos introduciendo relacionándolos con  conceptos que juzgamos los precedían. cuando decimos  ``preceder''contemplamos tanto el orden lógico de la construcción  como  el grado de abstracción de los objetos de estudio.

En esta unidad planteamos un salto cualitativo. Vamos abstraernos de la problemática que dió origen a la construcción de la medida en itengral de Lebesgue, esto es la noción de área, y consideraremos una teoría axiomática, donde postularemos como axiomas aquellas propiedades que se revelaron trascendentes en los capítulos anteriores. Este enfoque axiomático se abstrae a su vez de las entidades a las que pretendemos medir, en el sentido que ya no formularemos el concepto de medida para subconjuntos de $\mathbb{R}$, o el espacio euclideano $\mathbb{R}^n$. Introducieremos el concepto de \emph{espacio de medida}\index{Espacio de medida}  como una abstracción y veremos como este concepto induce un consecuente concepto de integral.  


Las nociones introducidas aquí fueron presentadas por primera vez por M. Frechet en el artículo del cual fue extraída la cita con la que comensamos el presente capítulo. 

La noción de medida abstracta es muy fructífera pues unifica multitud de instancias particulares de esta noción que aparecen en distintas áres de la matemática  además de contemplar la medida de Lebesgue. 

\section{Algebras, $\sigma$-álgebras y clases monótonas}

\begin{definicion}[Algebra de conjuntos]{def:algebra} \index{Algebra}
 Sea $X$ un conjunto y $\mathscr{A}\subset \mathcal{P}(X)$. Diremos que $\mathscr{A}$ es un \emph{álgebra} si:
 \begin{enumerate}
  \item $\emptyset\in\mathscr{A}$.
  \item $A\in\mathscr{A}\Rightarrow A^c\in\mathscr{A}$.
  \item $A_i\in\mathscr{A}$, $i=1,\ldots,n$, $\Rightarrow\bigcup_{i=1}^nA_i\in\mathscr{A}$.
 \end{enumerate}

\end{definicion}


\begin{ejercicio}{} Demostrar que los siguientes ejemplos definen álgebras de conjuntos.
 \begin{enumerate}
  \item La colección de todas las uniones de una cantidad finita de intervalos de $\mathbb{R}$, donde por intervalo inlcuímos tanto acotados como  no y tanto abiertos como cerrados como ninguno de ambos.
  \item Como en el ejemplo anterior, pero con los extremos de los intervalos en $\mathbb{Z}\cup \{\pm \infty\}$ o $\mathbb{Q}\cup \{\pm \infty\}$.
    \item Más generalmente aún, como en los ejemplos anteriores, pero con los extremos de los intervalos en $A\cup \{\pm \infty\}$, donde $A\subset\mathbb{R}$.
    \item Sea $X$ un conjunto cualquiera y sean $A_i\subset X$, $i=1,\ldots,n$, subconjuntos mutuamente disjuntos tales que $X=\bigcup_{i=1}^nA_i$. El álgebra que proponemos es $\mathscr{A}=\{\bigcup_{i\in F}A_i | F\subset\{1,\ldots,n\}\}$. Esto es todas las uniones posibles de los $A_i$. Eventualmente $F=\emptyset$ y la unión correspondiente es asumida igual a $\emptyset$. 
  
 \end{enumerate}
\end{ejercicio}
 \begin{ejercicio}{} Demostrar que $\mathscr{A}$  es un álgebra si y solo si
 \begin{enumerate}
  \item $\emptyset\in\mathscr{A}$.
  \item $A\in\mathscr{A}\Rightarrow A^c\in\mathscr{A}$.
  \item $A_i\in\mathscr{A}$, $i=1,\ldots,n$, $\Rightarrow\bigcap_{i=1}^nA_i\in\mathscr{A}$.
 \end{enumerate}
\end{ejercicio}

\begin{definicion}[Clases monótonas]{def:clase_monotona} \index{Clases monótonas}
 Sea $X$ un conjunto y $\mathscr{A}\subset\mathcal{P}(X)$. El conjunto $\mathscr{A}$ se llamará   \emph{clase monótona} si
 \begin{enumerate}
  \item $A_i\in\mathscr{A}$, $A_i\subset A_{i+1}$, $i=1,2,\ldots\Rightarrow \bigcup_{i=1}^{\infty}A_i\in\mathscr{A}$. 
    \item $A_i\in\mathscr{A}$, $A_i\supset A_{i+1}$, $i=1,2,\ldots\Rightarrow \bigcap_{i=1}^{\infty}A_i\in\mathscr{A}$. 
 \end{enumerate}

\end{definicion}

\begin{ejercicio}{} Demostrar que los siguientes ejemplos definen clases monótonas.
 \begin{enumerate}
  \item La colección de todos los intervalos de $\mathbb{R}$ de la forma $(a,+\infty)$ o $[a,+\infty)$ con $a\in [-\infty,+\infty)$.
  \item En $\mathbb{R}^n$ la colección de de todas las bolas, tanto cerradas o abiertas, de centro $0$ y radio $r\in[0,+\infty]$. 
  \item La colección de todos los subgrupos de un grupo dado $G$. 
 \end{enumerate}
\end{ejercicio}

Recordemos del capítulo anterior.
\begin{definicion}[$\sigma$-álgebra de conjuntos]{def:sigma_algebra} \index{$\sigma$-álgebra}
 Sea $X$ un conjunto y $\mathscr{A}\subset \mathcal{P}(X)$. Diremos que $\mathscr{A}$ es una \emph{$\sigma$-álgebra} si:
 \begin{enumerate}
  \item $\emptyset\in\mathscr{A}$.
  \item $A\in\mathscr{A}\Rightarrow A^c\in\mathscr{A}$.
  \item $A_i\in\mathscr{A}$, $i=1,\ldots$, $\Rightarrow\bigcup_{i=1}^{\infty}A_i\in\mathscr{A}$.
 \end{enumerate}

\end{definicion}


\begin{ejercicio}{} Demostrar que  $\mathscr{A}$ es $\sigma$-álgebra si y sólo si es clase monótona y álgebra.
 
\end{ejercicio}
\begin{ejercicio}{ejer:sigma_alg_equiv} Demostrar que  $\mathscr{A}$ es $\sigma$-álgebra si y sólo si es álgebra y satisface que 
\begin{itemize}
 \item $E_n\in\mathscr{A}$, $n=1,2,\ldots$ y $E_i\cap E_j=\emptyset$, cuando $i\neq j$ implican que $\bigcup_{n=1}^{\infty} E_n\in\mathscr{A}$.
\end{itemize}

 
\end{ejercicio}


\section{Medidas} 

\begin{definicion}{} Sea $X$ un conjunto y $\mathscr{A}$ una $\sigma$-álgebra. Una funcion $\mu:\mathscr{A}\to [0,+\infty]$ se llama una \emph{medida}\index{Medida} si  para toda colección numerable de subconjuntos $A_i\in\mathscr{A}$, $i=1,2,\ldots$, mutuamente disjuntos entre si, se satisface que 
$$\mu\left(\bigcup_{i=1}^{\infty}A_i\right)=\sum_{i=1}^{\infty}\mu\left(A_i\right).$$
Al triplete $(X,\mathscr{A},\mu)$ se lo denomina \emph{espacio de medida}.\index{Espacio de medida}. 
\end{definicion}


\begin{ejercicio}{}  Sea $(X,\mathscr{A},\mu)$ espacio de medida. Demostrar que se satisface las siguientes relaciones
\begin{enumerate}
 \item $\mu(\emptyset)=0$.
 \item $\mu(A\cup B)+\mu(A\cap B)=\mu(A)+\mu(B)$.
 \item Si $\mu(B)<\infty$ y $B\subset A$ entonces $\mu(A-B)=\mu(A)-\mu(B)$. 
\end{enumerate}

 
\end{ejercicio}

\begin{ejemplo}{} $(\mathbb{R},\mathscr{M},m)$, donde $\mathscr{M}$ denota la $\sigma$-algebra de los conjuntos medibles Lebesgue y $m$ la medida de Lebesgue, es un espacio de medida. Si en lugar de considerar la $\sigma$-álgebra $\mathscr{M}$ consideramos la $\sigma$-álgebra $\mathscr{B}$ de conjuntos medibles Borel, resulta en otro espacio de medida $(\mathbb{R},\mathscr{B},m)$, que no es más que la restricción de la medida a una sub-$\sigma$-álgebra.
 
\end{ejemplo}


\begin{ejercicio}[Medida de conteo]{def:med_cont} Sea $X$ es un conjunto y $\mathscr{A}=\mathcal{P}(X)$ la colección de todos los subconjuntos de $X$. Para $A\in\mathscr{A}$ escribamos $\mu(A)=\#A$, cuando $A$ es finito, y $\mu(A)=+\infty$ cuando $A$ no es finito. Demostrar que el triple $(X,\mathscr{A},\mu)$ es espacio de medida.
 
\end{ejercicio}
 
 
\begin{ejercicio}{def:med_cont} Sea $X=\mathbb{N}$  y $\mathscr{A}=\mathcal{P}(\mathbb{N})$ la colección de todos los subconjuntos de $\mathbb{N}$. Supongamos dada una función $f:\mathbb{N}\to [0,+\infty]$. Para $A\in\mathscr{A}$ escribamos 
\[\mu(A)=\sum_{n\in A}f(n), \] 
Demostrar que el triple $(\mathbb{N},\mathscr{A},\mu)$ es espacio de medida. ¿Qué resulta $\mu$ si $f(n)=1$ para todo $n$? ¿Qué condición debe satisfacer $f$ para que $\mu(A)<\infty$ para todo $A\subset\mathbb{N}$?
 
\end{ejercicio}

\begin{ejercicio}{def:med_cont_mult3} Sea $X=\mathbb{N}$  y $\mathscr{A}=\mathcal{P}(\mathbb{N})$ la colección de todos los subconjuntos de $\mathbb{N}$. Sea $k$ un natural fijo. Para $A\subset\mathbb{N}$ escribir

\[
 \mu(A)=\#\{ n\in A :  k|n \},
\]
es decir $\mu(A)$ cuenta cuantos multiplos de $k$ hay en $A$. Demostrar que $\mu$ es medida. Demostrar que esta medida es una instancia de las medidas introducidas en \eqref {def:med_cont}. 
\end{ejercicio}


Un ejemplo muy importante  es provisto por la siguiente proposición

\begin{proposicion}{prop:med_abs_cont}
 Sea $f:\mathbb{R}\to [0,+\infty]$ una función integrable y no negativa. El triplete
  $(\mathbb{R},\mathscr{M},\mu_f)$ es espacio de medida, donde $\mathscr{M}$ denota la $\sigma$-algebra de los conjuntos medibles Lebesgue y 
  \[\mu_f(A)=\int_A f(x)dx.\]
\end{proposicion}

\begin{demo}
 Sólo hay que demostrar que $\mu_f$ es una medida. Sean $A_i\in\mathscr{A}$, $i=1,\ldots$, mutuamente disjuntos. 
 
 \begin{ejercicio}{ejer:suma_simples}
  Verificar la siguiente relación
  \[
  \rchi_{\bigcup_{i=1}^{\infty}A_i}=\sum_{i=1}^{\infty}\rchi_{A_i}.
  \]


 \end{ejercicio}

 Luego por la intercambiabilidad entre integral y series de términos positivos

\begin{multline*}
 \mu_f\left(\bigcup_{i=1}^{\infty}A_i \right)=\int_{\bigcup_{i=1}^{\infty}A_i}f(x)dx=\int \rchi_{\bigcup_{i=1}^{\infty}A_i} f(x)dx\\=\int \sum_{i=1}^{\infty}\rchi_{A_i}dx= \sum_{i=1}^{\infty}\int\rchi_{A_i}dx= \sum_{i=1}^{\infty}\mu_f(A_i).
\end{multline*}

 
\end{demo}


\begin{ejercicio}[Delta de dirac]{ejer:delta_dirac}
 Sea $X$ un conjunto no vacío cualquiera, $a\in X$ un punto fijo y $\mathscr{A}=\mathcal{P}(X)$. Definimos $\delta_a:\mathscr{A}\to [0,+\infty]$ por
 \[
  \delta_a(A)=\left\{\begin{array}{ll} 1 &\text{ si } a\in A\\0 &\text{ si } a\notin A\end{array}\right.
 \]
 Demostrar que $(X,\mathscr{A},\delta)$ es un espacio de medida. La medida $\delta_a$
se denomina \emph{$\delta$ de Dirac}\index{$\delta$ de Dirac}\index[personas]{Dirac}. 
\end{ejercicio}
\marginnote{
\begin{center}
 \adjustimage{max size={0.9\linewidth}{0.9\paperheight}}{imagenes/dirac.jpg}\\
\end{center}
\small
Paul Adrien Maurice Dirac, (Brístol, Reino Unido, 8 de agosto de 1902-Tallahassee, Estados Unidos, 20 de octubre de 1984) fue un ingeniero eléctrico, matemático y físico teórico británico que contribuyó de forma fundamental al desarrollo de la mecánica cuántica y la electrodinámica cuántica.
}
\begin{definicion}[Completitud de medidas]{defi:med_completa}
 Un espacio de medida  se llama \emph{completo}\index{Medida!completa} si $A\subset B\in\mathscr{A}$ y $\mu(B)=0$ implican $A\in\mathscr{A}$.
\end{definicion}

\begin{ejemplo}{}  $(\mathbb{R},\mathscr{M},m)$ es un espacio de medida completo, mientras que $(\mathbb{R},\mathscr{B},m)$ no lo es.
\end{ejemplo}





\section{Medida exterior}

\begin{definicion}[Medida exterior]{defi:med_exterior}
Sea $X$ un conjunto no vacío. Una función $\mu^\star:\mathcal{P}(X)\to [0,+\infty]$ se denomina una \emph{medida exterior}\index{Medida!exterior} si satisface que  	
  \begin{description}
   \item[] $\mu^\star(\emptyset)=0$. 
   \item[Monotonía.] $A_1\subset A_2\Rightarrow \mu^\star(A_1)\leq \mu^\star(A_2)$ 
   \item[$\sigma$-subaditividad.] $A_j\subset X$, $j=1,2,\ldots$, $\Rightarrow  \mu^\star\left(\bigcup_{j=1}^{\infty}A_j\right)\leq\sum_{j=1}^{\infty}\mu^\star(A_j)$.
   \end{description}


\end{definicion}

\begin{ejemplo}{} La medida exterior que definimos sobre subconjuntos de $\mathbb{R}$ es obviamente una medida exterior en el sentido de la definición anterior. 
\end{ejemplo}

 



\begin{definicion}[Conjuntos medibles de Carathéodory]{defi:med_cara}
 Sea $\mu^\star$ una medida exterior sobre $X$ y $E\subset X$. Diremos que $E$ es \emph{medible en el sentido de Charathéodory} \index{Conjunto!Medible Carathéodory}\index[personas]{Carathéodory} si para todo $A\subset X$ se cumple que
 \begin{equation}\label{eq:cond_cara}
  \mu^\star(A)=\mu^\star(A\cap E)+\mu^\star(A-E).
 \end{equation}

\end{definicion}

 \begin{observacion} A los efectos de chequear si un conjunto es medible es suficiente probar que se satisface la desigualdad 
   \begin{equation}\label{eq:cond_caraII}
  \mu^\star(A)\geq\mu^\star(A\cap E)+\mu^\star(A-E).
 \end{equation}
 para todo $A$ con medida exterior finita.
 \end{observacion}
 
 
 \begin{teorema}{teo:med_cara} Si $\mu^\star$ es una medida exterior sobre $X$ y $\mathscr{A}$ el conjunto de todos los subconjuntos de $X$ que son medibles según Carathéodory. Entonces  $\mathscr{A}$ es una $\sigma$-álgebra y $\mu^\star$ restringido a $\mathscr{A}$ es una medida. El espacio de medida $(X,\mathscr{A},\mu^\star)$ es completo
  
 \end{teorema}
 
 \begin{demo} Que $\emptyset\in\mathscr{A}$ es una afirmación inmediata. 
 
 Si uno escribe la condición de Carathéodory para $E^c$ queda exactamente igual que la respectiva condición para $E$. Esta observación justifica que $E\in \mathscr{A}$ implica que $E^c\in\mathscr{A}$.
 
 Sean $E_1,E_2\in\mathscr{A}$ y $A\subset X$. Entonces
 \begin{multline*}
  \mu^\star(A)\geq  \mu^\star(A\cap E_2)+ \mu^\star(A\cap E_2^c)\\
  \geq \mu^\star(A\cap E_1\cap E_2)+ \mu^\star(A\cap E_2\cap E_1^c)\\
  + \mu^\star(A\cap E_2^c\cap E_1)+ \mu^\star(A\cap E_2^c\cap E_1^c)
 \end{multline*} 
Ahora los conjuntos $E_2\cap E_1, E_2\cap E_1^c$ y $E_1\cap E_2^c$ son mutuamente disjuntos  y su unión es $E_1\cup E_2$. Esta observación aplicada a los tres primeros términos del último miembro de la desigualdad anterior implica
\[
  \mu^\star(A)\geq \mu^\star(A\cap (E_1\cup E_2))+ \mu^\star(A\cap (E_1\cup E_2)^c).
\]
De esta desiguldad concluímos que $E_2\cup E_2\in\mathscr{A}$. Esto a su vez implica que $\mathscr{A}$ es (al menos) un álgebra. Queremos ver que en realidad es $\sigma$-álgebra.

Supongamos que $E_1\cap E_2=\emptyset$. Tomando $A= E_1\cup E_2$ en \eqref{eq:cond_caraII} obtenemos
\[
 \mu^\star(E_1\cup E_2)\geq \mu^\star(E_1)+\mu^\star(E_2).
 \]

\begin{ejercicio}{ejer:med_ext_premed} Generalizar la desigualdad anterior de la siguiente forma. Si $E_j$, $j=1,\ldots,n$ son mutuamente disjuntos entonces
\[
 \mu^\star(E_1\cup\cdots\cup E_2)\geq \mu^\star(E_1)+\cdots+\mu^\star(E_n).
 \]
 \end{ejercicio}

Siguiendo con la demostración, sean $E_n\in\mathscr{A}$, $n=1,2,\ldots$ una colección numerables de conjuntos mutuamente disjuntos en $\mathscr{A}$. Tomemos $G_n=\bigcup_{j=1}^nE_j$ y $G=\bigcup_{j=1}^{\infty}E_j$.

Como $\mathscr{A}$ es un álgebra, $G_n\in\mathscr{A}$. Además usando sucesivamente la condición de Carathéodory
\begin{equation}\label{eq:dem_cara1}
\begin{split}
  \mu^\star(G_n\cap A) &\geq  \mu^\star(G_n\cap A\cap E_n)+ \mu^\star(G_n\cap A\cap E_n^c)\\
 &=\mu^\star( A\cap E_n)+ \mu^\star(G_{n-1}\cap A)\\
 &\geq \mu^\star( A\cap E_n)+ \mu^\star(A\cap E_{n-1})+ \mu^\star(G_{n-2}\cap A)\\
&\hspace{1cm}\vdots\\
&\geq \sum_{j=1}^n\mu^\star(A\cap E_j).
\end{split}
\end{equation}

Ahora deducimos que para todo $A\subset X$


\begin{align*}
\mu^\star( A)&\geq  \mu^\star(A\cap G_n)+ \mu^\star(A\cap G_n^c) & (G_n\in\mathscr{A})\\
& \geq \sum_{j=1}^n\mu^\star(A\cap E_j)+\mu^\star(A\cap G^c) & (\text{Ecuación \eqref{eq:dem_cara1}}, G_n\subset G)\\ 
\end{align*}

Tomando límite cuando $n\to\infty$
\begin{align*}
\mu^\star( A)&\geq  \sum_{j=1}^{\infty}\mu^\star(A\cap E_j)+\mu^\star(A\cap G^c) &  \\ 
             &\geq  \mu^\star\left(A\cap \bigcup_{j=1}^{\infty} E_j\right)+\mu^\star(A\cap G^c) & (\sigma-\text{subaditividad de } \mu^\star)  \\ 
             &\geq  \mu^\star\left(A\cap G \right)+\mu^\star(A\cap G^c) &   \\ 
\end{align*}
Luego $G\in\mathscr{A}$. Ahora el Ejercicio \ref{ejer:sigma_alg_equiv} implican que $\mathscr{A}$ es $\sigma$-álgebra. Además por el Ejercio\ref {ejer:med_ext_premed}, para $E_j$, $j=1,\ldots$ mutuamente disjuntos en $\mathscr{A}$.
\begin{align*}
 \mu^\star\left(\bigcup_{j=1}^{\infty} E_j\right) &\geq \mu^\star\left(\bigcup_{j=1}^{n} E_j\right) & (\text{monotonía de } \mu^\star)\\ 
 &= \sum_{j=1}^{n}  \mu^\star\left(E_j\right) & (\text{Ejercicio \ref{ejer:sigma_alg_equiv}}) \\ 
\end{align*}

Tomando límite cuando $n\to\infty$ obtenemos 
\[\mu^\star\left(\bigcup_{j=1}^{\infty} E_j\right) \geq \sum_{j=1}^{\infty}  \mu^\star\left(E_j\right)\]
Como la desigualdad inversa a la anterior es siempre cierta por la $\sigma$-subaditividad queda demostrado que $\mu^\star$ es medida sobre $\mathscr{A}$ y finalizada la demostración del teorema.   
 \end{demo}

\section{Premedidas}

\begin{definicion}[Premedida]{defi:premedida}
 Sea $X$ un cojunto no vacío y $\mathscr{A}_0$ un álgebra de subconjuntos de $X$. Diremos que una función $\mu_0: \mathscr{A}_0\to [0,+\infty]$ es una \textbf{premedida}\index{Premedida} si satisface que
 \begin{itemize}
  \item  Si $E_j\in\mathscr{A}_0$, $j=1,2,\ldots$ son mutuamente disjuntos y $\bigcup_{j=1}^{\infty}E_j\in\mathscr{A}_0$ entonces
  \[\mu_0\left(\bigcup_{j=1}^{\infty} E_j\right) = \sum_{j=1}^{\infty}  \mu_0\left(E_j\right)\]
\end{itemize}

  
 
\end{definicion}


Podemos construir medidas a partir de premedidas.

\begin{lema}{lem:extension}
 Sea $\mu_0$ una premedida sobre el álgebra  $\mathscr{A}_0$ de subconjuntos de $X$. Definimos  $\mu^\star:\mathcal{P}(X)\to [0,+\infty]$ por
 \begin{equation}\label{eq:defi_med_ext}
  \mu^\star(E)=\inf\left\{ \sum_{j=1}^{\infty}\mu_0(E_j)\mid E\subset \bigcup_{j=1}^{\infty}E_j, E_j\in\mathcal{A}_0  \right\}
 \end{equation}
 Entonces $\mu^\star$ es una medida exterior que satisface que $\mu^\star(E)=\mu_0(E)$ para todo $E\in\mathcal{A}_0$ y que todo conjunto 
$E\in\mathcal{A}_0$ es medible en el sentido de Carathéodory. 
\end{lema}

 \begin{demo} 
  


\begin{ejercicio}{} Probar que $\mu^\star$ definida en \eqref{eq:defi_med_ext} define en efecto una medida exterior.
 
\end{ejercicio}

Veamos que la restricción de $\mu_{*}$ to $\mathcal{A}$ coincide con $\mu_{0}$. Supongamos que $E \in \mathcal{A}$. Siempre $\mu_{*}(E) \leq \mu_{0}(E)$ pues $E$ se cubre a si mismo. Probemos la desigualdad recíproca. Supongamos  $E \subset \bigcup_{j=1}^{\infty} E_{j}$, con $E_{j} \in \mathcal{A}$ para todo $j$. Definimos
$$
E_{k}^{\prime}=E \cap\left(E_{k}-\bigcup_{j=1}^{k-1} E_{j}\right)
.$$
Los conjuntos $E_{k}^{\prime}$ son mutuamente disjuntos, $E_{k}^{\prime}\in \mathcal{A}, E_{k}^{\prime} \subset E_{k}$ y $E=\bigcup_{k=1}^{\infty} E_{k}^{\prime}$. Por la definición de premedida:
$$
\mu_{0}(E)=\sum_{k=1}^{\infty} \mu_{0}\left(E_{k}^{\prime}\right) \leq \sum_{k=1}^{\infty} \mu_{0}\left(E_{k}\right)
$$
Luego  $\mu_{0}(E) \leq \mu_{*}(E)$.

Por último probemos que los conjuntos en $\mathcal{A}$ son medibles para $\mu_{*}$. Sea $A\subset X$, $E \in \mathcal{A}$ y $\epsilon>0$. Por definición existen  $E_{1}, E_{2}, \ldots$ en $\mathcal{A}$ con $A \subset \bigcup_{j=1}^{\infty} E_{j}$ y
$$
\sum_{j=1}^{\infty} \mu_{0}\left(E_{j}\right) \leq \mu_{*}(A)+\epsilon
$$
Como $\mu_{0}$  finitamente  aditiva en $\mathcal{A}$ 
$$
\begin{aligned}
\sum_{j=1}^{\infty} \mu_{0}\left(E_{j}\right) &=\sum_{j=1}^{\infty} \mu_{0}\left(E \cap E_{j}\right)+\sum_{j=1}^{\infty} \mu_{0}\left(E^{c} \cap E_{j}\right) \\
& \geq \mu_{*}(E \cap A)+\mu_{*}\left(E^{c} \cap A\right)
\end{aligned}
$$

$\epsilon$ es arbitrario,  $\mu_{*}(A) \geq \mu_{*}(E \cap A)+\mu_{*}\left(E^{c} \cap A\right)$ que termina por probar el teorema. \end{demo}

 
\begin{teorema}[Extensión premedidas]{teo:extension}
  Sea $\mu_0$ una premedida sobre el álgebra  $\mathscr{A}_0$ de subconjuntos de $X$. Entonces existe una extensión $\mu$ de $\mu_0$ a la $\sigma$-algebra   $\mathscr{A}$ generada por  $\mathscr{A}_0$.
\end{teorema}
\begin{proof}  La medida exterior $\mu_{*}$ inducida po $\mu_{0}$ define una medida $\mu$ sobre la $\sigma$-álgebra de los conjuntos medibles según  Carathéodory. Por el Lema anterior  $\mu$ es medida sobre $\mathscr{A}$ que extiende $\mu_{0}$. 

\end{proof}



\section{Medidas $\sigma$-finitas}

\begin{definicion}{} Un espacio de medida $(X,\mathcal{A},\mu)$ se llama \index{Medida!$\sigma$-finita}\emph{$\sigma$-finita}  si existen conjuntos medibles $E_n$, $n=1,\ldots$, de medida finita tales que $X=\bigcup_{n=1}^{\infty}E_n$. 
 
\end{definicion}

\begin{ejercicio}{}
 Demostrar que la medida de Lebesgue sobre $\mathbb{R}$ es $\sigma$-finita. Demotrar que la medida de conteo del ejercicio \ref{def:med_cont}  es $\sigma$-finita si y sólo si $X$ es a lo sumo numerable.
\end{ejercicio}


\section{Medida de Lebesgue-Stieltjes}

Haremos una construcción más general  que produce una gran familia de medidas en $\mathbb{R}$ cuyo dominio es el $\sigma$-álgebra de Borel $\mathcal{B}_{\mathbb {R}}$. Tales medidas se denominan \emph{medidas de Borel}\index{medida!Borel} en $\mathbb{R}$.

Para motivar las ideas, supongamos que $\mu$ es una medida de Borel finita en $\mathbb{R}$, y sea $F(x)=\mu((-\infty, x])$ . La función $F$  se llama la \emph{función de distribución}\index{función!distribución} de $\mu$.  Entonces $F$ es creciente a y continua a la derecha, ya que $(-\infty, x]=\ bigcap_{1}^{\infty}\left(-\infty, x_{n}\right]$ siempre que $x_{n} \searrow x$.  Además, si $b>a,(-\infty, b]=(-\infty, a] \cup(a, b]$, entonces $\mu((a, b])=F(b) -F(a)$. Nuestro procedimiento será darle la vuelta a este proceso y construir una medida $\mu$ a partir de una función creciente continua por la derecha $F$. El caso especial $F(x)=x$ producirá la medida habitual de Lebesgue.

Los bloques de construcción de nuestra teoría serán los intervalos abiertos por la izquierda y cerrados por la derecha en $\mathbb{R}$, es decir, conjuntos de la forma $(a, b]$ o $(a, \infty)$ o $\varnothing$, donde $-\infty \leq a<b<\infty$. En esta sección nos referiremos a tales conjuntos como intervalos semi-abiertos. Claramente, la intersección de dos intervalos semi-abiertos es un intervalo semi-abierto, y el complemento de un intervalo semi-abierto es un intervalo semi-abierto o la unión disjunta de dos intervalos semi-abiertos. La colección $\mathcal{A}$ de uniones disjuntas finitas de intervalos semi-abiertos es un álgebra, y la $\sigma$-álgebra generada por $\mathcal{A}$ es la $\sigma$-algebra de Borel  $\mathcal{B}_{\mathbb{R}}$.


\begin{proposicion}{}
 Sea $F: \mathbb{R} \rightarrow \mathbb{R}$ creciente y continua por la derecha. Si $\left(a_{j}, b_{j}\right]$ $(j=1, \ldots, n)$ son intervalos semi-abiertos disjuntos, sea
$$
\mu_{0}\left(\bigcup_{1}^{n}\left(a_{j}, b_{j}\right]\right)=\sum_{1}^{n}\left[F\left(b_{j}\right)-F\left(a_{j}\right)\right]
$$
y sea $\mu_{0}(\varnothing)=0$. Entonces $\mu_{0}$ es una premedida en el álgebra $\mathcal{A} .$
\end{proposicion}

\begin{proof}
 Primero debemos verificar que $\mu_{0}$ esté bien definida, ya que los elementos de $\mathcal{A}$ se pueden representar en más de una forma como uniones disjuntas de intervalos semi-abiertos. Si $\left\{\left(a_{j}, b_{j}\right]\right\}_{1}^{n}$ son disjuntos y $\bigcup_{1}^{n}\left( a_{j}, b_{j}\right]=(a, b]$, entonces, quizás después de volver a etiquetar el índice $j$, debemos tener $a=a_{1}<b_{1}=a_{2 }<b_{2}=\ldots<b_{n}=b$, entonces $\sum_{1}^{n}\left[F\left(b_{j}\right)-F\left(a_{ j}\right)\right]=$ $F(b)-F(a)$. Más generalmente, si $\left\{I_{i}\right\}_{1}^{n}$ y $ \left\{J_{j}\right\}_{1}^{m}$ son dos familias  finitas de intervalos semi-abiertos disjuntos tales que $\bigcup_{1}^{n} I_{i}=\bigcup_{1}^{n} J_{j}$:
 
$$
\sum_{i} \mu_{0}\left(I_{i}\right)=\sum_{i, j} \mu_{0}\left(I_{i} \cap J_{j}\right)= \sum_{j} \mu_{0}\left(J_{j}\right) .
$$
Por lo tanto, $\mu_{0}$ está bien definida y es finitamente aditiva por construcción.
Queda por demostrar que si $\left\{I_{j}\right\}_{1}^{\infty}$ es una secuencia de intervalos semi-abiertos disjuntos con $\bigcup_{1}^{\infty} I_ {j} \in$ $\mathcal{A}$ entonces $\mu_{0}\left(\bigcup_{1}^{\infty} I_{j}\right)=\sum_{1}^{\infty } \mu_{0}\left(I_{j}\right)$. Dado que $\bigcup_{1}^{\infty} I_{j}$ es una unión finita de  intervalos semi-abiertos, la sucesión $\left\{I_{j}\right\}_{1}^{\infty} $ se puede dividir en un número finito de subsucesiones de modo que la unión de los intervalos en cada subsucesión sea un único intervalo semi-abierto. Al considerar cada subsucesión por separado y usar la aditividad finita de $\mu_{0}$, podemos suponer que $\bigcup_{1}^{\infty} I_{j}$ es un intervalo semi-abierto $I=(a, b]$. En este caso, tenemos
$$
\mu_{0}(I)=\mu_{0}\left(\bigcup_{1}^{n} I_{j}\right)+\mu_{0}\left(I \backslash \bigcup_{1} ^{n} I_{j}\right) \geq \mu_{0}\left(\bigcup_{1}^{n} I_{j}\right)=\sum_{1}^{n} \mu_{ 0}\left(I_{j}\right) .
$$
Haciendo $n \rightarrow \infty$, obtenemos $\mu_{0}(I) \geq \sum_{1}^{\infty} \mu_0\left(I_{j}\right)$. 

Para probar la desigualdad inversa, supongamos primero que $a$ y $b$ son finitos, y fijemos $\epsilon>0$. Como $F$ es continua por la derecha, existe $\delta>0$ tal que $F(a+\delta)-F(a)<\epsilon$, y si $I_{j}=\left(a_{j} , b_{j}\right]$, para cada $j$ existe $\delta_{j}>0$ tal que $F\left(b_{j}+\delta_{j}\right)-F\left(b_{j}\right)<\epsilon 2^{-j}$ Los intervalos abiertos $\left(a_{j}, b_{j}+\delta_{j}\right)$ cubren el conjunto compacto $[a+\delta, b]$, por lo que hay un sub-cubrimiento finito. Al descartar cualquier $\left(a_{j}, b_{j}+\delta_{j}\right)$ que esté contenido en uno más grande y reetiquetando el índice $j$, podemos suponer que 
\begin{itemize}
\item  los intervalos $\left(a_{1}, b_{1}+\delta_{1}\right), \ldots,\left(a_{N}, b_{N}+\delta_{N}\right)$  cubren $[a+\delta, b]$,
\item  $b_{j}+\delta_{j} \in\left(a_{j+1}, b_{j+1}+\delta_{j+1}\right)$ para $j=1, \ldots , N-1$.
\end{itemize}
Pero entonces
$$
\begin{aligned}
\mu_{0}(I) &<F(b)-F(a+\delta)+\epsilon \\
& \leq F\left(b_{N}+\delta_{N}\right)-F\left(a_{1}\right)+\epsilon \\
&=F\left(b_{N}+\delta_{N}\right)-F\left(a_{N}\right)+\sum_{1}^{N-1}\left[F\left( a_{j+1}\right)-F\left(a_{j}\right)\right]+\epsilon \\
& \leq F\left(b_{N}+\delta_{N}\right)-F\left(a_{N}\right)+\sum_{1}^{N-1}\left[F\left (b_{j}+\delta_{j}\right)-F\left(a_{j}\right)\right]+\epsilon \\
&<\sum_{1}^{N}\left[F\left(b_{j}\right)+\epsilon 2^{-j}-F\left(a_{j}\right)\right]+ \epsilon \\
&<\sum_{1}^{\infty} \mu\left(I_{j}\right)+2 \epsilon .
\end{aligned}
$$
Dado que $\epsilon$ es arbitrario, demostramos el resultado cuando $a$ y $b$ son finitos. 

Si $a=-\infty$, para cualquier $M<\infty$ los intervalos $\left(a_{j}, b_{j}+\delta_{j}\right)$ cubren $[-M, b]$ , por lo que el mismo razonamiento da $F(b)-F(-M) \leq \sum_{1}^{\infty} \mu_{0}\left(I_{j}\right)+2 \epsilon$, mientras que si $b=\infty$, para cualquier $M<\infty$ obtenemos igualmente $F(M)-F(a) \leq \sum_{1}^{\infty} \mu_{0}\left( I_{j}\right)+2 \epsilon$. El resultado deseado sigue entonces dejando $\epsilon \rightarrow 0$ y $M \rightarrow \infty$.
\end{proof}


\section{Integración en espacio de medida}

\begin{definicion}[Funciones medibles]{def:func_medibles} Sea $(X,\mathcal{A},\mu)$ un espacio de medida.  Una función $f:X\to\overline{\rr}:=\rr\cup\{\pm\infty\}$ se llama \index{Función!medible}\emph{medible} si para todo $a\in\rr$ 
\[
 f^{-1}([-\infty,a))=\left\{x\in X\mid f(x)<a\right\}. 
\]

\end{definicion}

La mayoría de los resultados y definiciones  establecidos en el contexto de la medida de Lebesgue en $\rr$ se extienden si cambios al contexto de medidas abstractas. Enumeremos los más importantes.
\begin{itemize}
 \item El concepto de propiedad válida en casi todo punto.
 \item Funciones simples. son funciones de la forma
 \[
  \phi(x)=\sum_{k=1}^na_k\chi_{E_k},
 \]
 con $a_k\in\rr$ y $E_k\in\mathcal{A}$, $k=1,\ldots,n$. 


 \end{itemize}
 
 \begin{ejercicio}{} Demostrar que si $f:X\to\overline{\rr}$ es medibles entonces si
\end{ejercicio}
 
 
 \begin{itemize}
 \item Integral de funciones medibles no-negativas.
 \item Teorema de Beppo-Levi.
 \item Lema Fatou.
 \item Teorema de la convergencia mayorada de Lebesgue
\end{itemize}


\end{document}

\section{Medidas producto}

Hemos sido capaces de definir una medida sobre subconjuntos de $\rr$ que posee  las propiedades que a priori queríamos que tuviese, particularmente  la invariancia por traslaciones y tal que $\mu([0,1])=1$. El conjunto $\rr$ es importante pues es el modelo del espacio euclideano unidimensional. Pero tan naturales como este conjunto son los espacios euclideanos $n$-dimensionales $\rr^n$, sobre los cuales no tenemos definidas medidas, más que aquella definida en el ejemplo \ref{def:med_cont}. En estos espacios es util tener una medida  invariante por traslaciones y tal que $\mu([0,1]^n)=1$. Vamos a construit tales medidas en esta sección. Como $\rr^n$ es elproducto cartesiano de $n$ copias de $\rr$, seremos capaces de construir una medida allí si somo capaces de construir medidas sobre productos cartesianos de espacios de medidas.  


\begin{definicion}[Rectángulos medibles]{def:rect-med}
 Dados dos espacios de medida $(X_i,\mathscr{A}_i,\mu_i)$, $i=1,2$, un \index{rectángulo medible}\emph{rectángulo medible} es un conjunto de la forma $R=A_1\times A_2$, con $A_i\in\mathscr{A}_i$, $i=1,2$. Sea $\mathscr{A}_0$  la colección de todos los conjuntos que se pueden expresar como unión de una cantidad finita de restángulos medibles. 
\end{definicion}

\begin{ejercicio}{} Demostrar que $\mathscr{A}_0$ es una álgebra.
\end{ejercicio}




\begin{definicion}{} Si  $A\times B$ es un rectángulo medible definimos 
\[
 \mu_0(A\times B)=\mu_1(A)\mu_2(B).
\]
\end{definicion}




\begin{ejercicio}{} Supongamos que $A\times B$ es un rectángulo medible que es unión disjunta $\bigcup_{j=1}^{\infty}A_j\times B_j$ de otros rectángulos medibles $A_j\times B_j$, $j=1,\ldots$.  Demostrar que 
\begin{equation}\label{eq:chis_prod}
 \rchi_{A}\rchi_{B}=\sum_{j=1}^{\infty} \rchi_{A_j}\rchi_{B_j} 
\end{equation}

\end{ejercicio}

\begin{proposicion}{} Supongamos que $A\times B$ es un rectángulo medible que es unión disjunta $\bigcup_{j=1}^{\infty}A_j\times B_j$ de otros rectángulos medibles $A_j\times B_j$, $j=1,\ldots$.  Entonces
\begin{equation}\label{eq:chis_prod}
 \mu_0(A\times B)=\sum_{j=1}^{\infty} \mu_0(A_j\times B_j). 
\end{equation}

\end{proposicion}


\begin{definicion}{}
Si $C\in\mathscr{A}_0$ con $C=\bigcup_{j=1}^{n}A_j\times B_j$, $A_j\times B_j=\emptyset$, si $i\neq j$, definimos 
\[
 \mu_0(C)=\sum_{j=1}^{n}\mu_0(A_j\times B_j)
\]

\end{definicion}





\begin{proposicion}[Buena definición]{} La función $\mu_0$ esta bien definida, i.e. si $C$ admite dos representaciones distintas   $C=\bigcup_{j=1}^{n}A_j\times B_j=\bigcup_{j=1}^{n}A'_j\times B'_j$ como unión de rectángulos medibles mutuamente disjuntos entonces
 \[
\sum_{j=1}^{n}\mu_0(A_j\times B_j)=\sum_{j=1}^{n}\mu_0(A'_j\times B'_j)
\]
\end{proposicion}

\begin{proposicion}{} La función $\mu_0$ es una premedida para el algebra $\mathscr{A}_0$.
\end{proposicion}

\chapter*{Apéndice}

\section{Topología}

\begin{teorema}[Principio de encaje de intervalos\index{Intervalos encajados}]{}  Sea $I_n=[a_n,b_n]\subset\mathbb{R}$ una sucessión de intervalos con las siguientes propiedades
\begin{enumerate}
 \item $\forall n\in\mathbb{N}: I_n\subset I_{n+1},$
 \item $\lim\limits_{n\to\infty}(b_n-a_n)=0.$
\end{enumerate}
Entonces $\bigcap_{n=1}^{\infty}I_n$ consiste de uno, y solo un, punto $x\in\mathbb{R}$.
 
\end{teorema}

\begin{proof}
 
\end{proof}





\begin{teorema}[Heine-Borel]{}\index[personas]{Heine}\index[personas]{Borel} Toda sucesión acotada de $\mathbb{R}$ 
tiene una subsucesión convergente.
 
\end{teorema}

\begin{proof} Uso encajes de intervalos.
 
\end{proof}


\begin{definicion}[Continuidad uniforme \index{Continuidad uniforme}]{} Sea $f:A\subset\mathbb{R}\to\mathbb{R}$ una función. Diremos que $f$ es uniformemente continua si 
\[
 \forall \epsilon>0\exists \delta>0 \forall x,y\in A:|x-y|<\delta\Rightarrow |f(x)-f(y)|<\epsilon.
 \]

\end{definicion}

\begin{ejemplo} Varios ilustrando diferencia con continuidad
 
\end{ejemplo}


\begin{teorema}{} Sea $f:[a,b]\to\mathbb{R}$ continua. Entonces $f$ es uniformemente continua. 
 \end{teorema}
\begin{proof} Uso Heine-Borel
 \end{proof}


%\include{Apendice0}

 \nocite{*}
 \addcontentsline{toc}{chapter}{Bibliografía}
 
\bibliographystyle{apalike}
\bibliography{biblio}
 
 \printindex
\printindex[personas]
\printindex[simbolos]



\end{document}
