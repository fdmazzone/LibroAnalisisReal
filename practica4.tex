\documentclass{book}
\usepackage{amssymb,amsmath}
\usepackage{polyglossia}
\setmainlanguage{spanish} % Idioma principal
\usepackage{theorem}
\usepackage{times}
\usepackage{array}
\usepackage{graphicx}
\usepackage{hyperref}
\usepackage{multirow}
\usepackage{fancyhdr}
%\usepackage[cp1252]{inputenc}
\usepackage{hhline}
\usepackage{multicol}
\usepackage[a4paper,driver=xetex,top=4.5cm,head=4.5cm, bottom=2cm,%
layouthoffset=0mm, left=2.5cm, right=2.5cm,marginparwidth=0cm]{geometry}
%\usepackage{bm}
%\usepackage{tabular}
\usepackage{fontspec}
 \usepackage[breakable,many]{tcolorbox}
\defaultfontfeatures{Ligatures=TeX}
\usepackage{empheq}
 \setromanfont{Roboto Condensed}
 \usepackage{float}
 \usepackage{mathrsfs} 
%
%
% \renewcommand{\familydefault}{\sfdefault}
%\renewcommand{\familydefault}{\sfdefault}

%%%%%%%%%Estilo de la pagina%%%%%%%%%%%%%%%%%%%%%%%%%%%%%%%%%%%
%%%%%%%%%%%%%%%%%%%%%%%%%%%%%%%%%%%%%%%%%%%%%%%%%%%%%%%%%%%%%%%%%%
% \newcounter{ejer}
% 
% {\theorembodyfont{\normalfont}
% \newtheorem{ejercicio}[ejer]{Ejercicio}}

\newcommand{\rr}{\mathbb{R}}
\newcommand{\qq}{\mathbb{Q}}
\newcommand{\nn}{\mathbb{N}}
\newcommand{\ii}{\mathbb{I}}



\DeclareMathOperator{\atan2}{atan2}
%\DeclareMathOperator{\sen}{sen}
\DeclareMathOperator{\sign}{sign}
\DeclareMathOperator{\sn}{sn}
\DeclareMathOperator{\SO}{SO}
%\DeclareMathOperator{\arcsen}{arcsen}
\DeclareMathOperator{\Or}{O}

\usepackage[framemethod=TikZ]{mdframed}
%%%%%%%%%%%%%%%%%%%%%%%%%%%%%%
%Theorem

%% Ejercicio
\newcounter{ejer} \setcounter{ejer}{0}
\renewcommand{\theejer}{\arabic{ejer}}
\newenvironment{ejer}[2][]{%
\vspace{5pt}
\refstepcounter{ejer}%
\ifstrempty{#1}%
{%
% \mdfsetup{%
% frametitle={%
% \tikz[baseline=(current bounding box.east),outer sep=-0pt]
% \node[anchor=east,rectangle,fill=green!50]
{\noindent\bfseries Ejercicio~\theejer}.}
%
{%
% \mdfsetup{%
% frametitle={%
% \tikz[baseline=(current bounding box.east),outer sep=0pt]
% \node[anchor=east,rectangle,fill=green!50]
{\noindent\bfseries  Ejercicio~\theejer:~#1}.}%
%
%\mdfsetup{innertopmargin=10pt,linecolor=green!50,%
%linewidth=2pt,topline=true,%
%frametitleaboveskip=\dimexpr-\ht\strutbox\relax
%}
%\begin{mdframed}[]
\relax%
\label{#2}}{\vspace{5pt}}%\end{mdframed}}

%Theorem
\newcounter{theo}[chapter] \setcounter{theo}{0}
\renewcommand{\thetheo}{\arabic{section}.\arabic{theo}}
\newenvironment{theo}[2][]{%
\refstepcounter{theo}%
\ifstrempty{#1}%
{\mdfsetup{%
frametitle={%
\tikz[baseline=(current bounding box.east),outer sep=0pt]
\node[anchor=east,rectangle,fill=blue!20]
{\strut Teorema~\thetheo};}}
}%
{\mdfsetup{%
frametitle={%
\tikz[baseline=(current bounding box.east),outer sep=0pt]
\node[anchor=east,rectangle,fill=blue!20]
{\strut Teorema~\thetheo:~#1};}}%
}%
\mdfsetup{innertopmargin=10pt,linecolor=blue!20,%
linewidth=2pt,topline=true,%
frametitleaboveskip=\dimexpr-\ht\strutbox\relax
}
\begin{mdframed}[]\relax%
\label{#2}}{\end{mdframed}}
%%%%%%%%%%%%%%%%%%%%%%%%%%%%%%
%Lemma
\newcounter{lem}[chapter] \setcounter{lem}{0}
\renewcommand{\thelem}{\arabic{section}.\arabic{lem}}
\newenvironment{lem}[2][]{%
\refstepcounter{lem}%
\ifstrempty{#1}%
{\mdfsetup{%
frametitle={%
\tikz[baseline=(current bounding box.east),outer sep=0pt]
\node[anchor=east,rectangle,fill=green!20]
{\strut Lemma~\thelem};}}
}%
{\mdfsetup{%
frametitle={%
\tikz[baseline=(current bounding box.east),outer sep=0pt]
\node[anchor=east,rectangle,fill=green!20]
{\strut Lemma~\thelem:~#1};}}%
}%
\mdfsetup{innertopmargin=10pt,linecolor=green!20,%
linewidth=2pt,topline=true,%
frametitleaboveskip=\dimexpr-\ht\strutbox\relax
}
\begin{mdframed}[]\relax%
\label{#2}}{\end{mdframed}}
%%%%%%%%%%%%%%%%%%%%%%%%%%%%%%
%% Definicion
\newcounter{defini}[chapter] \setcounter{defini}{1}
\renewcommand{\thedefini}{\arabic{section}.\arabic{defini}}
\newenvironment{definicion}[2][]{%
\refstepcounter{defini}%
\ifstrempty{#1}%
{\mdfsetup{%
frametitle={%
\tikz[baseline=(current bounding box.east),outer sep=0pt]
\node[anchor=east,rectangle,fill=green!20]
{\strut Definición~\thedefini};}}
}%
{\mdfsetup{%
frametitle={%
\tikz[baseline=(current bounding box.east),outer sep=0pt]
\node[anchor=east,rectangle,fill=green!20]
{\strut Definición~\thedefini:~#1};}}%
}%
\mdfsetup{innertopmargin=10pt,linecolor=green!20,%
linewidth=2pt,topline=true,%
frametitleaboveskip=\dimexpr-\ht\strutbox\relax
}
\begin{mdframed}[]\relax%
\label{#2}}{\end{mdframed}}

%Proof
\newenvironment{prf}{\noindent\emph{Dem.}}{$\square$ \newline\vspace{5pt}}


%Corolario
\newcounter{cor}[chapter] \setcounter{cor}{0}
\renewcommand{\thecor}{\arabic{section}.\arabic{cor}}
\newenvironment{cor}[2][]{%
\refstepcounter{cor}%
\ifstrempty{#1}%
{\mdfsetup{%
frametitle={%
\tikz[baseline=(current bounding box.east),outer sep=0pt]
\node[anchor=east,rectangle,fill=green!20]
{\strut Corolario~\thelem};}}
}%
{\mdfsetup{%
frametitle={%
\tikz[baseline=(current bounding box.east),outer sep=0pt]
\node[anchor=east,rectangle,fill=green!20]
{\strut Corolario~\thelem:~#1};}}%
}%
\mdfsetup{innertopmargin=10pt,linecolor=green!20,%
linewidth=2pt,topline=true,%
frametitleaboveskip=\dimexpr-\ht\strutbox\relax
}
\begin{mdframed}[]\relax%
\label{#2}}{\end{mdframed}}

\tcbset{highlight math style={enhanced,
  colframe=red!60!black,colback=yellow!50!white,arc=4pt,boxrule=1pt,
  drop fuzzy shadow}}
  
  
  
  
  
  
  


\pagestyle{fancyplain}

 \renewcommand{\sectionmark}[1]
                 {\markright{\thesection\ #1}}


% \lhead[\fancyplain{}{\bfseries\thepage}]
%       {\fancyplain{}{\bfseries\rightmark}}
%
 \rhead[\fancyplain{}{\bfseries\leftmark}]{\fancyplain{}{\bfseries}}




 \lhead[\fancyplain{}{ \includegraphics[scale=.3]{EscudoUNLPam.png}}]{\fancyplain{}{ \includegraphics[scale=.3]{EscudoUNLPam.png}}}

\cfoot{}





  
  
  
  
  
  
  
\begin{document}


\hyphenation{excen-tri-ci-dad}


\begin{large}
\begin{bfseries} % \begin{scshape}
        \noindent Depto de Matem\'atica.\\
        Primer Cuatrimestre de 2022\\                                                                                                                                                                                                                                                                                                                                                
        Teoría de la Medida \\
        Práctica 4: Funciones medibles

%\end{scshape}
\end{bfseries}
\end{large}
\par\noindent\rule{\textwidth}{.5pt}

	%\begin{ejer}{} 
%Sea $f$ una  funci\'on medible sobre el conjunto medible $E$. %Mostrar que
%\begin{enumerate}
%\item $f^{-1}([c,d))=\{ x \in E: c\leq f(x)<d \}$ y 
%$f^{-1}((c,d])=\{ x \in E: c< f(x)\leq d \}$ son conjuntos %medibles.
%\\
%\underline{Sugerencia:} $f^{-1}([c,d))=f^{-1}([c,\infty]\cap %[-\infty,d))=f^{-1}([c,\infty])\cap f^{-1}( [-\infty,d)).$
%\item  $f^{-1}((c,d))=\{ x \in E: c< f(x)<d \}$  es %medible.
%\\
%\underline{Ayuda:} $f^{-1}([c,d)))=f^{-1}((c,\infty]\cap %[-\infty,d))=f^{-1}((c,\infty])\cap f^{-1}( %[-\infty,d)).$ 
%\item  $f^{-1}(\{\infty\})$ y $f^{-1}(\{-\infty\})$ son %conjuntos  medibles.
%\\
%%\underline{Sugerencia:} $f^{-1}({\infty})=\{ x \in E: f(x)=\infty \}=  \bigcap\limits_k \{ x \in E: f(x)>k\}.$ 
%\item  $f^{-1}(\{c\})=\{x\in E: f(x)=c\}$ es medible. 
%\\
%\underline{Ayuda:} $\{ x \in E: f(x)=c \}=\{x \in E: %f(x)\geq c\}  \cap \{ x \in E: f(x)\leq c\}.$ 
%\item Sea $G$ cualquier conjunto abierto de $\rr$. Entonces %$f^{-1}(G)$ es un subconjunto medible de $E$.
%\\
%\underline{Sugerencia:} $G=\bigcup\limits_k I_k$, $I_k$ %intervalos abiertos disjuntos y 
%$f^{-1}(G)=f^{-1}(\bigcup I_k)=\bigcup\ f^{-1}(I_k)$.
%\end{enumerate}
%\end{ejer}


\begin{ejer}{} 
Mostrar que las siguientes funciones son medibles sobre sus respectivos dominios y calcular 
$m(f^{-1}((c,\infty]))$.
\begin{enumerate}
\item  $f(x)=\frac{1}{x}$, para $0<x<1$. \;\;
\underline{Sugerencia:} $c<0,$\, $0\leq c<1$,\, $1\leq c$.
\item* 
{
\extrarowheight = -0.5ex
\renewcommand{\arraystretch}{1.8}
$f(x)=\left\{
\begin{array}{ll}
\frac{1}{x}&0<x\leq 1
\\
0&x=0
\end{array}
\right.$}
%\item $f(x)=\left\{
%\begin{array}{ll}
%1&x \in \qq \cap[0,1]
%\\
%0&x \in \ii\cap[0,1]
%\end{array}
%\right.
%$
\end{enumerate}
\end{ejer}


\begin{ejer}{} 
Mostrar que si $f$ es una funci\'on  medible sobre un conjunto medible $E$ y si $A$ es cualquier subconjunto medible de $E$, entonces $f$ es una funci\'on medible sobre $A$.
%\\
%\underline{Sugerencia:}  $\{x \in A: f(x)>c \}=\{ x\in E: %f(x)>c\}\cap A.$
\end{ejer}




\begin{ejer}{} 
Probar que si $f(x)$ es una función medible y $h$ es un vector de $\rr^d$, la ``función 
trasladada'' $f(x+h)$ es también medible.
\end{ejer}

%\begin{ejer}{} 
 %Demostrar que
 %: 
	%\begin{enumerate}
%\item una función $f$ continua en c.t.p es medible;
%\item 
%si $f$ es medible y $f=g$ en c.t.p, entonces $g$ es medible.
%	\end{enumerate}
%	\end{ejer}


\begin{ejer}{} 
Determinar si las siguientes funciones son simples en $[0,1]$:
\begin{enumerate}
\item 
{
\extrarowheight = -0.5ex
\renewcommand{\arraystretch}{1.8}
$\varphi(x)=
\left\{
\begin{array}{rl}
1& x=\frac{1}{n},\;n=1,2,\dots
\\
-1&x\neq \frac{1}{n}
\end{array}
\right.
$}
\item 
$\varphi(x)=
\left\{
\begin{array}{ll}
1& x \in \ii
\\
0&x\in \qq
\end{array}
\right.
$
\end{enumerate}
\end{ejer}

\begin{ejer}{}
 Para cada $k\in \nn$, sea
 {
\extrarowheight = -0.5ex
\renewcommand{\arraystretch}{1.8}
 $\varphi_k(x)=
 \left\{
 \begin{array}{lll}
  \frac{i-1}{2^k}    & \frac{i-1}{2^k} \leq x< \frac{i}{2^k}, & i=1,2,\ldots,k 2^k
  \\
    0  & x<0 &
    \\
    k &  x\geq k. &
 \end{array}
 \right.$}
 %con $k \in \nn$. 
 
 Dada $f\geq 0$  y  medible, se define $f_k=\varphi_k \circ f$. 
 Se verifica que cada $f_k$ es medible y que $\lim\limits_{k \to \infty} f_k=f$.
\\
Hallar $f_k$ para 
 \begin{enumerate}
     \item  $f$ igual a la funci\'on de Dirichlet;
     \item\label{it:item-2} $f=\alpha \chi_A$, $\alpha\in \rr$ y $\alpha>0$, $A$ medible;
     \item ?`puede ocurrir que $f=f_k$ para las funciones del inciso \ref{it:item-2}?
 \end{enumerate}
\end{ejer}


\begin{ejer}{} 
Calcular $f^+$, $f^-$, $f^+-f^-$, $f^{+} + f^-$ para
\begin{enumerate}
\item 
{
\extrarowheight = -0.5ex
\renewcommand{\arraystretch}{1.5}
$f(x)=\left\{
\begin{array}{lc}
\infty&x=\frac{-\pi}{2}
\\
\tan x &-\frac{\pi}{2}<x<\frac{\pi}{2}
\\
\infty&x=\frac{\pi}{2}
\end{array}
\right.$
\item $f(x)=\left\{
\begin{array}{ll}
x^2-1&x<1
\\
3-x&x\geq 1
\end{array}
\right.$}
\end{enumerate}
\end{ejer}




\begin{ejer}{} 
Exhibir una función no medible $f$ tal que $|f|$ es medible.
\end{ejer}




\begin{ejer}{} 
Sea $(f_k)$ una sucesión de  funciones medibles que converge a una función finita $f$
en c.t.p de un conjunto $E$ de medida finita. 
\\
Demostrar que para cada $\delta>0$  existe un conjunto
$E_{\delta}\subset E$ de medida menor que $\delta$, tal que $f_k$ converge uniformemente sobre 
$E-E_{\delta}$. 
{\it{(Teorema de Egorov)}}.
\\
\underline{Sugerencia:} Considérese para cada $i$ la 
sucesión decreciente de conjuntos
$$E_m^i=\bigcup_{k\geq m}E\Bigl(|f_k-f|\geq \frac{1}{i}\Bigr) \;\;\;m=1,2,3,...$$
y elíjase un índice $m_i$ tal que la medida de $E_{m_i}^i$ sea menor luego, llámese $E_{\delta}$ a la unión de estos conjuntos.
\end{ejer} 

\begin{ejer}{} 
Considerar la sucesi\'on de funciones $f_k(x)=\left[\cos(\frac{\pi}{x})\right]^{2k}$ para $x \in [0,1]$.
\begin{enumerate}
\item Demostrar que $f_k \to 0$ en c.t.p.
\item El \textit{Teorema de Egorov} dice que existe un conjunto $E_{\delta}$ tal que $f_k \xrightarrow{u} 0$
en $[0,1]-E_{\delta}$. 
Hallar $E_{\delta}$.
\end{enumerate}
\end{ejer}

%\begin{ejer} {}*
%\begin{enumerate} 
%\item Dada la sucesión
%{
%\extrarowheight = -0.5ex
%\renewcommand{\arraystretch}{2.0}
%$f_k(x)=\left\{
%\begin{array}{cl}
%k^2x &\mbox{si }x\in[0,\frac{1}{k}]
%\\
%-k^2x+2k &\mbox{si } x\in \left[\frac{1}{k},\frac{2}{k}\right] %
%\\
%&\mbox{en otro caso,}
%\end{array}
%\right.
%$}\\
%comprobar que $f_k \to 0$ en $[0,1]$. 
%?`Qué se puede decir acerca de la convergencia uniforme en %$[0,1]$?
%\item Sea $f_k=\chi_{(0,\frac{1}{k})}$ con $k\in\nn$. 
%Probar que $\{f_k\}_{\{k\in\nn\}}$ converge puntualmente a $0$ %pero que no converge uniformemente
%excepto un conjunto de medida nula.
%\end{enumerate}
%\end{ejer}

%\begin{ejer}{}  
%Considerar la funci\'on 
%$f(x)=\left\lfloor \frac{1}{x}\right\rfloor+\left\lfloor %\frac{1}{x-1}\right\rfloor$
%para $x \in [0,1]$.
%\begin{enumerate}
%\item Demostrar que $f$ es medible.
%\item Seg\'un el \textit{Teorema de Lusin} existe  un conjunto %compacto $E_{\varepsilon}\subset [0,1]$
%tal que $m\left([0,1]-E_{\varepsilon}\right)<\epsilon$ y %$f\left.\right|_{E_{\varepsilon}}$ es continua.
%Hallar un conjunto $E_{\varepsilon}$.
%\end{enumerate}
%\end{ejer}


%\begin{ejer}{}
% Probar que toda función medible $f$ coincide en c.t.p con una %función boreliana.
%\\
%\underline{Sugerencia:} suponer %primero que $f$ es la función %característica de un conjunto
%medible; luego, que $f$ es una %función simple;   finalmente que %$f\geq 0$.
%\end{ejer}





%\begin{ejer}{}
%Calcular $\underline{f_k}$, $\overline{f_k}$, $\limsup f_k$, %$\liminf f_k$ y $\lim f_k$ cuando sea
%apropiado para las siguientes sucesiones de funciones:
%\begin{enumerate}
%\item
%$f_k(x)=x^k$, \; para $0\leq x\leq 1.$
%\item $f_{2k}(x)=\left(1+\frac{x}{2k}\right)^{2k}$, %$f_{2k-1}(x)=\left(1-\frac{x}{2k-1}\right)^{2k-1}$, \;para %$0\leq x<\infty$.
%%\item $f_k(x)=\left\{ 
%%\begin{array}{ll}
%%1&x=\frac{i}{j}, \;i+j=k,\; 0\leq x\leq 1
%%\\
%%0& \mbox{en otro caso.}
%%\end{array}
%%\right.$
%\end{enumerate}
%\end{ejer}



\begin{ejer}{} Probar que si $f_k \stackrel{m}{\rightarrow}f$ \;y\; $g_k \stackrel{m}{\rightarrow}g$, entonces \;
$f_k+g_k \stackrel{m}{\rightarrow}f+g$.  
Y si, además, todas las funciones son finitas y $m(E)<\infty$, entonces $f_k.g_k \stackrel{m} \rightarrow f.g$
dentro del conjunto $E$.
\end{ejer}

\begin{ejer}{} 
Dada la sucesión $\{f_k(x)\}=\{\text{sen}^{2k} x\}$ definida sobre el intervalo $[-5 \pi, 5\pi]$, 
analizar  la convergencia puntual, uniforme y en medida.
\end{ejer}


%\begin{ejer}{}* 
%Probar que si $f_k \stackrel{m}{\rightarrow}f$, \;entonces existe una subsucesión %$(f_{k_i})$ que 
%converge a $f$ en c.t.p.
%\end{ejer}

\begin{ejer}{} 
 Siendo $E$ un subconjunto de medida finita de $\rr^d$,\;  para cada función medible $f$
y cada n\'umero real $t$, se define
$$\lambda_f(t)=m(\{x:x\in E,f(x)>t\})=m\,E(f>t).$$
Probar que 
	\begin{enumerate}
  \item $\lambda_f$ es decreciente y continua por la derecha;
  \item si $f_k \stackrel {m}{\rightarrow}f$, entonces $\lambda_{f_k} \rightarrow \lambda_f(t)$
  en cada punto $t$ donde $\lambda_f$ sea continua.
\\
  \underline{Sugerencia:}
de la inclusión $E(f_k>t)\subset E(f>t-\epsilon)\cup E(|f_k-f|>\epsilon) $
se deduce \;$\limsup\limits_{{k\rightarrow \infty}\atop} \lambda_{f_k}(t)\leq \lambda_{f}(t-)$\; y análogamente se prueba que 
\;$\lambda_f(t)\leq \liminf\limits_{k\rightarrow \infty} \lambda_{f_k}(t).\;$
	\end{enumerate}
\end{ejer}

%======================================================================================


\end{document}
