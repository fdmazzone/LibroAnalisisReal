\documentclass{book}
\usepackage{amssymb,amsmath}
\usepackage{polyglossia}
\setmainlanguage{spanish} % Idioma principal
\usepackage{theorem}
\usepackage{times}
\usepackage{array}
\usepackage{graphicx}
\usepackage{hyperref}
\usepackage{multirow}
\usepackage{fancyhdr}
%\usepackage[cp1252]{inputenc}
\usepackage{hhline}
\usepackage{multicol}
\usepackage[a4paper,driver=xetex,top=4.5cm,head=4.5cm, bottom=2cm,%
layouthoffset=0mm, left=2.5cm, right=2.5cm,marginparwidth=0cm]{geometry}
%\usepackage{bm}
%\usepackage{tabular}
\usepackage{fontspec}
 \usepackage[breakable,many]{tcolorbox}
\defaultfontfeatures{Ligatures=TeX}
\usepackage{empheq}
 \setromanfont{Roboto Condensed}
 \usepackage{float}
 \usepackage{mathrsfs} 
%
%
% \renewcommand{\familydefault}{\sfdefault}
%\renewcommand{\familydefault}{\sfdefault}

%%%%%%%%%Estilo de la pagina%%%%%%%%%%%%%%%%%%%%%%%%%%%%%%%%%%%
%%%%%%%%%%%%%%%%%%%%%%%%%%%%%%%%%%%%%%%%%%%%%%%%%%%%%%%%%%%%%%%%%%
% \newcounter{ejer}
% 
% {\theorembodyfont{\normalfont}
% \newtheorem{ejercicio}[ejer]{Ejercicio}}

\newcommand{\rr}{\mathbb{R}}
\newcommand{\qq}{\mathbb{Q}}
\newcommand{\nn}{\mathbb{N}}



\DeclareMathOperator{\atan2}{atan2}
%\DeclareMathOperator{\sen}{sen}
\DeclareMathOperator{\sign}{sign}
\DeclareMathOperator{\sn}{sn}
\DeclareMathOperator{\SO}{SO}
%\DeclareMathOperator{\arcsen}{arcsen}
\DeclareMathOperator{\Or}{O}

\usepackage[framemethod=TikZ]{mdframed}
%%%%%%%%%%%%%%%%%%%%%%%%%%%%%%
%Theorem

%% Ejercicio
\newcounter{ejer} \setcounter{ejer}{0}
\renewcommand{\theejer}{\arabic{ejer}}
\newenvironment{ejer}[2][]{%
\vspace{5pt}
\refstepcounter{ejer}%
\ifstrempty{#1}%
{%
% \mdfsetup{%
% frametitle={%
% \tikz[baseline=(current bounding box.east),outer sep=-0pt]
% \node[anchor=east,rectangle,fill=green!50]
{\noindent\bfseries Ejercicio~\theejer}.}
%
{%
% \mdfsetup{%
% frametitle={%
% \tikz[baseline=(current bounding box.east),outer sep=0pt]
% \node[anchor=east,rectangle,fill=green!50]
{\noindent\bfseries  Ejercicio~\theejer:~#1}.}%
%
%\mdfsetup{innertopmargin=10pt,linecolor=green!50,%
%linewidth=2pt,topline=true,%
%frametitleaboveskip=\dimexpr-\ht\strutbox\relax
%}
%\begin{mdframed}[]
\relax%
\label{#2}}{\vspace{5pt}}%\end{mdframed}}

%Theorem
\newcounter{theo}[chapter] \setcounter{theo}{0}
\renewcommand{\thetheo}{\arabic{section}.\arabic{theo}}
\newenvironment{theo}[2][]{%
\refstepcounter{theo}%
\ifstrempty{#1}%
{\mdfsetup{%
frametitle={%
\tikz[baseline=(current bounding box.east),outer sep=0pt]
\node[anchor=east,rectangle,fill=blue!20]
{\strut Teorema~\thetheo};}}
}%
{\mdfsetup{%
frametitle={%
\tikz[baseline=(current bounding box.east),outer sep=0pt]
\node[anchor=east,rectangle,fill=blue!20]
{\strut Teorema~\thetheo:~#1};}}%
}%
\mdfsetup{innertopmargin=10pt,linecolor=blue!20,%
linewidth=2pt,topline=true,%
frametitleaboveskip=\dimexpr-\ht\strutbox\relax
}
\begin{mdframed}[]\relax%
\label{#2}}{\end{mdframed}}
%%%%%%%%%%%%%%%%%%%%%%%%%%%%%%
%Lemma
\newcounter{lem}[chapter] \setcounter{lem}{0}
\renewcommand{\thelem}{\arabic{section}.\arabic{lem}}
\newenvironment{lem}[2][]{%
\refstepcounter{lem}%
\ifstrempty{#1}%
{\mdfsetup{%
frametitle={%
\tikz[baseline=(current bounding box.east),outer sep=0pt]
\node[anchor=east,rectangle,fill=green!20]
{\strut Lemma~\thelem};}}
}%
{\mdfsetup{%
frametitle={%
\tikz[baseline=(current bounding box.east),outer sep=0pt]
\node[anchor=east,rectangle,fill=green!20]
{\strut Lemma~\thelem:~#1};}}%
}%
\mdfsetup{innertopmargin=10pt,linecolor=green!20,%
linewidth=2pt,topline=true,%
frametitleaboveskip=\dimexpr-\ht\strutbox\relax
}
\begin{mdframed}[]\relax%
\label{#2}}{\end{mdframed}}
%%%%%%%%%%%%%%%%%%%%%%%%%%%%%%
%% Definicion
\newcounter{defini}[chapter] \setcounter{defini}{1}
\renewcommand{\thedefini}{\arabic{section}.\arabic{defini}}
\newenvironment{definicion}[2][]{%
\refstepcounter{defini}%
\ifstrempty{#1}%
{\mdfsetup{%
frametitle={%
\tikz[baseline=(current bounding box.east),outer sep=0pt]
\node[anchor=east,rectangle,fill=green!20]
{\strut Definición~\thedefini};}}
}%
{\mdfsetup{%
frametitle={%
\tikz[baseline=(current bounding box.east),outer sep=0pt]
\node[anchor=east,rectangle,fill=green!20]
{\strut Definición~\thedefini:~#1};}}%
}%
\mdfsetup{innertopmargin=10pt,linecolor=green!20,%
linewidth=2pt,topline=true,%
frametitleaboveskip=\dimexpr-\ht\strutbox\relax
}
\begin{mdframed}[]\relax%
\label{#2}}{\end{mdframed}}

%Proof
\newenvironment{prf}{\noindent\emph{Dem.}}{$\square$ \newline\vspace{5pt}}


%Corolario
\newcounter{cor}[chapter] \setcounter{cor}{0}
\renewcommand{\thecor}{\arabic{section}.\arabic{cor}}
\newenvironment{cor}[2][]{%
\refstepcounter{cor}%
\ifstrempty{#1}%
{\mdfsetup{%
frametitle={%
\tikz[baseline=(current bounding box.east),outer sep=0pt]
\node[anchor=east,rectangle,fill=green!20]
{\strut Corolario~\thelem};}}
}%
{\mdfsetup{%
frametitle={%
\tikz[baseline=(current bounding box.east),outer sep=0pt]
\node[anchor=east,rectangle,fill=green!20]
{\strut Corolario~\thelem:~#1};}}%
}%
\mdfsetup{innertopmargin=10pt,linecolor=green!20,%
linewidth=2pt,topline=true,%
frametitleaboveskip=\dimexpr-\ht\strutbox\relax
}
\begin{mdframed}[]\relax%
\label{#2}}{\end{mdframed}}

\tcbset{highlight math style={enhanced,
  colframe=red!60!black,colback=yellow!50!white,arc=4pt,boxrule=1pt,
  drop fuzzy shadow}}
  
  
  
  
  
  
  


\pagestyle{fancyplain}

 \renewcommand{\sectionmark}[1]
                 {\markright{\thesection\ #1}}


% \lhead[\fancyplain{}{\bfseries\thepage}]
%       {\fancyplain{}{\bfseries\rightmark}}
%
 \rhead[\fancyplain{}{\bfseries\leftmark}]{\fancyplain{}{\bfseries}}




 \lhead[\fancyplain{}{ \includegraphics[scale=.3]{EscudoUNLPam.png}}]{\fancyplain{}{ \includegraphics[scale=.3]{EscudoUNLPam.png}}}

\cfoot{}





  
  
  
  
  
  
  
\begin{document}


\hyphenation{excen-tri-ci-dad}


\begin{large}
\begin{bfseries} % \begin{scshape}
        \noindent Depto de Matem\'atica.\\
        Primer Cuatrimestre de 2022\\                                                                                                                                                                                                                                                                                                                                                
        Teoría de la Medida \\
        Práctica 2: Integral de Riemann

%\end{scshape}
\end{bfseries}
\end{large}
\par\noindent\rule{\textwidth}{.5pt}




\section{Práctica II: Integral de Riemann}

\begin{ejer}{} Obtener la fórmula del área de: 
\begin{enumerate}
\item rectángulos con lados  enteros, racionales  e irracionales; 
\item paralelogramos; 
\item triángulos; 
\item trapecios; 
\item polígonos regulares.
\end{enumerate}
\end{ejer}

\begin{ejer}{}  Explicar por qué si $P$ y $Q$ son particiones del mismo intervalo y $Q$ es un refinamiento de $P$
($Q\supseteq P$) y si $f$ es cualquier función acotada sobre el intervalo, entonces
\[
\underline{S}(P;f)\leq \underline{S}(Q;f)\leq \overline{S}(Q;f)\leq \overline{S}(P;f)
\]

\end{ejer}

 
\textbf{EN EL APUNTE:}
\begin{ejer}{} 
Sea $f:[a,b]\to\mathbb{R}$ una función acotada y $P,P'$ particiones de $[a,b]$ con $P\subset P'$. Demostrar que 
  \[
  \underline{S}(P,f)\leq \underline{S}(P',f)\quad\text{y}\quad \overline{S}(P',f)\leq \overline{S}(P,f).
 \]
 Inferir que para cualesquiera $P,P'$ (sin importar que una este o no contenida dentro de la otra)
   \[
\underline{S}(P,f)\leq \overline{S}(P,f).
 \]


\end{ejer}  

\begin{ejer}{} Considerar la función definida por 
\begin{equation}\label{eq:funcion-sumas-Darboux}
f(x)=\left
\{\begin{array}{ll}
1&x=0\\
x&0<x<1\\
0&x=1,
\end{array}
\right.
\end{equation}
y la partición $P=\left\{0,\frac{1}{4},\frac{1}{2},\frac{3}{4},1\right\}$ de $(0,1)$. 
\\
Hallar las sumas inferior y superior de Darboux $\underline{S}(P;f)$ y $\overline{S}(P;f)$.

\end{ejer}  

\begin{ejer}{} Usando la función $f$ definida por \eqref{eq:funcion-sumas-Darboux} y $\epsilon=\frac{1}{2}$, 
encontrar $\delta>0$ tal que para cualquier partición $P$ en intervalos de longitud menor que $\delta$, 
la diferencia entre $\overline{S}(P;f)$ y $\underline{S}(P;f)$ sea menor que $\frac{1}{2}$.

\end{ejer} 

\begin{ejer}{} Considerar la función 
\begin{equation}\label{eq:funcion-g}
g(x)=\sum_{n=1}^{\infty} \frac{1}{2^{2n-1}}\left\lfloor \frac{2^nx+1}{2}\right\rfloor\;\;0\leq x\leq 1,
\end{equation}
donde $\left\lfloor\alpha\right\rfloor$ denota el mayor entero menor o igual a $\alpha$.
\\
Mostrar que la serie converge para todo $x \in [0,1]$, que la función $g(x)$ es monótona creciente, que 
$g(0)=0$ y $g(1)=1$.
\\
Encontrar todos los puntos en los cuales la función $g$ es discontinua y en estos puntos calcular la diferencia
entre los límites laterales.
\end{ejer} 

\begin{ejer}{} Mostrar que la función $g$ definida en   \eqref{eq:funcion-g} es integrable de Riemann sobre $[0,1]$.
\end{ejer} 

\begin{ejer}{} Hallar el valor de $\int_0^1 g(x)\,dx$ siendo $g$ la función definida por \eqref{eq:funcion-g}. 
Indicar claramente los motivos que llevan a su conclusión.
\end{ejer}  

\begin{ejer}{} Encontrar todos los valores positivos de $\alpha$ para los cuales la integral impropia 
\[
\int_{-1}^1 \frac{1}{|x|^{\alpha}}\,dx
\]
tiene un valor. Explicar las razones que conducen a su conclusión.

\end{ejer}  

\begin{ejer}{} Probar que  $f(x)=\sum\limits_{n=1}^{\infty} \frac{((nx))}{n^2}$ converge uniformemente,
siendo
\[
((x))=\left\{
\begin{array}{ll}
x-\left\lfloor x\right\rfloor, &\left\lfloor x\right\rfloor  \leq x < \left\lfloor x\right\rfloor+\frac{1}{2}
\\
0&x=\left\lfloor x\right\rfloor+\frac{1}{2}
\\
x-\left\lfloor x\right\rfloor-1 & \left\lfloor x\right\rfloor+\frac{1}{2} <x< \left\lfloor x\right\rfloor+1.
\end{array}
\right.
\]


\end{ejer} 

\begin{ejer}{} Probar que $f$ es continua en $c$ si y sólo si la oscilación de $f$ en el punto $c$ es cero. 


\textbf{EN EL APUNTE: }

$f$ es continua en $x$ si y solo si $w(f;x)=0$.



\end{ejer}

\begin{ejer}{} Sea $f$ definida por $f(x)=\sen \left(\frac{1}{x}\right)$ si $x \neq 0$ y $f(0)=0$.
?`Cuál es la oscilación de $f$ en 0? Justificar su respuesta.

\end{ejer} 

\begin{ejer}{} Hallar la oscilación en $x=\frac{1}{3}$ de la función 
\[
g(x)=\left\{
\begin{array}{ll}
x & x\in \qq
\\
-x & x\notin\qq
\end{array}
\right.
\]
Justificar la respuesta.


\end{ejer} 

\begin{ejer}{}  Usando sumas de Darboux y \textbf{Python ????}, aproximar:  
\begin{enumerate}
\item $\int_1^2\frac{1}{x}\,dx $    y comparar con $\ln(2)$; 
\item  $\int_{-1}^{1}x^2\,dx$.\\
   ?`A qu\'e parecen aproximarse las sumas inferiores y superiores?
 \item  $\int_0^{\frac{1}{2}}\frac{1}{\sqrt{1-x^2}}\,dx$\\
    ?`Por qu\'e el resultado puede usarse para aproximar $\pi$?
\end{enumerate}    

\end{ejer}  

\begin{ejer}{} Usando sumas de Darboux obtener:
\begin{enumerate}
\item $\int_a^b x^2\,dx$;
\item $\int_a^b \cos x\,dx$;
\item $\int_a^b \frac{1}{x}\,dx=\ln b- \ln a$.
\end{enumerate}
\end{ejer}  

\textbf{EN EL APUNTE: }



\begin{ejer}{} 
Sea $0\leq a<b$ veamos que 
\[
 \int_a^b x^2 dx=\frac{b^3}{3}-\frac{a^3}{3}.
\]
{\em Ayuda:} Usar particiones uniformes y la fórmula $\sum\limits_{i=1}^nn^2= n(n+1)(2n+1)/6$.


\end{ejer}  

\begin{ejer}{} 
Sea $0\leq a<b$ y $n$ un entero negativo, veamos que
\[
 \int_a^b x^n dx=
 \begin{cases}
\frac{b^{n+1}}{n+1}-
\frac{a^{n+1}}{n+1} & \text{ si } n\neq -1\\
 \ln(b)-\ln(a) & \text{ si } n=-1\\
\end{cases}
\]

\end{ejer} 

\begin{ejer}{} 
 Sea $0\leq a<b\leq \pi$, veamos que
\[
 \int_a^b \sen x dx=-(\cos(b)-\cos(a)).
\]


\end{ejer} 

\begin{ejer}{} 
Sea $n(x)=\left
\{
\begin {array}{ll}
1&x=\frac{1}{n},\;\;n\in \nn.\\
0&\mbox{en otro caso.}
\end{array}
\right.$
\\
Probar que $n$ es integrable sobre $[0,1]$ y que $\int n(x)\,dx=0$.


\end{ejer}

\begin{ejer}{} 
Sea $p(x)=\left
\{
\begin {array}{ll}
0&x=0\\
\frac{1}{x}-\left\lfloor \frac{1}{x}\right\rfloor&\mbox{en otro caso}
\end{array}
\right.$
\\
Probar que $p$ es integrable sobre $[0,1]$.


\end{ejer} 

\begin{ejer}{} Hallar el contenido exterior de los siguientes conjuntos: 
\begin{enumerate}
\item $\qq \cap [0,1]$;
\item $[0,1]-\qq$;
\item $\left\{\frac{n}{n+1}\left. \right|\; n\in\nn\right\}$;
\item $\left\{\frac{2k-1}{2^n}\left. \right|\; n\in\nn, 1\leq k\leq 2^{n-1}\right\}$;
\item $\left\{\frac{k}{n}\left. \right|\; n\in\nn, \;k=1,2 \mbox{ \'o  } 3\right\}$;
\item $(0,1)\cup(3,4)$;
\item $\bigcup\limits_{n=1}^{\infty} \left(\frac{1}{2n},\frac{1}{2n-1}\right)$.
\end{enumerate}


\end{ejer}  

\begin{ejer}{} Probar que si $S$ tiene contenido exterior 0 y $T$ es cualquier conjunto acotado, entonces
\[
c_e(S\cup T)=c_e(T)
\]
donde $c_e$ se lee como \textit{contenido exterior}.

\end{ejer} 


\end{document}
